\documentclass[a4paper,twoside,12pt,AutoFakeBold]{ctexart}
\newcommand\specialsectioning{\setcounter{secnumdepth}{-2}}
\specialsectioning
\usepackage[center]{titlesec}
\usepackage{indentfirst}
\setlength{\parindent}{2em}
\usepackage{fancyhdr}
\pagestyle{fancy}%fancy style
\fancyhf{}%清空页眉页脚
\fancyhead[LE,RO]{\thepage}%页码位置:偶数页居左,奇数页居右
\fancyfoot[RO,RE]{\textit{Wuyunfei's note}}% 设置页脚:在每页的右下脚以斜体显示书名
\usepackage{graphicx}
\setlength{\headheight}{15pt}%解决页眉warnings
\usepackage{amsmath}
\renewcommand{\headrulewidth}{0pt} % 页眉与正文之间的水平线粗细
\renewcommand{\footrulewidth}{0pt}

\usepackage{changepage}%设置引用段落左右侧缩进
\newCJKfontfamily\mincho{IPAexMincho}%日语明朝体
\usepackage{tabularx}
\usepackage{hyperref}%设置超链接
\usepackage{float}
\title{马克思主义理论学习笔记}
\author{吴云飞}
\date{}
\usepackage{perpage}
\MakePerPage{footnote}
\begin{document}
\maketitle

\newpage

\tableofcontents

\newpage

\section{序言}
\begin{adjustwidth}{2em}{2em}
\qquad\fangsong 笔者是一名马克思主义理论的硕士研究生,本书记载了笔者学习马克思主义相关理论的过程,由于笔者个人的专业性有限,因而本书的主要目的是记录笔者的思考过程。本书主要以马克思主义的原著为分析原本(在后面的章节会编入一些专题类的论文,如数理马克思主义政治经济学\footnote{关于数理马克思主义政治经济学的部分,笔者将采取\textbf{个人讲义}的形式进行编写,为了使读者们能够不至于被复杂的数学推导过程唬住,以保证读者们的阅读兴趣,因此涉及到较为复杂的公式推理往往会略去一部分。}、生态政治经济学等部分),在此基础上,添加一些笔者对于某些论述的感悟与理解。

本书首先从恩格斯的国民经济学批判大纲开始,逐一探讨马克思主义理论家们所著的短篇文章。由于笔者的兴趣点主要集中于政治经济学的方面,因而对某些内容的理解难免会具有一定的片面性,又由于笔者个人学术水平的稚嫩,因而对某些观点的叙述难免会显得重复啰嗦,望请读者见谅。

无产阶级在进行斗争的过程中需要先进的思想作为自身的武器,知识不应被特权阶级(资产阶级)所垄断。且由于本书会随着笔者学习的深入而不断更新其内容,故决定将本书开源放置在本人的github上,读者们可以随时获取最新版本\footnote{本书采用LaTeX进行编写

\quad github地址:\url{https://github.com/Imheaven233/Wuyunfei}}。

本书以“原文”+“笔记”的形式进行编写,正如海涅所言:“箭一离开弦便不再属于射手了,言论一离开说话人的口,尤其是经过大量印刷之后,便不再属于他了”\footnote{见海涅《论德国宗教和哲学的历史》},因此笔者希望读者在本书有关\textbf{马恩原著}的章节中,能够先阅读原文,进行一定的思考,再去阅读笔者的笔记,以免由于笔者个人的粗浅理解影响读者对伟大思想家们深邃理论的认识。

\end{adjustwidth}

\newpage

\section{国民经济学批判大纲}

\subsection{1.原文}

国民经济学的产生是商业扩展的自然结果,随着它的出现,一个成熟的允许欺诈的体系、一门完整的发财致富的科学代替了简单的不科学的生意经。\footnote{弗·恩格斯大约写于1843年9月底或10月初—1844年1月中,载于1844年2月《德法年鉴》,原文是德文。}

这种从商人的彼此妒忌和贪婪中产生的国民经济学或发财致富的科学,在额角上带有最令人厌恶的自私自利的烙印。人们还有一种幼稚的看法,以为金银就是财富,因此必须到处从速禁止“贵”金属出口。各国像守财奴一样相互对立,双手抱住自己珍爱的钱袋,怀着妒忌心和猜疑心注视着自己的邻居。他们使用一切手段尽可能多地骗取那些与自己通商的民族的现钱,并使这些侥幸赚来的钱好好地保持在关税线以内。

如果完全彻底地实行这个原则,那就会葬送商业。因此,人们便开始跨越这个最初的阶段。他们意识到,放在钱柜里的资本是死的,而流通中的资本会不断增殖。于是,人们变得比较友善了,人们开始把自己的杜卡特\footnote{14—19世纪欧洲许多国家通用的金币。——编者注 }当做诱鸟放出去,以便把别人的杜卡特一并引回来,并且认识到,多花一点钱买甲的商品一点也不会吃亏,只要能以更高的价格把它卖给乙就行了。

重商主义体系就建立在这个基础上。商业的贪婪性已多少有所遮掩;各国多少有所接近,开始缔结通商友好条约,彼此做生意,并且为了获得更大的利润,甚至尽可能地互相表示友爱和亲善。但是实质上还是同从前一样贪财和自私,当时一切基于商业角逐而引起的战争就时时露出这种贪财和自私。这些战争也表明:贸易和掠夺一样,是以强权为基础的;人们只要认为哪些条约最有利,他们就甚至会昧着良心使用诡计或暴力强行订立这些条约。

贸易差额论是整个重商主义体系的要点。正因为人们始终坚持金银就是财富的论点,所以他们认为只有那最终给国家带来现金的交易才是赢利交易。为了说明这一点,他们以输出和输入作比较。如果输出大于输入,那么他们就认为这个差额会以现金的形式回到本国,国家也因这个差额而更富裕。因此经济学家的本事就是要设法使输出和输入到每年年底有一个顺差。为了这样一个可笑的幻想,竟有成千上万的人被屠杀!商业也有了它的十字军征讨和宗教裁判所。

18世纪这个革命的世纪使经济学也发生了革命。然而,正如这个世纪的一切革命都是片面的并且停留在对立的状态中一样,正如抽象的唯物主义和抽象的唯灵论相对立,共和国和君主国相对立,社会契约和神权相对立一样,经济学的革命也未能克服对立。到处依然存在着下述前提:唯物主义不抨击基督教对人的轻视和侮辱,只是把自然界当做一种绝对的东西来代替基督教的上帝而与人相对立;政治学没有想去检验国家的各个前提本身;经济学没有想去过问\textbf{私有制的合理性}的问题。因此,新的经济学只前进了半步;它不得不背弃和否认它自己的前提,不得不求助于诡辩和伪善,以便掩盖它所陷入的矛盾,以便得出那些不是由它自己的前提而是由这个世纪的人道精神得出的结论。这样,经济学就具有仁爱的性质;它不再宠爱生产者,而转向消费者了;它假惺惺地对重商主义体系的血腥恐怖表示神圣的厌恶,并且宣布商业是各民族、各个人之间的友谊和团结的纽带。一切都显得十分辉煌壮丽,可是上述前提马上又充分发挥作用,而且创立了与这种伪善的博爱相对立的马尔萨斯人口论,这种理论是迄今存在过的体系中最粗陋最野蛮的体系,是一种彻底否定关于仁爱和世界公民的一切美好言词的绝望体系;这些前提创造并发展了工厂制度和现代的奴隶制度,这种奴隶制度就它的无人性和残酷性来说不亚于古代的奴隶制度。新的经济学,即以亚当·斯密的《国富论》\footnote{亚·斯密《国民财富的性质和原因的研究》1776年伦敦版。——编者注}为基础的自由贸易体系,也同样是伪善、前后不一贯和不道德的。这种伪善、前后不一贯和不道德目前在一切领域中与自由的人性处于对立的地位。

可是,难道说亚当·斯密的体系不是一个进步吗?当然是进步,而且是一个必要的进步。为了使私有制的真实的后果能够显露出来,就有必要摧毁重商主义体系以及它的垄断和它对商业关系的束缚;为了使当代的斗争能够成为普遍的人类的斗争,就有必要使所有这些地域的和国家的小算盘退居次要的地位;有必要使私有制的理论抛弃纯粹经验主义的、仅仅是客观主义的研究方法,并使它具有一种也对结果负责的更为科学的性质,从而使问题涉及全人类的范围;有必要通过对旧经济学中包含的不道德加以否定的尝试,并通过由此产生的伪善——这种尝试的必然结果——而使这种不道德达于极点。这一切都是理所当然的。我们乐于承认,只有通过对贸易自由的论证和阐述,我们才有可能超越私有制的经济学,然而我们同时也应该有权指出,这种贸易自由并没有任何理论价值和实践价值。

我们所要评判的经济学家离我们的时代越近,我们对他们的判决就必定越严厉。因为斯密和马尔萨斯所看到的现成的东西只不过是一些片断,而在新近的经济学家面前却已经有了一个完整的体系;一切结论已经作出,各种矛盾已经十分清楚地显露出来,但是,他们仍不去检验前提,而且还是对整个体系负责。经济学家离我们的时代越近,离诚实就越远。时代每前进一步,为把经济学保持在时代的水平上,诡辩术就必然提高一步。因此,比如说,\textbf{李嘉图}的罪过比\textbf{亚当·斯密}大,而\textbf{麦克库洛赫}和\textbf{穆勒}的罪过又比李嘉图大。

新近的经济学甚至不能对重商主义体系作出正确的评判,因为它本身就带有片面性,而且还受到重商主义体系的各个前提的拖累。只有摆脱这两种体系的对立,批判这两种体系的共同前提,并从纯粹人的、普遍的基础出发来看问题,才能够给这两种体系指出它们的真正的地位。那时大家就会明白,贸易自由的捍卫者是一些比旧的重商主义者本身更为恶劣的垄断者。那时大家就会明白,在新经济学家的虚伪的人道背后隐藏着旧经济学家闻所未闻的野蛮;旧经济学家的概念虽然混乱,与攻击他们的人的口是心非的逻辑比较起来还是单纯的、前后一贯的;这两派中任何一派对另一派的指责,都不会不落到自己头上。因此,新的自由主义经济学也无法理解李斯特为什么要恢复重商主义体系\footnote{弗·李斯特《政治经济学的国民体系》第 1卷《国际贸易、贸易政策和德国关税同盟》1841年斯图加特—蒂宾根版。——编者注},而这件事我们却觉得很简单。前后不一贯的和具有两面性的自由主义经济学必然要重新分解为它的基本组成部分。正如神学不回到迷信,就得前进到自由哲学一样,贸易自由必定一方面造成垄断的恢复,另一方面造成私有制的消灭。

自由主义经济学达到的唯一肯定的进步,就是阐述了私有制的各种规律。这种经济学确实包含这些规律,虽然这些规律还没有被阐述为最后的结论,还没有被清楚地表达出来。由此可见,在涉及确定生财捷径的一切地方,就是说,在一切严格意义的经济学上的争论中,贸易自由的捍卫者们是正确的。当然,这里指的是与支持垄断的人争论,而不是与反对私有制的人争论,因为正如英国社会主义者早就在实践中和理论上证明的那样\footnote{指约·弗·布雷、威·汤普森、约·瓦茨和他们的著作:布雷《劳动的不公正现象及其解决办法,或强权时代和公正时代》1839年利兹版;汤普森《最能促进人类幸福的财富分配原理的研究》1824年伦敦版;瓦茨《政治经济学家的事实和臆想:科学原则述评,去伪存真》1842年曼彻斯特—伦敦版。——编者注},反对私有制的人能够从经济的观点比较正确地解决经济问题。

因此,我们在批判国民经济学时要研究它的基本范畴,揭露自由贸易体系所产生的矛盾,并从这个矛盾的两个方面作出结论。 

国民财富这个用语是由于自由主义经济学家努力进行概括才产生的。只要私有制存在一天,这个用语便没有任何意义。英国人的“国民财富”很多,他们却是世界上最穷的民族。人们要么完全抛弃这个用语,要么采用一些使它具有意义的前提。国民经济学,政治经济学,公共经济学等用语也是一样。在目前的情况下,应该把这种科学称为私经济学,因为在这种科学看来,社会关系只是为了私有制而存在。 

私有制产生的最直接的结果就是\textbf{商业},即彼此交换必需品,亦即买和卖。在私有制的统治下,这种商业与其他一切活动一样,必然是经商者收入的直接源泉;就是说,每个人必定要尽量设法贱买贵卖。因此,在任何一次买卖中,两个人总是以绝对对立的利益相对抗;

这种冲突带有势不两立的性质,因为每一个人都知道另一个人的意图,知道另一个人的意图是和自己的意图相反的。因此,商业所产生的第一个后果是:一方面互不信任,另一方面为这种互不信任辩护,采取不道德的手段来达到不道德的目的。例如,商业的第一条原则就是对一切可能降低有关商品的价格的事情都绝口不谈,秘而不宣。由此可以得出结论:在商业中允许利用对方的无知和轻信来取得最大利益,并且也同样允许夸大自己的商品本来没有的品质。总而言之,商业是合法的欺诈。任何一个商人,只要他说实话,他就会证明实践是符合这个理论的。

重商主义体系在某种程度上还具有某种纯朴的天主教的坦率精神,它丝毫不隐瞒商业的不道德的本质。我们已经看到,它怎样公开地显露自己卑鄙的贪婪。18世纪民族间的相互敌视、可憎的妒忌以及商业角逐,都是贸易本身的必然结果。社会舆论既然还不具有人道精神,那么何必要掩饰从商业本身的无人性的和充满敌意的本质中所产生的那些东西呢?

但是,当\textbf{经济学的路德}\footnote{马克思在《1844年经济学哲学手稿》中对这个提法作了解释,见本卷第178—179页。——编者注},即亚当·斯密,批判过去的经济学的时候,情况大大地改变了。时代具有人道精神了,理性起作用了,道德开始要求自己的永恒权利了。强迫订立的通商条约、商业战争、民族间的严重孤立状态与前进了的意识异常激烈地发生冲突。新教的伪善代替了天主教的坦率。斯密证明,人道也是由商业的本质产生的,商业不应当是“纠纷和敌视的最丰产的源泉”,而应当是“各民族、各个人之间的团结和友谊的纽带”(参看《国富论》第4卷第3章第2节);理所当然的是,商业总的说来对它的一切参加者都是有利的。

斯密颂扬商业是人道的,这是对的。世界上本来就没有绝对不道德的东西;商业也有对道德和人性表示尊重的一面。但这是怎样的尊重啊!当中世纪的强权,即公开的拦路行劫转到商业时,这种行劫就变得具有人道精神了;当商业上以禁止货币输出为特征的第一个阶段转到重商主义体系时,商业也变得具有人道精神了。现在连这种体系本身也变得具有人道精神了。当然,商人为了自己的利益必须与廉价卖给他货物的人们和高价买他的货物的人们保持良好的关系。因此,一个民族要是引起它的供应者和顾客的敌对情绪,就太不明智了。它表现得越友好,对它就越有利。这就是商业的人道,而滥用道德以实现不道德的意图的伪善方式就是自由贸易体系引以自豪的东西。伪君子叫道:难道我们没有打倒垄断的野蛮吗?难道我们没有把文明带往世界上遥远的地方吗?难道我们没有使各民族建立起兄弟般的关系并减少了战争次数吗?不错,这一切你们都做了,然而你们是怎样做的啊!你们消灭了小的垄断,以便使一个巨大的根本的垄断,即所有权,更自由地、更不受限制地起作用;你们把文明带到世界的各个角落,以便赢得新的地域来扩张你们卑鄙的贪欲;你们使各民族建立起兄弟般的关系——但这是盗贼的兄弟情谊;你们减少了战争次数,以便在和平时期赚更多的钱,以便使各个人之间的敌视、可耻的竞争战争达到登峰造极的地步!你们什么时候做事情是从纯粹的人道出发,是从普遍利益和个人利益之间的对立毫无意义这种意识出发的呢?你们什么时候讲过道德,而不图谋私利,不在心底隐藏一些不道德的、利己的动机呢?

自由主义的经济学竭力用瓦解各民族的办法使敌对情绪普遍化,使人类变成一群正因为每一个人具有与其他人相同的利益而互相吞噬的凶猛野兽——竞争者不是凶猛野兽又是什么呢?自由主义的经济学做完这个准备工作之后,只要再走一步——使家庭解体——就达到目的了。为了实现这一点,它自己美妙的发明即工厂制度助了它一臂之力。共同利益的最后痕迹,即家庭的财产共有被工厂制度破坏了,至少在这里,在英国已处在瓦解的过程中。孩子一到能劳动的时候,就是说,到了九岁,就靠自己的工钱过活,把父母的家只看做一个寄宿处,付给父母一定的膳宿费。这已经是很平常的事了。还能有别的什么呢?从构成自由贸易体系的基础的利益分离,还能产生什么别的结果呢?一种原则一旦被运用,它就会自行贯穿在它的一切结果中,不管经济学家们是否乐意。

然而,经济学家自己也不知道他在为什么服务。他不知道,他的全部利己的论辩只不过构成人类普遍进步的链条中的一环。他不知道,他瓦解一切私人利益只不过替我们这个世纪面临的大转变,即人类与自然的和解以及人类本身的和解开辟道路。

商业形成的第一个范畴是价值。关于这个范畴和其他一切范畴,在新旧两派经济学家之间没有什么争论,因为直接热衷于发财致富的垄断主义者没有多余时间来研究各种范畴。关于这类论点的所有争论都出自新近的经济学家。

靠种种对立活命的经济学家当然也有一种双重的价值:抽象价值(或实际价值)和交换价值。关于实际价值的本质,英国人和法国人萨伊进行了长期的争论。前者认为生产费用是实际价值的表现,后者则说什么实际价值要按物品的效用来测定。这个争论从本世纪初开始,后来停息了,没有得到解决。这些经济学家是什么问题也解决不了的。

这样,英国人——特别是麦克库洛赫和李嘉图——断言,物品的抽象价值是由生产费用决定的。请注意,是抽象价值,不是交换价值,不是 exchangeable value,不是商业价值;至于商业价值,据说完全是另外一回事。为什么生产费用是价值的尺度呢?请听!请听!因为在通常情况下,如果把竞争关系撇开,没有人会把物品卖得低于它的生产费用。没有人会卖吧?在这里,既然不谈商业价值,我们谈“卖”干什么呢?一谈到“卖”,我们就要让我们刚才要撇开的商业重新参加进来,而且是这样一种商业!一种不把主要的东西即竞争关系考虑在内的商业!起初我们有一种抽象价值,现在又有一种抽象商业,一种没有竞争的商业,就是说有一个没有躯体的人,一种没有产生思想的大脑的思想。难道经济学家根本没有想到,一旦竞争被撇开,那就保证不了生产者正是按照他的生产费用来卖自己的商品吗?多么混乱啊!

还不仅如此!我们暂且认为,一切都像经济学家所说的那样。假定某人花了很大的力气和巨大的费用制造了一种谁也不要的毫无用处的东西,难道这个东西的价值也同生产费用一样吗?经济学家回答说,绝对没有,谁愿意买这种东西呢?于是,我们立刻不仅碰到了萨伊的声名狼藉的效用,而且还有了随着“买”而来的竞争关系。经济学家是一刻也不能坚持他的抽象的——这是做不到的。不仅他所竭力避开的竞争,而且连他所攻击的效用,随时都可能突然出现在他面前。抽象价值以及抽象价值由生产费用决定的说法,恰恰都只是抽象的非实在的东西。

我们再一次暂且假定经济学家是对的,那么在不把竞争考虑在内的情况下,他又怎样确定生产费用呢?我们研究一下生产费用,就可以看出,这个范畴也是建立在竞争的基础上的。在这里又一次表明经济学家是无法贯彻他的主张的。

如果我们转向萨伊的学说,我们也会发现同样的抽象。物品的效用是一种纯主观的根本不能绝对确定的东西,至少它在人们还在对立中徘徊的时候肯定是不能确定的。根据这种理论,生活必需品应当比奢侈品具有更大的价值。在私有制统治下,竞争关系是唯一能比较客观地、似乎能大体确定物品效用大小的办法,然而恰恰是竞争关系被撇在一边。但是,只要容许有竞争关系,生产费用也就随之产生,因为没有人会卖得低于他自己在生产上投入的费用。因此,在这里也是对立的一方不情愿地转到另一方。

让我们设法来澄清这种混乱吧!物品的价值包含两个因素,争论的双方都要强行把这两个因素分开,但正如我们所看到的,这是徒劳的。价值是生产费用对效用的关系。价值首先是用来决定某种物品是否应该生产,即这种物品的效用是否能抵偿生产费用。然后才谈得上运用价值来进行交换。如果两种物品的生产费用相等,那么效用就是确定它们的比较价值的决定性因素。

这个基础是交换的唯一正确的基础。可是,如果以这个基础为出发点,那么又该谁来决定物品的效用呢?单凭当事人的意见吗?这样总会有一人受骗。或者,是否有一种不取决于当事人双方、不为当事人所知悉、只以物品固有的效用为依据的规定呢?这样,交换就只能强制进行,并且每一个人都认为自己受骗了。不消灭私有制,就不可能消灭物品固有的实际效用和这种效用的规定之间的对立,以及效用的规定和交换者的自由之间的对立;而私有制一旦被消灭,就无须再谈现在这样的交换了。到那个时候,价值概念的实际运用就会越来越限于决定生产,而这也是它真正的活动范围。

然而,目前的情况怎样呢?我们看到,价值概念被强行分割了,它的每一个方面都叫嚷自己是整体。一开始就为竞争所歪曲的生产费用,应该被看做是价值本身。纯主观的效用同样应该被看做是价值本身,因为现在不可能有第二种效用。要把这两个跛脚的定义扶正,必须在两种情况下都把竞争考虑在内;而这里最有意思的是:在英国人那里,竞争代表效用而与生产费用相对立,在萨伊那里则相反,竞争带来生产费用而与效用相对立。但是,竞争究竟带来什么样的效用和什么样的生产费用!它带来的效用取决于偶然情况、时尚和富人的癖好,它带来的生产费用则随着需求和供给的偶然比例而上下波动。

实际价值和交换价值之间的差别基于下述事实:物品的价值不同于人们在买卖中为该物品提供的那个所谓等价物,就是说,这个等价物并不是等价物。这个所谓等价物就是物品的价格,如果经济学家是诚实的,他就会把等价物一词当做“商业价值”来使用。但是,为了使商业的不道德不过于明显地暴露出来,他总得保留一点假象,似乎价格和价值以某种方式相联系。说价格由生产费用和竞争的相互作用决定,这是完全正确的,而且是私有制的一个主要的规律。经济学家的第一个发现就是这个纯经验的规律;接着他从这个规律中抽去他的实际价值,就是说,抽去竞争关系均衡时、供求一致时的价格,这时,剩下的自然只有生产费用了,经济学家就把它称为实际价值,其实只是价格的一种规定性。但是,这样一来,经济学中的一切就被本末倒置了:价值本来是原初的东西,是价格的源泉,倒要取决于价格,即它自己的产物。大家知道,正是这种颠倒构成了抽象的本质。关于这点,请参看费尔巴哈的著作。\footnote{路·费尔巴哈《关于哲学改革的临时纲要》,见《德国现代哲学和政论界轶文集》1843年苏黎世—温特图尔版第64—71页。——编者注}

在经济学家看来,商品的生产费用由以下三个要素组成:生产原材料所必需的土地的地租,资本及其利润,生产和加工所需要的劳动的报酬。但人们立即就发现,资本和劳动是同一个东西,因为经济学家自己就承认资本是“积蓄的劳动”\footnote{亚·斯密《国民财富的性质和原因的研究》1828年爱丁堡版第2卷第94页。——编者注}。这样,我们这里剩下的就只有两个方面,自然的、客观的方面即土地和人的、主观的方面即劳动。劳动包括资本,并且除资本之外还包括经济学家没有想到的第三要素,我指的是简单劳动这一肉体要素以外的发明和思想这一精神要素。经济学家与发明的精神有什么关系呢?难道没有他参与的一切发明就不会落到他手里吗?有哪一件发明曾经使他花费过什么?因此,他在计算他的生产费用时为什么要为这些发明操心呢?在他看来,财富的条件就是土地、资本、劳动,除此以外,他什么也不需要。科学是与他无关的。尽管科学通过贝托莱、戴维、李比希、瓦特、卡特赖特等人送了许多礼物给他,把他本人和他的生产都提到空前未有的高度,可是这与他有何相干呢?他不懂得重视这些东西,科学的进步超出了他的计算。但是,在一个超越利益的分裂——正如在经济学家那里发生的那样——的合理状态下,精神要素自然会列入生产要素,并且会在经济学的生产费用项目中找到自己的位置。到那时,我们自然会满意地看到,扶植科学的工作也在物质上得到报偿,会看到,仅仅詹姆斯·瓦特的蒸汽机这样一项科学成果,在它存在的头50年中给世界带来的东西就比世界从一开始为扶植科学所付出的代价还要多。

这样,我们就有了两个生产要素——自然和人,而后者还包括他的肉体活动和精神活动。现在我们可以回过来谈谈经济学家和他的生产费用。

经济学家说,凡是无法垄断的东西就没有价值。这个论点以后再详细研究。如果我们说:凡是无法垄断的东西就没有价格,那么,这个论点对于以私有制为基础的状态而言是正确的。如果土地像空气一样容易得到,那就没有人会支付地租了。既然情况不是这样,而是在一种特殊情况下被占有的土地的面积是有限的,那人们就要为一块被占有的即被垄断的土地支付地租或者按照售价把它买下来。令人感到奇怪的是,在这样弄明白了土地价值的产生以后,还得听经济学家说什么地租是付租金的土地的收入和值得费力耕种的最坏的土地的收入之间的差额。大家知道,这是李嘉图第一次充分阐明的地租定义。\footnote{大·李嘉图《政治经济学和赋税原理》1817年伦敦版第54页。——编者注}当人们假定需求的减少马上影响地租并立刻使相应数量的最坏耕地停止耕种的时候,这个定义实际上是正确的。但情况并不是这样,因此这个定义是有缺陷的;况且这个定义没有包括地租产生的原因,仅仅由于这一点,这个定义就已经站不住脚了。反谷物法同盟盟员托·佩·汤普森上校在反对这个定义时,又把亚当·斯密的定义\footnote{亚·斯密《国民财富的性质和原因的研究》1828年爱丁堡版第1卷第237—242页。——编者注}搬了出来并加以论证。据他说,地租是谋求使用土地者的竞争和可支配的土地的有限数量之间的关系。在这里,这至少又回到地租产生的问题上来了;但是,这个解释没有包括土壤肥力的差别,正如上述的定义忽略了竞争一样。\footnote{托·佩·汤普森《真正的地租理论,驳李嘉图先生等》,见他的《政治习作及其他》1842年伦敦版第4卷第404页。——编者注}

这样一来,同一个对象又有了两个片面的因而是不完全的定义。正如研究价值概念时一样,在这里我们也必须把这两个定义结合起来,以便得出一个正确的、来自事物本身发展的、因而包括了实践中的一切情况的定义。地租是土地的收获量即自然方面(这方面又包括自然的肥力和人的耕作即改良土壤所耗费的劳动)和人的方面即竞争之间的相互关系。经济学家会对这个“定义”摇头;当他们知道这个定义包括了有关这个问题的一切时,他们会大吃一惊的。

土地占有者无论如何不能责备商人。

他靠垄断土地进行掠夺。他利用人口的增长进行掠夺,因为人口的增长加强了竞争,从而抬高了他的土地的价值。他把不是通过他个人劳动得来的、完全偶然地落到他手里的东西当做他个人利益的源泉进行掠夺。他靠出租土地、靠最终攫取租地农场主的种种改良的成果进行掠夺。大土地占有者的财富日益增长的秘密就在于此。

认定土地占有者的获得方式是掠夺,即认定人人都有享受自己的劳动产品的权利或不播种者不应有收获,这样的公理\footnote{亚·斯密《国民财富的性质和原因的研究》1828年爱丁堡版第1卷第85—86页。——编者注}并不是我们的主张。第一个公理排除抚育儿童的义务;第二个公理排除任何世代的生存权利,因为任何世代都得继承前一世代的遗产。确切地说,这些公理都是由私有制产生的结论。要么实现由私有制产生的一切结论,要么抛弃私有制这个前提。

甚至最初的占有本身,也是以断言老早就存在过共同占有权为理由的。因此,不管我们转向哪里,私有制总会把我们引到矛盾中去。

土地是我们的一切,是我们生存的首要条件;出卖土地,就是走向自我出卖的最后一步;这无论过去或直至今日都是这样一种不道德,只有自我出让的不道德才能超过它。最初的占有土地,少数人垄断土地,所有其他的人都被剥夺了基本的生存条件,就不道德来说,丝毫也不逊于后来的土地出卖。

如果我们在这里再把私有制撇开,那么地租就恢复它的本来面目,就归结为实质上可以作为地租基础的合理观点。这时,作为地租而与土地分离的土地价值,就回到土地本身。这个价值是依据面积相等的土地在花费的劳动量相等的条件下所具有的生产能力来计算的;这个价值在确定产品的价值时自然是作为生产费用的一部分计算在内的,它像地租一样是生产能力对竞争的关系,不过是对真正的竞争,即对某个时候会展开的竞争的关系。

我们已经看到,资本和劳动最初是同一个东西;其次,我们从经济学家自己的阐述中也可以看到,资本是劳动的结果,它在生产过程中立刻又变成了劳动的基质、劳动的材料;可见,资本和劳动的短暂分开,立刻又在两者的统一中消失了;但是,经济学家还是把资本和劳动分开,还是坚持这两者的分裂,他只在资本是“积蓄的劳动”这个定义\footnote{亚·斯密《国民财富的性质和原因的研究》1828年爱丁堡版第2卷第94页。——编者注}中承认它们两者的统一。由私有制造成的资本和劳动的分裂,不外是与这种分裂状态相应的并从这种状态产生的劳动本身的分裂。这种分开完成之后,资本又分为原有资本和利润,即资本在生产过程中所获得的增长额,虽然实践本身立刻又将这种利润加到资本上,并把它和资本投入周转中。甚至利润又分裂为利息和本来意义上的利润。在利息中,这种分裂的不合理性达到顶点。贷款生息,即不花劳动单凭贷款获得收入,是不道德的,虽然这种不道德已经包含在私有制中,但毕竟还是太明显,并且早已被不持偏见的人民意识看穿了,而人民意识在认识这类问题上通常总是正确的。所有这些微妙的分裂和划分,都产生于资本和劳动的最初的分开和这一分开的完成,即人类分裂为资本家和工人。这一分裂正日益加剧,而且我们将看到,它必定会不断地加剧。但是,这种分开与我们考察过的土地同资本和劳动分开一样,归根结底是不可能的。我们根本无法确定在某种产品中土地、资本和劳动各占多少分量。这三个量是不可通约的。土地出产原材料,但这里并非没有资本和劳动;资本以土地和劳动为前提,而劳动至少以土地,在大多数场合还以资本为前提。这三者的作用截然不同,无法用任何第四种共同的尺度来衡量。因此,如果在当前的条件下,将收入在这三种要素之间进行分配,那就没有它们固有的尺度,而只有由一个完全异己的、对它们来说是偶然的尺度即竞争或者强者狡诈的权利来解决。地租包含着竞争;资本的利润只有由竞争决定,至于工资的情况怎样,我们立刻就会看到。

如果我们撇开私有制,那么所有这些反常的分裂就不会存在。利息和利润的差别也会消失;资本如果没有劳动、没有运动就是虚无。利润把自己的意义归结为资本在决定生产费用时置于天平上的砝码,它仍是资本所固有的部分,正如资本本身将回到它与劳动的最初统一体一样。

劳动是生产的主要要素,是“财富的源泉”\footnote{亚·斯密《国民财富的性质和原因的研究》1828年爱丁堡版第1卷第9—10页。——编者注},是人的自由活动,但很少受到经济学家的重视。正如资本已经同劳动分开一样,现在劳动又再度分裂了;劳动的产物以工资的形式与劳动相对立,它与劳动分开,并且通常又由竞争决定,因为,正如我们所看到的,没有一个固定的尺度来确定劳动在生产中所占的比重。只要我们消灭了私有制,这种反常的分离就会消失;劳动就会成为它自己的报酬,而以前被让渡的工资的真正意义,即劳动对于确定物品的生产费用的意义,也就会清清楚楚地显示出来。

我们知道,只要私有制存在一天,一切终究会归结为竞争。竞争是经济学家的主要范畴,是他最宠爱的女儿,他始终娇惯和爱抚着她,但是请看,在这里出现的是一张什么样的美杜莎的怪脸。

私有制的最直接的结果是生产分裂为两个对立的方面:自然的方面和人的方面,即土地和人的活动。土地无人施肥就会荒芜,成为不毛之地,而人的活动的首要条件恰恰是土地。其次,我们看到,人的活动又怎样分解为劳动和资本,这两方面怎样彼此敌视。这样,我们已经看到的是这三种要素的彼此斗争,而不是它们的相互支持;现在,我们还看到私有制使这三种要素中的每一种都分裂。一块土地与另一块土地对立,一个资本与另一个资本对立,一个劳动力与另一个劳动力对立。换句话说,因为私有制把每一个人隔离在他自己的粗陋的孤立状态中,又因为每个人和他周围的人有同样的利益,所以土地占有者敌视土地占有者,资本家敌视资本家,工人敌视工人。在相同利益的敌对状态中,正是由于利益的相同,人类目前状态的不道德已经达到极点,而这个极点就是竞争。

竞争的对立面是垄断。垄断是重商主义者战斗时的呐喊,竞争是自由主义经济学家厮打时的吼叫。不难看出,这个对立面也是完全空洞的东西。每一个竞争者,不管他是工人,是资本家,或是土地占有者,都必定希望取得垄断地位。每一个较小的竞争者群体都必定希望为自己取得垄断地位来对付所有其他的人。竞争建立在利益基础上,而利益又引起垄断;简言之,竞争转为垄断。另一方面,垄断挡不住竞争的洪流;而且,它本身还会引起竞争,正如禁止输入或高额关税直接引起走私一样。竞争的矛盾和私有制本身的矛盾是完全一样的。单个人的利益是要占有一切,而群体的利益是要使每个人所占有的都相等。因此,普遍利益和个人利益是直接对立的。竞争的矛盾在于:每个人都必定希望取得垄断地位,可是群体本身却因垄断而一定遭受损失,因此一定要排除垄断。此外,竞争已经以垄断即所有权的垄断为前提——这里又暴露出自由主义者的虚伪——,而且只要所有权的垄断存在着,垄断的所有权也同样是正当的,因为垄断一经存在,它就是所有权。可见,攻击小的垄断,保留根本的垄断,这是多么可鄙的不彻底啊!前面我们已经提到过经济学家的论点,凡是无法垄断的东西就没有价值,因此,凡是不容许垄断的东西就不可能卷入这个竞争的斗争;如果我们再把经济学家的这个论点引到这里来,那么我们关于竞争以垄断为前提的论断,就被证明是完全正确的了。

竞争的规律是:需求和供给始终力图互相适应,而正因为如此,从未有过互相适应。双方又重新脱节并转化为尖锐的对立。供给总是紧跟着需求,然而从来没有达到过刚好满足需求的情况;供给不是太多,就是太少,它和需求永远不相适应,因为在人类的不自觉状态下,谁也不知道需求和供给究竟有多大。如果需求大于供给,价格就会上涨,因而供给似乎就会兴奋起来;只要市场上供给增加,价格又会下跌,而如果供给大于需求,价格就会急剧下跌,因而需求又被激起。情况总是这样;从未有过健全的状态,而总是兴奋和松弛相更迭——这种更迭排斥一切进步——一种达不到目的的永恒波动。这个规律永远起着平衡的作用,使在这里失去的又在那里获得,因而经济学家非常欣赏它。这个规律是他最大的荣誉,他简直百看不厌,甚至在一切可能的和不可能的条件下都对它进行观察。然而,很明显,这个规律是纯自然的规律,而不是精神的规律。这是一个产生革命的规律。经济学家用他那绝妙的供求理论向你们证明“生产永远不会过多”\footnote{亚·斯密《国民财富的性质和原因的研究》1828年爱丁堡版第1卷第97页。——编者注},而实践却用商业危机来回答,这种危机就像彗星一样定期再现,在我们这里现在是平均每五年到七年发生一次。 80年来,这些商业危机像过去的大瘟疫一样定期来临,而且它们造成的不幸和不道德比大瘟疫所造成的更大(参看威德《中等阶级和工人阶级的历史》1835年伦敦版第211页)。当然,这些商业革命证实了这个规律,完完全全地证实了这个规律,但不是用经济学家想使我们相信的那种方式证实的。我们应该怎样理解这个只有通过周期性的革命才能为自己开辟道路的规律呢?这是一个以当事人的无意识活动为基础的自然规律。如果生产者自己知道消费者需要多少,如果他们把生产组织起来,并且在他们中间进行分配,那么就不会有竞争的波动和竞争引起危机的倾向了。你们有意识地作为人,而不是作为没有类意识的分散原子进行生产吧,你们就会摆脱所有这些人为的无根据的对立。但是,只要你们继续以目前这种无意识的、不假思索的、全凭偶然性摆布的方式来进行生产,那么商业危机就会继续存在;而且每一次接踵而来的商业危机必定比前一次更普遍,因而也更严重,必定会使更多的小资本家变穷,使专靠劳动为生的阶级人数以增大的比例增加,从而使待雇劳动者的人数显著地增加——这是我们的经济学家必须解决的一个主要问题——,最后,必定引起一场社会革命,而这一革命,经济学家凭他的书本知识是做梦也想不到的。

由竞争关系造成的价格永恒波动,使商业完全丧失了道德的最后一点痕迹。至于价值就无须再谈了。这种似乎非常重视价值并以货币的形式把价值的抽象推崇为一种特殊存在物的制度,本身就通过竞争破坏着一切物品所固有的任何价值,而且每日每时改变着一切物品相互的价值关系。在这个漩涡中,哪里还可能有建立在道德基础上的交换呢?在这种持续地不断涨落的情况下,每个人都必定力图碰上最有利的时机进行买卖,每个人都必定会成为投机家,就是说,都企图不劳而获,损人利己,算计别人的倒霉,或利用偶然事件发财。投机者总是指望不幸事件,特别是指望歉收,他们利用一切事件,例如,当年的纽约大火灾\footnote{指1835年12月16日在纽约发生的火灾。——编者注 };而不道德的顶点还是交易所中有价证券的投机,这种投机把历史和历史上的人类贬低为那种用来满足善于算计或伺机冒险的投机者的贪欲的手段。但愿诚实的、“正派的”商人不以“我感谢你上帝”等表面的虔诚形式摆脱交易所投机。这种商人和证券投机者一样可恶,他也同他们一样地投机倒把,他必须投机倒把,竞争迫使他这样做,所以他的买卖也与证券投机者的勾当一样不道德。竞争关系的真谛就是消费力对生产力的关系。在一种与人类相称的状态下,不会有除这种竞争之外的别的竞争。社会应当考虑,靠它所支配的资料能够生产些什么,并根据生产力和广大消费者之间的这种关系来确定,应该把生产提高多少或缩减多少,应该允许生产或限制生产多少奢侈品。但是,为了正确地判断这种关系,判断从合理的社会状态下能期待的生产力提高的程度,请读者参看英国社会主义者的著作\footnote{约·弗·布雷《劳动的不公正现象及其解决办法,或强权时代和公正时代》1839年利兹版;威·汤普森《最能促进人类幸福的财富分配原理的研究》1824年伦敦版;约·瓦茨《政治经济学家的事实和臆想:科学原则述评,去伪存真》1842年曼彻斯特—伦敦版。——编者注}并部分地参看傅立叶的著作\footnote{沙·傅立叶《关于四种运动和普遍命运的理论》1841年巴黎第2版和《经济的和协作的新世界,或按情欲分类的引人入胜的和合乎自然的劳动方式的发现》1829年巴黎版。——编者注}。

在这种情况下,主体的竞争,即资本对资本、劳动对劳动的竞争等等,被归结为以人的本性为基础并且到目前为止只有傅立叶作过差强人意的说明的竞赛\footnote{沙·傅立叶《关于四种运动和普遍命运的理论》1841年巴黎第2版第175、244—245、265和434—436页。——编者注},这种竞赛将随着对立利益的消除而被限制在它特有的和合理的范围内。

资本对资本、劳动对劳动、土地对土地的斗争,使生产陷于高烧状态,使一切自然的合理的关系都颠倒过来。要是资本不最大限度地展开自己的活动,它就经不住其他资本的竞争。要是土地的生产力不经常提高,耕种土地就会无利可获。要是工人不把自己的全部力量用于劳动,他就对付不了自己的竞争者。总之,卷入竞争斗争的人,如果不全力以赴,不放弃一切真正人的目的,就经不住这种斗争。一方的这种过度紧张,其结果必然是另一方的松弛。在竞争的波动不大,需求和供给、消费和生产几乎彼此相等的时候,在生产发展过程中必定会出现这样一个阶段,在这个阶段,生产力大大过剩,结果,广大人民群众无以为生,人们纯粹由于过剩而饿死。长期以来,英国就处于这种荒诞的状况中,处于这种极不合理的情况下。如果生产波动得比较厉害——这是这种状态的必然结果——,那么就会出现繁荣和危机、生产过剩和停滞的反复交替。经济学家从来就解释不了这种怪诞状况;为了解释这种状况,他发明了人口论,这种理论和当时这种贫富矛盾同样荒谬,甚至比它更荒谬。经济学家不敢正视真理,不敢承认这种矛盾无非是竞争的结果,因为否则他的整个体系就会垮台。

在我们看来,这个问题很容易解释。人类支配的生产力是无法估量的。资本、劳动和科学的应用,可以使土地的生产能力无限地提高。按照最有才智的经济学家和统计学家的计算(参看艾利生的《人口原理》第1卷第 1、2章),“人口过密”的大不列颠在十年内,将使粮食生产足以供应六倍于目前人口的需要。资本日益增加,劳动力随着人口的增长而增长,科学又日益使自然力受人类支配。这种无法估量的生产能力,一旦被自觉地运用并为大众造福,人类肩负的劳动就会很快地减少到最低限度。要是让竞争自由发展,它虽然也会起同样的作用,然而是在对立之中起作用。一部分土地进行精耕细作,而另一部分土地——大不列颠和爱尔兰的3 000万英亩好地——却荒芜着。一部分资本以难以置信的速度周转,而另一部分资本却闲置在钱柜里。一部分工人每天工作 14或16小时,而另一部分工人却无所事事,无活可干,活活饿死。或者,这种分立现象并不同时发生:今天生意很好,需求很大,这时,大家都工作,资本以惊人的速度周转着,农业欣欣向荣,工人干得累倒了;而明天停滞到来,农业不值得费力去经营,大片土地荒芜,资本在正在流动的时候凝滞,工人无事可做,整个国家因财富过剩、人口过剩而备尝痛苦。

经济学家不能承认事情这样发展是对的,否则,他就得像上面所说的那样放弃自己的全部竞争体系,就得认识到自己把生产和消费对立起来、把人口过剩和财富过剩对立起来是荒诞无稽的。但是,既然事实是无法否认的,为了使这种事实与理论一致,就发明了人口论。

这种学说的创始人马尔萨斯断言,人口总是威胁着生活资料,一当生产增加,人口也以同样比例增加,人口固有的那种其繁衍超过可支配的生活资料的倾向,是一切贫困和罪恶的原因。因此,在人太多的地方,就应当用某种方法把他们消灭掉:或者用暴力将他们杀死,或者让他们饿死。可是这样做了以后,又会出现一个空隙,这个空隙又会马上被另一次繁衍的人口填满,于是,以前的贫困又开始到来。据说在任何条件下都是如此,不仅在文明的状态下,而且在自然的状态下都是如此;新荷兰\footnote{澳大利亚的旧称。——编者注}平均每平方英里只有一个野蛮人,却也和英国一样,深受人口过剩的痛苦。简言之,要是我们愿意首尾一贯,那我们就得承认:当地球上只有一个人的时候,就已经人口过剩了。从这种阐述得出的结论是:正因为穷人是过剩人口,所以,除了尽可能减轻他们饿死的痛苦,使他们相信这是无法改变的,他们整个阶级的唯一出路是尽量减少生育,此外就不应该为他们做任何事情;或者,如果这样做不行,那么最好还是像“马尔库斯”所建议的那样,建立一种国家机构,用无痛苦的办法把穷人的孩子杀死;按照他的建议,每一个工人家庭只能有两个半小孩,超过此数的孩子用无痛苦的办法杀死。施舍被认为是犯罪,因为这会助长过剩人口的增长;但是,把贫穷宣布为犯罪,把济贫所变为监狱——这正是英国通过“自由的”新济贫法已经做的——,却算是非常有益的事情。的确,这种理论很不符合圣经关于上帝及其创造物完美无缺的教义,但是“动用圣经来反驳事实,是拙劣的反驳!”\footnote{托·卡莱尔《宪章运动》1840年伦敦版第109页。——编者注}

我是否还需要更详尽地阐述这种卑鄙无耻的学说,这种对自然和人类的恶毒诬蔑,并进一步探究其结论呢?在这里我们终于看到,经济学家的不道德已经登峰造极。一切战争和垄断制度所造成的灾难,与这种理论相比,又算得了什么呢?要知道,正是这种理论构成了自由派的自由贸易体系的拱顶石,这块石头一旦坠落,整个大厦就倾倒。因为竞争在这里既然已经被证明是贫困、穷苦、犯罪的原因,那么谁还敢对竞争赞一词呢?

艾利生在上面引用过的著作中动摇了马尔萨斯的理论,他诉诸土地的生产力,并用以下的事实来反对马尔萨斯的原理:每一个成年人能够生产出多于他本人消费所需的东西。如果不存在这一事实,人类就不可能繁衍,甚至不可能生存;否则成长中的一代依靠什么来生活呢?\footnote{阿·艾利生《人口原理及其和人类幸福的关系》1840年爱丁堡—伦敦版第33—82页。——编者注}可是,艾利生没有深入事物的本质,因而他最后也得出了同马尔萨斯一样的结论。他虽然证明了马尔萨斯的原理是不正确的,但未能驳倒马尔萨斯据以提出他的原理的事实。

如果马尔萨斯不这样片面地看问题,那么他必定会看到,人口过剩或劳动力过剩是始终与财富过剩、资本过剩和地产过剩联系着的。只有在整个生产力过大的地方,人口才会过多。从马尔萨斯写作时起\footnote{托·罗·马尔萨斯《人口原理》第1版于1798年在伦敦出版。——编者注},任何人口过剩的国家的情况,尤其是英国的情况,都极其明显地证实了这一点。这是马尔萨斯应当从总体上加以考察的事实,而对这些事实的考察必然会得出正确的结论;他没有这样做,而是只选出一个事实,对其他事实不予考虑,因而得出荒谬的结论。他犯的第二个错误是把生活资料和就业手段混为一谈。人口总是威胁着就业手段,有多少人能够就业,就有多少人出生,简言之,劳动力的产生迄今为止由竞争的规律来调节,因而也同样要经受周期性的危机和波动,这是事实,确定这一事实是马尔萨斯的功绩。\footnote{托·罗·马尔萨斯《人口原理》1826年伦敦版第1卷第18—21页。——编者注}然而,就业手段并不就是生活资料。就业手段由于机器力和资本的增加而增加,这是仅就其最终结果而言;而生活资料,只要生产力稍有提高,就立刻增加。这里暴露出经济学的一个新的矛盾。经济学家所说的需求不是现实的需求,他所说的消费只是人为的消费。在经济学家看来,只有能够为自己取得的东西提供等价物的人,才是现实的需求者,现实的消费者。但是,如果事实是这样:每一个成年人生产的东西多于他本人所消费的东西;小孩像树木一样能够绰绰有余地偿还花在他身上的费用——难道这不是事实?——,那么就应该认为,每一个工人必定能够生产出远远多于他所需要的东西,因此,社会必定会乐意供给他所必需的一切;同时也应该认为,大家庭必定是非常值得社会向往的礼物。但是,由于经济学家观察问题很粗糙,除了以可触摸的现金向他支付的东西以外,他不知道还有任何别的等价物。他已深陷在自己的对立物中,以致连最令人信服的事实也像最科学的原理一样使他无动于衷。

我们干脆用扬弃矛盾的方法消灭矛盾。只要目前对立的利益能够融合,一方面的人口过剩和另一方面的财富过剩之间的对立就会消失,关于一国人民纯粹由于富裕和过剩而必定饿死这种不可思议的事实,这种比一切宗教中的一切奇迹的总和更不可思议的事实就会消失,那种认为土地无力养活人们的荒谬见解也就会消失。这种见解是基督教经济学的顶峰,——而我们的经济学本质上是基督教经济学,这一点我可以用任何命题和任何范畴加以证明,这个工作在适当的时候我会做的;马尔萨斯的理论只不过是关于精神和自然之间存在着矛盾和由此而来的关于二者的堕落的宗教教条在经济学上的表现。我希望也在经济学领域揭示这个对宗教来说并与宗教一起早就解决了的矛盾的虚无性。同时,如果马尔萨斯理论的辩护人事先不能用这种理论的原则向我解释,一国人民怎么能够纯粹由于过剩而饿死,并使这种解释同理性和事实一致起来,那我就不会认为这种辩护是站得住脚的。

可是,马尔萨斯的理论却是一个推动我们不断前进的、绝对必要的中转站。我们由于他的理论,总的来说由于经济学,才注意到土地和人类的生产力,而且我们在战胜了这种经济学上的绝望以后,就保证永远不惧怕人口过剩。我们从马尔萨斯的理论中为社会变革汲取到最有力的经济论据,因为即使马尔萨斯完全正确,也必须立刻进行这种变革,原因是只有这种变革,只有通过这种变革来教育群众,才能够从道德上限制繁殖本能,而马尔萨斯本人也认为这种限制是对付人口过剩的最有效和最简易的办法。\footnote{托·罗·马尔萨斯《人口原理》1826年伦敦版第2卷第255—269页。——编者注}我们由于这个理论才开始明白人类的极端堕落,才了解这种堕落依存于竞争关系;这种理论向我们指出,私有制如何最终使人变成了商品,使人的生产和消灭也仅仅依存于需求;它由此也指出竞争制度如何屠杀了并且每日还在屠杀着千百万人;这一切我们都看到了,这一切都促使我们要用消灭私有制、消灭竞争和利益对立的办法来消灭这种人类堕落。

然而,为了驳倒对人口过剩普遍存在的恐惧所持的根据,让我们再回过来谈生产力和人口的关系。马尔萨斯把自己的整个体系建立在下面这种计算上:人口按几何级数1+2+4+8+16+32……增加,而土地的生产力按算术级数1+2+3+4+5+6增加。\footnote{同上,第1卷第11页。——编者注 }差额是明显的、触目惊心的,但这是否对呢?在什么地方证明过土地的生产能力是按算术级数增加的呢?土地的扩大是受限制的。好吧。在这个面积上使用的劳动力随着人口的增加而增加。即使我们假定,由于增加劳动而增加的收获量,并不总是与劳动成比例地增加,这时仍然还有一个第三要素,一个对经济学家来说当然是无足轻重的要素——科学,它的进步与人口的增长一样,是永无止境的,至少也是与人口的增长一样快。仅仅一门化学,光是汉弗莱·戴维爵士和尤斯图斯·李比希两人,就使本世纪的农业获得了怎样的成就?可见科学发展的速度至少也是与人口增长的速度一样的;人口与前一代人的人数成比例地增长,而科学则与前一代人遗留的知识量成比例地发展,因此,在最普通的情况下,科学也是按几何级数发展的。而对科学来说,又有什么是做不到的呢?当“密西西比河流域有足够的荒地可容下欧洲的全部人口”\footnote{约·瓦茨《政治经济学家的事实和臆想》1842年曼彻斯特—伦敦版第21页。——编者注}的时候,当地球上的土地才耕种了三分之一,而这三分之一的土地只要采用现在已经人所共知的改良耕作方法,就能使产量提高五倍、甚至五倍以上的时候,谈论什么人口过剩,岂不是非常可笑的事情。

这样,竞争就使资本与资本、劳动与劳动、土地占有与土地占有对立起来,同样又使这些要素中的每一个要素与其他两个要素对立起来。力量较强的在斗争中取得胜利。要预卜这个斗争的结局,我们就得研究一下参加斗争的各方的力量。首先,土地占有或资本都比劳动强,因为工人要生活就得工作,而土地占有者可以靠地租过活,资本家可以靠利息过活,万不得已时,也可以靠资本或资本化了的土地占有过活。其结果是:劳动得到的仅仅是最必需的东西,仅仅是一点点生活资料,而大部分产品则为资本和土地占有所得。此外,较强的工人把较弱的工人,较大的资本把较小的资本,较大的土地占有把小土地占有从市场上排挤出去。实践证实了这个结果。大家都知道,大厂主和大商人比小厂主和小商人占优势,大土地占有者比只有一摩尔根土地的占有者占优势。其结果是:在通常情况下,按照强者的权利,大资本和大土地占有吞并小资本和小土地占有,就是说,产生了财产的集中。在商业危机和农业危机时期,这种集中就进行得更快。一般说来,大的财产比小的财产增长得更快,因为从收入中作为占有者的费用所扣除的部分要小得多。这种财产的集中是一个规律,它与所有其他的规律一样,是私有制所固有的;中间阶级必然越来越多地消失,直到世界分裂为百万富翁和穷光蛋、大土地占有者和贫穷的短工为止。任何法律,土地占有的任何分割,资本的任何偶然的分裂,都无济于事,这个结果必定会产生,而且就会产生,除非在此之前全面变革社会关系、使对立的利益融合、使私有制归于消灭。

作为当今经济学家主要口号的自由竞争,是不可能的事情。垄断至少具有使消费者不受欺骗的意图,虽然它不可能实现这种意图。消灭垄断就会为欺骗敞开大门。你们说,竞争本身是对付欺骗的办法,谁也不会去买坏的东西;照这样说来,每个人都必须是每一种商品的行家,而这是不可能的,由此可见,垄断是必要的,这种必要性也在许多商品中表现出来。药房等等必须实行垄断。最重要的商品即货币恰好最需要垄断。每当流通手段不再为国家所垄断的时候,这种手段就引起商业危机,因此,英国的经济学家,其中包括威德博士,也认为在这里有实行垄断的必要。\footnote{约·威德《中等阶级和工人阶级的历史》1835年伦敦第3版第152—160页。——编者注 }但是,垄断也不能防止假币。随便你站在问题的哪一方面,一方面的困难与另一方面的困难都不相上下。垄断引起自由竞争,自由竞争又引起垄断;因此,二者一定都失败,而且这些困难只有在消灭了产生这二者的原则时才能消除。

竞争贯穿在我们的全部生活关系中,造成了人们今日所处的相互奴役状况。竞争是强有力的发条,它一再促使我们的日益陈旧而衰退的社会秩序,或者更正确地说,无秩序状况活动起来,但是,它每努力一次,也就消耗掉一部分日益衰败的力量。竞争支配着人类在数量 上的增长,也支配着人类在道德上的进步。谁只要稍微熟悉一下犯罪统计,他就会注意到,犯罪行为按照特有的规律性年年增加,一定的原因按照特有的规律性产生一定的犯罪行为。工厂制度的扩展到处引起犯罪行为的增加。我们能够精确地预计一个大城市或者一个地区每年会发生的逮捕、刑事案件,以至凶杀、抢劫、偷窃等事件的数字,在英国就常常这样做。这种规律性证明犯罪也受竞争支配,证明社会产生了犯罪的需求,这个需求要由相应的供给来满足;它证明由于一些人被逮捕、放逐或处死所形成的空隙,立刻会有其他的人来填满,正如人口一有空隙立刻就会有新来的人填满一样;换句话说,它证明了犯罪威胁着惩罚手段,正如人口威胁着就业手段一样。别的且不谈,在这种情况下对罪犯的惩罚究竟公正到什么程度,我让我的读者去判断。我认为这里重要的是:证明竞争也扩展到了道德领域,并表明私有制使人堕落到多么严重的地步。

在资本和土地反对劳动的斗争中,前两个要素比劳动还有一个特殊的优越条件,那就是科学的帮助,因为在目前情况下连科学也是用来反对劳动的。例如,几乎一切机械发明,尤其是哈格里沃斯、克朗普顿和阿克莱的棉纺机,都是由于缺乏劳动力而引起的。对劳动的渴求导致发明的出现,发明大大地增加了劳动力,因而降低了对人的劳动的需求。1770年以来英国的历史不断地证明了这一点。棉纺业中最近的重大发明——自动走锭纺纱机——就完全是由于对劳动的需求和工资的提高引起的;这项发明使机器劳动增加了一倍,从而把手工劳动减少了一半,使一半工人失业,因而也就降低另一半工人的工资;这项发明破坏了工人对工厂主的反抗,摧毁了劳动在坚持与资本作力量悬殊的斗争时的最后一点力量(参看尤尔博士《工厂哲学》第2卷\footnote{安·尤尔《工厂哲学:或论大不列颠工厂制度的科学、道德和商业的经济》1835年伦敦修订第2版第366—373页。——编者注})。诚然,经济学家说,归根结底,机器对工人是有利的,因为机器能够降低生产费用,因而替产品开拓新的更广大的市场,这样,机器最终还能使失业工人重新就业。这完全正确,但是,劳动力的生产是受竞争调节的;劳动力始终威胁着就业手段,因而在这些有利条件出现以前就已经有大量寻求工作的竞争者等待着,于是有利的情况形同虚构,而不利的情况,即一半工人突然被剥夺生活资料而另一半工人的工资被降低,却决非虚构,这一点为什么经济学家就忘记了呢?发明是永远不会停滞不前的,因而这种不利的情况将永远继续下去,这一点为什么经济学家就忘记了呢?由于我们的文明,分工无止境地增多,在这种情况下,一个工人只有在一定的机器上被用来做一定的细小的工作才能生存,成年工人几乎在任何时候都根本不可能从一种职业转到另一种新的职业,这一点为什么经济学家又忘记了呢?

考虑到机器的作用,我有了另一个比较远的题目即工厂制度;但是,现在我既不想也没有时间来讨论这个题目。不过,我希望不久能够有机会来详细地阐述这个制度的极端的不道德,并且无情地揭露经济学家在这里表现得十分出色的那种伪善。 

\newpage

\subsection{2.笔记}
恩格斯认为国民经济学的产生就是商业扩展的结果,是一种自然的现象,是现实的商业发展的理论化的表现。而这种早期的理论表现便是重商主义。

恩格斯指出,\textbf{贸易差额论}是整个重商主义体系的要点。恩格斯说,要从“纯粹人的、普遍的基础出发来看问题”。但是当时的恩格斯似乎没有意识到,“纯粹人的、普遍的”基础本身或许就不存在。

恩格斯认为在批判国民经济学时要研究它的基本范畴,揭露自由贸易体系所产生的矛盾。这一点是正确的,强有力的批判正是要深入问题的实质,从内部瓦解敌人。

在这里,恩格斯提出了著名的“两个和解”。原文是这么说的:

“\begin{fangsong}
经济学家自己也不知道他在为什么服务。他不知道,他的全部利己的论辩只不过构成人类普遍进步的链条中的一环。他不知道,他瓦解一切私人利益只不过替我们这个世纪面临的大转变,即人类与自然的和解以及人类本身的和解开辟道路。    
\end{fangsong}”

关于“两个和解”这一命题,许多学者对它的认识是处于一定的混乱之中的,这一命题似乎被过分地滥用了,退一步说,至少是被过分地局限于它的字面意思了。接下来我说的话或许会引起一些争议,但还是希望读者可以深入思考一下。目前,存在部分研究人员认为恩格斯所述的\textbf{人类本身的和解}不过是对于个别主体而言的\textbf{自我和解},是在伦理学意义上的道德呼吁。笔者认为这种理解仅仅局限于外在经验,而并没有深入问题的实质,在这里,\textbf{人类}并非是具体的个人,而是一种\textbf{超感性的集合},是一种\textbf{抽象的}人的集合。从这个意义上而言,“人类本身的和解”绝不是对于个别主体而言的“自我和解”,恰恰相反,它是代表着人类的不同主体之间的矛盾的消除,且这种矛盾的消除不但不是通过温和的自我和解而实现的,相反,这种矛盾是通过对抗性的冲突而消解的。因此,事实上,恩格斯在这里真正想要论述的是,在私有制下,问题的本质不在于要实现“两个和解”\footnote{因为只有在私有制的框架下才会产生人与自然、人与人之间的对立,\textbf{私有制的现实性便是这种和解的非现实性}。},而在于变革那使得人与自然之间的对立存在的社会条件,即资本主义的社会制度。
\newpage
\section{1844年经济学哲学手稿}
\subsection{1.原文节选}
\subsubsection{私有财产的关系}
〔……〕〔XL〕构成他的资本的利息。因此,资本是完全失去自身的人这种情况在工人身上主观地存在着,正像劳动是失去自身的人这种情况在资本身上客观地存在着一样。但是,工人不幸而成为一种活的、因而是贫困的资本,这种资本只要一瞬间不劳动便失去自己的利息,从而也失去自己的生存条件。作为资本,工人的价值按照需求和供给而增长,而且,从肉体上来说,他的存在、他的生命也同其它任何商品一样,过去和现在都被看成是商品的供给。工人生产资本,资本生产工人,因而工人生产自身,而且人做为工人、做为商品就是这整个运动的产物。人只不过是工人,并且做为工人,他的特性只有在这些特性对异己的资本来说是存在的时候才存在。但是,因为资本和工人彼此是异己的,从而处于漠不关心的、外部的和偶然的相互关系中,所以这种异己性也必然现实地表现出来。因此,资本一但想到-不管是必然地还是任意地想到-不再对工人存在,工人自己对自己说来便不再存在:他没有工作,因而也没有工资,并且因为他不是作为人,而是作为工人存在,所以他就会被人埋葬,会饿死,等等。工人有当他对自己作为资本存在的时候,才作为工人存在;而他只有当某种资本对他存在的时候,才作为资本存在。资本的存在便是他的存在、他的生活,资本的存在以一种他无法干预的方式来规定他的生活的内容。因此,国民经济学不知道有失业的工人,即处于这种劳动关系之外的劳动人。小偷、骗子、乞丐,失业的、快饿死的、贫穷的和犯罪的劳动人,他们都是些在国民经济学看来并不存在,而只有在其它人眼中,在医生、法官、掘墓人、乞丐管理人等等的眼中才存在的人物;他们是一些国民经济学领域之外游荡的幽灵。因此,在国民经济学看来,工人的需要不过是维持工人在劳动期间的生活的需要,而且只限于保持工人后代不致死绝的程度。因此,工资就与其它任何生产工具的保养和维修,与资本连同利息的再生产所需要的一般资本的消费,与为了保持车轮运转而加的润滑油,具有完全相同的意义。可见,工资是资本和资本家的必要费用之一,并且不得不超出这个必要的需要。因此,英国工厂主在1834年实行济贫法以前,把工人靠济贫税得到的社会救济金从他的工资中扣除,并且把这种救济金看作工资的一个组成部分,这种做法是完全合乎逻辑的。

生产不仅把人当作商品、当作商品人、当作具有商品的规定的人生产出来;它依照这个规定把人当作即在精神上又在肉体上非人化的存在物生产出来。-工人和资本家的不道德、退化、愚钝。-这种生产的产品是自我意识的和自主活动的商品……商品人……李嘉图、穆勒等人比斯密和萨伊进了一大步,他们把人的存在-人这种商品的或高或低的生产率-说成是无关紧要的,甚至是有害的。在他们看来,生产的真正目的不是一笔资本养活多少工人,而是它带来多少利息,每年总共积攒多少钱。同样,现代英国国民经济学的一个合乎逻辑的大进步是,它把劳动提高为国民经济学的惟一原则,同时十分清楚地揭示了工资和资本利息之间的反比例关系,指出资本家通常只有通过降低工资才能增加收益,反之则降低收益。不是对消费者诈取,而是资本家和工人彼此诈取,才是正常的关系。

私有财产的关系潜在地包含着作为劳动的私有财产的关系和作为资本的私有财产的关系,以及这两种表现的相互关系。一方面是作为劳动,即作为对自身、对人和自然界因而也对意识和生命表现说来完全异己的活动的人的活动的生产,是人作为单纯的劳动人的抽象存在,因而这种劳动人每天都可能由他的充实的无沦为绝对的无,沦为他的社会的从而也是现实的非存在。另一方面是作为资本的人的活动的对象的生产,在这里,对象的一切自然的社会的规定性都消失了,在这里私有财产丧失了自己的自然的和社会的特质(因而也丧失了一切政治的和社会的幻象,甚至连表面上的人的关系也没有了),在这里同一个资本在各种不同的自然的和社会的存在中始终是同一的,而完全不管它的现实内容如何。劳动和资本的这种对立一到达极端,就必然成为全部私有财产关系的顶点、最高阶段和灭亡。因此,现代英国国民经济学的又一重大成就是:它指明了地租是最坏耕地的利息和最好耕地的利息之间的差额,揭示了土地所有者的浪漫主义想象-他的所谓社会重要性和他的利益同社会利益的一致性,而这一点是亚当·斯密继重农学派之后主张过的;它预料到并且准备了这样一个现实的运动:使土地所有者变成极其普通的、平庸的资本家,从而使对立简化和尖锐化,并加速这种对立的消灭。这样一来,作为土地的土地,作为地租的地租,就失去自己的等级的差别而变成毫无意义的,或者毋宁说,只表示货币意义的资本和利息。

资本和土地的差别,利润和地租的差别,这二者和工资的差别,工业和农业之间、私有的不动产和动产之间的差别,仍然是历史的差别,而不是基于事物本质的差别。这种差别是资本和劳动之间的对立形成和产生的一个固定环节。同不动的地产相反,在工业等等中只表现出工业产生的方式以及工业在其中得到发展的那个与农业的对立。这种差别只要在下述情况下就作为特殊种类的活动,作为一个本质的、重要的、包括全部生活的差别而存在:工业(城市生活)同地产(贵族生活(封建生活))对立而形成,并且工业本身在垄断、行会、同业公会和社团等形式中还带有自己对立面的封建性质;而在这些形式的规定内,劳动还具有表面上的社会意义,现实的共同体的意义,还没有达到对自己的内容漠不关心,没有达到完全自为的存在的地步,就是说,还没有从其它一切存在中抽象出来,从而也还没有成为获得行动自由的资本。

但是,获得行动自由的、本身有单独构成的工业和获得行动自由的资本是劳动的必然发展。工业对它的对立面的支配立即表现在作为真正工业活动的农业的产生上,而过去农业是把主要工作交给土地和耕种这块土地的奴隶去做的。随着奴隶转化为自由工人即雇佣工人,地主本身便实际上转化为工厂主、资本家,而这种转化最初是通过租地农场主这个中介环节实现的。但是,租地农场主是土地所有者的代表,是土地所有者的公开秘密;只有依靠租地农场主,土地所有者才有经济上的存在,才能作为私有者存在,-因为他的土地的地租只有依靠租地农场主的竞争才能获得。因此,地主通过租地农场主本质上已经变成普通的资本家。而这种情况也必然在现实中发生:经营农业的资本家即租地农场主必然要成为地主,或者相反。租地农场主的以产业形式牟利就是土地所有者的以产业形式牟利,因为前者的存在设定后者的存在。

但是,当土地所有者和资本家想起自己的对立面的产生,回想起自己的来历时:土地所有者把资本家看做自己的骄傲起来的、发了财的、昨天的奴隶,并且看出他对自己这个资本家的威胁;而资本家则把土地所有者看作自己游手好闲的、残酷无情的(自私自利的)、昨天的主人;他知道土地所有者会使他这个资本家受损害,虽然土地所有者今天的整个社会地位、财产和享受都应归功于工业;资本家把土地所有者看成自由的工业和摆脱任何自然规定的自由的资本的直接对立面。他们之间的这种对立是极其激烈的,并且双方各自说出对方的真相。只要看一看不动产对动产的攻击,并且反过来看一看,动产对不动产的攻击,就对双方的卑鄙行有一个明确的概念。土地所有者炫耀他的财产的贵族渊源、封建的往昔纪念(怀旧)、他的诗意的回忆、他的耽于幻想的气质、他的政治上的重要性等等,而如果他用国民经济学的语言来表达,那末他就会说:只有农业才是生产的。同时,他把自己的对手描绘为狡黠诡诈的,兜售叫卖的,吹毛求疵的,坑蒙拐骗的,贪婪成性的,见钱眼开的,图谋不轨的,没有心干和丧尽天良的,背离社会和出卖社会利益的,放高利贷的,牵线撮合的,奴颜婢膝的,阿谀奉承的,圆滑世故的,招摇撞骗的,冷漠生硬的,制造、助长和纵容竞争、赤贫和犯罪的、败坏一切社会纲纪的,没有廉耻、没有原则、没有诗意、没有实体、心灵空虚的贪财恶棍(此外,见其中的重农学派贝尔加斯的著作,对他,卡米耶·德穆兰在自己的杂志《法国革命和布拉班特革命》中曾经严厉地批评过;并见冯·芬克、兰齐措勒、哈勒、莱奥\footnote{见爱好夸张的老年黑格尔派神学家丰克的著作,他满眼含泪,按照莱奥先生的说法讲述了在废除农奴制时一个奴隶如何不肯不再充当贵族的财产。并见尤斯图斯·莫泽尔的《爱国主义的幻想》,这些幻想的特色是它们一刻也没有超出循规蹈矩的庸人的那种小市民的、“家传的”、平庸的狭隘眼界;虽然如此,它们仍不失为纯粹的幻想。这个矛盾也使这些幻想如此投合德国人的口味。}、科泽加滕。)

动产也显示工业和运动的奇迹,它是现代之子,现代的合法的嫡子;它很遗憾自己的对手是一个不理解自己本质(而这是完全对的),想用粗野的、不道德的暴力和农奴制来代替道德的资本和自由的劳动的蠢人;它把他描绘成用率直坦诚、一本正经、为普遍利益服务、坚贞不渝这些假面具来掩盖缺乏活动能力、贪得无餍的享乐欲、自私自利、斤斤计较和居心不良的唐·吉诃德。它宣布他的对手是诡计多端的垄断者;它从历史发展上并用嘲讽的口气历数他的以罗曼蒂克的城堡为作坊的下流、残忍、挥霍、淫逸、寡廉鲜耻、无法无天和大逆不道,来给他的怀旧、他的诗意、他的幻想浇冷水。

据说,动产已经使人民获得了政治的自由,解脱了束缚市民社会的桎梏,把各领域彼此联成一体,创造了博爱的商业、纯粹的道德、温文尔雅的教养;它给人民以文明的需要来代替粗陋的需要,并提供了满足需要的手段;而土地所有者这个无所事事的、只会碍事的粮食投机商则抬高人民最必须的生活资料的价格,从而迫使资本家提高工资而不能提高生产力;因此,土地所有者妨碍国民年收入的增长,阻碍资本的积累,从而减少人民就业和国家增加财富的可能性;最后使这种可能性完全消失,引起普遍的衰退,并且像高利贷一样剥削现代文明的一切利益,而没有对它做丝毫贡献,甚至不放弃自己的封建偏见。最后,让土地所有者来看一看自己的租地农场主-对土地所有者来说,农业和土地本身仅仅作为赐给他的财源而存在,-并且让他说说,他是不是这样一个一本正经的、非凡的、狡猾的无赖:不管他曾怎样反对工业和商业,也不管他曾怎样絮絮叨叨地数说历史的回忆以及伦理的和政治的目的,他早已在心理并且在实际上属于自由的工业和可爱的商业了。土地所有者实际上提出的有利于自己的一切,只有用在耕作者(资本家和雇农)身上才是符合事实的,而土地所有者不如说是耕作者的敌人;因此,土地所有者作了不利于自身的论证。据说,没有资本,地产就是死的、无价值的物质。资本的文明的胜利恰恰在于,资本发现并促进使人的劳动代替死的物而成为财富的源泉。(见保罗·路易·库里埃、圣西门、加尼耳、李嘉图、穆勒、麦克库洛赫、德斯杜特·德·特拉西和米歇尔·舍伐利埃的著作。)

从现实的发展进程中(这里插一句)必然产生出资本家对土地所有者的胜利,即发达的私有财产对不发达的、不完全的私有财产的胜利,正如一般说来运动必然战胜不动,公开的、自觉的卑鄙行为必然战胜隐蔽的、不自觉的卑鄙行为,贪财欲必然战胜享乐欲,直认不讳的、老于世故的、孜孜不息的、精明机敏的开明利己主义必然战胜眼界狭隘的的、一本正经的、懒散的、幻想的、迷信利己主义,货币必然战胜其它形式的私有财产一样。

那些多少觉察到完成的自由工业、完成的纯粹道德和完成的博爱商业的危险的国家,企图阻止地产变成资本,却完全白费力气。

与资本不同,地产是还带有地域的和政治的偏见的私有财产、资本,是还没有完全摆脱同周围世界的纠缠而达到自身的资本,即还没有完成的资本。它必然要在它的世界发展过程中达到它的抽象的即纯粹的表现。

私有财产的关系是劳动、资本以及二者的关系。这个关系的各个成分所必定经历的运动是:

第一:二者直接的或间接的统一。

起初,资本和劳动还是统一的;后来,他们虽然分离和异化,却作为积极的条件而互相促进和互相推动。

〔第二〕:二者的对立。它们互相排斥;工人知道资本家是自己的非存在,反过来也是这样;每一方都力图剥夺另一方的存在。

〔第三〕:二者各自同自身对立。资本=积累的劳动=劳动。作为这样的东西,资本分解为自身和自己的利息,而利息又分解为利息和利润。资本家完全成为牺牲品。他沦为工人阶级,正像工人-但只是例外地-成为资本家一样。劳动是资本的要素,是资本的费用,因而,工资是资本的牺牲。

劳动分解为自身和工资。工人本身是资本、商品。

敌对性的相互对立。

\subsubsection{国民经济学中反映的私有财产的本质}
私有财产的主体本质,作为自为地存在着的活动、作为主体、作为个人的私有财产,就是劳动。因此,十分明显,只有把劳动视为自己的原则——亚当·斯密——,也就是说,不再认为私有财产仅仅是人之外的一种状态的国民经济学,只有这种国民经济学才应该被看成私有财产的现实能量和现实运动的产物(这种国民经济学是私有财产的在意识中自为地形成的独立运动,是现代工业本身),现代工业的产物;而另一方面,正是这种国民经济学促进并赞美了这种工业的能量和发展,使之变成意识的力量。因此,在这种揭示了-在私有制范围内-财富的主体本质的启蒙国民经济学看来,那些认为私有财产对人来说仅仅是对象性的本质的货币主义体系和重商主义体系地拥护者,是拜物教徒、天主教徒。所以,恩格斯有理由把亚当·斯密称作国民经济学的路德。正像路德认为宗教、信仰为外部世界的本质并以此反对天主教异教一样,正像他把宗教观念变成人的内在本质,从而扬弃了外在的宗教观念一样,正像他把教士移到俗人心中,因而否定了俗人之外存在的教士一样,由于私有财产体现为在人本身中,而人本身被认为是私有财产的本质,因而在人之外并且不依赖于人的财富,也就是只以外在方式来保存和保持的财富被扬弃了,换言之,财富这种外在的、无思想的对象性就被扬弃了,但正因为这个缘故,人本身被当成了私有财产的规定,就像在路德那里被当成了宗教的规定一样。因此,以劳动为原则的国民经济学,在承认人的假象下,毋宁说不过是彻底实现对人的否定而已,因为人本身已不再同私有财产的外在本质处于外部的紧张关系中,而人本身却成了私有财产的这种紧张的本质。以前是人之外的存在——人的实际外化——的东西,现在仅仅变成外化的行为,变成了外在化。因此,如果说上述国民经济学是从表面上承认人、人的独立性、自主活动等等开始,并由于把私有财产转为人自身的本质而能够不再束缚于作为存在于人之外的本质的私有财产的那些地域性的、民族的等等的规定,从而发挥一种世界主义的、普遍的、摧毁一切界限和束缚的能量,以便自己作为惟一的政策、普遍性、界限和束缚取代这些规定,-那末,国民经济学在它往后的发展过程中必定抛弃这种伪善性,而使自己的犬儒主义充分表现出来。它实际上也是这样做的——它不顾这种学说使它陷入的那一切表面上的的矛盾——,它十分片面地,因而也更加明确和彻底地发挥了关于劳动是财富的惟一本质的论点,然而它表明,这个学说的结论与上述原来的观点相反,不如说是敌视人的;最后,它还致命地打击了私有财产和财富的最后的个别的、自然的、不依赖于劳动运动存在的形式即地租,打击了这种已经完全成了经济的东西因而对国民经济学无法反抗的封建所有的制的表现。(李嘉图学派。)从斯密经过萨伊到李嘉图、穆勒等等,国民经济学的犬儒主义不仅相对地增长了(因为工业所造成的后果在后面这些人面前以更发达和更充满矛盾的形式表现出来),而且肯定地说,他们总是自觉地在人的异化方面比他们的先驱者走的更远,但这只是因为他们的科学发展的更加彻底、更加真实罢了。因为他们把具有活动形式的私有财产变为主体,就是说,既使人成为本质,又同时使作为某种非存在物〔Unwesen〕的人成为本质,所以,现实中的矛盾就完全符合他们视为原则的那个充满矛盾的本质。支离破碎的〔II〕工业现实不仅没有推翻,相反地,却证实了他们的自身支离破碎的原则。他们的原则本来就是这种支离破碎状态的原则。

魁奈医生的重农主义学说是从重商主义体系到亚当·斯密的过渡。重农学派直接是封建所有制在国民经济学上的解体,但正因为如此,它同样直接是封建所有制在国民经济学上的变革、恢复,不过它的语言这时不再是封建的,而且是经济学的了。全部财富被归结为土地和耕作(农业)。土地还不是资本,它还是资本的一种特殊的存在形式,这种存在形式应当在它的自然特殊性中并且由于它的这种自然特殊性才具有意义。但是,土地毕竟是一种普遍的自然的要素,而重商主义体系只承认贵金属是财富的存在。因此,财富的对象、财富的材料立即获得了自然界范围内的最高普遍性,因为它们作为自然界仍然是直接对象性的财富。而土地只有通过劳动、耕种才对人存在。因而,财富的主体本质已经移入劳动中。但农业同时是唯一的生产的劳动。因此,劳动还不是从它的普遍性和抽象性上被理解的,它还是同一种作为它的材料的特殊自然要素结合在一起的,因而,它也还是仅仅在一种特殊的、自然规定的存在形式中被认识的。因此,劳动不过是人的一种特定的、特殊的外化,正像劳动产品还被看作一种特定的财富-与其说来源于劳动本身,不如说来源于自然界的财富。在这里,土地还被看作不依赖于人的自然存在,还没有被看作资本,也就是说,还没有被看作劳动本身的要素。相反地,劳动却表现为土地的要素。但是,因为这里把过去的仅仅作为对象存在的外部财富的拜物教归结为一种极其简单的自然要素,而且已经承认-虽然只是部分地、以一种特殊的方式承认-财富的本质就在于财富的主体存在,所以,认出财富的普遍本质,并因此把具有完全绝对性即抽象性的劳动提高为原则,是一个必要的进步。人们向重农学派证明,从经济学观点即唯一合理的观点来看,农业同任何其它一切生产部门毫无区别,因此,财富的本质不是某种特定的劳动,不是与某种特殊要素结合在一起的、某种特殊的劳动表现,而是一般劳动。

重农学派既然把劳动宣布为财富的本质,也就否定了特殊的、外在的、仅仅是对象性的财富。但是,在重农学派看来,劳动首先只是地产的主体本质(重农学派是以那种在历史上占统治地位并得到公认的财产为出发点的);他们认为,只有地产才成为外化的人。他们既然把生产(农业)宣布为地产的本质,也就消除了地产的封建性质;但由于他们宣布农业是唯一的生产,他们对工业世界持否定态度,并且承认封建制度。

十分明显,那种与地产相对立的、即作为工业而确立下来的工业的主体本质一旦被理解,那末这种本质就同时也包含着自己的那个对立面。因为正像工业包含着已被扬弃了的地产一样,工业的主体本质也同时包含着地产的主体本质。

地产是私有财产的第一个形式,而工业在历史上最初仅仅作为财产的一个特殊种类与地产相对立,或者不如说它是地产的被释放了的奴隶,同样,在科学地理解私有财产的主体本质,理解劳动时,这一过程也在重演。而劳动起初只作为农业劳动出现,然后才作为一般劳动得到承认。

一切财富都成了工业的财富,成了劳动的财富,而工业是完成了的劳动,正像工厂制度是工业即劳动的发达的本质,而工业资本是私有财产的完成了的客观形式一样。

我们看到,只有这时私有财产才能完成它对人的统治,并以最普遍的形式成为世界历史性的力量。

\subsubsection{共产主义}
但是,无产和有产的对立,只要还没有把它理解为劳动和资本的对立,它还是一种无关紧要的对立,一种没有从它的能动关系上、它的内在关系上来理解的对立,还没有作为矛盾来理解的对立。这种对立即使没有私有财产的进一步的运动也能以最初的形式表现出来,如在古罗马、土耳其等。所以它还不表现为私有财产本身设定的对立。但是,作为财产之排除的劳动,即私有财产的主体本质,和作为劳动之排除的资本,即客体化的劳动,-这就是作为上述对立发展到矛盾关系的、因而促使矛盾得到解决的能动关系的私有财产。

补入同一页。自我异化的扬弃同自我异化走的是同一条道路。最初,对私有财产只是从它的客体方面来考察,-但是劳动仍然被看成它的本质。因此,它的存在方式就是“本身“应被消灭的资本(蒲鲁东。)或者,劳动的特殊方式,即划一的、分散的因而是不自由的劳动,被理解为私有财产的有害性和它同人相异化的存在的根源-傅立叶,他和重农学派一样,也把农业劳动看成至少是最好的劳动,而圣西门则相反,他把工业劳动本身说成本质,因此他渴望工业家独占统治,渴望改善工人状况。最后,共产主义是扬弃私有财产的积极表现;起先它是作为普遍的私有财产出现的。共产主义是从私有财产的普遍性来看私有财产关系,因而共产主义

(1)在它的最初的形式中不过是私有财产关系的普遍化和完成。这样的共产主义以双重的形式表现出来:首先,物质的财产对它的统治力量如此之大,以致它想把不能被所有人作为私有财产占有的一切都消灭;它想用强制的方式把才能等等舍弃。在它看来,物质的直接占有是生活和存在的惟一目的;工人这个范畴并没有被取消,而是被推广到一切人身上;私有财产关系仍然是整个社会同实物世界的关系;最后,用普遍的私有财产来反对私有财产的这个运动是以一种动物的形式表现出来的:用公妻制(也就是把妇女变成公有的和共有的财产)来反对婚姻(它确实是一种排它性的私有财产的形式)。人们可以说,公妻制这种思想暴露了这个完全粗陋的和无思想的共产主义的秘密。正像妇女从婚姻转向普遍卖淫\footnote{卖淫不过是工人普遍卖的一个特殊表现而已,因为这种卖淫是一种不仅包括卖淫者,而且包括逼人卖淫的关系,而且后者的下流无耻远为严重,所以,资本家等等,也包括到卖淫这一范畴中。}一样,财富即人的对象性的本质的整个世界也从它同私有者的排它性的婚姻关系转向它同整个社会的普遍卖淫关系。这种共产主义,由于到处否定人的个性,只不过是私有财产的彻底表现,私有财产就是这种否定。普遍的和作为权力形成起来的忌妒,是贪财欲所采取的并且仅仅是用另一种方式来满足自己的隐蔽形式。任何私有财产,就它本身而言,至少都对较富裕的私有财产怀有忌妒和平均化欲望,这种忌妒和平均化欲望甚至构成竞争的本质。粗陋的共产主义不过是这个忌妒和这种从想象的最低限度出发的平均化的顶点。它具有一个特定的、有限制的尺度。对整个文化和文明的世界的抽象否定,向贫穷的、需求不高的人-他不仅没有超越私有财产的水平,甚至从来没有达到私有财产的水平-的非自然的〔IV〕简单状态的倒退,恰恰证明私有财产的这种扬弃决不是真正的占有。

共同性只是劳动的共同性以及由共同的资本即作为普遍的资本家的共同体所支付的工资的平等的共同性。这种关系的两个方面被提高到想象的普遍性的程度:劳动是每个人的本分,而资本是共同体的公认的普遍性和力量。

拿妇女当作共同淫欲的虏获物和婢女来对待,这表现了人在对待自身方面的无限的退化,因为这种关系的秘密在男人对妇女的关系上,以及在对直接的、自然的、类的关系的理解方式上,都毫不含糊地、确凿无疑地、明显地、露骨地表现出来的。人和人之间的直接的、自然的、必然的关系是男人对妇女的关系。在这种自然的、类的关系中,人同自然的关系直接就是人和人之间的关系,而人和人之间的关系直接就是人同自然的关系,就是他自己的关于自然的规定。因此,这种关系通过感性的形式,作为一种显而易见的事实,表现出人的本质在何种程度上对人来说成为自然,或者自然在何种程度上成了人具有的人的本质。因而,从这种关系就可以判断人的整个文化教养程度。从这种关系的性质就可以看出,人在何种程度上对自己来说成为并把自身理解为类存在物、人。男人对妇女的关系是人和人之间最自然的关系。因此,这种关系表明人的自然的行为在何种程度上成为了人的行为,或者,人的本质在何种程度上对人来说成了自然的本质,他的人的本性在何种程度上对他来说成为了自然。这种关系还表明,人具有的需要在何种程度上成为了人的需要,也就是说,别人作为人在何种程度上对他说来成了需要,他作为个人的存在在何种程度上同时又是社会存在物。

由此可见,对私有财产的最初积极的扬弃,即粗陋的共产主义,不过是想把自己作为积极的共同体确定下来的私有财产的卑鄙性的一种表现形式。

(2)共产主义(a)按政治性质是民主的或专制的;(b)是废除国家的,但同时是尚未完成的,并且仍然处于私有财产即人的异化的影响下。这两种形式的共产主义都已经把自己理解为人向自身的还原或复归,理解为人的自我异化的扬弃;但是它还没有弄清楚私有财产的积极的本质,也还不理解所具有的人的本性,所以它还受私有财产的束缚和感染。它虽然已经理解私有财产这一概念,但是还不理解它的本质。

(3)共产主义是私有财产即人的自我异化的积极的扬弃,因而是通过人并且为了人而对人的本质的真正占有;因此,它是人向自身、向社会的即合乎人性的人的复归,这种复归是完全的,自觉的和在以往发展的全部财富的范围内生成的。这种共产主义,作为完成了的自然主义,等于人道主义,而作为完成了的人道主义,等于自然主义,它是人和自然界之间、人和人之间的矛盾的真正解决,是存在和本质、对象化和自我确证、自由和必然、个体和类之间的斗争的真正解决。它是历史之谜的解答,而且知道自己就是这种解答。

〔V〕因此,历史的全部运动,既是这种共产主义的现实的产生活动即它的经验存在的诞生活动,同时,对它的能思维的意识说来,又是它的被理解到和被认识到的生成运动。而上述尚未完成的共产主义从各个同私有财产相对立的历史形式中为自己寻找历史的证明,从现存的事物中寻找证明,同时从运动中抽出个别环节(卡贝、维尔加德尔等人尤其喜欢卖弄这一套),把它们作为自己的历史的纯种的证明固定下来;但是,它这样做恰好证明:历史运动的绝大部分是同它的论断相矛盾的,如果说它曾经存在过,那末它的这种过去的存在恰恰反驳了对本质的奢求。

不难看到,整个革命运动必然在私有财产的运动中,即在经济的运动中,为自己既找到经验的基础,也找到理论的基础。

这种物质的、直接感性的私有财产,是异化了的人的生命的物质的、感性的表现。私有财产的运动-生产和消费-是迄今为止全部生产的运动的感性展现,也就是说,是人的实现或人的现实。宗教、家庭、国家、法、道德、科学、艺术等等,都不过是生产的一些特殊的方式,并且受生产的普遍规律的支配。因此,对私有财产的积极的扬弃,作为对人的生命的占有,是对一切异化的积极的扬弃,从而是人从宗教、家庭、国家等等向自己的人的即社会的存在的复归。宗教的异化本身只是发生在人内心深处的意识领域中,而经济的异化则是现实生活的异化,-因此异化的扬弃包括两个方面。不言而喻,在不同的民族那里,这一运动从哪个领域开始,这要看一个民族的真正的、公认的生活主要是在意识领域中还是外部世界中进行,这种生活更多地是观念的生活还是现实的生活。共产主义是从无神论开始的(欧文),而无神论最初还远不是共产主义;那种无神论毋宁说还是一个抽象。所以,无神论的博爱最初还只是哲学的、抽象的博爱,而共产主义的博爱则从一开始就是现实的和直接追求实效的。

我们已经看到,在被积极扬弃的私有财产的前提下,人如何生产人-他自己和别人;直接体现他的个性的对象如何是他自己为别人的存在,同时是这个别人的存在,而且也是这个别人为他的存在。但是,同样,无论劳动的材料是作为主体的人,都既是运动的结果,又是运动的出发点(并且二者必须是这个出发点,私有财产的历史必然性就在于此)。因此,社会性质是整个运动的普遍性质;正像社会本身生产作为人的人一样,人也生产社会。活动和享受,无论就其内容或就其存在方式来说,都是社会的,是社会的活动和社会的享受。自然界的人的本质只有对社会的人来说才是存在的;因为只有在社会中,自然界对人来说才是人与人联系的纽带,才是他为别人的存在和别人为他的存在,才是人的现实的生活要素;只有在社会中,自然界才是人自己的人的存在的基础。只有在社会中,人的自然的存在对他说来才是他的人的存在,而自然界对他来说才成为人。因此,社会是人同自然界的完成了的本质的统一,是自然界的真正复活,是人的实现了的自然主义和自然界的实现了的人道主义。

〔VI〕社会的活动和社会的享受决不仅仅存在于直接共同的活动和直接共同的享受这种形式中,虽然共同的活动和共同的享受,即直接通过同别人的实际交往表现出来和得到确证的那种活动和享受,在社会性的上述直接表现以这种活动或这种享受的内容的本质为根据并且符合其本性的地方都会出现。

甚至当我从事科学之类的活动,即从事一种我只是在很少情况下才能同别人进行直接联系的活动的时候,我也是社会的,因为我是作为人活动的。不仅我的活动所需的材料,甚至思想家用来进行活动的语言本身,都是作为社会的产品给予我的,而且我本身的存在就是社会的活动;因此,我从自身所做出的东西,是我从自身为社会做出的,并且意识到我自己是社会存在物。

我的普遍意识不过是以现实共同体、社会存在物为生动形式的那个东西的理论形式,而在今天,普遍意识是现实生活的抽象,并且作为这样的抽象是与现实生活相敌对的。因此,我的普遍意识的活动本身也是我作为社会存在物的理论存在。

首先应当避免重新把“社会“当作抽象的东西同个人对立起来。个人是社会存在物。因此,他的生命表现,即使不采取共同的、同其它人一起完成的生命表现这种直接形式,也是社会生活的表现和确证。人的个人生活和类生活并不是各不相同的,尽管个人生活的存在方式是——必然是——类生活的较为特殊的或者较为普遍的方式,而类生活必然是较为特殊的或者较为普遍的个人生活。

作为类意识,人确证自己的现实的社会生活,并且只是在思维中复现自己的现实存在;反之,类存在则在类意识中确证自己,并且在自己的普遍性中作为思维着的存在物自为地存在着。

因此,人是一个特殊的个体,并且正是他的特殊性使他成为一个个体,成为一个现实的、单个的社会存在物,同样,他也是总体,观念的总体,被思考和被感知的社会的自为的主体存在,正如他在现实中既作为对社会存在的直观和现实感受而存在,又作为人的生命表现的总体而存在一样。

可见,思维和存在虽有区别,但同时彼此又处于统一中。

死似乎是类对特定的个体的冷酷无情的胜利,并且似乎是同它们的统一相矛盾的;但是特定的个体不过是一个特定的类存在物,而作为这样的存在物是迟早要死的。

(4)私有财产不过是下述情况的感性表现:人变成了对自己来说是对象性的,同时,确切的说,变成了异己的和非人的对象;他的生命表现就是他的生命的外化,他的现实化就是他的非现实化,就是异己的现实。同样,私有财产的积极的扬弃,也就是说,为了人并且通过人对人的本质和人的生命、对象性的人和人的作品的感性的占有,不应当仅仅被理解为所有、拥有,不应当仅仅被理解为直接的、片面的享受。人以一种全面的方式,也就是说,作为一个完整的人,占有自己的全面的本质。人同世界的任何一种人的关系-视觉、听觉、嗅觉、味觉、触觉、思维、直观、情感、愿望、活动、爱-总之,他的个体的一切器官,正像在形式上直接是社会的器官的那些器官一样,〔IVII〕是通过自己的对象性关系,即通过自己同对象的关系而对对象的占有,对人的现实的占有;这些器官同对象的关系,是人的现实的实现\footnote{因此,正像人的本质规定和活动是多种多样的一样,人的现实也是多种多样的。},是人的能动和人的受动,因为按人的方式来理解的受动,是人的一种自我享受。

私有制使我们变得如此愚蠢和片面,以致一个对象,只有当它为我们拥有的时候,也就是说,当它对我们来说作为资本而存在,或者它被我们直接占有,被我们吃、喝、穿、住等等的时候,简言之,在它被我们使用的时候,才是我们的。尽管私有制本身又把占有的这一切直接实现仅仅看作生活手段,而它们作为手段为之服务的那种生活是私有制的生活-劳动和资本化。

因此,一切肉体的和精神的感觉都被这一切感觉的单纯异化即拥有的感觉所代替。人这个存在物必须被归结为这种绝对的贫困,这样他才能从自身产生出他的内在丰富性。(关于拥有这个范畴,见《二十一印张》文集中赫斯的论文。)

因此,对私有财产的扬弃,是人的一切感觉和特性的彻底解放;但这种扬弃之所以是这种解放,正是因为这些感觉和特性无论在主体上还是在客体上都变成人的。眼睛变成了人的眼睛,正像眼睛的对象变成了社会的、人的、由人并为了人创造出来的对象一样。因此,感觉通过自己的实践直接变成了理论家。感觉为了物而同物发生关系,但物本身却是对自身和对人的一种对象性的、人的关系\footnote{只有当物按人的方式同人发生关系时,我才能在实践上按人的方式同物发生关系。},反过来也是这样。因此,需要和享受失去了自己的利己主义性质,而自然界失去了自己的纯粹的有用性,因为效用成了人的效用。

同样,别人的感觉和享受也形成了我自己的占有。因此,除了这些直接的器官外,还以社会的形式形成社会的器官。例如,直接同别人交往的活动等等,成了我的生命表现的器官和对人的生命的一种占有方式。

不言而喻,人的眼睛和野性的、非人的眼睛得到的享受不同,人的耳朵与野性的耳朵得到的享受不同,如此等等。

我们知道,只有当对象对人来说成为人的对象或者说成为对象性的人的时候,人才不致在自己的对象中丧失自身。只有当对象对人来说成为社会的对象,人本身对自己来说成为社会的存在物,而社会在这个对象中对人来说成为本质的时候,这种情况才是可能的。

因此,一方面,随着对象性的现实在社会中对人说来到处成为人的本质力量的现实,成为人的现实,因而成为人自己的本质力量的现实,一切对象对他说来也就成为他自身的对象化,成为确证和实现他的个性的对象,成为他的对象,而这就是说,对象成为他自身。对象如何对他来说成为他的对象,这取决于对象的性质以及与之相适应的本质力量的性质;因为正是这种关系的规定性形成一种特殊的、现实的肯定方式。眼睛对对象的感觉不同于耳朵,眼睛的对象是不同于耳朵的对象的。每一种本质力量的独特性,恰好就是这种本质力量的独特的本质,因而也是它的对象化的独特方式,它的对象性的、现实的、活生生的存在的独特方式。因此,人不仅通过思维,〔VIII〕而且以全部感觉在对象中肯定自己。

另一方面,即从主体方面来看:只有音乐才能激起人的音乐感;对于没有音乐感的耳朵说来,最美的音乐也毫无意义,不是对象,因为我的对象只能是我的一种本质力量的确证,也就是说,它只能像我的本质力量作为一种主体能力自为地存在着那样才对我而存在,因为任何一个对象对我的意义(它只是对那个与它相适应的感觉来说才有意义)恰好都以我的感觉所及的程度为限。所以社会的人的感觉不同于非社会的人的感觉。只是由于人的本质客观地展开的丰富性,主体的、人的感性的丰富性,如有音乐感的耳朵、能感受形式美的眼睛,总之,那些能成为人的享受的感觉,即确证自己是人的本质力量的感觉,才一部份发展起来,一部分产生出来。因为,不仅五官感觉,而且所谓精神感觉、实践感觉(意志、爱等等),一句话,人的感觉、感觉的人性,都是由于它的对象的存在,由于人化的自然界,才产生出来的。

五官感觉的形成是以往全部世界历史的产物。囿于粗陋的实际需要的感觉,也只具有有限的意义。对于一个挨饿的人来说并不存在人的食物形式,而只有作为食物的抽象存在;食物同样也可能具有最粗糙的形式,而且不能说,这种进食活动与动物的进食活动有什么不同。忧心忡忡的、贫穷的人甚至对最美丽的景色都没有什么感觉;贩卖矿物的商人只看到矿物的商业价值,而看不到矿物的美和独特性;他没有矿物学的感觉。因此,一方面为了使人的感觉成为人的,另一方面为了创造同人的本质和自然界的本质的全部丰富性相适应的人的感觉,无论从理论方面还是从实践方面来说,人的本质的对象化都是必要的。

通过私有财产及其富有和贫困-或物质的和精神的富有和贫困-的运动,正在生成的社会发现这种形式所需的全部材料;同样,已经生成的社会,创造着具有人的本质的这种全部丰富性的人,创造着具有丰富的、全面而深刻的感觉的人作为这个社会的恒久的现实。

我们看到,主观主义和客观主义,唯灵主义和唯物主义,活动和受动,只是在社会状态中才失去它们彼此间的对立,并从而失去它们作为这样的对立面的存在;我们看到,理论的对立本身的解决,只有通过实践方式,只有借助于人的实践力量,才是可能的;因此,这种对立的解决决不只是认识的任务,而是一个现实生活的任务,而哲学未能解决这个任务,正是因为哲学把这仅仅看作理论的任务。

我们看到,工业的历史和工业的已经产生的对象性的存在,是一本打开了的关于人的本质力量的书,是感性地摆在我们面前的人的心理学;对这种心理学人们至今还没有从它同人的本质的联系,而总是仅仅从有用性这种外在关系来理解,因为在异化范围内活动的人们仅仅把人的普遍存在、宗教、或者具有抽象普遍性质的历史,如政治、艺术和文学等等,〔IX〕理解为人的本质力量的现实性和人的类活动。在通常的、物质的工业中(人们可以把这种工业看成是上述普遍运动的一部分,正像可以把这个运动本身看成是工业的一个特殊部分一样,因为全部人的活动迄今都是劳动,也就是工业,就是同自身相异化的活动)人的对象化的本质力量以感性的、异己的、有用的对象的形式,以异化的形式呈现在我们面前。如果心理学还没有打开这本书即历史的这个恰恰最容易感知的、最容易理解的部份,那末这种心理学就不能成为内容确实丰富的和真正的科学。如果科学从人的活动的如此广泛的丰富性中只知道那种可以用“需要“、“一般需要!”的话来表达的东西,那末人们对于这种高傲地撇开人的劳动的这一巨大部分而不感觉自身不足的科学究竟应该怎样想呢?

自然科学展开了大规模的活动并且占有了不断增多的材料。但是哲学对自然科学也始终是疏远的,正像自然科学对哲学也始终是疏远的一样。过去把它们暂时结合起来,不过是离奇的幻想。存在着结合的意志,但缺少结合的能力。甚至历史学也只是顺便地考虑到自然科学,仅仅把它看作是启蒙、有用性和某些伟大发现的因素。然而,自然科学却通过工业日益在实践上进入人的生活,改造人的生活,并为人的解放做准备,尽管它不得不直接地使非人化充分发展。工业是自然界同人之间,因而也是自然科学同人之间的现实的历史关系。因此,如果把工业看成人的本质力量的公开的展示,那末,自然界的人的本质,或者人的自然的本质,也就可以理解了;因此,自然科学将失去它的抽象物质的或者不如说是唯心主义的方向,并且将成为人的科学的基础,正像它现在已经-尽管以异化的形式-成了真正人的生活的基础一样;至于说生活还有别的什么基础,科学还有别的什么基础-这根本就是谎言。在人类历史中即在人类社会的形成过程中生成的自然界是人的现实的自然界;因此,通过工业-尽管以异化的形式-形成的自然界,是真正的、人本学的自然界。

感性(见费尔巴哈)必须是一切科学的基础。科学只有从感性意识和感性需要这两种形式的感性出发,因而,科学只有从自然界出发,才是现实的科学。可见,全部历史是为了使“人“成为感性意识的对象和使“人作为人“的需要成为“自然的、感性的“需要而做准备的历史(发展史)。历史本身是自然史的即自然界生成为人这一过程的一个现实部分。自然科学往后将包括关于人的科学,正像关于人的科学包括自然科学一样:这将是一门科学。

〔X〕人是自然科学的直接对象;因为直接的感性自然界,对人说来直接地就是人的感性(这是同一个说法),直接地就是另一个对他来说感性地存在着的人;因为他自己的感性,只有通过另一个人,才对他本身说来是人的感性。但是自然界是关于人的科学的直接对象。人的第一个对象-人-就是自然界、感性;而那些特殊的、人的、感性的本质力量,正如它们只有在自然对象中才能得到客观的实现一样,只有在关于一般自然界的科学中才能获得它们的自我认识。思维本身的要素,思想的生命表现的要素,即语言,是感性的自然界。自然界的社会的现实,和人的自然科学或关于人的自然科学,是同一个说法。

我们看到,富有的人和富有的人的需要代替了国民经济学上的富有和贫困。富有的人同时就是需要有总体的人的生命表现的人,在这样的人的身上,他自己的实现作为内在的必然性、作为需要而存在。不仅人的富有,而且人的贫困,——在社会主义的前提下——同样具有人的因而是社会的意义。贫困是被动的纽带,它使人感觉到需要最大的财富即别人。因此,对象性的本质在我身上的统治,我的本质活动的感性爆发,是激情,从而激情在这里就成了我的本质的活动。

(5)任何一个存在物只有当它用自己的双脚站立的时候,才认为自己是独立的,而且只有当它依靠自己而存在的时候,它才是用自己的双脚站立的。靠别人恩典为生的人,把自己看成一个从属的存在物。但是,如果我不仅靠别人维持我的生活,而且别人还创造了我的生活,别人还是我的生活的泉源,那末,我就完全靠别人的恩典为生;如果我的生活不是我自己的创造,那末,我的生活就必定在自身之外有这样一个根源。所以,创造是一个很难从人民意识中排除的观念。自然界和人的通过自身的存在,对人民意识来说是不能理解的,因为这种存在是同实际生活的一切明摆着的事实相矛盾的。

大地创造说,受到了地球构造学(即说明地球的形成、生成是一个过程、一种自我产生的科学)的致命打击。自然发生说是对创世说的唯一实际的驳斥。

对个别人讲讲亚理士多德已经说过的下面这句话,当然是容易的:你是你的父亲和你的母亲所生;这就是说,在你身上,两个人的性的结合即人的类行为生产了人。因而,你看到,人的肉体的存在也要归功于人。所以,你应该不是仅仅注意一个方面即无限的过程,由于这个过程你会进一步发问:谁生出了我的父亲?谁生出了他的祖父?等等。你还应该紧紧盯住这个无限过程中的那个可以直接感觉到的循环运动,由于这个运动,人通过生儿育女使自身重复出现,因而人始终是主体。

但是,你会回答说:我承认这个循环运动,那末你也要向我承认那个无限的过程,这过程使我不断追问,直到我提出问题,谁生出了第一个人和整个自然界?我只能对你做如下的回答:你的问题本身就是抽象的产物。请你问一下自己,你是怎样想到这个问题的;请你问一下自己,你的问题是不是来自一个因为荒谬而使我无法回答的观点。请你问一下自己,那个无限的过程本身对理性的思维说来是否存在。既然你提出自然界和人的创造问题,那末你也就把人和自然界抽象掉了。你假定它们是不存在的,然而你却希望我向你证明它们是存在的。那我就对你说:放弃你的抽象,你也就会放弃你的问题,或者,你想坚持自己的抽象,你就要贯彻到底,如果你设想人和自然界是不存在的,〔XI〕那末你就要设想你自己也是不存在的,因为你自己也是自然界和人。不要那样想,也不要那样向我提问,因为你一旦那样想,那样提问,你就会把自然界的和人的存在抽象掉,这是没有任何意义的。也许你是一个假定一切都不存在,而自己却想存在的利己主义者吧?

你可能反驳我说:我并不想假定自然界等等不存在;我是问你自然界的形成过程,正像我问解剖学家骨骼如何形成等等一样。

但是,因为在社会主义的人看来,整个所谓世界历史不外是人通过人的劳动而诞生的过程,是自然界对人来说的生成过程,所以,关于他通过自身而诞生、关于他的形成过程,他有直观的、无可辩驳的证明。因为人和自然界的实在性,即人对人说来作为自然界的存在以及自然界对人来说作为人的存在,已经成为实际的、可以通过感觉直观的,所以,关于某种异己的存在物,关于凌驾于自然界和人之上的存在物的问题,即包含着对自然界和人的非实在性的承认的问题,实际上已经成为不可能的了。无神论,作为对这种非实在性的否定,已不再有任何意义,因为无神论是对神的否定,并且正是通过这种否定而肯定人的存在;但是,社会主义作为社会主义,已经不再需要这样的中介;它是从把人和自然界看作本质这种理论上和实践上的感性认识开始的。社会主义是人的不再以宗教的扬弃为中介的积极的自我意识,正像现实生活是人的不再以私有财产的扬弃即共产主义为中介的积极的现实一样。共产主义是作为否定的否定的肯定,因此,它是人的解放和复原的一个现实的、对下一段历史发展说来是必然的环节。共产主义是最近将来的必然的形式和有效的原则。但是,共产主义本身并不是人的发展的目标,并不是人的社会的形式。

\subsubsection{对黑格尔的辩证法和整个哲学的批判}
〔XI〕(6)在这一部分,为了便于理解和论证,对黑格尔的整个辩证法,特别是《现象学》和《逻辑学》中有关辩证法的叙述,以及最后对最近的批判运动同黑格尔的关系略作说明,也许是适当的。

现代德国的批判,着意研究旧世界的内容,而且批判的发展完全拘泥于所批判的材料,以致对批判的方法采取完全非批判的态度,同时,对于我们如何对待黑格尔辩证法,这一表面上看来是形式的问题,而实际上是本质的问题,则完全缺乏认识。对于现代的批判同黑格尔的整个哲学,特别是同辩证法的关系问题是如此缺乏认识,以致像施特劳斯和布鲁诺·鲍威尔这样的批判家——前者是完完全全地,后者在自己的《符类福音作者》中(与施特劳斯相反,它在这里用抽象的人的“自我意识“代替了“抽象的自然界“的实体),甚至在《基督教真相》中,至少有可能完全地——仍然拘泥于黑格尔的逻辑学。例如《基督教真相》一书中说︰

\begin{fangsong}
“自我意识设定世界、设定差别,并且在它所创造的东西中创造自身,因为它重新扬弃了它的创造物同它自身的差别。因为它只是在创造活动中和运动中才是自己本身,——这个自我意识在这个运动中似乎就没有自己的目的了“,等等。或者说:“他们〈法国唯物主义者〉还未能看到,宇宙的运动只有作为自我意识的运动,才能实际成为自为的运动,从而达到同自身的统一。”    
\end{fangsong}

这些说法连语言上都和黑格尔的观点毫无区别,而且毋宁说是在逐字逐句重述黑格尔的观点。

〔XII〕鲍威尔在他的《自由的正义事业》一书中对格鲁培先生提出的“那末逻辑学的情况如何呢?”这一唐突的问题避而不答,却让他去问未来的批评家。这表明,鲍威尔在进行批判活动(鲍威尔《复类福音作者》)时对于黑格尔辩证法关系是多么缺乏认识,而且在物质的批判活动之后也还缺乏这种认识。

但是,即使现在,在费尔巴哈不仅在收入《轶文集》的《纲要》中,而且更详细地在《未来哲学》中从根本上推翻了旧的辩证法和哲学之后;在不能完成这一事业的上述批判,反而认为这一切事业已经完成,并且宣称自己是“纯粹的、坚决的、绝对的、洞察一切的批判“之后;在批判以唯灵论的狂妄自大态度把整个历史运动归结为世界的其他部分(它把这个世界与它自身对立起来而归入“群众”这一范畴)和它自身的关系,并且把一切独断的对立销融于它自身的聪明和世界的愚蠢之间、批判的基督和作为“群氓”的人类之间的一个独断的对立中之后;在批判每日每时以群众的愚钝来证明它本身的超群出众之后;在批判终于宣称这样一天——那时整个正在堕落的人类将聚集在批判面前,由批判加以分类,而每一人类都将得到一份贫困证明书——即将来临,即以这种形式宣告批判的末日审判之后;在批判于报刊上宣布它既对人的感觉又·对它自己独标一格地雄踞其上的世界具有优越性,而且只是不时从它那好讥讽嘲笑的口中发出奥林帕斯诸神的哄笑声之后,——在以批判的形式消逝着的唯心主义(青年黑格尔主义)做出这一切滑稽可笑的动作之后,这种唯心主义甚至一点也没想到现在已经到了同自己的母亲,即黑格尔辩证法批判地画出界限的时候,甚至也〔丝毫〕未能表明它对费尔巴哈辩证法的批判态度。这是对自身持完全非批判的态度。

费尔巴哈是唯一对黑格尔辩证法采取严肃的、批判的态度的人;只有他在这个领域内作出了真正的发现,总之他真正克服了旧哲学。费尔巴哈成就的伟大以及他把这种成就贡献给世界时所表现的那种谦虚纯朴,同批判所持的相反的态度恰成惊人的对照。

费尔巴哈的伟大功绩在于:

(1)证明了哲学不过是变成思想的并且经过思维加以阐明的宗教,不过是人的本质的异化的另一种形式和存在方式;因此,哲学同样应当受到谴责。

(2)创立了真正的唯物主义和现实的科学,因为费尔巴哈也使“人与人之间的“社会关系成了理论的基本原则。

(3)他把基于自身并且积极地以自身为根据的肯定的东西,同自称是绝对肯定的东西的那个否定的否定对立起来。

费尔巴哈这样解释了黑格尔的辩证法(从而论证了要从肯定的东西,即从感觉确定的东西出发);

黑格尔从异化出发(在逻辑上就是从无限的东西、抽象的普遍的东西出发),从实体出发,从绝对的和不变的抽象出发,就是说,说得更通俗些,他从宗教和神学出发。

第二,他扬弃了无限的东西,设定了现实的、感性的、实在的、有限的、特殊的东西(哲学,对宗教和神学的扬弃)。

第三,他重新扬弃了肯定的东西,重新恢复了抽象、无限的东西。宗教和神学的恢复。

由此可见,费尔巴哈把否定的否定仅仅看作哲学同自身的矛盾,看作在否定神学(超验性等等)之后又肯定神学的哲学,即同自身相对立而肯定神学的哲学。

否定的否定所包含的肯定,或自我肯定和自我确证,被认为是对自身还不能确信,因而自身还受对立面影响的、对自身怀疑因而需要证明的肯定,即被认为是还没有用自己的存在证明自身的、还没有被承认的〔XIII〕肯定;可见,感觉确定的、以自身为基础的肯定,是同这种肯定直接地而非间接地对立着的。\footnote{马克思在这里加了一句话:“费尔巴哈把否定的否定、具体概念看做在思维中超越自身的和作为思维而想直接成为直观、自然界、现实的思维。”——编者注}

费尔巴哈还把否定的否定、具体概念看作在思维中超越自身的和作为思维而想直接成为直观、自然界、现实的思维

但是,由于黑格尔根据否定的否定所包含的肯定方面把否定的否定看成真正的和惟一的肯定的东西,而根据它所包含的否定方面把它看成一切存在的唯一真正的活动和自我实现的活动,所以他只是为那种历史的运动找到抽象的、逻辑的、思辨的表达,这种历史还不是作为既定的主体的人的现实的历史,而只是人的产生的活动、人的形成的历史。

我们既要说明这一运动在黑格尔那里所采取的抽象形式,也要说明这一和现代的批判相反的运动,同费尔巴哈的《基督教的本质》一书所描述的同一过程的区别;或者更正确些说,要说明这一在黑格尔那里还是非批判的运动所具有的批判形式。

现在看一看黑格尔的体系。必须从黑格尔的《现象学》即从黑格尔哲学的真正诞生地和秘密开始。

现象学

(A)自我意识。

I.意识。(a)感性确定性,或“这一个”和意谓。(b)知觉,或具有特性的事物和幻觉。(c)力和知性,现象和超感觉世界。

II.自我意识。自身确定性的真理。(a)自我意识的独立性和非独立性,主人和奴隶(b)自我意识的自由。斯多葛主义,怀疑主义,苦恼的意识。

III.理性。理性的确定性和真理。(a)观察的理性;对自然界和自我的意识的观察(b)理性的自我意识通过自身来实现。快乐和必然性。心的规律和自大狂。德行和世道。(c)自在和自为地实在的个性。精神的动物界和欺骗,或事情本身。立法的理性。审核法律的理性。

(B)精神。

I.真的精神;伦理。II.自我异化的精神,教养。III.确定自身的精神,道德。

(C)宗教。自然宗教,艺术宗教,启示宗教。

(D)绝对知识。

因为黑格尔的《哲学全书》以逻辑学,以纯粹的思辨的思想开始,而以绝对知识,以自我意识的、理解自身的哲学或绝对的即超人的抽象精神结束,所以整整一部《哲学全书》不过是哲学精神的展开的本质,是哲学精神的自我对象化;而哲学精神不过是在它的自我异化内部通过思维理解,即抽象地理解自身话的、异化的宇宙精神。逻辑学是精神的货币,是人和自然界的思辨的、思想的价值——人和自然界的同一切现实的规定性毫不相干的生成的因而是非现实的本质,——是外化的因而是从自然界和现实的人抽象出来的思维,即抽象思维。——这种抽象思维的外在性就是……自然界,就像自然界对这种抽象思维所表现的那样。自然界对抽象思维说来是外在的,是抽象思维的自我丧失;而抽象思维也是外在地把自然界作为抽象的思想来理解,然而是作为外化的、抽象的思维来理解。——最后,精神,这个回到自己的诞生地的思维,这种思维在它终于发现自己和肯定自己就是绝对知识,因而就是绝对的即抽象的精神之前,在它获得自己的自觉的、与自身相符合的存在之前,它作为人类学的、现象学的、心理学的、伦理的、艺术的、宗教的精神,总还不是自身,因为它的现实存在就是抽象。

黑格尔有双重错误。

第一个错误在黑格尔哲学的诞生地《现象学》中表现的最为明显。例如,当他把财富、国家权力等等看成同人的本质相异化的本质时,这只是就它们的思想形式而言……它们是思想本质,因而只是纯粹的即抽象的哲学思维的异化。因此,整个运动是以绝对知识结束的。这些从对象中异化出来的并且以现实性自居而与之对立的,恰恰是抽象的思维。哲学家——他本身是异化的人的抽象形象——把自己变成异化的世界的尺度。因此,全部外化历史和外化的全部消除,不过是抽象的、绝对的〔XVII〕思维的生产史,即逻辑的思辨的思维的生产史。因而,异化——它从而构成这种外化的以及这种外化之扬弃的真正意义——是在自在和自为之间、意识和自我意识之间、客体和主体之间的对立,就是说,是抽象思维同感性的现实,或现实的感性在思想本身范围内的对立。其它一切对立及其运动,不过是这种唯一有意义的对立的外观、外壳、公开形式,这些唯一有意义的对立构成其它世俗对立的含义。在这里,不是人的本质以非人的方式同自身对立的对象化,而是人的本质以不同于抽象思维的方式,并且同抽象思维对立的对象化,被当作异化的被设定的和应该扬弃的本质。

〔XVIII〕因此,对于人的已成为对象而且是异己对象的本质力量的占有,首先不过是那种在意识中、在纯思维中即在抽象中发生的占有,是对这些作为思想和思想运动的对象的占有;因此,在《现象学》中,尽管已有一个完全否定的和批判的外表,尽管实际上已包含着往往早在后来发展之前就先进行的批判,黑格尔晚期著作的那种非批判的实证主义,和同样非批判的唯心主义——现有经验在哲学上的分解和恢复——已经已一种潜在的方式,作为萌芽、潜能和秘密存在着了。其次,因此,要求把对象世界归还给人——例如,有这样一种理解︰感性意识不是抽象的感性意识,而是人的感性的意识;宗教、财富等等不过是人的对象化的异化了的现实,是客体化了的人的本质力量的异化了的现实;因此,宗教、财富等等不过是通向真正的人的现实的道路,——这种对人的本质力量的占有或对这一过程的理解,在黑格尔那里是这样表现的:感性、宗教、国家权力等等是精神的本质,因为只有精神才是人的真正的本质,而精神的真正的形式则是思维着的精神,逻辑的、思辨的精神。自然界的人性和历史所创造的自然界——人的产品——的人性,就表现在它们是抽象精神的产物,所以,在这个限度内,它们是精神的环节即思想本质。可见,《现象学》是一种隐蔽的、自身还不清楚的、神秘化的批判;但是,由于《现象学》紧紧抓住人的异化,——尽管人只是以精神的方式出现的,——所以它潜在地包含着批判的一切要素,而且这些要素往往已经以远远超过黑格尔观点的方式准备好和加过工了。关于“苦恼的意识”、“诚实的意识”、“高尚的意识和卑鄙的意识”的斗争等等、等等这些章节,包含着对宗教、国家、市民生活等整个领域的批判的要素,但还是通过异化的形式。正像本质、对象表现为思想本质一样,主体也始终是意识或自我意识,或者更正确些说,对象仅仅表现为抽象的意识,而人仅仅表现为自我意识。因此,在《现象学》中出现的异化的各种不同形式,不过是意识和自我意识的不同形式,正像抽象的意识本身(对象就被看成这样的意识)仅仅是自我意识的一个差别环节一样,这一运动的结果表现为自我意识和意识的同一,绝对知识,那种已经不是向外部而是仅仅在自身内部进行的抽象思维活动,也就是说,其结果是纯思想的辩证法。〔XVIII〕

〔XXIII〕因此,黑格尔的《现象学》及其最后成果——作为推动原则和创造原则的否定性的辩证法——的伟大之处首先在于,黑格尔把人的自我产生看作一个过程,把对象化看作非对象化,看作外化和这种外化的扬弃;因而,他抓住了劳动的本质,把对象性的人、现实的因而是真正的人,理解为他自己的劳动的成果。人同作为类存在物的自身发生现实的、能动的关系,或者说,人作为现实的类存在物即作为人的存在物的实现,只有通过下述途径才有可能:人实际上把自己的类的力量统统发挥出来(这又是只有通过人类的全部活动、只有作为历史的结果才有可能),并且把这些力量当作对象来对待,而这首先又是只有通过异化的形式才有可能。

我们将以《现象学》的最后一章——绝对知识——来详细说明黑格尔的片面性和局限性。这一章既概括地阐述了《现象学》的精神、包含《现象学》同思辨的辩证法的关系,也概括地阐述了黑格尔对这二者及其相互关系的理解。

且让我们先指出一点︰黑格尔站在现代国民经济学家的立场上。他把劳动看作人的本质,看作人的自我确证的本质;他只看到劳动的积极的方面,而没有看到它的消极的方面。劳动是人在外化范围内或者作为外化的人的自为的生成。黑格尔唯一知道并承认的劳动是抽象的精神的劳动。因此,黑格尔把一般说来构成哲学的本质的那个东西,即知道自身的人的外化,或者思考自身的、外化的科学看成劳动的本质;因此,同以往的哲学相反,他能把哲学的各个环节加以总括,并且把自己的哲学描述成这种哲学。其它哲学家做过的事情——把自然界和人类生活的各个环节看作自我意识的,以至抽象的自我意识的环节,黑格尔则认为是哲学本身所做的事情。因此,他的科学是绝对的。

现在让我们转到我们的本题上来。

绝对知识。《现象学》的最后一章。

主要之点就在于︰意识的对象无非就是自我意识;或者说,对象不过是对象化的自我意识、作为对象的自我意识(把人和自我意识等同起来)。

因此,问题就在于克服意识的对象。对象性本身被认为是人的异化了的、同人的本质(自我意识)不相适应的关系。因此,重新占有在异化规定下作为异己的东西产生的、人的对象性的本质,这不仅具有扬弃异化的意义,而且有扬弃对象性的意义,这就是说,因此,人被看成非对象性的、唯灵论的存在物。

黑格尔对克服意识的对象的运动作了如下的描述︰

对象不仅表现为向自我〔das Selbst〕复归的东西(在黑格尔看来,这是对第一运动的片面的,即只抓住了一个方面的理解)。把人和自我等同起来。而自我不过是被抽象地理解和通过抽象产生出来的人。人是自我的〔selbstisch〕。人的眼睛、人的耳朵等等都是自我的;人的每一种本质力量在人身上都具有自我性这种特性。但正因为这样,说自我意识具有眼睛、耳朵、本质力量,就完全错了。毋宁说,自我意识是人的自然的即人的眼睛等等的质,而并非人的自然是〔XXIV〕自我意识的质。

被抽象化和被固定化的自我,就是作为抽象的利己主义者的人,就是在自己的纯粹抽象中被提升到思维的利己主义(下文还要提到这一点)。

人的本质,人,在黑格尔看来是和自我意识等同的。因此,人的本质的一切异化都不过是自我意识的异化。自我意识的异化没有被看作人的本质的现实异化的表现,即在知识和思维中反映出来的这种异化的表现。相反地,现实的即真实出现的异化,就其潜藏在内部最深处的——并且只有哲学才能揭示出来的——本质来说,不过是现实的、人的本质即自我意识的异化现象。因此,掌握了这一点的科学就叫作现象学。因此,对异化了的对象性本质的全部重新占有,都表现为把这种本质合并于自我意识:掌握了自己本质的人,仅仅是掌握了对象性本质的自我意识。因此,对象向自我的复归就是对象的重新占有。

意识的对象的克服可全面表述如下:

(1)对象本身对意识说是正在消逝的东西;

(2)自我意识的外化就是设定物性;

(3)这种外化不仅有否定的意义,而且有肯定的意义;

(4)它不仅对我们有这种意义或者说自在地有这种意义,而且对意识本身也有这种意义;

(5)对象的否定,或对象的自我扬弃,对意识所以有肯定的意义(或者说,它所以知道对象的这种虚无性),是由于意识把自身外化了,因为意识在这种外化中把自身设定为对象,或者说,由于自为的存在的不可分割的统一性,而把对象设定为自身。

(6)另一方面,这里同时包含着另一个环节,即意识扬弃这种外化和对象性,同样也把它们收回到自身,因而,它在自己的异在本身中也就是在自己那里;

(7)这就是意识的运动,因而也就是意识的各个环节的总体;

(8)意识必须依据自己的各个规定的总体对待对象,同样也必须依据这个总体的每一个规定来把握对象。意识的各个规定的这种总体使对象自在地成为精神的本质,而对于意识来说,对象所以真正成为精神的本质,是由于把对象(这个总体)的每一个别规定理解为自我的规定,或者说,是由于对这些规定采取了上述的精神的态度。

关于(1)。——所谓对象本身对意识来说是正在消逝的东西,就是上面提到的对象向自我的复归。

关于(2)。——自我意识的外化设定物性。因为人等于自我意识,所以人的外化的、对象性的本质即物性(即对他来说是对象的那个东西,而只有对他来说是本质的对象,并因而是他的对象性的本质的那个东西,才是他的真正的对象。既然被当作主体的不是现实的人本身,因而也不是自然——人是人的自然——而只是人的抽象,即自我意识,所以,物性只能是外化的自我意识),等于外化的自我意识,而物性是由这种外化设定的。一个有生命的、自然的、具备并赋有对象性的即物质的本质力量的存在物,既拥有他的本质的现实的、自然的对象,而他的自我外化又设定一个现实的、但以外在性的形式表现出来因而不属于他的本质的、极其强大的对象世界,这是十分自然的。这里并没有什么不可捉摸的和神秘莫测的东西。相反的情况倒是神秘莫测的。但是,同样明显的是,自我意识通过自己的外化所能设定的只是物性,即只是抽象物、抽象的物,而不是现实的物。〔XXVI〕此外还很明显的是:物性因此对自我意识来说绝不是什么独立的、实质的东西,而只是纯粹的创造物,是自我意识所设定的东西,这个被设定的东西并不证实自己,而只是证实设定这一行动,这一行动在一瞬间把自己的能力作为产物固定下来,使它表面上具有独立的、现实的本质的作用——但仍然只是一瞬间。

当现实的、肉体的的、站在坚实的呈圆形的地球上呼出和吸入一切自然力的人,通过自己的外化把自己现实的、对象性的本质力量设定为异己的对象时,这种设定并不是主体;它是对象性的本质力量的主体性,因此这些本质力量的活动也必须是对象性的活动。对象性的存在物进行对象性活动,如果它的本质规定中不包含对象性的东西,它就不进行对象性活动。它所以只创造或设定对象,因为它本身是被对象所设定的,因为它本来就是自然界。因此,并不是它在设定这一行动中从自己的“纯粹的活动”转而创造对象,而是它的对象性的产物仅仅证实了它的对象性活动,证实了它的活动是对象性的自然存在物的活动。

我们在这里看到,彻底的自然主义或人道主义,既不同于唯心主义,也不同于唯物主义,同时又是把这两者结合的真理。我们同时也看到,只有自然主义能够理解世界历史的行动。
 
人直接地是自然存在物。人作为自然存在物,而且作为有生命的自然存在物,一方面具有自然力、生命力,是能动的自然存在物;这些力量作为天赋和才能、作为欲望存在于人身上;另一方面,人作为自然的、肉体的、感性的、对象性的存在物,同动植物一样,是受动的、受制约的和受限制的存在物,也就是说,他的欲望的对象是作为不依赖于他的对象而存在于他之外的;但这些对象是他的需要的对象;是表现和确证他的本质力量所不可缺少的、重要的对象。说人是肉体的、有自然力的、有生命的、现实的、感性的、对象性的存在物,这就等于说,人是有现实的、感性的对象作为自己本质的即自己生命表现的对象;或者说,人只有凭借现实的、感性的对象才能表现自己的生命。说一个东西是对象性的、自然的、感性的,又说,在这个东西之外有对象、自然界、感觉,或者说,它本身对于第三者来说是对象、自然界、感觉,这都是同一个意思。饥饿是自然的需要;因而为了使自身得到满足,使自身解除饥饿,它需要自身之外的自然界、自身之外的对象。饥饿是我的身体对某一对象的公认的需要,这个对象存在于我的身体之外、是使我的身体得以充实并使本质得以表现所不可缺少的。太阳是植物的对象,是植物所不可缺少的、确证它的生命的对象,正像植物是太阳的对象,是太阳的唤醒生命的力量的表现,是太阳的对象性的本质力量的表现一样。

一个存在物如果在自身之外没有自己的自然界,就不是自然存在物,就不能参加自然界的生活。一个存在如果在自身之外没有对象,就不是对象性的存在物。一个存在物如果本身不是第三存在物的对象,就没有任何存在物作为自己的对象,也就是说,它没有对象性的关系,它的存在就不是对象性的存在。

〔XXVII〕非对象性的存在物是非存在物〔Unwesen〕

假定一种存在物本身既不是对象,又没有对象。这样的存在物首先将是一个唯一的存在物,在它之外没有任何东西存在,它孤零零地独自存在着。因为,只要有对象存在于我之外,只要我不是独自存在着,那末我就是和在我之外存在的对象不同的它物、另一个现实。因而,对这个第三对象来说,我是和他不同的另一个现实,也就是说,我是它的对象。这样,一个存在物如果不是另一个存在物的对象,那末就要以不存在任何一个对象性的存在物存在为前提。只要我有一个对象,这个对象就以我作为它的对象。但是,非对象性的存在物,是一种非现实的、非感性的、只是思想上的,即只是虚构出来的存在物,是抽象的东西。说一个东西是感性的即现实的,这是说,它是感觉的对象,是感性的对象,从而在自己之外有感性的对象,有自己的感性的对象。说一个东西是感性的,就是指它是受动的。

因此,人作为对象性的、感性的存在物,是一个受动的存在物;因为它感到自己是受动的,所以是一个有激情的存在物。激情、热情是人强烈追求自己的对象的本质力量。

但是,人不仅仅是自然存在物,而且是人的自然存在物,就是说,是自为地存在着的存在物,因而是类存在物。他必须既在自己的存在中也在自己的知识中确证并表现自身。因此,正像人的对象不是直接呈现出来的自然对象一样,直接地存在着的、客观地存在着的人的感觉,也不是人的感性、人的对象性。自然界,无论是客观的还是主观的,都不是直接同人的存在物相适合地存在着。

正像一切自然必须产生一样,人也有自己的形成过程即历史,但历史对人来说是被认识到的历史,因而它作为形成过程是一种有意识地扬弃自身的形成过程。历史是人的真正的自然史。——(关于这一点以后还要回过头来谈。)

第三,由于物性的这种设定本身不过是一种外观,一种与纯粹活动的本质相矛盾的行为,所以这种设定必然重新被扬弃,物性必然被否定。

关于第(3)、(4)、(5)、(6)。——(3)意识的这种外化不仅有否定的意义,而且也有肯定的意义。(4)它不仅对我们有这种肯定的意义或者说自在地有这种肯定的意义,而且对它即意识本身也有这种肯定的意义。(5)对象的否定,或对象的自我扬弃,对意识所以有肯定的意义,或者说,它所以知道对象的这种虚无性,是由于意识把自身外化了,因为意识在这种外化中知道自己是对象,或者说,由于自为存在的的不可分割的统一性而知道对象就是它自身。(6)另一方面,这里同时包含着另一个环节,即意识既扬弃这种外化和对象性,同样也把它们收回到自身,因此,它在自己的异在本身中也就是在自身。

我们已经看到,异化的对象性本质的占有,或在异化——它必然从漠不关心的异己性发展到现实的、敌对的异化——这个规定下的对象性的扬弃,在黑格尔看来,同时或甚至主要地具有扬弃对象性的意义,因为并不是对象的一定的性质,而是它的对象性的性质本身,对自我意识来说成为一种障碍和异化。因此,对象是一种否定的东西、自我扬弃的东西,是一种虚无性。对象的这种虚无性对意识来说不仅有否定的意义,而且有肯定的意义,因为对象的这种虚无性正是它自身的非对象性的即〔XXVIII〕抽象的自我确证。对于意识本身来说,对象的虚无性所以有肯定的意义,是因为意识知道这种虚无性、这种对象性本质是它自己的自我外化,知道这种虚无性只是由于它的自我外化才存在……

意识的存在方式,以及对意识说来某个东西的存在方式,这就是知识。知识是意识的惟一的行动。因此,只要意识知道某个东西,那末这个东西对意识来说就生成了。知识是意识的惟一的、对象性的关系。——意识所以知道对象的虚无性,就是说知道对象同它没有区别,对象对它说来是非存在,是因为意识知道对象是它的自我外化,就是说,意识所以知道自己(作为对象的知识),是因为对象只是对象的外观、障眼的云雾,而就它的本质来说不过是知识本身,知识把自己同自身对立起来,从而把某种虚无性,即在知识之外没有任何对象性的某种东西同自己对立起来;或者说,知识知道,当它与某个对象发生关系时,它只是在自身之外,使自身外化;它知道它本身只表现为对象,或者说,对它来说表现为对象的那个东西仅仅是它本身。

另一方面,用黑格尔的话来说,这里同时还包含着另一个环节,即自我意识既扬弃这种外化和对象性,同样也把它们收回到自身,因此,它在自己的异在本身中就是在自身。

这段议论汇集了思辨的一切幻想。

第一,意识、自我意识在自己的异在本身中就是在自身。因此,自我意识——或者,如果我们在这里撇开黑格尔的抽象而用人的自我意识来代替自我意识,——从而可以说人的自我意识在自己的异在本身中,也就是在自身。这里首先包含着:意识,也就是作为知识的知识、作为思维的思维,直接地冒充为异于自身的他物,冒充为感性、现实、生命,——在思维中超越自身的思维(费尔巴哈)。这里所以包含着这一方面,是因为仅仅作为意识的意识,所碰到的障碍不是异化的对象性,而是对象性本身。

第二,这里包含着:因为有自我意识的人认为精神世界——或人的世界在精神上的普遍存在——是自我外化并加以扬弃,所以他仍然重新通过这个外化的形态确证精神世界,把这个世界冒充为自己的真正的存在,恢复这个世界,假称他在自己的异在本身中也就是在自身。因此,在扬弃例如宗教之后,在承认宗教是自我外化的产物之后,他仍然在作为宗教的宗教中找到自身的确证。黑格尔的虚假的实证主义,即他那只是徒有其表的批判主义的根源就在于此,这也就是费尔巴哈所说的宗教或神学的设定、否定和恢复,然而这应当以更一般的形式来表述。因此,理性在作为非理性的非理性中也就是在自身。一个认识到自己在法、政治等等中过着外化生活的人,就是在这种外化生活本身中过着自己的真正的人的生活。因此,与自身相矛盾的,既与知识又与对象的本质相矛盾的自我肯定、自我确证,是真正的知识和真正的生活。

因此,现在不用再谈黑格尔对宗教、国家等等的适应了,因为这种谎言是他的原则的谎言。

〔XXIX〕如果我知道宗教是外化的人的自我意识,那末我也知道,在作为宗教的宗教中得到确证的不是我的自我意识,而是我的外化的自我意识。这就是说,我知道我的属于自身的、属于我的本质的自我意识,不是在宗教中,倒是在被消灭、被扬弃的宗教中得到确证的。

因而,在黑格尔那里,否定的否定不是通过否定假象本质来确证真正的本质,而是通过否定假象本质来确证假象本质,或者说,来确证同自身相异化的本质,换句话说,否定的否定是否定作为在人之外的、不依赖于人的对象性本质的这种假象本质,并使它转化为主体。

因此,把否定和保存即肯定结合起来的扬弃起着一种独特的作用。

例如,在黑格尔法哲学中,扬弃了的私法等于道德,扬弃了的道德等于家庭,扬弃了的家庭等于市民社会,扬弃了的市民社会等于国家,扬弃了的国家等于世界历史。在现实中,私法、道德、家庭、市民社会、国家等等依然存在着,它们只是变成环节,变成人的存在和存在方式,这些存在方式不能孤立地发挥作用,而是互相消融,互相产生等等。它们是运动的环节。

在它们的现实存在中,它们的这种运动的本质是隐蔽的。这种本质只是在思维中、在哲学中才表露、显示出来;因此,我的真正的宗教存在是我的宗教哲学的存在,我的真正的政治存在是我的法哲学的存在,我的真正的自然存在是我的自然哲学的存在,我的真正艺术存在是我的艺术哲学的存在,我的真正的人的存在是我的哲学的存在。同样,宗教、国家、自然界、艺术的真正存在,就是宗教哲学、自然哲学、国家哲学、艺术哲学。但是,如果只有宗教哲学等等对我来说才是真正的宗教存在,那末我就只有作为宗教哲学家才算是真正信教的,而这样一来我就否定了现实的宗教信仰和现实的信教的人。但是我同时又确证了它们:一方面,是在我自己的存在中或在我使之与它们相对立的那个异己的存在中,因为异己的存在仅仅是它们本身的哲学的表现,另一方面,则是在它们自己的最初形式中,因为在我看来它们不过是虚假的异在、譬喻,是隐蔽在感性外壳下面的它们自己的真正存在即我的哲学的存在的形式。

同样地,扬弃了的质等于量,扬弃了的量等于度,扬弃了的度等于本质,扬弃了的本质等于现象,扬弃了的现象等于现实,扬弃了的现实等于概念,扬弃了的概念等于客观性,扬弃了的客观性等于绝对观念,扬弃了的绝对观念等于自然界,扬弃了的自然界等于主观精神,扬弃了的主观精神等于伦理的客观精神,扬弃的伦理精神等于艺术,扬弃了的艺术等于宗教,扬弃了的宗教等于绝对知识。

一方面,这种扬弃是思想上的本质的扬弃,也就是说,思想上的私有财产在道德的思想中的扬弃。而且因为思维自以为直接就是和自身不同的另一个东西,即感性的现实,从而认为自己的活动也是感性的现实的活动,所以这种思想上的扬弃,在现实中没有触动自己的对象,却以为实际上克服了自己的对象;另一方面,因为对象对于思维说来现在已成为一个思维环节,所以对象在自己的现实中也被思维看作思维本身的即自我意识的、抽象的自我确证。

〔XXIX〕因此,从一方面说,黑格尔在哲学中加以扬弃的存在,并不是现实的宗教、国家、自然界、而是已经成为知识的对象的宗教本身,即教义学;法学、国家学、自然科学也是如此。因此,从一方面来说,黑格尔既同现实的本质相对立,也同直接的、非哲学的科学或这种本质的非哲学的概念相对立。因此,黑格尔是同它们的通用的概念相矛盾的。

另一方面,信奉宗教等等的人可以在黑格尔那里找到自己的最后的确证。

现在应该考察——在异化这个规定之内——黑格尔辩证法的积极的环节。

(a)扬弃是把外化收回到自身的、对象性的运动。——这是在异化的范围内表现出来的关于通过扬弃对象性本质的异化来占有对象性本质的见解;这是异化的见解,它主张人的现实的对象化,主张人通过消灭对象世界的异化的规定、通过在对象世界的异化存在中扬弃对象世界而现实地占有自己的对象性本质,正像无神论作为神的扬弃,就是理论的人道主义的生成,而共产主义作为私有财产的扬弃,就是要求归还真正人的生命即人的财产,就是实践的人道主义的生成一样;或者说,无神论是以扬弃宗教作为自己的中介的人道主义,共产主义则是以扬弃私有财产作为自己的中介的人道主义。只有通过扬弃这种中介——但这种中介是一个必要的前提——积极地从自身开始的即积极的人道主义才能产生。

然而,无神论、共产主义绝不是人所创造的对象世界的消逝、舍弃和丧失,即绝不是人的采取对象形式的本质力量的消逝、舍弃和丧失,绝不是返回到非自然的、不发达的简单状态去的贫困。恰恰相反,它们倒是人的本质的或作为某种现实东西的人的本质的现实的生成,对人来说的真正的实现。

这样,因为黑格尔理解到——尽管又是通过异化的方式——有关自身的否定具有的积极意义,所以同时也把人的自我异化、人的本质的外化、人的非对象化和非现实化理解为自我获得、本质的表现、对象化、现实化。简单地说,他——在抽象的范围内——把劳动理解为人的自我产生的行动,把人对自身的关系理解为对异己存在物的关系,把作为异己存在物的自身的实现理解为生成着的类意识和类生活。

(b)但是,撇开上述颠倒的说法不谈,或者更正确地说,作为上述颠倒说法的结果,在黑格尔看来,这种行动,第一,仅仅具有形式的性质,因为它是抽象的,因为人的本质本身仅仅被看作抽象的、思维着的本质,即自我意识。

第二,因为这种观点是形式的和抽象的,所以外化的扬弃成为外化的确证,或者说,在黑格尔看来,自我产生、自我对象化的运动,作为自我外化和自我异化的运动,是绝对的因而也是最后的、以自身为目的的、安于自身的、达到自己本质的人的生命表现。

因此,这个运动在其抽象〔XXXI〕形式上,作为辩证法,被看成真正人的生命;而因为它毕竟是人的生命的抽象、异化,所以它被看成神性的过程,然而是人的神性的过程,——一个与人自身有区别的、抽象的、纯粹的、绝对的本质本身所经历的过程。

第三,这个过程必须有一个承担者、主体;但主体只作为结果出现;因此,这个结果,即知道自己是绝对自我意识的主体,就是神,就是绝对精神,就是知道自己并且实现自己的观念。现实的人和现实的自然界不过是成为这个隐蔽的非现实的人和这个非现实的自然界的谓语、象征。因此,主语和宾语之间的关系被绝对地相互颠倒了:这就是神秘的主体——客体,或笼罩在客体上的主体性,作为过程的绝对主体,作为使自身外化并且从这种外化返回到自身的、但同时又把外化收回到自身的主体,以及作为这一过程的主体;这就是在自身内部的纯粹的、不停息的旋转。

关于第一点:对人的自我产生的或自我对象化的行动的形式的和抽象的理解。

因为黑格尔把人和自我意识等同起来,所以人的异化了的对象,人的异化了的、本质的现实性,不外就是异化的意识,只是异化的思想,是异化的抽象的因而无内容的和非现实的表现,即否定。因此,外化的扬弃也不外是对这种无内容的抽象进行抽象的、无内容的扬弃,即否定的否定。因此,自我对象化的内容丰富的、活生生的、感性的、具体的活动,就成为这种活动的纯粹抽象——绝对的否定性,而这种抽象也被抽象地固定下来并且被想象为独立的活动,即干脆被想象为活动。因为这种所谓否定性无非就是上述现实的、活生生的行动的抽象的无内容的形式,所以它的内容也只能是形式的、抽去一切内容而产生的内容。因此,这就是普遍的,抽象的,适合任何内容的,从而既超脱任何内容同时又恰恰对任何内容都有效的,脱离现实的精神和现实的自然界的抽象形式、思维形式、逻辑范畴。(下文我们将阐明绝对的否定性的逻辑内容。)

黑格尔在这里——在它的思辨的逻辑学里——所完成的积极的东西在于;独立于自然界和精神的特定概念、普遍的固定的思维形式,是人的本质普遍异化的必然结果,因而也是人的思维普遍异化的必然结果;因此,黑格尔把它们描绘成抽象过程的各个环节,并且把它们连贯起来了。例如,扬弃了的存在是本质,扬弃了的本质是概念,扬弃了的概念……是绝对观念。然而,绝对观念究竟是什么呢?如果绝对观念不愿意再去重头经历全部抽象活动,不想再满足于充当种种抽象的总体或充当理解自我的抽象,那末,绝对观念也要再一次扬弃自身。但是,把自我理解为抽象的抽象,知道自己是无;它必须放弃自身,放弃抽象,从而达到那恰恰是它的对立面的本质,达到自然界。因此,全部逻辑学都证明,抽象思维本身是无,绝对观念本身是无,只有自然界才是某物。

〔XXXII〕绝对观念、抽象观念

\begin{fangsong}
“从它与自身统一这一方面来考察就是直观”(黑格尔《全书》第3版第222页),它“在自己的绝对真理中决心把自己的特殊性这一环节,或最初的规定和异在这一环节,即作为自己的反应的直接观念,从自身释放出去,也就是说,把自身作为自然界从自身释放出去”
\end{fangsong}

举动如此奇妙而怪诞、使黑格尔分子伤透了脑筋的这整个观念,无非始终是抽象,即抽象思维者,这种抽象由于经验而变得聪明起来,并且弄清了它的真相,于是在某些——虚假的甚至还是抽象的——条件下决心放弃自身,而用自己的异在,即特殊的东西、特定的东西,来代替自己的在自身的存在(非存在),代替自己的普遍性和不确定性;决心把那只是作为抽象、作为思想物而隐藏在它里面的自然界从自身释放出去,也就是说,决心抛弃抽象而去看一看摆脱了它的自然界。直接成为直观的抽象观念,无非始终是那种放弃自身并且决心成为直观的抽象思维。从逻辑学到自然哲学的这整个过度,无非是对抽象思维者来说如此难以实现、因而由他作了如此牵强附会的描述的从抽象到直观的过渡。有一种神秘的感觉驱使哲学家从抽象思维转向直观,那就是厌烦,就是对内容的渴望。

(同自身相异化的人,也就是同自己的本质即同自己的自然的和人的本质相异化的思维者。因此,他的那些思想是居于自然界和人之外的僵化的精灵。黑格尔把这一切僵化的精灵统统禁锢在他的逻辑学里,先是把它们每一个都看成否定,即人的思维的外化,然后又把它们看成否定的否定,即看成这种外化的扬弃,看成人的思维的现实的表现;但是,这种否定的否定——尽管仍然被束缚在异化中——,它一部分是使原来那些僵化的精灵在它们的异化中恢复,一部分是停留于最后的行动中,也就是在作为这些僵化的精灵的真实存在的外化中自身同自身相联系\footnote{这就是说,黑格尔用那在自身内部绕圈的抽象行动来代替这些僵化的抽象概念;于是,他就有了这样的贡献:他指明了原来属于各个哲学家的一切不适当的概念的诞生地,把他们综合起来,并且创造出一个在自己整个范围内穷尽一切的抽象作为批判的对象,以代替某种特定的抽象。(我们在下面将会看到,黑格尔为什么把思维同主体分隔开来;但就是现在也已经很清楚:如果没有人,那末人的本质表现也不可能是人的,因此思维也不能被看作是人的本质表现,即在社会、世界和自然界生活的有眼睛、耳朵等等的人和自然的主体的本质表现。)
};一部分则由于这种抽象理解了自身并且对自身感到无限的厌烦,所以,在黑格尔那里放弃抽象的、只在思维中运动的思维,即无眼、无牙、无耳、无一切的思维,便表现为决心承认自然界是本质并且转而致力于直观。)

〔XXXIII〕但是,被抽象地理解的,自为的,被确定为与人分割开来的自然界,对人来说也是无。不言而喻,这位决心转向直观的抽象思维者是抽象地直观自然界的。正向自然界曾经被思维者禁锢于他的这种对他来说也是隐密的和不可思议的形式即绝对观念、思想物中一样,现在,当他把自然界从自身释放出去时,他实际上从自身释放出去的只是这个抽象的自然界,只是自然界的思想物——不过现在具有这样一种意义,即这个自然界是思想的异在,是现实的、被直观的、有别于抽象思维的自然界——,只是自然界的思想物。或者,如果用人的语言来说,抽象思维者在他直观自然界时了解到,他在神性的辩证法中以为是从无、从纯抽象中创造出来的那些本质——在自身中转动的并且在任何地方都不向现实看一看的思维劳动的纯粹产物——无非是自然界诸规定的抽象概念。因此,对他来说整个自然界不过是在感性的、外在的形式下重复逻辑的抽象概念而已。他重新把自然界分解为这些抽象概念。因此,他对自然界的直观不过是他把对自然界的直观抽象化的确证行动,不过是他有意识地重复他的抽象概念的产生过程。例如,时间等于自身同自身相联系的否定性(前引书;第238页)。扬弃了的运动即物质——在自然形式中——同扬弃了的生成即定在相符合。光是反射于自身的自然形式。像月亮和彗星这样的物体,是对立物的自然形式,按照《逻辑学》,这种对立物一方面是以自身为根据的肯定的东西,而另一方面又是以自身为根据的否定的东西。地球是作为对立物的否定性统一的逻辑根据的自然形式,等等。

作为自然界的自然界,这是说,就它还在感性上不同于它自身所隐藏的神秘的意义而言,与这些抽象概念分割开来并并与这些抽象概念不同的自然界,就是无,是证明自己为无的无,是无意义的,或者只具有应被扬弃的外在性的意义。

\begin{fangsong}
“有限的目的论的观点包含着一个正确的前提,即自然界本身并不包含着绝对的目的。”(第225页)
\end{fangsong}

自然界的目的就在于对抽象的确证。

\begin{fangsong}
“结果自然界成为具有异在形式的观念。既然观念在这里表现为对自身的否定或外在于自身的东西,那末自然界并非只在相对的意义上对这种观念说来是外在的,而是外在性构成这样的规定,观念在其中表现为自然界。”(第227页)
\end{fangsong}

这里不应该把外在性理解为显露在外的并且对光、对感性的人敞开的感性;在这里应该把外在性理解为外化,理解为不应有的偏差、缺陷。因为真实的东西毕竟是观念。自然界不过是观念的异在的形式。而既然抽象思维是本质,那末外在于它的东西,就其本质来说,不过是某种外在的东西。抽象思维者同时承认感性、同在自身中转动的思维相对立的外在性,是自然界的本质。但是,它同时又把这种对立说成这样,即自然界的这种外在性,自然界同思维的对立,是自然界的缺陷;就自然界不同于抽象而言,自然界是个有缺陷的存在物。〔XXXIV〕一个不仅对我来说、在我的眼中有缺陷而且本身就有缺陷的存在物,在它自身之外有一种为它所缺少的东西。这就是说,它的本质是不同于它自身的另一种东西。因此,对抽象思维者说来,自然界必须扬弃自身,因为他已经把自然界设定为潜在地被扬弃的本质。

\begin{fangsong}
“对我们来说,精神以自然界为自己的前提,精神是自然界的真理,因而对自然界来说,精神也是某种绝对第一性的东西。在这个真理中自然界消逝了,结果精神成为达到其自为的存在的观念,而概念则既是观念的客体,又是观念的主体。这种同一性是绝对的否定性,因为概念在自然界中有自己的完满的外在的客观性,但现在它的这种外化被扬弃了。而概念在这种外化中成了与自己同身的东西。因此,概念只有作为从自然界的回归才是这种同一性。”(第392页)

“启示,作为抽象观念,是向自然界的直接的过渡,是自然界的生成,而作为自由精神的启示,则是自由精神把自然界设定为自己的世界,——这种设定,作为反思,同时又是把世界假定为独立的自然界。概念中的启示,是精神把自然界创造为自己的存在,而精神在这个存在中获得自己的自由的确证和真实性。”“绝对的东西是精神;这是绝对的东西的最高定义。”〔XXXIV〕
\end{fangsong}

\newpage
\subsection{2.笔记}
马克思认为私有财产的关系是劳动、资本以及二者的关系。他指出劳动与资本之间的关系经历了三个阶段:(1)直接的或间接的统一(2)对立(3)各自同自身的对立。

在这里(1)和(2)是很好理解的:资本和劳动之间的关系最初便是一体的,资本就是劳动,劳动就是资本,而随着历史的发展,资本与劳动之间显现出了对立性的关系,劳动服从于资本。但是为什么马克思又说劳动与资本又会经历各自同自身相对立的阶段呢?对于(3)而言,马克思原文是这么叙述的:

\begin{fangsong}
    “〔第三〕:二者各自同自身对立。资本=积累的劳动=劳动。作为这样的
东西,资本分解为自身和自己的利息,而利息又分解为利息和利润。资本家
完全成为牺牲品。他沦为工人阶级,正像工人-但只是例外地-成为资本
家一样。劳动是资本的要素,是资本的费用,因而,工资是资本的牺牲。”
\end{fangsong}

事实上,在分析这段话的时候,我们要明白马克思在这一语境中所规定的前提,马克思在指出劳动与资本之间关系所经历的阶段之时,他是在\textbf{私有财产}这一框架内讲述的。也即是说,在(1)(2)(3)这三个阶段中,劳动与资本之间发生的关系一直处于\textbf{私有财产关系}的自身运动之中。

因此,我们可以这么理解马克思所指出的(3)这一阶段,即(3)是(2)的尖锐化。马克思在这里并没有认为对于\textbf{私有财产}而言,它自身蕴含着解决问题的答案,相反,马克思认为私有财产的自我运动所指向的是其自身的毁灭(即自我矛盾的不可克服)。换句话说,马克思认为私有财产的发展本身就是一种历史的异化过程,资本起初作为一种人的力量(资本与劳动的统一)逐步走向同人的对立。在这一对立的过程中,资本起初是同作为代表着劳动的劳动者相对立,后来甚至同\textbf{一切人}相对立:1.就劳动者阶级而言,资本对于劳动的需求导致劳动者之间互相竞争,劳动阶级内部便产生了互相对立的因素。2.就资产阶级而言,资本的逐利性导致资本之间的斗争,资本家阶级内部便出现了区域性的斗争\footnote{诚然,在面对阶级整体利益时,资产阶级还是比较团结的,但是他们内部的斗争依旧存在着,因为对于资产阶级中的个体而言,\textbf{竞争的目的就是为了垄断},这在现实社会中已经有许多经验性的例子了。},部分资本家便作为牺牲品而下落到无产阶级的队伍中。

因此,从这个意义上来说,马克思甚至不仅仅是站在无产阶级的立场上去谈论私有财产的运动,而是站在整个人类社会的角度上谈论私有财产的运动。在私有财产的框架下,资本不仅对于无产者而言是一种束缚,且对于资本家而言也是他们的枷锁。\textbf{因为一切人都被异化了。}

马克思认为在国民经济学中反映的是私有财产的本质,这和恩格斯的《国民经济学批判大纲》中的观点类似。私有财产的主体本质就是劳动,也即是说私有财产的本质是某种\textbf{人的属性}。因此,马克思进一步指出,那些仅仅认为私有财产是某种对人而言的对象物的观念是\textbf{拜物教}。在这里马克思的原文是这么说的:

\begin{fangsong}
“因此,
在这种揭示了——在私有制范围内——财富的主体本质的启蒙国民经济学看来,
那些认为私有财产对人来说仅仅是对象性的本质的货币主义体系和重商主
义体系地拥护者,是拜物教徒、天主教徒。所以,恩格斯有理由把亚当·斯
密称作国民经济学的路德。正像路德认为宗教、信仰为外部世界的本质并
以此反对天主教异教一样,正像他把宗教观念变成人的内在本质,从而扬
弃了外在的宗教观念一样,正像他把教士移到俗人心中,因而否定了俗人
之外存在的教士一样,由于私有财产体现为在人本身中,而人本身被认为
是私有财产的本质,因而在人之外并且不依赖于人的财富,也就是只以外
在方式来保存和保持的财富被扬弃了,换言之,财富这种外在的、无思想的
对象性就被扬弃了,但正因为这个缘故,人本身被当成了私有财产的规定,
就像在路德那里被当成了宗教的规定一样。”
\end{fangsong}

马克思在这里用了类比的手法形象地说明了私有财产与人之间的关系。私有财产与人之间的关系恰如宗教与人之间的关系,私有财产的神秘面纱——即它的\textbf{至高性}\footnote{这里笔者用了“至高性”一词,可以这么理解:私有财产相对于人而言所具有控制力,似乎它天生就具有某种对人的控制力量,仿佛至高的天神一样。}原则,并非某种\textbf{脱离人之外}的虚幻力量,相反,私有财产的本质就是人的本质,国民经济学做的工作正是揭露了私有财产与人之间关系的面纱。但是国民经济学所做的工作并不是真正具有革命性的,因为它仅仅揭露了一种既定的现实,而并没有对这种现实进行更深入的批判。国民经济学使私有财产与人之间的\textbf{分离}转变成了私有财产与人之间的\textbf{统一},但这种\textbf{统一}是\textbf{颠倒的统一}。因此,对于国民经济学自身而言,它是矛盾的、伪善的,它没有看到它所声称的私有财产与人的统一与现实的私有财产与人的统一二者是主客颠倒的,它声称财富的本质是人的本质,以承认人的独立性与个性,但现实情况是人的本质似乎变成了财富的本质,人相较于财富本身而言并不具备独立性与个性。我们可以很清晰地看到马克思在原文中是这么叙述的:

\begin{fangsong}
“……因此,以劳动为原则的国民经
济学,在承认人的假象下,毋宁说不过是彻底实现对人的否定而已,因为
人本身已不再同私有财产的外在本质处于外部的紧张关系中,而人本身却成了私有财产的这种紧张的本质。以前是人之外的存在——人的实际外化
——的东西,现在仅仅变成外化的行为,变成了外在化。因此,如果说上述
国民经济学是从表面上承认人、人的独立性、自主活动等等开始,并由于把
私有财产转为人自身的本质而能够不再束缚于作为存在于人之外的本质的
私有财产的那些地域性的、民族的等等的规定,从而发挥一种世界主义的、
普遍的、摧毁一切界限和束缚的能量,以便自己作为惟一的政策、普遍性、
界限和束缚取代这些规定,-那末,国民经济学在它往后的发展过程中必
定抛弃这种伪善性,而使自己的犬儒主义充分表现出来。”

……

因为他们把具有活动形式的私有财产变为主体,就是说,既使人成为本质,
又同时使作为某种非存在物〔Unwesen〕的人成为本质,所以,现实中的矛
盾就完全符合他们视为原则的那个充满矛盾的本质。支离破碎的工
业现实不仅没有推翻,相反地,却证实了他们的自身支离破碎的原则。他们
的原则本来就是这种支离破碎状态的原则。”
\end{fangsong}



\section{德谟克利特的自然哲学和伊壁鸠鲁的自然哲学的差别(马克思博士论文)}

\subsection{1.原文节选}
\subsubsection{序言}
这篇论文如果当初不是预定作为博士论文,那么它一方面可能会具有更加严格的科学形式,另一方面在某些叙述上也许会少一点学究气。但是,由于一些外在的原因,我只能让它以这种形式付印。此外,我认为,在这篇论文里我已经解决了一个在希腊哲学史上至今尚未解决的问题。

专家们知道,关于这篇论文的对象没有任何先前的著作可供参考。西塞罗和普卢塔克所说过的废话,到现在人们一直在照样重复。伽桑狄虽然把伊壁鸠鲁从教父们和整个中世纪即实现了非理性的时代所加给他的禁锢中解救了出来,但是在自己的阐述\footnote{指皮·伽桑狄编的《第欧根尼·拉尔修,第10卷:<论伊壁鸠鲁的生平、习惯和见解>注释本》1649年里昂版。——编者注}里也只提供了一个有趣的方面。他竭力要使他的天主教的良心同他的异端知识相适应,使伊壁鸠鲁同教会相适应,这当然是白费气力。这就好比是想在希腊拉伊丝的姣美的身体上披上一件基督教修女的黑衣。确切地说,伽桑狄是自己在向伊壁鸠鲁学习哲学,他不能向我们讲授伊壁鸠鲁哲学。

不妨把这篇论文仅仅看作是一部更大著作\footnote{马克思后来没有写出这部著作。——编者注}的先导,在那部著作中我将联系整个希腊思辨详细地阐述伊壁鸠鲁主义\footnote{伊壁鸠鲁主义是公元前4—3世纪产生于古希腊的一个哲学派别,因其创始人伊壁鸠鲁在雅典自家花园中宣讲他的原子说和无神论,也称为花园学派。伊壁鸠鲁哲学包括物理学(关于自然界的学说)、准则学(关于认识的学说)和伦理学。伊壁鸠鲁对自然界作了唯物主义的解释,进一步发展了德谟克利特的原子论学说,认为原子有三种运动:直线式的下落运动、脱离直线的偏斜运动和由此产生的碰撞运动,从而提出了物质运动的内在源泉思想。在认识论上,他认为感觉是判断真理的标准。在伦理学方面,伊壁鸠鲁认为,由于组成人的灵魂的原子具有脱离直线作偏斜运动的倾向,因而人的行为有可能脱离命定的必然性,获得意志和行为的自由。他斥责对神的崇拜和迷信,蔑视命运,强调事在人为,认为人类行为的目的是从痛苦和恐惧中解放出来,求得快乐,快乐是幸福生活的目的,是善的唯一标准,人应当通过哲学认识自然和人生,用理性规划自己的生活。在社会政治观点方面。他首先提出了原始的社会契约说。——11、212、453。},斯多亚主义\footnote{斯多亚主义是公元前4—3世纪产生于古希腊的一个哲学派别,因其创始人芝诺通常在雅典集市的画廊讲学,又称画廊学派(画廊的希腊文是“stoa”)。斯多亚派哲学分为逻辑学、物理学和伦理学,以伦理学为中心,逻辑学和物理学只是为伦理学提供基础。这个学派主要宣扬服从命运并带有浓厚宗教色彩的泛神论思想,其中既有唯物主义倾向,又有唯心主义思想。早期斯多亚派认为认识来源于对外界事物的感觉,但又承认关于神、善恶、正义等的先天观念。他们把赫拉克利特的火和逻各斯看成一个东西,认为宇宙实体既是物质性的,同时又是创造一切并统治万物的世界理性,也是神、天命和命运,或称自然。人是自然的一部分,也受天命支配,人应该顺应自然的规律而生活,即遵照理性和道德而生活。合乎理性的行为就是德行,只有德行才能使人幸福。人要有德行,成为善人,就必须用理性克制情欲,达到清心寡欲以至无情无欲的境界。中期斯多亚派强调社会责任、道德义务,加强了道德生活中的禁欲主义倾向。晚期斯多亚派宣扬安于命运,服从命运,认为人的一生注定是有罪的、痛苦的,只有忍耐和克制欲望,才能摆脱痛苦和罪恶,得到精神的安宁和幸福。晚期斯多亚派的伦理思想为基督教的兴起准备了思想条件。——11、212、452。}和怀疑主义\footnote{怀疑主义是公元前4—3世纪产生于古希腊的一个哲学派别,代表人物有皮浪、阿克西劳、卡内亚德、埃奈西德穆及恩披里柯。怀疑派哲学是对客观世界和客观真理是否存在、能否认识表示怀疑的哲学学说。它认为事物是不可认识的,因为对每一事物都可以有两种相互排斥的意见;既然人们什么也不能确定,就应该放弃判断,放弃认识,平心静气地求得精神的安宁。怀疑主义揭示了人们认识中的矛盾,在认识史上有一定的意义。但是它反对唯物主义,不相信理性的力量,否定科学知识,实际上为宗教迷信和神秘主义的传播提供了条件。——11、212。}这一组哲学。这篇论文在形式方面和其他方面的缺点在那里将被消除。

虽然黑格尔大体上正确地规定了上述各个体系的一般特点,但是一方面,由于他的哲学史——一般说来哲学史只能从它开始——的令人惊讶的庞大和大胆的计划,使他不能深入研究个别细节;另一方面,黑格尔对于他主要称之为思辨的东西的观点,也妨碍了这位巨人般的思想家认识上述那些体系对于希腊哲学史和整个希腊精神的重大意义。这些体系是理解希腊哲学的真正历史的钥匙。关于它们同希腊生活的联系,在我的朋友科本的著作《弗里德里希大帝和他的敌人》\footnote{科本《弗里德里希大帝和他的敌人》是献给“朋友、特里尔的卡尔·马克思”的。科本在该书1840年版第39页上写道:“伊壁鸠鲁主义、斯多亚主义和怀疑主义是古代有机体的神经系统、肌肉系统和内脏系统,它们的直接的自然的统一决定了古代的美和道德,它们也随着古代的衰亡而瓦解。”——11。}中有较深刻的提示。

如果说这里以附录的形式增加了一篇评普卢塔克对伊壁鸠鲁神学的论战的文章,那么这样做,是因为这场论战不是什么个别的东西,而是代表着一定的方向,因为它本身就很恰当地表明了神学化的理智对哲学的态度。

此外,在这篇评论中,对于普卢塔克把哲学带上宗教法庭的立场是如何地错误,我还没有谈到。关于这点,无需任何论证,只要从大卫·休谟那里引证一段话就够了:

\begin{fangsong}
    “如果人们迫使哲学在每一场合为自己的结论辩护,并在对它不满的任何艺术和科学面前替自己申辩,对理应到处都承认享有最高权威的哲学来说,当然是一种侮辱。这就令人想起一个被指控犯了背叛自己臣民的叛国罪的国王。”\footnote{大·休谟《人性论》德文译本1790年哈雷版第1卷第485页。——编者注}
\end{fangsong}

只要哲学还有一滴血在自己那颗要征服世界的、绝对自由的心脏里跳动着,它就将永远用伊壁鸠鲁的话向它的反对者宣称:

\begin{fangsong}
    “渎神的并不是那抛弃众人所崇拜的众神的人,而是把众人的意见强加于众神的人。”\footnote{马克思根据第欧根尼·拉尔修《论哲学家的生平》第10卷第123页,引用了伊壁鸠鲁致梅诺伊凯乌斯的信中的一段话;这段引文以及下面引自埃斯库罗斯著作的引文,马克思是用希腊文摘抄的。——12。}
\end{fangsong}

哲学并不隐瞒这一点。普罗米修斯的自白

\begin{fangsong}
    “总而言之,我痛恨所有的神”\footnote{鲍威尔在1841年4月12日的信中建议马克思不要把超出“哲学发展”的埃斯库罗斯《被锁链锁住的普罗米修斯》中的诗句(即“我痛恨所有的神”)放进博士论文,认为这样将不利于马克思谋求波恩大学的教职。——12。}
\end{fangsong}

 就是哲学自己的自白,是哲学自己的格言,表示它反对不承认人的自我意识是最高神性的一切天上的和地上的神。不应该有任何神同人的自我意识相并列。

对于那些以为哲学在社会中的地位似乎已经恶化因而感到欢欣鼓舞的可怜的懦夫们,哲学又以普罗米修斯对众神的侍者海尔梅斯所说的话来回答他们:

    \begin{fangsong}
 我绝不愿像你那样甘受役使,来改变自己悲惨的命运,
    你好好听着,我永不愿意!
    是的,宁可被缚在崖石上,
    也不为父亲宙斯效忠,充当他的信使。”       
    \end{fangsong}

普罗米修斯是哲学历书上最高尚的圣者和殉道者。

1841年3月于柏林

\subsubsection{第一部分~德谟克利特的自然哲学和伊壁鸠鲁的自然哲学的一般差别}
\textbf{一、论文的对象}

 希腊哲学看起来似乎遇到了一出好的悲剧所不应遇到的结局,即平淡的结局。在希腊,哲学的客观历史似乎在亚里士多德这个希腊哲学中的马其顿王亚历山大那里就停止了,甚至勇敢坚强的斯多亚派也没有取得像斯巴达人在他们的庙宇里所取得的那样的胜利:他们把雅典娜紧紧捆在海格立斯身旁,使她不能逃走。

伊壁鸠鲁派、斯多亚派、怀疑派几乎被看作一种不合适的附加物,同他们的巨大的前提很不相称。伊壁鸠鲁哲学似乎是德谟克利特的物理学和昔勒尼派\footnote{昔勒尼派是公元前5世纪古希腊小苏格拉底学派之一,其创始人为苏格拉底的学生,昔勒尼的亚里斯提卜。这个学派接受并发展了苏格拉底关于善的概念,认为人们天生追求的快乐就是善,人不能被快乐所支配,应该主宰快乐。昔勒尼派的学说对伊壁鸠鲁的伦理学有一定的影响。——15。}的道德思想的混合物;斯多亚主义好像是赫拉克利特的自然思辨和昔尼克派\footnote{昔尼克派是公元前3世纪古希腊的一个主张自然主义的哲学学派,又译“犬儒学派”,由安提斯泰尼所创立。这个学派崇尚自然,但是把自然和社会绝对对立起来,认为一切人间的文明享受都是有害的,理想的生活应是极端简朴的原始生活,提出“德行本身就是幸福”,主张屏弃一切人间的享受。昔尼克派对后来的斯多亚派有一定的影响。——15。}的伦理世界观的结合,也许再加上一点亚里士多德的逻辑学;最后,怀疑主义则仿佛是同这两种独断主义相对立的必不可免的祸害。这样,人们在把这些哲学说成是更加片面而更具有倾向性的折衷主义时,也就不自觉地把它们同亚历山大里亚哲学\footnote{亚历山大里亚哲学指公元3—6世纪流行于古罗马的新柏拉图主义,是由亚历山大里亚的阿·萨卡创立的一个唯心主义哲学流派。亚历山大里亚哲学以柏拉图哲学为基础,吸收了亚里士多德学派、斯多亚派和毕达哥拉斯学派和东方宗教神秘主义的观点,反映了罗马帝国衰亡时期奴隶主贵族的思想。——15。}联系在一起。最后,亚历山大里亚哲学则被看成是一种完全的幻想和混乱——一种紊乱,在这种紊乱里据说最多只能承认意向的普遍性。

的确,有一种老生常谈的真理,说发生、繁荣和衰亡是一个铁环,一切与人有关的事物都注定包含于其中,并且必定要绕着它走一圈。所以,说希腊哲学在亚里士多德那里达到极盛之后,接着就衰落了,这也没有什么可惊奇之处。不过英雄之死与太阳落山相似,而和青蛙因胀破了肚皮致死不同。

此外,发生、繁荣和衰亡是极其一般、极其模糊的观念,要把一切东西都塞进去固然可以,但要借助这些观念去理解什么东西却办不到。死亡本身已预先包含在生物中,因此对死亡的形态也应像对生命的形态那样,在固有的特殊性中加以考察。

最后,如果我们回顾一下历史,难道伊壁鸠鲁主义、斯多亚主义和怀疑主义是一些特殊现象吗?难道它们不是罗马精神的原型,即希腊迁移到罗马去的那种形态吗?难道它们不具有性格十分刚毅的、强有力的、永恒的本质,以致连现代世界也不得不承认它们享有充分的精神上的公民权吗?

我强调指出这一点,只是为了唤起对于这些体系的历史重要性的记忆。但是,这里要研究的并不是它们对于整个文化的一般意义;这里要研究的是它们同更古老的希腊哲学的联系。

有人认为,希腊哲学是以两类不同的折衷主义体系为终结的,其中一类是伊壁鸠鲁主义、斯多亚主义和怀疑主义这一组哲学,另一类统称为亚历山大里亚的思辨,难道这种看法不应促使人们至少联系这种关系去加以探讨吗?其次,在正在向总体发展的柏拉图哲学和亚里士多德哲学之后,出现了一些新的体系,它们不以这两种丰富的精神形态为依据,而是进一步往上追溯到最简单的学派:在物理学方面转向自然哲学家,在伦理学方面转向苏格拉底学派,难道这不是值得注意的现象吗?再者,在亚里士多德之后出现的体系,仿佛都可以在往昔找到它们现成的基础,这种说法有何根据呢?把德谟克利特和昔勒尼派、赫拉克利特和昔尼克派结合在一起,这又怎样予以说明呢?在伊壁鸠鲁派、斯多亚派和怀疑派那里,自我意识的一切环节都得到充分表现,不过每个环节都表现为一种特殊的存在,难道这是偶然的吗?这些体系合在一起形成自我意识的完整结构,这也是偶然的吗?最后,希腊哲学借以神话般地从七贤\footnote{七贤是指米利都的泰勒斯、米蒂利尼的皮达科斯、普林纳的比亚士、雅典的梭伦、斯巴达的奇仑、科林斯的柏连德、罗得岛的克莱奥布洛斯。——17。}开始,并且仿佛作为这一哲学的中心点,作为这一哲学的造物主体现在苏格拉底身上的形象,我指的是哲人——σοφοξ——的形象,这种形象被上述那些体系说成是真正科学的现实,难道这也是偶然的吗?

在我看来,如果说那些较早的体系对于希腊哲学的内容较为重要、较有意义的话,那么亚里士多德以后的体系,主要是伊壁鸠鲁派、斯多亚派和怀疑派这一组学派则对希腊哲学的主观形式,对其性质较为重要、较有意义。然而正是这种主观形式,即这些哲学体系的精神承担者,由于它们的形而上学的规定,直到现在几乎完全被遗忘了。

关于伊壁鸠鲁派、斯多亚派和怀疑派哲学的全部概况,以及它们与较早的和较晚的希腊思辨的总体关系,我打算在一部更为详尽的著作里加以阐述。\footnote{马克思后来没有写出这部著作。——编者注}

在这里,好像通过一个例子,并且也只从一个方面,即从它们与较早的思辨的联系方面,来阐述这种关系也就足够了。

我选择了伊壁鸠鲁的自然哲学同德谟克利特的自然哲学的关系作为这样一个例子。我并不认为这是一个最便当的出发点。因为,一方面人们有一个根深蒂固的旧偏见,即把德谟克利特的物理学和伊壁鸠鲁的物理学等同起来,以致把伊壁鸠鲁所作的修改看作只是一些随心所欲的臆造;另一方面,就具体情况来说,我又不得不去研究一些看起来好像无关紧要的细枝末节。但是,正因为这种偏见同哲学的历史一样古老,而二者之间的差别又极其隐蔽,好像只有用显微镜才能发现它们,所以,尽管德谟克利特的物理学和伊壁鸠鲁的物理学之间有着联系,但是证实存在于它们之间的贯穿到极其细微之处的本质差别就显得特别重要了。在细微之处可以证实的东西,当各种情况在更大范围表现出来的时候就更容易加以说明了,相反,如果只作极其一般的考察,就会令人怀疑所得出的结论究竟是否在每一个别场合都能得到证实。

\textbf{二、对德谟克利特的物理学和伊壁鸠鲁的物理学的关系的判断}

一般地说,我的见解和前人的见解关系怎样,只要粗略地考察一下古代人对德谟克利特的物理学和伊壁鸠鲁的物理学的关系的判断,就一目了然了。

斯多亚派的波西多尼乌斯、尼古拉和索蒂昂指责伊壁鸠鲁,说他把德谟克利特关于原子的学说和亚里斯提卜关于快乐的学说冒充为他自己的学说。学院派的科塔问西塞罗:“在伊壁鸠鲁的物理学中究竟有什么东西不是属于德谟克利特的呢?伊壁鸠鲁诚然改变了一些地方,但大部分是重复德谟克利特的话。”西塞罗自己也说:“伊壁鸠鲁在他特别夸耀的物理学中,是一个地道的门外汉,其中大部分是属于德谟克利特的;在伊壁鸠鲁离开德谟克利特的地方,在他想加以改进的地方,他都损害和败坏了德谟克利特。”不过,虽然有许多人指责伊壁鸠鲁诽谤德谟克利特,但是,据普卢塔克说,莱昂泰乌斯断言,伊壁鸠鲁很尊敬德谟克利特,因为德谟克利特在他之前就宣示了正确的学说,因为德谟克利特更早发现了自然界的本原。在《论哲学家的见解》\footnote{《论哲学家的见解》虽然以普卢塔克的名义流传下来,但并不是出自他的手笔,而是另一位不知名的作者写的。——19。}这一著作中,伊壁鸠鲁被称为按照德谟克利特的观点探究哲理的人。普卢塔克在他的著作《科洛特》中走得更远。当他依次将伊壁鸠鲁同德谟克利特、恩培多克勒、巴门尼德、柏拉图、苏格拉底、斯蒂尔蓬、昔勒尼派和学院派加以比较时,他力求得出这样的结论:“伊壁鸠鲁从整个希腊哲学里吸收的是错误的东西,而对其中正确的东西他并不理解。”《论信从伊壁鸠鲁不可能有幸福的生活》这篇论文也充满了类似的敌意的暗讽。

古代作家的这种不利的见解,在教父们那里仍然保留着。我在附注里只引证了亚历山大里亚的克莱门斯这位教父的一句话,在谈到伊壁鸠鲁时特别值得提到他,因为他把使徒保罗警告人们提防一般哲学的话说成是警告人们提防伊壁鸠鲁哲学的话,说这种哲学连天意之类的东西都没有幻想过。但是,人们一般都倾向于指责伊壁鸠鲁有剽窃行为,在这方面塞克斯都·恩披里柯表现得最为突出,他企图把荷马和厄皮卡尔摩斯的一些完全不相干的语句,硬说成是伊壁鸠鲁哲学的主要来源。

众所周知,近代作家大体上也同样认为,伊壁鸠鲁作为一个自然哲学家,仅仅是德谟克利特的剽窃者。莱布尼茨有一段话大致可以代表他们的见解:

“关于这个伟大人物<德谟克利特>,我们所知道的东西,几乎只是伊壁鸠鲁从他那里抄袭来的,而伊壁鸠鲁又往往不能从他那里抄袭到最好的东西。”

因此,如果说西塞罗认为,伊壁鸠鲁败坏了德谟克利特的学说,但他至少还承认伊壁鸠鲁有改进德谟克利特学说的愿望,还有看到这个学说的缺点的眼力;如果说普卢塔克认为他观点前后不一贯,认为他对坏的东西有一种天生的偏爱,因而对他的愿望也表示怀疑,那么,莱布尼茨则甚至否认他具有善于摘录德谟克利特的能力。

不过,大家一致认为,伊壁鸠鲁的物理学是从德谟克利特那儿抄袭来的。

\textbf{三、把德谟克利特的自然哲学和伊壁鸠鲁的自然哲学等同起来所产生的困难}

除了历史的证据之外,许多情况也说明德谟克利特和伊壁鸠鲁的物理学的同一性。原子和虚空这两个本原无可争辩地是相同的。只是在个别的规定中,任意的、因而是非本质的差别看来才占统治地位。

不过,这样就留下一个奇特的、无法解开的谜。两位哲学家讲授的是同一门科学,并且采用的是完全相同的方式,但是——多么不合逻辑啊!——在一切方面,无论涉及这门科学的真理性、可靠性及其应用,还是涉及思想和现实的一般关系,他们都是截然相反的。我说他们是截然相反的,现在我将尽力证明这一点。

 (A)德谟克利特关于人类知识的真理性和可靠性的判断看来很难弄清楚。他有一些自相矛盾的语句,或者不如说,不是这些语句,而是德谟克利特的观点自相矛盾。特伦德伦堡在为亚里士多德心理学作的注释里说,知道这个矛盾的不是亚里士多德,而是晚近的作家,这个说法事实上是不正确的。在亚里士多德的心理学中有这样的话:“德谟克利特认为灵魂和理性是同一个东西,因为在他看来,现象是真实的东西。”与此相反,亚里士多德在《形而上学》中却说:“德谟克利特断言,或者没有东西是真实的,或者真实的东西对我们是隐蔽的。”亚里士多德的这几段话难道不是自相矛盾吗?如果现象是真实的东西,那么真实的东西怎么会是隐蔽的呢?只有现象和真理互相分离的地方,才开始有隐蔽的东西。但是第欧根尼·拉尔修说,有人曾把德谟克利特算作怀疑主义者。他们引证了他的一句名言:“实际上,我们什么也不知道,因为真理隐藏在深渊里。”类似的意见在塞克斯都·恩披里柯那里也可以看到。

德谟克利特的这种怀疑主义的、不确定的和内部自相矛盾的观点,在他规定原子和感性的现象世界的相互关系的方式中不过是得到了进一步的发展。

一方面,感性现象不是原子本身所固有的。它不是客观现象,而是主观的假象。“真实的本原是原子和虚空;其余的一切都是意见、假象。”“只有按照意见才有冷,只有按照意见才有热,而实际上只有原子和虚空。”因此,一实际上不是由若干原子组成,而是“任何一看起来都好像是由于原子的结合而形成的”。因此,只有通过理性才能看见本原,由于本原微小到肉眼都无法看见,所以它们甚至被称为观念。不过另一方面,感性现象是唯一真实的客体,并且“感性知觉”就是“理性”,而这种真实的东西是变化着的、不稳定的,它是现象。但是,说现象是真实的东西,这就自相矛盾了。因此,时而把这一面,时而把另一面当作主观的或客观的东西。这样矛盾似乎就被消除了,因为矛盾着的两个方面分别被分配给两个世界了。德谟克利特因而就把感性的现实变成主观的假象;不过,从客体的世界被驱逐出去的二律背反,却仍然存在于他自己的自我意识内,在自我意识里原子的概念和感性直观互相敌对地冲突着。

可见,德谟克利特并没有能摆脱二律背反。这里还不是阐明二律背反的地方,只要明白不能否认它的存在就够了。

 让我们反过来听听伊壁鸠鲁是怎么说的。

他说:哲人对事物采取独断主义的态度,而不采取怀疑主义的态度。是的,哲人比大家高明之处,正在于他对自己的认识深信不疑。“一切感官都是真实东西的报道者。”“没有什么东西能够驳倒感性知觉;同类的感性知觉不能驳倒同类的感性知觉,因为它们有相同的效用,而不同类的感性知觉也不能驳倒不同类的感性知觉,因为它们并不是对同一个东西作出判断;概念也不能驳倒感性知觉,因为概念依赖于感性知觉”,这是在《准则》中所说的话。当德谟克利特把感性世界变成主观假象时,伊壁鸠鲁却把它变成客观现象。而且在这里他是有意识地作出这种区别的,因为他断言,他赞成同样的原则,但是并不主张把感性的质看作是仅仅存在于意见中的东西。

因此,既然对伊壁鸠鲁来说感性知觉是标准,客观现象又符合于感性知觉,那么只好承认那使西塞罗耸耸肩膀的话是正确的结论:“太阳在德谟克利特看来是很大的,因为他是一个有学问的人,并且是对几何学有了完备知识的人;太阳在伊壁鸠鲁看来约莫有两英尺大,因为据他判断,太阳就是看起来那么大。”

(B)德谟克利特和伊壁鸠鲁关于科学的可靠性和科学对象的真实性的理论见解上的这种差别,体现在这两个人的不同的科学活动和实践中。

在德谟克利特那里,原则是不在现象中表现的,它始终是没有现实性和处于存在之外的,但是,他认为感性知觉的世界是实在的和富有内容的世界。这个世界虽然是主观的假象,但正因为如此,它才脱离原则而保持着自己的独立的现实性;同时作为唯一实在的客体,它本身具有价值和意义。因此,德谟克利特被迫进行经验的观察。他不满足于哲学,便投入实证知识的怀抱。我们已听说过,西塞罗称他为博学之士。他精通物理学、伦理学、数学,各个综合性科目\footnote{各个综合性科目在古代是指语言文学、辩证法、论辩术、音乐、算术、几何、天文学。——23、69。},各种技艺。第欧根尼·拉尔修所列举的德谟克利特的著作的目录就足以证明他的博学。而由于博学的特点是要努力扩大视野,搜集资料,到外部世界去探索,所以,我们就看见德谟克利特走遍半个世界,以便积累经验、知识和观感。德谟克利特自夸道:“在我的同时代人中,我游历的地球上的地方最多,考察了最遥远的东西;我到过的地区和国家最多,我听过的有学问的人的讲演也最多;而在勾画几何图形并加以证明方面,没有人超过我,就连埃及的所谓土地测量员也未能超过我。”

德米特里在《同名作家传》中,安提西尼在《论哲学家的继承》中都说,德谟克利特曾游历埃及并向祭司学习几何学,曾游历波斯,拜访迦勒底人,并且说他曾到达红海。有些人还说,他曾在印度会见过裸体智者,并且到过埃塞俄比亚。一方面求知欲使他不能平静,另一方面对真实的即哲学的知识的不满足,迫使他外出远行。他认为是真实的那种知识是没有内容的;而能向他提供内容的知识却没有真实性。古代人述说的关于德谟克利特的轶事可能是一种传闻,但是一种真实的传闻,因为它描述了德谟克利特的本质的矛盾。据说德谟克利特自己弄瞎了自己的眼睛,以使感性的目光不致蒙蔽他的理智的敏锐。这就是那个照西塞罗的说法走遍了半个世界的人。但是他没有获得他所寻求的东西。

伊壁鸠鲁则以一个相反的形象出现在我们面前。

伊壁鸠鲁在哲学中感到满足和幸福。他说:“要得到真正的自由,你就必须为哲学服务。凡是倾心降志地献身于哲学的人,用不着久等,他立即就会获得解放,因为服务于哲学本身就是自由。”因此,他教导说:“青年人不应该耽误了对哲学的研究,老年人也不应该放弃对哲学的研究。因为谁要使心灵健康,都不会为时尚早或者为时已晚。谁如果说研究哲学的时间尚未到来或者已经过去,那么他就像那个说享受幸福的时间尚未到来或者已经过去的人一样。”德谟克利特不满足于哲学而投身于经验知识的怀抱,而伊壁鸠鲁却轻视实证科学,因为按照他的意见,这种科学丝毫无助于达到真正的完善。他被称为科学的敌人,语言文学的轻视者。人们甚至骂他无知。在西塞罗的书中曾提到,有一个伊壁鸠鲁派说:“但是,不是伊壁鸠鲁没有学识,而是那些以为直到老年还应去背诵那些连小孩不知道都觉得可耻的东西的人,才是无知的人。”

可是,德谟克利特努力从埃及的祭司、波斯的迦勒底人和印度的裸体智者那里寻求知识,而伊壁鸠鲁却以他从未有过任何教师,他是一个自学者而自豪。据塞涅卡叙述,伊壁鸠鲁曾经说过,有些人努力寻求真理而无需任何人的帮助。作为这种人当中的一个,他自己为自己开辟了道路。他最称赞那些自学者。其他的人在他看来是第二流的人物。德谟克利特感觉到必须走遍世界各地,而伊壁鸠鲁却只有两三次离开他在雅典的花园到伊奥尼亚去,不是为了研究,而是为了访友。最后,德谟克利特由于对知识感到绝望而弄瞎了自己的眼睛,伊壁鸠鲁却在感到死亡临近之时洗了一个热水澡,要求喝醇酒,并且嘱咐他的朋友们忠实于哲学。

(C)不能把刚才所指出的那些差别归因于两位哲学家的偶然的个性;它们所体现的是两个相反的方向。我们看到,前面表现为理论意识方面的差别的东西,现在表现为实践活动方面的差别了。

最后,我们来考察一下表现思想同存在的关系,两者的相互关系的反思形式。哲学家在他所规定的世界和思想之间的一般关系中,只是为自己把他的特殊意识同现实世界的关系客观化了。

德谟克利特把必然性看作现实性的反思形式。关于他,亚里士多德说过,他把一切都归结为必然性。第欧根尼·拉尔修报道说,一切事物所由以产生的那种原子旋涡就是德谟克利特的必然性。《论哲学家的见解》的作者关于这点说得更为详细:“在德谟克利特看来,必然性是命运,是法,是天意,是世界的创造者。物质的抗击、运动和撞击就是这种必然性的实体。”类似的说法也出现在斯托贝的《自然的牧歌》里和欧塞比乌斯的《福音之准备》第6卷里。在斯托贝的《伦理的牧歌》里还保存着德谟克利特的一句话,在欧塞比乌斯的第14卷中这句话几乎被一字不差地重复了一遍,即:人们给自己虚构出偶然这个幻影,——这正是他们自己束手无策的表现,因为偶然和强有力的思维是敌对的。同样,西姆普利齐乌斯认为,亚里士多德在一个地方谈到一种取消偶然的古代学说时,也就是指德谟克利特而言的。

与此相反,伊壁鸠鲁说:“被某些人当作万物主宰的必然性,并不存在,无宁说有些事物是偶然的,另一些事物则取决于我们的任意性。必然性是不容劝说的,相反,偶然是不稳定的。所以,宁可听信关于神灵的神话,也比当物理学家所说的命运的奴隶要好些,因为神话还留下一点希望,即由于敬神将会得到神的保佑,而命运却是铁面无情的必然性。应该承认偶然,而不是像众人所认为的那样承认神。”“在必然性中生活,是不幸的事,但是在必然性中生活,并不是一种必然性。通向自由的道路到处都敞开着,这种道路很多,它们是便捷易行的。因此,我们感谢上帝,因为在生活中谁也不会被束缚住。控制住必然性本身倒是许可的。”

在西塞罗的书中曾提到过,伊壁鸠鲁派的韦莱关于斯多亚派哲学说过类似的话:“有一种哲学像年迈而又无知的妇人们一样认为,一切都由于命运而发生,我们应该怎样评价这种哲学呢?……伊壁鸠鲁拯救了我们,使我们获得了自由。”

为了避免承认任何必然性,伊壁鸠鲁甚至否定了选言判断。

不错,也有人断言,德谟克利特使用过偶然,但是,在西姆普利齐乌斯谈到这一点的两个地方中,一个地方却使另一个地方变得可疑,因为它清楚地表明,不是德谟克利特使用了偶然这一范畴,而是西姆普利齐乌斯把这一范畴作为结论强加给德谟克利特。西姆普利齐乌斯是这样说的,德谟克利特并没有指出一般的创造世界的原因,因此看来他是把偶然当作原因。但是,这里问题并不在于内容的规定,而在于德谟克利特有意识地使用过的那种形式。欧塞比乌斯的报道也与此相似:德谟克利特把偶然当作一般的东西和神性的东西的主宰,并断言这里一切都由于偶然而发生,同时他又把偶然从人的生活和经验的自然中排除掉,并斥责它的宣扬者愚蠢无知。

在这里,一方面我们看到,这纯粹是基督教主教迪奥尼修斯的臆断,另一方面,我们又看到,在一般的东西和神性的东西开始的地方,德谟克利特的必然性概念同偶然便没有差别了。

因此,从历史上看有一个事实是确实无疑的:德谟克利特使用必然性,伊壁鸠鲁使用偶然,并且每个人都以论战的激烈语调驳斥相反的观点。

这种差别的主要后果表现在对具体的物理现象的解释方式上。

在有限的自然里,必然性表现为相对的必然性,表现为决定论。而相对的必然性只能从实在的可能性中推演出来,这就是说,存在着一系列的条件、原因、根据等等,这种必然性是通过它们作为中介的。实在的可能性是相对必然性的展现。我们看到,德谟克利特曾使用过它。让我们从西姆普利齐乌斯那里引证一些材料来作证。

如果一个人感到口渴,喝了水并变得精神舒畅了,那么德谟克利特不会认为偶然是原因,而会认为渴是原因。因为尽管他讲到世界的创造时看来曾使用过偶然这一范畴,但他毕竟断言,在每个个别现象中偶然不是原因,而只是指出别的原因。例如,挖掘财宝是找到财宝的原因,或者种植橄榄树是橄榄树生长的原因。

德谟克利特在采用这种解释方式来研究自然时所表现的热情和严肃性,以及他认为寻找根据的意图所具有的重要意义,都在他下面这句自白里坦率地表达了出来:“我发现一个新的因果联系比获得波斯国的王位还要高兴!”

伊壁鸠鲁与德谟克利特又正相反。偶然是一种只具有可能性价值的现实性,而抽象的可能性则正是实在的可能性的反面。实在的可能性就像知性那样被限制在严格的限度里;而抽象的可能性却像幻想那样是没有限制的。实在的可能性力求证明它的客体的必然性和现实性;而抽象的可能性涉及的不是被说明的客体,而是作出说明的主体。只要对象是可能的,是可以想象的就行了。抽象可能的东西,可以想象的东西,不会妨碍思维着的主体,也不会成为这个主体的界限,不会成为障碍物。至于这种可能性是否会成为现实,那是无关紧要的,因为这里感兴趣的不是对象本身。

因此,伊壁鸠鲁在解释具体的物理现象时表现出一种非常冷淡的态度。

这一点在他给皮托克勒斯的信中可以看得更清楚,这封信我们后面还要加以考察。这里只须注意一下伊壁鸠鲁对先前的物理学家的意见的态度就够了。在《论哲学家的见解》的作者及斯托贝引证哲学家们关于星球的实体、太阳的体积和形状以及诸如此类的东西的不同观点的地方,他们谈到伊壁鸠鲁时总是说:他不反对这类意见中的任何一种意见;在他看来,所有的意见都可能是对的,他坚持可能的东西。的确,伊壁鸠鲁甚至对那种从实在的可能性出发的、为理智所规定的、因而带有片面性的解释方法,也加以驳斥。

因此,塞涅卡在他的《自然问题》中说道:伊壁鸠鲁断言,所有这些原因都可能存在,除此之外他还力图提出一些别的解释,并斥责那些断言在这些原因中只存在某一种原因的人,因为要给只是根据推测推论出来的东西下一个必然的判断,是一种冒险。我们可以看到,这里没有探讨客体的实在根据的兴趣。问题只在于使那作出说明的主体得到安慰。由于一切可能的东西都被看作是符合抽象可能性性质的可能的东西,于是很显然,存在的偶然就仅仅转化为思维的偶然了。伊壁鸠鲁所提出的唯一的规则,即“解释不应该同感性知觉相矛盾”是不言而喻的,因为抽象可能的东西正在于摆脱矛盾,因此矛盾是应该防止的。最后,伊壁鸠鲁承认,他的解释方法的目的在于求得自我意识的心灵的宁静\footnote{心灵的宁静(Ataraxy)是古希腊伦理学的概念,又译“不动心”。在伊壁鸠鲁的伦理学中,这是生活的最高理想,是通过认识自然、摆脱对神和死亡的恐惧而达到内心自由的哲人所处的至善状态。——28、57。},而不在于对自然的认识本身。

这里他的态度也是同德谟克利特完全对立的,这当然就不用再加以证明了。

因此,我们看到,这两个人在每一步骤上都是互相对立的。一个是怀疑主义者,另一个是独断主义者;一个把感性世界看作主观假象,另一个把感性世界看作客观现象。把感性世界看作主观假象的人注重经验的自然科学和实证的知识,他表现了进行实验、到处寻求知。识和外出远游进行观察的不安心情。另一个把现象世界看作实在东西的人,则轻视经验,在他身上体现了在自身中感到满足的思维的宁静和从内在原则中汲取自己知识的独立性。但是还有更深的矛盾。把感性自然看作主观假象的怀疑主义者和经验主义者,从必然性的观点来考察自然,并力求解释和理解事物的实在的存在。相反,把现象看作实在东西的哲学家和独断主义者到处只看见偶然,而他的解释方法无宁说是倾向于否定自然的一切客观实在性。在这些对立中似乎存在着某种颠倒的情况。

但是很难设想的是,这两个处处彼此对立的人会主张同一种学说。而他们毕竟看起来是互相紧密联系着的。

说明他们两人之间的一般关系,是下一章的课题。\footnote{按博士论文的目录,下面应该是第1部分第4章、第5章,但这两章的手稿没有找到。——29。}

\subsubsection{第二部分~论德谟克利特的物理学和伊壁鸠鲁的物理学的具体差别}

 伊壁鸠鲁认为原子在虚空中有三种运动。一种运动是直线式的下落;另一种运动起因于原子偏离直线;第三种运动是由于许多原子的互相排斥而引起的。承认第一种和第三种运动是德谟克利特和伊壁鸠鲁共同的;可是,原子脱离直线而偏斜却把伊壁鸠鲁同德谟克利特区别开来了。

对于这种偏斜运动,很多人都加以嘲笑。西塞罗一接触到这个论题,尤其有说不完的意见。例如,他曾说过这样一段话:“伊壁鸠鲁断言,原子由于自己的重量而作直线式的下落;照他的意见,这是物体的自然运动。后来,他又忽然想到,如果一切原子都从上往下坠落,那么一个原子就始终不会和另一个原子相碰。于是他就求助于谎言。他说,原子有一点点偏斜,但这是完全不可能的。据说由此就产生了原子之间的复合、结合和凝聚,结果就形成了世界、世界的一切部分和世界所包含的一切东西。且不说这一切都是幼稚的虚构,伊壁鸠鲁甚至没有达到他所要达到的目的。”在西塞罗《论神之本性》一书的第1卷中,我们看到他的另一种说法:“由于伊壁鸠鲁懂得,如果原子由于它们本身的重量而下落,那么我们对什么都无能为力,因为原子的运动是被规定了的、是必然的,于是,他臆造出了一个逃避必然性的办法,这种办法是德谟克利特所没有想到的。伊壁鸠鲁说,虽然原子由于它们的重量和重力从上往下坠落,但还是有一点点偏斜。作出这种论断比不能为自己所主张的东西进行辩护还不光彩。”

皮埃尔·培尔也同样地判断说:

\begin{fangsong}
    “在他〈即伊壁鸠鲁〉之前,人们只承认原子有由重力和排斥所引起的运动。伊壁鸠鲁设想,原子甚至在虚空中便稍微有点偏离直线,他说,因此便有了自由……必须附带指出,这并不是使他臆造出这个偏斜运动的唯一动机;偏斜运动还被他用来解释原子的碰撞,因为他当然看到,如果假定一切原子都以同一速度从上而下作直线运动,那就永远无法解释原子碰撞的可能性,这样一来,世界就不可能产生,所以,原子必然偏离直线。”
\end{fangsong}

这些论断究竟确实到什么程度,我暂且放下不提。但是,任何人一眼就可以看出,现代的一位伊壁鸠鲁批评者绍巴赫却错误地理解了西塞罗,因为他说:

\begin{fangsong}
    “一切原子由于重力,即根据物理的原因,平行地往下落,但是由于互相排斥而获得了另一种运动,按西塞罗的说法(《论神之本性》第1卷第25页),这就是由偶然原因,而且是向来就起作用的偶然原因产生的一种倾斜的运动。”
\end{fangsong}

 第一,在前面引证的那一段话里,西塞罗并未把排斥看作是倾斜方向的根据,相反,却认为倾斜方向是排斥的根据。第二,他并没有说到偶然原因,相反,他指责伊壁鸠鲁没有提到任何原因;可见,同时把排斥和偶然原因都看作是倾斜方向的根据,这本身就是自相矛盾的。所以,他说的至多只是排斥的偶然原因,而不是倾斜方向的偶然原因。

此外,在西塞罗和培尔的论断中,有一个极其显著的特点必须立即指出。这就是,他们给伊壁鸠鲁加上一些彼此互相排斥的动机:似乎伊壁鸠鲁承认原子的偏斜,有时是为了说明排斥,有时是为了说明自由。但是,如果原子没有偏斜就不会互相碰撞,那么用偏斜来论证自由就是多余的,因为正如我们在卢克莱修那里所看到的那样,只有在原子的互相碰撞是决定论的和强制的时候,才开始有自由这个对立面。如果原子没有偏斜就互相碰撞,那么用偏斜来论证排斥就是多余的。我认为这种矛盾之所以产生,是由于像西塞罗和培尔那样,把原子偏离直线的原因理解得太表面化和太无内在联系了。一般说来,在所有古代人中,卢克莱修是唯一理解了伊壁鸠鲁的物理学的人,在他那里我们可以看到一种较为深刻的阐述。

现在我们来考察一下偏斜本身。

正如点在线中被扬弃一样,每一个下落的物体也在它所划出的直线中被扬弃。这与它所特有的质完全没有关系。一个苹果落下时所划出的垂直线和一块铁落下时所划出的一样。因此,每一个物体,就它处在下落运动中来看,不外是一个运动着的点,并且是一个没有独立性的点,一个在某种定在中——即在它自己所划出的直线中——丧失了个别性的点。所以,亚里士多德对毕达哥拉斯派正确地指出:“你们说,线的运动构成面,点的运动构成线,那么单子的运动也会构成线了。”因此,从这种看法出发得出的结论是,无论就单子或原子来说,因为它们处在不断的运动中,所以,它们两者都不存在,而是消失在直线中;因为只要我们把原子仅仅看成是沿直线下落的东西,那么原子的坚实性就还根本没有出现。首先,如果把虚空想象为空间的虚空,那么,原子就是抽象空间的直接否定,因而也就是一个空间的点。那个与空间的外在性相对立、维持自己于自身之中的坚实性即强度,只有通过这样一种原则才能达到,这种原则是否定空间的整个范围的,而这种原则在现实自然界中就是时间。此外,如果连这一点也不赞同的话,那么,既然原子的运动构成一条直线,原子就纯粹是由空间来规定的了,它就会被赋予一个相对的定在,而它的存在就是纯粹物质性的存在。但是我们已经看到,原子概念中所包含的一个环节便是纯粹的形式,即对一切相对性的否定,对与另一定在的任何关系的否定。同时我们曾指出,伊壁鸠鲁把两个环节客观化了,它们虽然是互相矛盾的,但是两者都包含在原子概念中。

在这种情况下,伊壁鸠鲁如何能实现原子的纯粹形式规定,即如何能实现把每一个被另一个定在所规定的定在都加以否定的纯粹个别性概念呢?

由于伊壁鸠鲁是在直接存在的范围内进行活动,所以一切规定都是直接的。因此,对立的规定就被当作直接现实性而互相对立起来。

但是,同原子相对立的相对的存在,即原子应该给予否定的定在,就是直线。这一运动的直接否定是另一种运动,因此,即使从空间的角度来看,也是脱离直线的偏斜。

原子是纯粹独立的物体,或者不如说是被设想为像天体那样的有绝对独立性的物体。所以,它们也像天体一样,不是按直线而是按斜线运动。下落运动是非独立性的运动。

因此,伊壁鸠鲁以原子的直线运动表述了原子的物质性,又以脱离直线的偏斜实现了原子的形式规定,而这些对立的规定又被看成是直接对立的运动。

所以,卢克莱修正确地断言,偏斜打破了“命运的束缚”,并且正如他立即把这个思想运用于意识那样,关于原子也可以这样说,偏斜正是它胸中能进行斗争和对抗的某种东西。

但是,西塞罗指责伊壁鸠鲁说:“他甚至没有达到他编造这一理论所要达到的目的;因为如果一切原子都作偏斜运动,那么原子就永远不会结合;或者一些原子作偏斜运动,而另一些原子则作直线运动。这就等于我们必须事先给原子指出一定的位置,即哪些原子作直线运动,哪些原子作偏斜运动。”

这种指责是有道理的,因为原子概念中所包含的两个环节被看成是直接不同的运动,因而也就必须属于不同的个体,——这是不合逻辑的说法,但它也合乎逻辑,因为原子的范围是直接性。

伊壁鸠鲁很清楚地感觉到这里面所包含的矛盾。因此,他竭力把偏斜尽可能地说成是非感性的。偏斜是“既不在确定的地点,也不在确定的时间”发生的,它发生在小得不能再小的空间里。

其次,西塞罗,据普卢塔克说,还有几个古代人,责难伊壁鸠鲁,说按照他的学说,发生原子的偏斜是没有原因的;西塞罗并且说,对于一个物理学家来说,再也没有比这更不光彩的事情了。但是,首先,西塞罗所要求的物理的原因会把原子的偏斜拖回到决定论的范围里去,而偏斜正是应该超出这种决定论的。其次,在原子中未出现偏斜的规定之前,原子根本还没有完成。追问这种规定的原因,也就是追问使原子成为本原的原因,——这一问题,对于那认为原子是一切事物的原因,而它本身没有原因的人来说,显然是毫无意义的。

最后,如果说培尔依据奥古斯丁的权威(不过这个权威同亚里士多德和其他古代人相比,是无足轻重的,据奥古斯丁说,德谟克利特曾赋予原子以一个精神的原则)责备伊壁鸠鲁,说他想出了一个偏斜来代替这个精神的原则,那么可以反驳他说:原子的灵魂只是一句空话,而偏斜却表述了原子的真实的灵魂即抽象个别性的概念。

我们在考察原子脱离直线而偏斜的结论之前,还必须着重指出一个极其重要、至今完全被忽视的环节。

这就是,原子脱离直线而偏斜不是特殊的、偶然出现在伊壁鸠鲁物理学中的规定。相反,偏斜所表现的规律贯穿于整个伊壁鸠鲁哲学,因此,不言而喻,这一规律出现时的规定性,取决于它被应用的范围。

抽象的个别性只有从那个与它相对立的定在中抽象出来,才能实现它的概念——它的形式规定、纯粹的自为存在、不依赖于直接定在的独立性、一切相对性的扬弃。须知为了真正克服这种定在,抽象的个别性就应该把它观念化,而这只有普遍性才有可能做到。

因此,正像原子由于脱离直线,偏离直线,从而从自己的相对存在中,即从直线中解放出来那样,整个伊壁鸠鲁哲学在抽象的个别性概念,即独立性和对同他物的一切关系的否定,应该在它的存在中予以表述的地方,到处都脱离了限制性的定在。

因此,行为的目的就是脱离、离开痛苦和困惑,即获得心灵的宁静。所以,善就是逃避恶,而快乐就是脱离痛苦。最后,在抽象的个别性以其最高的自由和独立性,以其总体性表现出来的地方,那里被摆脱了的定在,就合乎逻辑地是全部的定在,因此众神也避开世界,对世界漠不关心,并且居住在世界之外。

人们曾经嘲笑伊壁鸠鲁的这些神,说它们和人相似,居住在现实世界的空隙中,它们没有躯体,但有近似躯体的东西,没有血,但有近似血的东西;它们处于幸福的宁静之中,不听任何祈求,不关心我们,不关心世界,人们崇敬它们是由于它们的美丽,它们的威严和完美的本性,并非为了谋取利益。

不过,这些神并不是伊壁鸠鲁的虚构。它们曾经存在过。这是希腊艺术塑造的众神。西塞罗,作为一个罗马人,有理由嘲笑它们,但是,当普卢塔克说:这种关于神的学说能消除恐惧和迷信,但是并不给人以愉快和神的恩惠,而是使我们和神处于这样一种关系中,就像我们和希尔卡尼亚海的鱼所处的关系一样,从这种鱼那里我们既不期望受到损害,也不期望得到好处,——当他说这番话时,作为一个希腊人,他已完全忘记了希腊人的观点。理论上的宁静正是希腊众神性格上的主要因素。亚里士多德也说:“最好的东西不需要行动,因为它本身就是目的。”

现在我们来考察一下从原子的偏斜中直接产生出来的结论。这种结论表明,原子否定一切这样的运动和关系,在这些运动和关系中原子作为一个特殊的定在为另一定在所规定。这个意思可以这样来表达:原子脱离并且远离了与它相对立的定在。但是,这种偏斜中所包含的东西——即原子对同他物的一切关系的否定——必须予以实现,必须以肯定的形式表现出来。这一点只有在下述情况下才有可能发生,即与原子发生关系的定在不是什么别的东西,而是它本身,因而也同样是一个原子,并且由于原子本身是直接地被规定的,所以就是众多的原子。于是,众多原子的排斥,就是卢克莱修称之为偏斜的那个“原子规律”\footnote{“原子规律”这一术语在卢克莱修的《物性论》中并未出现,显然是马克思自己提出来的。他在写《关于伊壁鸠鲁哲学的笔记》时,为了给伊壁鸠鲁的“原子偏离直线”下定义,曾研究卢克莱修的观点,并使用了这个术语。——36。}的必然实现。但是,由于这里每一个规定都被设定为特殊的定在,所以,除了前面两种运动以外,又增加了作为第三种运动的排斥。卢克莱修说得对,如果原子不是经常发生偏斜,就不会有原子的冲击,原子的碰撞,因而世界永远也不会创造出来。因为原子本身就是它们的唯一客体,它们只能自己和自己发生关系;或者如果从空间的角度来表述,它们只能自己和自己相撞,因为当它们和他物发生关系时,它们在这种关系中的每一个相对存在都被否定了;而这种相对的存在,正如我们所看到的那样,就是它们的原始运动,即沿直线下落的运动。所以,它们只是由于偏离直线才相撞。这与单纯的物质分裂毫不相干。

而事实上,直接存在的个别性,只有当它同他物发生关系,而这个他物就是它本身时,才按照它的概念得到实现,即使这个他物是以直接存在的形式同它相对立的。所以一个人,只有当他与之发生关系的他物不是一个不同于他的存在,相反,这个他物本身即使还不是精神,也是一个个别的人时,这个人才不再是自然的产物。但是,要使作为人的人成为他自己的唯一现实的客体,他就必须在他自身中打破他的相对的定在,即欲望的力量和纯粹自然的力量。排斥是自我意识的最初形式;因此,它是同那种把自己看作是直接存在的东西、抽象个别的东西的自我意识相适应的。

所以,在排斥中,原子概念实现了,按这个概念来看,原子是抽象的形式,但是其对立面同样也实现了,按其对立面来看,原子就是抽象的物质;因为那原子与之发生关系的东西虽然是原子,但是一些别的原子。但是,如果我同我自己发生关系,就像同直接的他物发生关系一样,那么我的这种关系就是物质的关系。这是可能设想的最极端的外在性。因此,在原予的排斥中,表现在直线下落中的原子的物质性和表现在偏斜中的原子的形式规定,都综合地结合起来了。

同伊壁鸠鲁相反,德谟克利特把那对于伊壁鸠鲁来说是原子概念的实现的东西,变成一种强制的运动,一种盲目必然性的行为。在上面我们已经看到,他把由原子的互相排斥和碰撞所产生的旋涡看作是必然性的实体。可见,他在排斥中只注意到物质方面,即分裂、变化,而没有注意到观念方面,按观念方面来说,在排斥中一切同他物的关系都被否定了,而运动被设定为自我规定。关于这一点,我们可以从下面的事实看得很清楚:他通过虚空的空间完全感性地把同一个物体想象成分裂为许多物体的东西,就像金子被碎成许多小块一样。这样一来,他几乎没有把一理解为原子概念。

亚里士多德正确地反驳他说:“因此,应该对断言原初物体永远在虚空中和无限中运动的留基伯和德谟克利特说,这是哪一种运动,什么样的运动适合这些物体的本性。因为如果每一个元素都是被另一个元素强行推动的,那么,每一个元素除了强制的运动之外必然还有一种自然的运动;而这种最初的运动应该不是强制的运动,而是自然的运动。否则就会发生无止境的递进。”

因此,伊壁鸠鲁的原子偏斜说就改变了原子王国的整个内部结构,因为通过偏斜,形式规定显现出来了,原子概念中所包含的矛盾也实现了。所以,伊壁鸠鲁最先理解了排斥的本质,虽然是在感性形式中,而德谟克利特则只认识到它的物质存在。

因此,我们还发现伊壁鸠鲁应用了排斥的一些更具体的形式。在政治领域里,那就是契约,在社会领域里,那就是友谊,友谊被称赞为最崇高的东西。

(后面的章节略)
\subsubsection{新序言(片段)}
我献给公众的这篇论文,是一篇旧作,它当初本应包括在一篇综述伊壁鸠鲁主义、斯多亚主义和怀疑主义哲学的著作里\footnote{接着马克思删掉了下面这句话:“但是,由于从事目前更具有直接意义的政治和哲学方面的著作,暂时不允许我完成对那些哲学的综述,由于我不知道何时才有机会重新回到这一题目上来,我只好满足于……”——编者注},鉴于我正在从事性质完全不同的政治和哲学方面的研究,目前我无法完成这一著作。\footnote{接着马克思删掉了下面这句话:“伊壁鸠鲁主义、斯多亚主义、怀疑主义哲学,即自我意识哲学,既被以前的哲学家当作非思辨哲学加以屏弃,也被那些同样在编写哲学史的有学识的学究当作……加以屏弃。”——编者注}

只是现在,伊壁鸠鲁派、斯多亚派和怀疑派的体系为人们所理解的时代才算到来了。他们是自我意识的哲学家。这篇论文至少将表明,迄今为止这项任务解决得多么不够。

\newpage
\subsection{2.笔记}
需要明确的一点是,当时马克思处在青年黑格尔派自我意识哲学的阵营之中。在当时的马克思看来,只有肯定人的自我意识的绝对性才会产生“人的解放”的哲学。\footnote{马克思为什么要探讨关于“人的解放”呢?这是因为在当时的普鲁士社会,专制政治受到传统基督教的深刻影响,人民在此状况下普遍受到宗教威权的束缚,因此迫切寻得一种手段使得人民突破这种束缚,以实现自身的解放。}而这种对于自我意识的绝对性的肯定在马克思的博士论文中表现为对伊壁鸠鲁哲学思想中的“原子偏斜”、“原子排斥”与“自由”的再解释。

马克思指出,伊壁鸠鲁的原子论与德谟克利特的原子论是截然不同的两种学说,并非专家们所认为的伊壁鸠鲁拙劣地“抄袭”了德谟克利特。马克思认为,在伊壁鸠鲁的原子论中,原子的偏斜并不是为了说明碰撞,而是为了说明自由,换句话说,原子的偏斜是与原子的碰撞在物理学层面\textbf{完全无关}的规定性\footnote{有的专家会认为伊壁鸠鲁所构建的原子偏斜运动是既为了说明碰撞又为了说明自由,而马克思则认为伊壁鸠鲁所构建的原子偏斜运动与原子的碰撞完全无关。}。什么是在物理学层面\textbf{完全无关}的规定性呢?就是说原子的偏斜完全不影响原子的碰撞。大家或许有疑问,原子的偏斜如果不影响原子的碰撞的话,那么原子是如何能够在沿直线下落的过程中发生碰撞呢?学过经典物理学的朋友们一定明白,平行下落的物体在下落过程中是永远不会发生碰撞的,若是碰撞,一定是因为某一物体的偏斜导致该物体与其他物体处于非平行的状态而发生的。因此,若是要理解马克思所认为的在伊壁鸠鲁原子论中的原子的三种运动形式,就一定要跳出纯粹物理学的框架,将其理解为通过直观的运动形式所表示的\textbf{规定性}。也即是说,在马克思看来,伊壁鸠鲁原子论中的原子的三种运动形式并不是在经验性的意义上说明了存在有一些\textbf{原子微粒},这些经验性的\textbf{原子微粒}构成了世界的全部,而是说明了作为世界本源的原子,是某种\textbf{非经验性的存在},换句话说,原子是作为世界本源的那个\textbf{非经验性存在}的\textbf{直观表现形式}。在这种意义上,全部问题便迎刃解决了:既然伊壁鸠鲁原子论中的原子的三种运动形式不是在物理学的意义上说明\textbf{经验性物体}的运动,而是代表了三种不同的规定性,那末伊壁鸠鲁的原子学说便是与德谟克利特的原子学说完全不同的两种理论,因为“原子的偏斜”并不是对德谟克利特原子学说的进一步完善(即对原子碰撞的进一步说明),而是与德谟克利特原子学说完全无关的规定性。因此,在这种意义上,伊壁鸠鲁原子论中的“原子沿直线下落”、“原子的偏斜”、“原子的碰撞”便可以分别理解成“必然性”、“必然性的否定”、“矛盾的实现”。

既然在伊壁鸠鲁的原子论中原子的运动形式所代表的是某种规定性的直观表现,因此对于“原子沿直线下落”而言可将其理解为通过某种空间的规定性(即下落)所揭示的\textbf{定在},这种\textbf{定在}事实上在马克思看来表明了某种必然性,更进一步说,对于作为世界本源的原子而言,这种定在便反映了\textbf{纯粹的必然性}\footnote{这里需要额外解释一下以方便读者的理解,\textbf{定在}反映了某种限制性,因此对于个别物而言,其\textbf{定在}便是体现了对该物的限制——即某种\textbf{相对的必然性},而\textbf{“原子”}这一范畴在伊壁鸠鲁的原子论中代表着\textbf{非经验性的世界本源},因此\textbf{“原子”的定在}便体现了对全部世界的限制——即\textbf{纯粹的必然性。}}。

顺着这个思路,接下来我们再来考察一下伊壁鸠鲁原子论中的“原子的偏斜”与“原子的碰撞”。如前文所述,“原子沿直线下落”所代表的是\textbf{“纯粹的必然性”},那末,“原子的偏斜”便是对这种\textbf{“纯粹必然性”}的否定,即\textbf{“纯粹的偶然性”、“抽象的自由”、“纯粹的自由”}。分析至此一切都明了了,“原子沿直线下落”与“原子的偏斜”这两种运动形式在伊壁鸠鲁的原子论中代表着原子自身所具有的\textbf{“纯粹的必然性”}与\textbf{“纯粹的自由”}这两种规定性。也即是说,作为世界本源的那个\textbf{“超经验性的原子”}本身就内在的包含着矛盾与冲突,它只能与自己发生作用,它便是它唯一的客体,这一规定性在直观性的形式上便表示为“原子的碰撞”。至此,原子的概念才算真正实现了,“纯粹的必然性”代表着原子的“肉体”,而“纯粹的自由”则代表着原子的“灵魂”,只有当“灵魂”与“肉体”不一致的时候,即“灵魂”对抗“肉体”的时候,“自我意识”才会产生,这便是马克思所说的\begin{fangsong}“排斥是自我意识的最初形式”    \end{fangsong}。

“灵魂”和“肉体”之间的关系用更哲学的话来说就是“形式”与“质料”或“抽象个别性”与“抽象普遍性”的关系。可以肯定地说,马克思在写这篇博士论文的时候是站在一种激进的自我意识哲学立场之上的,他想通过对伊壁鸠鲁原子论的再解释,在哲学层面给予“抽象个别性”以至高的地位,以便对抗那束缚着人民的宗教威权(某种虚幻的普遍性)。但是笔者感到这里似乎有一种“知其不可而为之”的悲怆,因为就现实层面而言,“抽象个别性”所代表的那种“个别性的自我意识”很难触动笼罩在整个社会之上的普遍性阴影。当时的马克思或许没有意识到,这种“抽象个别性”与“抽象普遍性”之间的斗争是局限于观念领域的斗争,而观念本身又是社会物质领域的映射,因此这种不触及实际物质领域的观念斗争是缺乏现实性的。

因此,为了使得斗争真正具有成效,即这种斗争获得真正的现实性,自我意识哲学自身必须加以扬弃,必须要从观念领域的批判转向物质领域的批判。而关于这一方面的工作内容,体现在马克思之后的著作之中。
\newpage
\section{哲学的贫困}
\subsection{1.原文节选}
\subsubsection{第二章~政治经济学的形而上学}
\begin{center}
    \textbf{第一节~方法}
\end{center}

现在我们已在德国中心!我们一方面谈论政治经济学,同时又要谈形而上学。这一次,我们也只是跟着蒲鲁东先生的“矛盾”走。刚才他迫使我们讲英国话,使我们在相当程度上变成了英国人。现在场面变了。蒲鲁东先生使我们回到我们亲爱的祖国,使我们不由得又变成了德国人。

\textbf{如果说有一个英国人把人变成帽子,那末,有一个德国人就把帽子变成了观念。这个英国人就是李嘉图,一位银行巨子,杰出的经济学家;这个德国人就是黑格尔,柏林大学的一位专任哲学教授。}法国最末一个专制君主和法兰西王朝没落的代表者路易十五有一个御医,这个人同时又是法国的第一个经济学家。这位御医,这位经济学家是预言法国资产阶级必然要取得胜利的先知。魁奈医生使政治经济学成为一门科学;他在自己的名著“经济表”中概括地叙述了这门科学。除了已经有的对该表的一千零一个注解以外,我们还找到医生本人做的一个注解。这就是附有“七个重要说明”的“经济表的分析”。

蒲鲁东先生是魁奈医生第二,他是政治经济学的形而上学方面的魁奈。

但是在黑格尔看来,形而上学同整个哲学一样,可以概括在方法里面。所以我们必须设法弄清楚蒲鲁东先生那套至少同“经济表”一样含糊不清的方法。因此,我们做了七个比较重要的说明。如果蒲鲁东博士不满意我们的说明,那没关系,他可以扮演修道院长勃多的角色,亲自写一篇“经济学—形而上学方法解说”。

\begin{center}
\textbf{第一个说明}    
\end{center}

\begin{fangsong}
    “这里我们论述的不是适应时间次序的历史,而是适应观念顺序的历史。各经济阶段或范畴有时候是同时出现,有时候又是颠倒的……不过,经济理论有它自己的逻辑顺序和理性中的一定系列,经济理论的这种次序,如所预期的那样,已被我们发现了。”(蒲鲁东,“贫困的哲学”第一卷第145和146页)
\end{fangsong}

蒲鲁东先生把这些冒牌的黑格尔词句扔向法国人,毫无问题是想吓唬他们一下。这样一来,我们就要同两个人打交道:首先是蒲鲁东先生,其次是黑格尔。蒲鲁东先生和其它经济学家有什么不同呢?黑格尔在蒲鲁东先生的政治经济学中又起什么作用呢?

经济学家们都把分工、信用、货币等资产阶级生产关系说成是固定不变的、永恒的范畴。蒲鲁东先生有了这些完全形成的范畴,他想给我们说明所有这些范畴、原理、规律、观念、思想的形成情况和来历。

经济学家们向我们解释了生产怎样在上述关系下进行,但是没有说明这些关系本身是怎样产生的,也就是说,没有说明产生这些关系的历史运动。由于蒲鲁东先生把这些关系看成原理、范畴和抽象的思想,所以他只要把这些思想(它们在每一篇政治经济学论文末尾已经按字母表排好)编一下次序就行了。经济学家的材料是人的生动活泼的生活;蒲鲁东先生的材料则是经济学家的教条。但是,既然我们忽略了生产关系(范畴只是它在理论上的表现)的历史发展,既然我们只希望在这些范畴中看到观念、不依赖实际关系而自生的思想,那末,我们就只得到纯理性的运动中去找寻这些思想的来历了。纯粹的、永恒的、无人身的理性怎样产生这些思想呢?它是怎样造成这些思想的呢?

假如在黑格尔主义方面我们具有蒲鲁东先生那种大无畏精神,我们就会说,理性在自身中把自己和自身区分开来。这是什么意思呢?因为无人身的理性在自身之外既没有可以安置自己的地盘,又没有可与自己对置的客体,也没有自己可与之结合的主体,所以它只得把自己颠来倒去:安置自己,把自己跟自己对置起来,自己跟自己结合——安置、对置、结合。用希腊语来说,这就是:正题、反题、合题。对于不懂黑格尔语言的读者,我们将告诉他们一个神圣的公式:肯定、否定、否定的否定。这就是措词的含意。固然这不是天书(请蒲鲁东先生不要见怪),然而却是脱离了个体的纯理性的语言。这里看到的不是一个用普通方式说话和思考的普通个体,而是没有个体的纯粹普通方式。

在抽象的最后阶段(因为这里谈的是抽象,而不是分析),一切事物都成为逻辑范畴,这用得着奇怪吗?如果我们抽掉构成某座房屋特性的一切,抽掉建筑这座房屋所用的材料和构成这座房屋特点的形式,结果只剩下一个一般的物体;如果把这一物体的界限也抽去,结果就只有空间了;如果再把这个空间的向度抽去,最后我们就只有同纯粹的数量,即数量的逻辑范畴打交道了,这用得着奇怪吗?用这种方法把每一个物体的一切所谓偶性(有生命的或无生命的,人类的或物类的)抽去,我们就有理由说,在抽象的最后阶段,作为实体的将是一些逻辑范畴。所以形而上学者认为进行抽象就是进行分析,越远离物体就是日益接近物体和深入事物。这些形而上学者说,我们世界上的事物只不过是逻辑范畴这种底布上的花彩;在他们自己看来,这种说法是正确的。哲学家和基督教徒不同之处正是在于:基督徒只知道逻各斯的化身,不管什么逻辑不逻辑;而哲学家那里则有无数这种化身。既然如此,那末一切存在物,一切生活在地上和水中的东西经过抽象都可以归结为逻辑范畴,因而整个现实世界都淹没在抽象世界之中,即淹没在逻辑范畴的世界之中,这又有什么奇怪呢?

一切存在物,一切生活在地上和水中的东西,只是由于某种运动才得以存在、生活。例如,历史的运动创造了社会关系,工业的运动给我们提供了工业产品,等等。

正如我们通过抽象把一切事物变成逻辑范畴一样,我们只要抽去各种各样的运动的一切特征,就可得到抽象形态的运动,纯粹形式上的运动,运动的纯粹逻辑公式。既然我们把逻辑范畴看做一切事物的实体,那末也就不难设想,我们在运动的逻辑公式中已找到了一种绝对方法,它不仅说明每一个事物,而且本身就包含每个事物的运动。

关于这种绝对方法,黑格尔这样说过:

\begin{fangsong}
    “方法是任何对象所不能抗拒的一种绝对的、唯一的、最高的、无限的力量;这是理性企图在每一个事物中发现和认识自己的意向。”(“逻辑学”第三卷[60])
\end{fangsong}

既然把任何一种事物都归结为逻辑范畴,任何一个运动、任何一种生产行为都归结为方法,那末,由此自然得出一个结论,产品和生产、对象和运动的任何总和都可以归结为应用的形而上学。黑格尔为宗教、法等做过的事情,蒲鲁东先生也想在政治经济学上如法泡制。

那末,这种绝对方法到底是什么呢?是运动的抽象。运动的抽象是什么呢?是抽象形态的运动。抽象形态的运动是什么呢?是运动的纯粹逻辑公式或者纯理性的运动。纯理性的运动又是怎么回事呢?就是它安置自己,把自己跟自己对置,自相结合,就是它把自己规定为正题、反题、合题,或者就是它自我肯定、自我否定和否定自我否定。

理性怎样进行自我肯定,或者它怎样把自己形成这种或那种特定的范畴呢?这已经是理性自己及其辩护人的事情了。

但是理性一旦把自己作为正题安置下来,这个正题、这个思想就会自相对置,分为两个互相矛盾的思想,即肯定和否定,“是”和“否”。这两个包含在反题中的对抗因素的斗争,形成辩证运动。“是”转化为“否”,“否”转化为“是”。“是”同时成为“是”和“否”,“否”同时成为“否”和“是”。对立面就是通过这种方式互相均衡,互相中和,互相抵消。这两个彼此矛盾的思想的融合,就形成一个新的思想,即它们的合题。这个新的思想又分为两个彼此矛盾的思想,而这两个思想又融合成新的合题。这种增殖过程就构成思想群。同简单的范畴一样,思想群也遵循这个辩证运动,它也有另一个与自己矛盾的群为自己的反题。从这两个思想群中产生出新的思想群,即它们的合题。

正如从简单范畴的辩证运动中产生群一样,从群的辩证运动中产生系列,从系列的辩证运动中又产生整个体系。

把这个方法运用到政治经济学的范畴上面,就会得出政治经济学的逻辑学和形而上学,换句话说,就会把人所共知的经济范畴翻译成人们不大知道的语言,这种语言使人觉得这些范畴似乎是刚从充满纯粹理性的头脑中产生的,好象这些范畴单凭辩证运动才互相产生、互相联系、互相交织。请读者不要害怕这个形而上学以及它那一大堆范畴、群、系列和体系,尽管蒲鲁东先生费了九牛二虎之力想爬上矛盾体系的顶峰,可是他从来没有超越过头两级即简单的正题和反题,而且这两级他仅仅爬上过两次,其中有一次还跌了下来。

在这以前我们谈的只是黑格尔的辩证法。下面我们要看到蒲鲁东先生怎样把它降低到极可怜的程度。黑格尔认为,世界上过去发生的一切和现在还在发生的一切,就是他自己的思维中发生的一切。因此,历史的哲学仅仅是哲学的历史,即他自己的哲学的历史。没有“适应时间次序的历史”,只有“观念在理性中的顺序”。他以为他是在通过思想的运动建设世界;其实,他只是根据自己的绝对方法把所有人们头脑中的思想加以系统的改组和排列而已。

\begin{center}
    \textbf{第二个说明}
\end{center}

经济范畴只不过是生产方面社会关系的理论表现,即其抽象。真正的哲学家蒲鲁东先生对事物的理解是颠倒的,他认为现实关系只是睡在“人类的无人身的理性”怀抱里(正如这位哲学家蒲鲁东先生告诉我们的)的一些原理和范畴的化身。

经济学家蒲鲁东先生非常明白,人们是在一定的生产关系范围内制造呢绒、麻布和丝织品的。但是他不明白,这些一定的社会关系同麻布、亚麻等一样,也是人们生产出来的。社会关系和生产力密切相联。随着新生产力的获得,人们改变自己的生产方式,随着生产方式即保证自己生活的方式的改变,人们也就会改变自己的一切社会关系。手工磨产生的是封建主为首的社会,蒸汽磨产生的是工业资本家为首的社会。

人们按照自己的物质生产的发展建立相应的社会关系,正是这些人又按照自己的社会关系创造了相应的原理、观念和范畴。

所以,这些观念、范畴也同它们所表现的关系一样,不是永恒的。它们是历史的暂时的产物。

生产力的增长、社会关系的破坏、思想的产生都是不断变动的,只有运动的抽象即“不死的死”才是停滞不动的。

\begin{center}
    \textbf{第三个说明}
\end{center}

每一个社会中的生产关系都形成一个统一的整体。蒲鲁东先生把种种经济关系看做同等数量的社会阶段,认为这些阶段一个产生一个,一个来自一个,正如反题来自正题一样;认为这些阶段在自己的逻辑顺序中实现着人类的无人身的理性。

这个方法的唯一短处就是:蒲鲁东先生在考察其中任何一个阶段时,都不能不靠其它一些社会关系来说明,可是当时这些社会关系尚未被他用辩证运动产生出来。当蒲鲁东先生后来借助纯粹理性使其它阶段产生出来时,却又把它们当成初生的婴儿,忘记它们和第一个阶段是同样年老了。

因此,要构成被他看做一切经济发展基础的价值,就非有分工、竞争等等不可。然而当时这些关系在一定的系列中、在蒲鲁东先生的理性中以及逻辑顺序中根本还不存在。

谁用政治经济学的范畴构筑某种思想体系的大厦,谁就是把社会体系的各个环节割裂开来,就是把社会的各个环节变成同等数量的互相连接的单个社会。其实,单凭运动、顺序和时间的逻辑公式怎能向我们说明一切关系同时存在而又互相依存的社会机体呢?

\begin{center}
    \textbf{第四个说明}
\end{center}

现在我们看一看蒲鲁东先生把黑格尔的辩证法应用到政治经济学上去的时候,把它变成了什么样子。

蒲鲁东先生认为,任何经济范畴都有好坏两个方面。他看范畴就象小资产者看历史伟人一样:拿破仑是一个大人物;他行了许多善,但是也作了许多恶。

蒲鲁东先生认为,好的方面和坏的方面,益处和害处加在一起就构成每个经济范畴所固有的矛盾。

应当作的是:保存好的方面,消除坏的方面。

奴隶制是同其它任何经济范畴一样的一个经济范畴。因此,它也有两个方面。我们抛开奴隶制的坏的方面不谈,且来看看它的好的方面。自然,这里谈的只是直接奴隶制,即苏里南、巴西和北美南部各州的黑人奴隶制。

同机器、信用等等一样,直接奴隶制是资产阶级工业的基础。没有奴隶制就没有棉花;没有棉花现代工业就不可设想。奴隶制使殖民地具有价值,殖民地产生了世界贸易,世界贸易是大工业的必备条件。可见,奴隶制是一个极重要的经济范畴。

没有奴隶制,北美这个进步最快的国家就会变成宗法式的国家。如果从世界地图上把北美划掉,结果看到的是一片无政府状态,现代贸易和现代文明十分衰落的情景。消灭奴隶制就等于从世界地图上抹掉美洲\footnote{注:这对1847年说来是完全正确的。当时美国的对外贸易主要限于输入移民和工业产品,输出棉花和烟草,即南部奴隶劳动的产物。北部各州主要是为奴隶占有制各州生产谷物和肉类。直至北部开始生产供输出用的谷物和肉类,并且成为工业国,而美洲棉花的垄断又遇到印度、埃及、巴西等国的激烈竞争的时候,奴隶制才有可能废除。而且当时,奴隶制的废除曾引起南部的破产,因为南部还没有以印度和中国隐蔽的苦力奴隶制代替公开的黑人奴隶制。——弗·恩·(恩格斯在1885年德文版上加的注)}。

因为奴隶制是一个经济范畴,所以它总是列入各民族的社会制度中。现代各民族只是在本国内把奴隶制掩饰一下,而在新大陆却赤裸裸地公开推行奴隶制。

蒲鲁东先生将用什么办法挽救奴隶制呢?他提出的任务是:保存这个经济范畴的好的方面,消除其坏的方面。

黑格尔没有需要提出任务。他只有辩证法。蒲鲁东先生从黑格尔的辩证法那里只学得了术语。而蒲鲁东先生自己的辩证运动只不过是机械地划分出好、坏两面而已。

我们暂且把蒲鲁东先生当做一个范畴看待,看一看他的好的方面和坏的方面,他的长处和短处。

如果说,与黑格尔比较,他的长处是提出任务并且保留为人类最大幸福而解决这些任务的权利,那末,他也有一个短处:当他想用辩证法引出一个新范畴时,却毫无所获。两个矛盾方面的共存、斗争以及融合成一个新范畴,就是辩证运动的实质。谁要给自己提出消除坏的方面的任务,就是立即使辩证运动终结。我们看到的已经不是由于矛盾本性而自我安置和自相对置的范畴,而是在范畴的两个方面中间激动、挣扎和冲撞的蒲鲁东先生。

这样,蒲鲁东先生就陷入了用正当方法难以摆脱的困境,于是他用尽全力一跳,便跳到一个新范畴的领域中。这时在他那惊异的目光面前便出现了理性中的一定系列。

他抓住第一个到手的范畴,随心所欲地给它一种特性:把应该清除的范畴的缺陷消除。例如,如果相信蒲鲁东先生的话,捐税可以消除垄断的缺陷,贸易差额可以消除捐税的缺陷,土地所有权可以消除信用的缺陷。

这样,蒲鲁东先生把所有经济范畴逐一取来,把一个范畴用作另一个范畴的消毒剂,用矛盾和矛盾的消毒剂的混合物写成两卷矛盾,并且恰当地称为“经济矛盾的体系”。

\begin{center}
    \textbf{第五个说明}
\end{center}

\begin{fangsong}
    “在绝对理性中,所有这些观念……是同样简单和普遍的……实际上我们只有靠我们的观念搭成的一种脚手架才能达到科学境地。但是,真理本身并不以这些辩证的图形为转移,而且不受我们智能的种种组合的束缚。”(蒲鲁东,第二卷第97页)
\end{fangsong}

这样,一个急转弯(现在我们才知道其中奥妙)就使政治经济学的形而上学变成了幻想!蒲鲁东先生的话从来没有说得这样公正。当然,如果把辩证运动的全部过程归结为简单地对比善和恶,归结为提出任务来消除恶并且把一个范畴用作另一个范畴的消毒剂,那末范畴就失去自己的独立运动;观念就“不再发生作用”;它就没有内在的生命。它既不能把自己安置为范畴,也不能把自己分解为范畴。范畴的顺序成了一种脚手架。辩证法已不是绝对理性的运动了。辩证法没有了,代替它的至多不过是最纯粹的道德而已。

当蒲鲁东先生谈到理性中的一定系列即范畴的逻辑顺序的时候,他肯定地说,他不是想论述适应时间次序的历史,即蒲鲁东先生所认为的范畴在其中出现的历史顺序。他认为那时一切都在理性的纯粹以太中进行。一切都应当通过辩证法从这种以太中产生。现在当实际应用这种辩证法的时候,理性却背叛了他。蒲鲁东先生的辩证法背弃了黑格尔的辩证法,于是蒲鲁东先生只得承认,他用以说明经济范畴的次序和这些经济范畴在其中相互产生的次序是不相适应的。经济的进化不再是理性本身的进化了。

那末,蒲鲁东先生给了我们什么呢?是现实的历史、即蒲鲁东先生所认为的范畴适应着时间次序在其中出现的那种顺序吗?不是。是在观念本身中进行的历史吗?更不是。这就是说,他既没有给我们范畴的世俗历史,也没有给我们范畴的神圣历史!那末,到底他给了我们什么历史呢?是他本身矛盾的历史。让我们来看看这些矛盾怎样行进以及它们怎样拖着蒲鲁东先生吧。

在未研究这一点(这是第六个重要说明的引子)之前,我们应当再作一个比较次要的说明。

我们暂且和蒲鲁东先生一同假定:现实的历史,适应时间次序的历史是观念、范畴和原理在其中出现的那种历史顺序。

每个原理都有其出现的世纪。例如,与权威原理相适应的是11世纪,与个人主义原理相适应的是18世纪,推其因果,我们应当说,不是原理属于世纪,而是世纪属于原理。换句话说,不是历史创造原理,而是原理创造历史。但是,如果为了顾全原理和历史我们再进一步自问一下,为什么该原理出现在11世纪或者18世纪,而不出现在其它某一世纪,我们就必然要仔细研究一下:11世纪的人们是怎样的,18世纪的人们是怎样的,在每个世纪中,人们的需求、生产力、生产方式以及生产中使用的原料是怎样的;最后,由这一切生存条件所产生的人与人之间的关系是怎样的。难道探讨这一切问题不就是研究每个世纪中人们的现实的、世俗的历史,不就是把这些人既当成剧作者又当成剧中人物吗?但是,只要你们把人们当成他们本身历史的剧中人物和剧作者,你们就是迂回曲折地回到真正的出发点,因为你们抛弃了最初作为出发点的永恒的原理。

至于蒲鲁东先生,他一直还在思想家所走的这条迂回曲折的道路上缓慢行进,离开历史的康庄大道还有一大段路程。

\begin{center}
    \textbf{第六个说明}
\end{center}

我们且沿着这条迂回曲折的道路跟蒲鲁东先生走下去。

假定被当做不变规律、永恒原理、理想范畴的经济关系先于人们的生动活跃的生活而存在;再假定这些规律、这些原理、这些范畴自古以来就睡在“人类的无人身的理性”的怀抱里。我们已经看到,在这一切一成不变的、停滞不动的永恒下面没有历史可言,即使有,至多也只是观念中的历史,即反映在纯理性的辩证运动中的历史。蒲鲁东先生谈到辩证运动中的各种观念不能自相“区分”时,把运动的一切影子和影子(它们可以造成某种类似历史的东西)的一切运动一概抹熬。他没有这样做,反而把自己的无能归罪于历史,埋怨一切,甚至连法国话也埋怨起来。

哲学家蒲鲁东先生告诉我们:“我们说什么东西出现或者什么东西产生,这种说法是不确切的,无论是在文明里还是在宇宙中,自古以来一切就存在着、活动着……整个社会经济也是如此。”(蒲鲁东,第二卷第102页)

在蒲鲁东先生的体系中起作用并且使蒲鲁东先生本人也起作用的矛盾的实力竟大到这样程度,以至他本想说明历史,但却不得不否定历史;本想说明社会关系的顺次出现,但却根本否定某种东西可以出现;本想说明生产及其一切阶段,但却否定某种东西可以生产出来。

这样,在蒲鲁东先生看来,再没有什么历史,也没有什么观念的顺序了;可是,他那本自称为“适应观念顺序的历史”的大作却继续存在。怎样才能找到一个公式(因为蒲鲁东先生就是公式的人物)帮助他一跳就越过这一切矛盾呢?

为了做到这一点,他发明了一种新理性,这既不是绝对的、纯粹的和纯真的理性,也不是生活在不同历史时期的活跃的人们的普通的理性;这是一种十分特殊的理性,是作为人的社会的理性,是称为人类的这种主体的理性,这种理性在蒲鲁东先生的笔下有时也被写为“社会天才”、“普遍理性”以及“人类理性”。然而这种名目繁多的理性都是蒲鲁东先生的个人理性,它有一切好的和坏的方面,有消毒剂也有任务。

“人类理性不创造真理”,真理蕴藏在绝对的永恒的理性的深处。它只能发现真理。但是直到现在它所发现的真理是不完备的,不充足的,而且是矛盾的。经济范畴是人类理性、社会天才所发现和揭示出来的真理,所以也是不完备的并含有矛盾的萌芽。在蒲鲁东先生以前,社会天才只看见对抗因素而未发现综合公式,虽然两者同时潜藏在绝对理性里面。既然经济关系只是这些不充足的真理、这些不完备的范畴、这些矛盾的概念在人世间的实现,因此,它们本身就包含着矛盾,并且有好坏两个方面。

社会天才的任务是发现完备的真理、完整无缺的概念、排除二律背反的综合公式。这就再一次说明,为什么蒲鲁东先生想象中的这个社会天才不得不从一个范畴跑到另一个范畴,尽管已经有了一整套范畴,但是直到现在还不能从上帝那里,从绝对理性那里得到一个综合公式:

\begin{fangsong}
    “首先,社会(社会天才)\footnote{括号里的话是马克思的。——译者注}假定一个原始的事实,提出一个假设……一个真正的二律背反,它的对抗性结果在社会经济中展开就象它们作为后果可以在精神上被推论出来一样,所以工业运动在各方面随着观念的演绎分为两道洪流:一道是有益行为的洪流,一道是有害结果的洪流……为了和谐地构成这个两重性的原理和解决这个二律背反,社会就产生第二个二律背反,随后很快地又产生第三个二律背反;社会天才将一直这样行进,直到它用尽自己的全部矛盾(尽管未曾得到证实,但是我料想,人类固有的矛盾是有止境的),一跳而回到它自己原来的各种论点并在唯一的公式中将自己的全部任务加以解决时为止。”(第一卷第133页)
\end{fangsong}

正如以前反题变成消毒剂一样,现在正题将变成假设。但是,蒲鲁东先生这种术语上的交换现在再也不能使我们感到惊奇了。人类理性最不纯洁,因为它只具有不完备的见解,每走一步都要遇到新的待解决的任务。人类理性在绝对理性中发现的以及作为第一个正题的否定的每一个新的正题,对它说来都是一个合题,并且被它相当天真地当做一个任务的解决。这个理性就这样在不断变换的矛盾中乱窜,直至它达到了矛盾的终点,发觉这一切正题和合题不过是相互矛盾的假设时为止。在极度混乱的状态下,“人类理性、社会天才一跳而回到它自己原来的各种论点并在唯一的公式中将自己的全部任务加以解决”。

假设只是为了某种特定的目的而设立的。通过蒲鲁东先生之口讲话的社会天才首先给自己提出的目的,就是消除每个经济范畴的一切坏的东西,使它只保留好的东西。他认为,好的东西,最高的幸福,真正的实际目的就是平等。为什么社会天才只要平等,而不要不平等或友爱、不要天主教或别的什么原理呢?因为“人类之所以实现这么多特殊的假设,正是由于考虑到一个最高的假设”,这个最高的假设就是平等。换句话说,因为平等是蒲鲁东先生的理想。他以为分工、信用、工厂,一句话,一切经济关系都仅仅是为了平等的利益才被发明的,但是结果它们往往对平等不利。由于历史和蒲鲁东先生的臆测步步发生矛盾,所以他得出结论说,有矛盾存在。即使是有矛盾存在,那也只存在于他的固定观念和现实运动之间。

从此以后,肯定平等的就是每个经济关系的好的方面,否定平等和肯定不平等的就是坏的方面。每一个新的范畴都是社会天才为了消除前一个假设所产生的不平等而作的假设。总之,平等是原始的意向、神秘的趋势、天命的目的,社会天才在经济矛盾的圈子里旋转时从来没有忽略过它。因此,天命是一个火车头,用它拖蒲鲁东先生的全部经济行囊前进远比用他那走了气的纯粹理性要好得多。我们这位著者在论捐税一章之后,用了整整一章来写天命。

天命,天命的目的,这是当前用以说明历史进程的一个响亮字眼。其实这个字眼不说明任何问题。它至多不过是一种修辞形式,是冗长地重述事实的若干方式之一。

大家知道,英国工业的发展提高了苏格兰地产的价值。英国工业为羊毛开辟了新的销售市场。要生产大量的羊毛,必须把耕地变成牧场。要这样做就必须集中地产。要集中地产就必须消灭世袭租佃者的小农庄,使成千上万的租佃者离开家园,让放牧几百万只羊的少数牧羊人来居住。这样,由于耕地接连不断地变成牧场,结果苏格兰的地产使羊群赶走了人。如果现在你们说,羊群赶走人就是苏格兰土地私有制度的天命的目的,那末,你们就会得到天命的历史。

当然,平等趋势是我们这个世纪所特有的。但是,说以往各世纪及其完全不同的需求、生产数据等等都是为实现平等而遵照天命行事,这首先就是把我们这个世纪的人和生产数据当做过去世纪的人和生产数据看待,否认世世代代不断改变前代所获得的成果的历史运动。经济学家们很清楚,同是一件东西对甲说来是成品,对乙说来只是从事另一种生产的原料。

如果你们同蒲鲁东先生一道假定:社会天才制造出,或者更确切些说随兴制造出封建主,是为了达到把耕者变为负有义务的和彼此平等的劳动者这一天命的目的,那末,你们就是把目的和人换了一下,这种做法和为了达到恶意的满足(即羊群赶走人)而在苏格兰确立土地私有制的天命比较起来,毫不逊色。

可是,蒲鲁东先生既然对天命表现出那样亲切的关怀,我们就介绍他看一看维尔纽夫-巴尔热蒙的“政治经济学历史”,此人也是追求天命的目的。但他这个目的已经不是平等,而是天主教了。

\begin{center}
    \textbf{第七个即最后一个说明}
\end{center}

经济学家们在论断中采用的方式是非常奇怪的。他们认为只有两种制度:一种是人为的,一种是天然的。封建制度是人为的,资产阶级制度是天然的。在这方面,经济学家很象那些把宗教也分为两类的神学家。一切异教都是人们臆造的,而他们自己的教则是神的启示。经济学家所以说现存的关系(资产阶级生产关系)是天然的,是想以此说明,这些关系正是使生产财富和发展生产力得以按照自然规律进行的那些关系。因此,这些关系是不受时间影响的自然规律。这是应当永远支配社会的永恒规律。于是,以前是有历史的,现在再也没有历史了。以前所以有历史,是由于有过封建制度,由于在这些封建制度中有一种和经济学家称为自然的、因而是永恒的资产阶级社会生产关系完全不同的生产关系。

封建主义也有过自己的无产阶级,即包含着资产阶级的一切萌芽的农奴等级。封建的生产也有两个对抗的因素,人们称为封建主义的好的方面和坏的方面,可是,却没想到结果总是坏的方面占优势。正是坏的方面引起斗争,产生形成历史的运动。假如在封建主义统治时代,经济学家看到骑士的德行、权利和义务之间美妙的协调、城市中的宗法式的生活、乡村中家庭手工业的繁荣、各同业公会、商会和行会中所组织的工业的发展,总而言之,看到封建主义的这一切好的方面而深受感动,抱定目的要消除这幅图画上的一切阴暗面(农奴状况、特权、无政府状态),那末结果会怎样呢?引起斗争的一切因素就会灭绝,资产阶级的发展在萌芽时就会被切断。经济学家就会给自己提出把历史一笔勾销的荒唐任务。

资产阶级得势以后,也就谈不到封建主义的好的方面和坏的方面了。资产阶级把它在封建主义统治下发展起来的生产力掌握起来。一切旧的经济形式、一切与之相适应的市民关系以及作为旧日市民社会的正式表现的政治制度都被粉碎了。

这样,为了正确地判断封建的生产,必须把它当做以对抗为基础的生产方式来考察。必须指出,财富怎样在这种对抗中间形成,生产力怎样和阶级对抗同时发展,这些阶级中一个代表着社会上坏的、否定的方面的阶级怎样不断地成长,直到它求得解放的物质条件最后成熟。这难道不是说,生产方式、生产力在其中发展的那些关系并不是永恒的规律,而是同人们及其生产力发展的一定水平相适应的东西,人们生产力的一切变化必然引起他们的生产关系的变化吗?由于最重要的是不使文明的果实(已经获得的生产力)被剥夺,所以必须粉碎生产力在其中产生的那些传统形式。从此以后,从前的革命阶级将成为保守阶级。

资产阶级开始自己的历史发展时就有一个本身是封建时期无产阶级残存物的无产阶级存在。资产阶级在其历史发展过程中不可避免地要发展它的对抗性质,起初这种性质或多或少是掩饰起来的,只是处于隐蔽状态。随着资产阶级的发展,在它的内部发展着一个新的无产阶级,即现代无产阶级。无产阶级同资产阶级之间展开了斗争,在双方尚未感觉、注意、重视、理解、承认并公开宣告以前,这个斗争最初仅表现为局部的暂时的冲突,表现为一些破坏行为。另一方面,如果说现代资产阶级的全体成员由于组成一个与另一个阶级相对立的阶级而有共同的利益,那末,由于他们互相对立,他们的利益又是对立的,对抗的。这种利益上的对立是由他们的资产阶级生活的经济条件产生的。资产阶级运动在其中进行的那些生产关系的性质绝不是一致的单纯的,而是两重的;在产生财富的那些关系中也产生贫困;在发展生产力的那些关系中也发展一种产生压迫的力量;只有在不断消灭资产阶级个别成员的财富和形成不断壮大的无产阶级的条件下,这些关系才能产生资产者的财富,即资产阶级的财富;这一切都一天比一天明显了。

这种对抗性质表现得越明显,经济学家们,这些资产阶级生产的学术代表就越和他们自己的理论发生分歧,于是形成了各种学派。

宿命论的经济学家,在理论上对他们所谓的资产阶级生产的否定方面采取漠不关心的态度,正如资产者在实践中对他们赖以取得财富的无产者的疾苦漠不关心一样。这个宿命论学派有古典派和浪漫派两种。古典派如亚当·斯密和李嘉图,他们代表着一个还在同封建社会的残余进行斗争、力图清洗经济关系上的封建残污、扩大生产力、使工商业具有新的规模的资产阶级。从他们的观点看来,参加这一斗争并专心致力于这一狂热活动的无产阶级只是经受着暂时的偶然的苦难,并且它自己也把这些苦难当做暂时的。亚当·斯密和李嘉图这样的经济学家是当代的历史学家,他们的使命只是表明在资产阶级生产关系下如何获得财富,只是将这些关系表述为范畴和规律并证明这些规律和范畴比封建社会的规律和范畴更便于进行财富的生产。在他们看来,贫困只不过是一种暂时的病痛,正如自然界中新生出东西来和工业上新东西出现时的情况一样。

浪漫派属于我们这个时代,这时资产阶级同无产阶级处于直接对立状态,贫困象财富那样大量产生。这时,经济学家便以饱食的宿命论者的姿态出现,他们自命高尚、蔑视那些用劳动创造财富的活人机器。他们的一言一语都仿照他们的前辈,可是,前辈们的漠不关心只是出于天真,而他们的漠不关心却已成为卖弄风情了。

其次是人道学派,这个学派对现时生产关系的坏的方面倒是放在心上的。为了不受良心的责备,这个学派想尽量缓和现有的对比;他们对无产者的苦难以及资产者之间的剧烈竞争表示真诚的痛心;他们劝工人安分守己,好好工作,少生孩子;他们建议资产阶级节制一下生产热情。这个学派的全部理论建立在理论和实践、原理和结果、观念和应用、内容和形式、本质和现实、法和事实、好的方面和坏的方面之间无限的区别上面。

博爱学派是完善的人道学派。他们否认对抗的必然性;他们愿意把一切人都变成资产者;他们愿意实现理论,因为这种理论与实践不同而且本身不会包含对抗。毫无疑问,在理论上把现实中每一步都要遇到的矛盾撇开不管并不困难。那样一来,这种理论就会变成理想化的现实。因此,博爱论者愿意保存那些表现资产阶级关系的范畴,而不要那种构成这些范畴的实质并且同这些范畴分不开的对抗。博爱论者以为,他们是在严肃地反对资产者的实践,其实,他们自己比任何人都更象资产者。

正如经济学家是资产阶级的学术代表一样,社会主义者和共产主义者是无产者阶级的理论家。在无产阶级尚未发展到足以确立为一个阶级,因而无产阶级同资产阶级的斗争尚未带政治性以前,在生产力在资产阶级本身的怀抱里尚未发展到足以使人看到解放无产阶级和建立新社会必备的物质条件以前,这些理论家不过是一些空想主义者,他们为了满足被压迫阶级的需求,想出各种各样的体系并且力求探寻一种革新的科学。但是随着历史的演进以及无产阶级斗争的日益明显,他们在自己头脑里找寻科学真理的做法便成为多余的了;他们只要注意眼前发生的事情,并且有意识地把这些事情表达出来就行了。当他们还在探寻科学和只是创立体系的时候,当他们的斗争才开始的时候,他们认为贫困不过是贫困,他们看不出它能够推翻旧社会的革命的破坏的一面。但是一旦看到这一面,这个由历史运动产生并且充分自觉地参与历史运动的科学就不再是空论,而是革命的科学了。

现在再来谈谈蒲鲁东先生。

每一种经济关系都有其好的一面和坏的一面;只有在这一点上蒲鲁东先生没有背叛自己。他认为好的方面由经济学家来揭示,坏的方面由社会主义者来揭发。他从经济学家那里借用了永恒经济关系的必然性这一看法;从社会主义者那里借用了使他们在贫困中只看到贫困的那种幻想。他对两者都表示赞成,企图拿科学权威当靠山。而科学在他的观念里已成为某种微不足道的科学公式了;他无休止地追逐公式。正因为如此,蒲鲁东先生自以为他既批判了政治经济学,也批判了共产主义;其实他远在这两者之下。说他在经济学家之下,因为他作为一个哲学家,自以为有了神秘的公式就用不着深入纯经济的细节;说他在社会主义者之下,因为他既缺乏勇气,也没有远见,不能超出(哪怕是思辨地也好)资产者的眼界。

他希望成为一种合题,结果只不过是一种总合的错误。

他希望充当科学泰斗,凌驾于资产者和无产者之上,结果只是一个小资产者,经常在资本和劳动、政治经济学和共产主义之间摇来摆去。

\newpage

\subsection{2.笔记}
\begin{fangsong}
    “现在我们已在德国中心!我们一方面谈论政治经济学,同时又要谈形而上学。”
\end{fangsong}
马克思如是说道。

英国的古典政治经济学是一门发源于经验的学科,而到了德国,便给它赋予了哲学的色彩。但由于德国的形而上学的传统,以至于其对政治经济学的理解处于一种颠倒的状态。古典经济学家\footnote{这里指的是相较于马克思那个时代而言的古典经济学家}们是从直接的社会现实中考察社会生产关系的,但是这种考察仅仅是经验意义上的概括,并没有进一步阐明经验关系的背后动因,因而这种考察在哲学的意义上是缺乏历史性的。当然,在当时的社会状况下,古典政治经济学所做的经验归纳是在一定程度上反映社会现实的,但由于这种经济学本身就是对社会现实的“同义反复”,因此这种经济学对于现实社会的理解往往是处于一种非革命性的状态,它看不到社会内部的本质的矛盾,以至于这种经济学丧失了远见性。

虽然古典政治经济学的直接对象就是活生生的社会现实,但他们所做的直接工作又是对这种现实进行直接的抽象,因此古典政治经济学已经开始远离真正的现实,虽然古典经济学家们自身并没有意识到。而当形而上学开始正大光明地介入古典政治经济学之中时,形而上学所作的工作便是对抽象的抽象,将生动的现实转变为死沉的抽象概念。这时古典国民经济学家们便可声称范畴的不变性,便可以用一种非历史性的经济学取代现实的历史。

事实上,对于经济范畴而言,这些范畴不过是历史上存在着的生产关系的理论表现(也就是现实的直接抽象)。而黑格尔的辩证法是将这种“现实的抽象”转变为“抽象的现实”,并承认抽象的自我运动。而蒲鲁东则是拙劣地模仿黑格尔的辩证法的形式,承认抽象\footnote{这里的“抽象”是“运动”的主语}的运动,但仅仅把抽象\footnote{同上}的运动理解为自己思维的运动,也即是说,蒲鲁东只是庸俗地把自己的个人想法加到历史的进程中,从而彻底扭曲了“自我运动”的含义,退回到了非辩证的主观唯心主义之中。

%从这里继续(第三个、第四个说明)
因此,蒲鲁东至少可以说是犯了两个致命的错误。第一,蒲鲁东没有领会黑格尔的辩证法,他仅仅把观念的自我运动理解成自己主观的想法、按照自己逻辑的演绎。第二,蒲鲁东没有意识到黑格尔的辩证法本身就是主客颠倒的辩证法(虽然他没有完全理解辩证法是什么),他把由现实社会的生产关系抽象出来的各概念之间的关系理解为真实存在的关系,而把现实社会本身仅仅理解为对这种关系的趋近、映射。因此,对于蒲鲁东而言,他是无法理解理性逻辑之外的事物,正如马克思在原文中所述的那样:

\begin{fangsong}
“……蒲鲁东先生在考察其中任何一个阶段时,都不能不靠其它一些社会关系来说明,可是当时这些社会关系尚未被他用辩证运动产生出来。当蒲鲁东先生后来借助纯粹理性使其它阶段产生出来时,却又把它们当成初生的婴儿,忘记它们和第一个阶段是同样年老了。”
\end{fangsong}

此外,由于蒲鲁东将概念的自我运动仅仅理解为自己主观的想法,因此他以一种机械的形式将事物的辩证运动改造成了事物好的方面\textbf{消灭}坏的方面的运动。也即是说,蒲鲁东的政治经济学遵循着这样的准则:在不改变现存的经济范畴的基础上,保存现存经济范畴的好的方面,消除其坏的方面。事实上,蒲鲁东的政治经济学主张是一种小资产阶级改良主义的思想。这种思想的问题根源于其背后遵循的那种哲学思想,这种思想本质上是非辩证的,因为对于辩证运动而言,事物(或范畴——即事物在观念上的映射)是在矛盾中发展的,也即是说企图割裂矛盾而实现事物的发展是不可能的。举个例子,对于\textbf{爱情}这一范畴而言,其有好的方面和坏的方面,当我们把它的内在矛盾割裂(把好的方面和坏的方面相分离),而只留住好的方面,那末,爱情便是一种虚假的幻想,是一种不存在现实世界的美好童话,这便是对爱情这一范畴非辩证的割裂,所能达到的结果就是陷入虚假的幻想之中。而什么是真正辩证的运动呢?即不去割裂爱情这一范畴的内在矛盾,而是在矛盾中发展爱情这一范畴本身,即爱情的好的方面和坏的方面同时被保留下来,二者在斗争中运动着,使爱情扬弃为\textbf{婚姻},对于婚姻这一范畴而言,它显然不是某种虚假的幻想,它同样保留着好的方面和坏的方面,从爱情到婚姻的扬弃,便是一种辩证运动。对于蒲鲁东而言,他主张在资本主义经济制度下,将好的方面保留下来,而不要坏的方面,这就如同要求现实社会中的爱情只保留好的方面一样,是不切实际的。

马克思指出,蒲鲁东所认为的通过人类理性所产生的一系列二律背反(或者说人类理性所不断解决的一系列正题与合题),不过是理性的矛盾假设。对于蒲鲁东而言,“社会天才”便是他做出的一个假设,这个假设的限定条件便是“社会天才”可以消除每个经济范畴一切坏的东西,而使它保留好的东西。诚然,我们可以从这一假设出发,遵循着“理性的逻辑”去发展“历史”,但这种“历史”仅仅是一种幻想的历史,它存在于著作家的小说中,不存在于真实的社会。造成这种观念与现实之间的二律背反的原因在于假设本身与现实世界之间的二律背反,事实上,关于这一点,是一个本体论层面的问题:当蒲鲁东做出这样的一个原初的假设之时,他就预设了一个唯心主义的世界观,他就承认了世界存在有一个支配一切的理性的本源,即理性自身存在的合理性——至少是理性自身存在的发生学机制——是不容置疑的。然而问题就在于,理性本身并非是先在的,就像人类通过(物质的或精神的)活动建构出了一个世界一样,理性本身同样也是通过诸多非理性的活动被建构出来的。

总的来说,蒲鲁东的假设是不成立的。某一经济范畴中好的方面与坏的方面本就是一体两面的,就像大枣有益于脾胃而有害于牙齿一样,若是强行保留其好的方面而去除坏的方面(即蒲鲁东所追求的“合题”),就是“囫囵吞枣”,既无益于牙齿也无益于脾胃。

因此,对于蒲鲁东,马克思如是评价道:

\begin{fangsong}
    “他希望成为一种合题,结果只不过是一种总合的错误。

    他希望充当科学泰斗,凌驾于资产者和无产者之上,结果只是一个小资产者,经常在资本和劳动、政治经济学和共产主义之间摇来摆去。”
\end{fangsong}




\newpage
\section{《政治经济学批判》导言}
\subsection{1.原文}
\begin{center}
 I生产,消费,分配,交换(流通)    
\end{center}
\subsubsection{1~生产}
\textbf{(a)面前的对象,首先是物质生产。}

在社会中进行生产的个人,因而,这些个人的一定社会性质的生产,自然是出发点。被斯密和李嘉图当作出发点的单个的孤立的猎人和渔夫,应归入18世纪鲁宾逊故事的毫无想象力的虚构,鲁宾逊故事决不像文化史家设想的那样,仅仅是对极度文明的反动和想要回到被误解了的自然生活中去。同样,卢梭的通过契约来建立天生独立的主体之间的相互关系和联系的社会契约论,也不是奠定在这种自然主义的基础上的,这是错觉,只是美学上大大小小的鲁宾逊故事的错觉。这倒是对于16世纪以来就进行准备,而在18世纪大踏步走向成熟的”市民社会”的预感。在这个自由竞争的社会里,单个的人表现为了摆脱了自然联系等等,后者在过去历史时代使他成为一定的狭隘人群的附属物。这种18世纪的个人,一方面是封建社会形式解体的产物,另一方面是16世纪以来新兴生产力的产物,而在18世纪的预言家看来(斯密和李嘉图还完全以这些预言家为依据),这种个人是一种理想,他的存在是过去的事;在他们看来,这种个人不是历史的结果,而是历史的起点。因为,按照他们关于人类天性的看法,合乎自然的个人并不是从历史中产生的,而是由自然造成的。这样的错觉是到现在为止的每个新时代所具有的。斯图亚特在许多方面同18世纪对立并做为贵族比较多地站在历史上,从而避免了这种局限性。

我们愈往前追溯历史,个人,也就是进行生产的个人,就显得愈不独立,愈从属于一个更大的整体:最初还是十分自然地在家庭和扩大成为氏族的家庭中;后来是在由氏族间的冲突和融合而产生的各种形式的公社中。只有到十八世纪,在”市民社会”中,社会结合的各种形式,对个人说来,才只是达到他私人目的手段,才是外在的必然性。但是,产生这种孤立的个人的观点的时代,正是具有迄今为止最发达的社会关系(从这种观点来看是一般关系)的时代。人是最名符其实的社会动物,不仅是一种合群的动物,而且是只有在社会中才能独立的动物。孤立的一个人在社会之外进行生产-这是罕见的事,’偶然落到荒野中的已经内在地具有社会力量的文明人或许能做到-就像许多个人不再一起生活和彼此交谈而竟有语言发展一样,是不可思议的。在这方面无须多说。十八世纪的人们有这种荒诞无稽的看法本是可以理解的,如果不是巴师夏,凯里和蒲鲁东等人又把这种看法郑重其事地引进最新的经济学中来,这一点本来可以完全不提。蒲鲁东等人自然乐于用编造神话的办法,来对一种他不知道历史来源的经济关系做历史哲学的说明,说什么这种观念对亚当及普罗米修斯已经是现成的,后来他就被付诸实行等等。再没有比这类想入非非的陈腔滥调更加乏味的了。

因此,说到生产,总是指在一定社会发展阶段上的生产-社会个人的生产。因而,好象只要一说到生产,我们或者就要把历史发展过程在它的各个阶段上一一加以研究,或者一开始就要声明,我们只的是某个一定的历史时代,例如,是现代资产阶级生产-这种生产事实上是我们研究的本题。可是,生产的一切时代有某些共同标,共同规定。生产一般是一个抽象,但是只要它真正把共同点提出来,定下来,免得我们重复,它就是一个合理的抽象。不过,这个一般,或者说,经过比较而抽出来的共同点,本身就是有另一些是几个时代共有的,[有些]规定是最新时代和最古时代共有的,没有它们,任何生产都无从设想;如果说最发达语言的有些规律和规定也是最不发达语言所有的,但是构成语言发展的恰恰是有别于这一般和共同点的差别,那末,对生产一般适用的种种规定所以要抽出来,也正是为了不致因见到统一(主体是人,客体是自然,这总是一样的,这里已经出现了统一)就忘记了本质的差别。而忘记这种差别,正是那些证明现存社会关系永存与和谐的现代经济学家的全部智慧所在。例如,他们说,没有生产工具,哪怕这种生产工具不过是手,任何生产都不可能。没有过去的,累积下来的劳动,哪怕这种劳动不过是由于反复操作而累聚在野蛮人手上的技巧,任何生产都不可能。资本,别的不说,也是生产工具,也是过去的,客体化了的劳动。可见资本是一种一般的,永存的自然关系;这就是说,如果我们恰好抛开了正是使”生产工具”,”累积下来的劳动”成为资本的那个特殊的话。因此,生产关系的全部历史,例如在凯里看来,是历代政府的恶意篡改。

如果没有生产一般,也就没有一般的生产。生产总是一个特殊的生产部门-如农业,畜牧业,制造业等,或者是他们的总体。可是,政治经济学不是工艺学。生产的一般规定在一定社会阶段上对特殊生产形式的关系,留待别处(后面)再说。

最后,生产也不只是特殊的生产,而始终是一定的社会体及社会的主体在或广或窄由各生产部门组成的总体中活动着。科学的叙述对现实运动的关系,也还不是这里所要说的。生产一般。特殊生产部门。生产的总体。

现在时髦的做法,是在经济学的开头摆上一个总论部份-就是标题为《生产》的那部份(参看约翰,斯图亚特,穆勒的著作),用来论述一切生产的一般条件。

这个总论部份包括或者好像应当包括:

(1)进行生产所必不可缺少的条件。因此,这实际上不过是要说明一切生产的基本要素。可是,我们将会知道,实际上归纳起来不过是几个十分简单的规定,却扩展成浅薄的同义反复。

(2)或多或少促进生产的条件,如像亚当。斯密所说的前进的和停滞的社会状态。要把这些在斯密那里作为提示而具有价值的东西提升到科学意义上来,就得研究各个民族的发展过程终生产率程度不同的各个时期-这种研究超出本题应有的范围,但就属于本题范围来说,在叙述竞争,累积等等时是要谈到的。照一般的提法,答案总是这样一个一般的说法:一个工业民族,当它一般地达到它的历史高峰的时候,也就达到它的生产高峰。实际上,一个民族的工业高峰是在它还不是以既得利益为要务,而是以争取利益为要务的时候。在这一点上,美国人胜过英国人。或者是这样的说法:例如,某一些种族,素质,气候,自然条件如离海远近,土地肥沃程度等等,比另外一些更有利于生产。这又是同义反复,即财富的主客观因素越是在更高的程度上具备,财富就越容易创造。

但是,经济学家在这个总论部份所真正要谈的并不是这一切。相反,照他们的意见,生产不同于分配等等(参看穆勒的著作),应当被描写成局限在脱离历史而独立的永恒自然规律之内的事情,于是资产阶级关系就被乘机当作社会一般的颠扑不破的自然规律偷偷地塞了进来。这是整套手法的多少有意识的目的。反之,在分配上,好象人们事实上可以随心所欲。即使根本不谈生产和分配的这种粗暴割裂与生产与分配的现实关系,下面这一点总应当是一开始就明白的:无论在不同社会阶段上分配如何不同,总是可以像在生产中那样提出一些共同的规定来,可以把一切历史差别混合和融化在一般人类规律之中。例如,奴隶,农奴,雇佣工人都得到一定量的食物,使他们能够作为奴隶,农奴和雇佣工人来生存。靠贡赋生活的征服者,靠税收生活的官吏,靠地租生活的土地占有者,靠施舍生活的僧侣,或者靠什一税生活的教士,都得到一份社会产品,而决定这一份产品的规律不同于决定奴隶等等那一份产品的规律。一切经济学家在这个项目下提出的两个要点是:(1)所有制,(2)司法,警察等对所有制的保护,对此要极简单地答复一下:

关于第一点,一切生产都是个人在一定社会形式中并藉这种社会形式而进行的对自然的占有。在这个意义上,说所有制(占有)是生产的一个条件,那是同义反复。但是,可笑的是从这里一步就跳到所有制的一定形式,如私有制。(而且还把对立的形式即无所有作为条件。)历史却表明,公有制是原始形式(如印度人,斯拉夫人,古克尔特人等等),这种形式在公社所有制形式下还长期起着显着的作用。至于财富在这种还是那种所有制形式下能更好地发展的问题,还根本不是这里所要谈的。可是,如果说在任何所有制都不存在的地方,就谈不到任何生产,因此也就谈不到任何社会,那末,这是同义反复。什么也不据为己有的占有,是自相矛盾。

关于第二点,对既得物的保护等等。如果把这些滥调还原为它们的实际内容,它们所表示的就比它们的说教者所知道的还多。就是说,每种生产形式都产生出它所特有的法权关系,统治形式等等。粗率和无知之处正在于把有机地联系着的东西看成是彼此偶然发生关系的,纯粹反射联系中的东西,资产阶级经济学家只 糊地感到,在现代警察制度下,比在例如强权下能更好地进行生产,他们只是忘记了,强权也是一种法权,而且强者的法权也以另一种形式继续存在于他们的”法治国家”中。

当与生产的一定阶段相应的社会状态刚刚产生或者已经衰亡的时候,自然会出现生产上的紊乱,虽然程度和影响有所不同。

总之:一切生产阶段所共同的,被思维当作一般规定而确定下来的规定,是存在的,但是所谓一切生产的一般条件,不过是这些抽象要素,用这些抽象要素不可能理解任何一个现实的历史的生产阶段。

\subsubsection{2~生产与分配,交换,消费的一般关系}

在进一步分析生产之前,必须观察一下经济学家拿来与生产并列的几个项目。

敷浅的表象是:在生产中,社会成员占有(开发,改造)自然产品供人类需要;分配决定个人分取这些产品的比例;交换给个人带来它享用分配给他的一份去换取的那些特殊产品;最后,在消费中,产品变成享受的对象,个人占有的对象。生产创造出适合需要的对象;分配依照社会规律把它们分配;交换依照个人需要把已经分配的东西再分配;最后,在消费中,产品脱离这种社会运动,直接变成个人需要的对象和仆役,被享受而满足个人需要。因而,生产表现为起点,消费表现为终点,分配和交换表现为中间环节,这中间环节又是二重的,因为分配被规定为从社会出发的要素,交换被规定为从个人出发的要素。在生产中,人客体化,在人中,物主体化;在分配中,社会以一般的,居于支配地位的规定的形式,担任生产和消费之间的媒介;在交换中,生产和消费由偶然的个人的规定性来媒介。

分配决定产品归个人的比例(分量);交换决定个人对于分配给自己的一份所要求的产品。

生产,分配,交换,消费因此形成一个正归的三段论法;生产是一般,分配和交换是特殊,消费是个别,全体由此结合在一起。这当然是一种联系,然而是一种敷浅的联系。生产决定于一般的自然规律,分配决定于社会的偶然情况,因此它能够或多或少地对生产起促进作用;交换作为形式上的社会运动介于两者之间;而消费这个不仅被看成终点而且被看成最后目地的结束行为,除了它又反过来作用于起点并重新引起整个过程之外,本来不属于经济学的范围。

反对政治经济学家的人们,-不论这些反对者是不是他们的同行,-责备他们把联系着的东西粗野地割裂了,这些反对者或者是同他们站在同一个基础上,或者是在他们之下。最庸俗不过的责备就是,说政治经计学家过于重视生产,把它当作目的本身。说分配也是同样重要的。这种责备的立足点恰恰是那种把分配当作与生产并列的独立自主的领域的经济见解。或者是这样的责备,说媒有把这些要素放在其统一中来理解。好象这种割裂不是从现实中进到教科书中去的,而相反地是从教科书进到现实中去的,好像这里的问题是要把概念作辩证的平衡,而不是解释现实的关系!

\textbf{(a)[生产和消费]}

生产直接也是消费。双重的消费,主体的和客体的:个人在生产当中发展自己的能力,也在生产行为中支出和消耗这种能力,同自然的生殖是生命力的一种消耗完全一样。第二,生产资料的消费,生产资料被使用,被消耗,一部分(如在燃烧中)重新分解为一般元素。原料的消费也是这样,原料不再保持自己的自然形状和特性,这种自然形状和特性倒是消耗掉了。因此,生产行为本身就它的一切要素来说也是消费行为。不过,这一点是经济学家所承认的,他们把直接与消费同一的生产,直接与生产合一的消费,称作生产的消费。生产和消费的这种同一性,归结起来是斯宾诺莎的命题:”规定即否定”。但是,提出生产的消费这个规定,只是为了把与生产同一的消费跟原来意义上的消费区别开来,后面这种消费被理解为起消灭作用的与生产相对的对立面,我们且观察一下这个原来意义上的消费。

消费直接也是生产,正如自然界中的元素和化学物质的消费是植物的生产一样。例如,吃喝是消费形式之一,人吃喝就生产自己的身体,这是明显的事。而对于以这种或那种形式从某一方面来生产人的其它任何消费形式也都可以这样说。消费的生产。可是,经济学却说,这种与消费同一的生产是第二种生产,是靠消灭第一种生产的产品引起的。在第一种生产中,生产者物化,在第二种生产中,生产者所创造的物人化。因此,这种消费的生产,-虽然它是生产和消费的直接统一-是与原来意义上的生产根本不同的。生产同消费合而为一和消费同生产合而为一的这种直接统一,并不排斥它们的直接两立。

可见,生产直接是消费,消费直接是生产。每一方直接是它的对方。可是同时在两者之间存在着一种媒介运动。生产媒介着消费,它创造出消费的材料,没有生产,消费就没有对象。但是消费也媒介着生产,因为正式消费替产品创造了主体,产品对这个主体才是产品。产品在消费中才得到最后完成。一条铁路,如果没有通车,不被磨损,不被消费,它只是可能性的铁路,不是现实的铁路。没有生产,就没有消费,但是,没有消费,也就没有生产,因为如果这样,生产就没有目的。消费从两方面生产着生产。

(1)因为只是在消费中产品才成为现实的产品,例如,一件衣服由于穿的行为才现实地成为衣服;一间房屋无人居住,事实上就不成为现实的房屋;因此,产品不同于单纯的自然对象,它在消费中才证实自己是产品,才成为产品。消费是在把产品消灭的时候才使产品最后完成,因为产品之所以是产品,不是它做为物化了的活动,而只是做为活动着的主体的对象。

(2)因为消费创造出新的生产的需要,因而创造出生产的观念上的内在动机,后者是生产的前提。消费创造出生产的动力;它也创造出在生产中做为决定目的的东西而发生作用的对象。如果说,生产在外部提供消费的对象是显而易见的,那末,同样显而易见的是,消费在观念上提出生产的对象,做为内心的意象,作为需要,做为动力和目的。消费创造出还是在主观形式上的生产对象。没有需要,就没有生产。而消费则把需要再生产出来。

与此相应,就生产方面来说:

(1)它为消费提供材料,对象。消费而无对象,不成其为消费;因而,生产在这方面创造出,生产出消费。

(2)但是,生产为消费创造的不只是对象。它也给予消费以消费的规定性,消费的性质,使消费得以完成。正如消费使产品得以完成其为产品一样,生产使消费得以完成。首先,对象不是一般的对象,而是一定的对象,是必须用一定的而又是由生产本身所媒介的方式来消费的。饥饿总是饥饿,但是用刀叉吃熟肉来解除的饥饿不同于用手,指甲和牙齿啃生肉来解除的饥饿。因此,不仅消费的对象,而且消费的方式,不仅客体方面,而且主体方面,都是生产所生产的。所以,生产创造消费者。

(3)生产不仅为需要提供材料,而且它也为材料提供需要。在消费脱离了它最初的自然粗陋状态和直接状态之后,-如果停留在这种状态,那也是生产停滞在自然粗陋状态的结果,-消费本身做为动力是靠对象做媒介的。消费对于对象所感到的需要,是对于对象的知觉所创造的。艺术对象创造出懂得艺术和能够欣赏美的大众,-任何其它产品也都是这样。因此,生产不仅做为主体生产对象,而且也为对象生产主体。

因此,生产生产着消费:(1)是由于生产为消费创造材料,(2)是由于生产决定消费的方式,(3)是由于生产靠它起初当作对象生产出来的产品在在消费者身上引起需要。因而,它生产出消费的对象,消费的方式和消费的动力。同样,消费生产出生产者的素质,因为它在生产者身上引起追求一定目的的需要。

因此,消费和生产之间的同一性表现在三方面:

(1)直接的同一性:生产是消费;消费是生产。消费的生产。生产的消费。政治经济学家把两者都称为生产的消费,可是还做了一个区别。前者表现为再生产,后者表现为生产的消费。关于前者的一切研究是关于生产的劳动或非生产的劳动的研究;关于后者的研究是关于生产的消费或非生产的消费的研究。

(2)每一方表现为对方的手段;以对方为媒介;这表现为他们的相互依存;这是一个运动,它们通过这个运动彼此发生关系,表现为互不可缺,但又各自处于对方之外。生产为消费创造作为外在对象的材料;消费为生产创造作为内在对象,作为目的的需要。没有生产就没有消费;没有消费就没有生产。这在经济学中以多种多样的形式表现出来。

(3)生产不仅直接是消费,消费也不仅直接是生产;而且生产不仅是消费的手段,消费不仅是生产的目的,-就是说,每一方都为对方提供对象,生产为消费提供外在的对象,消费为生产提供想象的对象;两者的每一方不仅直接就是对方,不仅媒介着对方,而且,两者的每一方当自己实现时也就创造对方,把自己当作对方创造出来。消费完成生产行为,只是在消费使产品最后完成其为产品的时候,在消费把它消灭,把它的独立的物体形式毁掉的时候;在消费使得在最初生产行为中发展起来的素质通过反复的需要达到完美的程度的时候;所以,消费不仅是使产品成为产品的最后行为,而且也是使生产者成为生产者的最后行为。另一方面,生产生产出消费,是在生产创造出消费的一定方式的时候,然后是在生产把消费的动力,消费能力本身当作需要创造出来的时候。这和第三项所说的这个最后的同一性,经济学在论述需求和供给,对象和需要,社会创造的需要和自然需要的关系时,曾多次加以解释。

这样看来,对于一个黑格尔主义者来说,把生产和消费同一起来,是最简单不过的事。不仅社会主义美文学家这样做过,而且平庸的经济学家也这样做过,萨伊就是个例子;他的说法是,就一个民族来说,它的生产也就是它的消费。或者,就人类一般来说,也是这样。施托尔希指出过萨伊的错误,因为例如一个民族,不是把自己的产品全部消费掉,而是还要创造生产资料等等,固定资本等等。此外,把社会当作一个单独的主体来观察,是对它做了不正确的观察,思辨式的观察。就一个主体来说,生产和消费表现为一个行为的两个要素。这里要强调的主要之点是:如果我们把生产和消费看做一个主体的或者许多单个个人的活动,它们无论如何表现为一个过程的两个要素,在这个过程中,生产是实际的起点,因而也是居于支配地位的要素。消费,做为必需,做为需要,本身就是生产活动的一个内在要素。但是生产活动是实现起点,因而也是实现的居于支配地位的要素,是整个过程借以从新进行的行为。个人生产出一个对象,因消费了它而再回到自己身上,然而,他是作为生产的个人,把自己再生产的个人。所以,消费表现为生产的要素。

但是,在社会中,产品一经完成,生产者对产品的关系就是一种外在的关系,产品回到主体,取决于主体对其它个人的关系。他不是直接获得产品。如果说他是在社会中生产,那末直接占有产品也不是他的目的。在产品和生产者之间插进了分配,分配借社会规律决定生产者在产品世界中的份额,因而插在生产和消费之间。

那末,分配是否作唯独立的领域,处于生产之旁和生产之外呢?

\textbf{(b)[生产和分配]}

如果看看普通的经济学著作,首先令人注目的是,在这些著作里什么都被提出两次。举例来说,在分配上出现的是地租,工资,利息和利润,而在生产上做为生产要素出现的是土地,劳动,资本。说到资本,一看就清楚,它被提出了两次:(1)当作生产要素;(2)当作收入源泉,当作决定一定的分配形式的东西。利息和利润,就它们做为资本增殖和扩大的形式,因而做为资本自身的生产的要素来说,本身也出现在生产中。利息和利润作为分配形式,是以资本作为生产要素为前提的。他们是以资本作为生产要素为前提的分配方式。它们又是资本的再生产方式。

同样,工资也是在另一个项目中被考察的雇佣劳动:在一处作为生产要素的劳动所具有的规定性,在另一处表现为分配的规定。如果劳动不是规定为雇佣劳动,那末,它参与产品分配的方式,也就不表现为工资,如在奴隶制度下就是这样。最后,地租-我们直接地来看地产参与产品分配的最发达形式-的前提,是作为生产要素的大地产(其实是大农业),而不是通常的土地,就像工资的前提不是通常的劳动一样。所以,分配关系和分配方式只是表现为生产要素的背面。个人以雇佣劳动的形式参与生产,就以工资形式参与产品,生产成果的分配。分配的结构完全取决于生产的结构,分配本身就是生产的产物,不仅就对象说是如此,而且就形式说也是如此。就对象说,能分配的只是生产的成果,就形式说,参与生产的一定形式决定分配的特定形式,决定参与分配的形式。把土地放在生产上来谈,把地租放在分配上来谈,等等,简直是幻觉。

因此,像李嘉图那样的经济学家,最受责备的就是他们眼中只有生产,他们却专门把分配规定为经济学的对象,因为他们本能地把分配形式看成是一定社会中的生产要素得以确定的最确切的表现。

在单个的个人面前,分配自然表现为一种社会规律,这种规律决定他在生产中-指他在其中进行生产的那个生产-的地位,因而分配先于生产。这个个人一开始就没有资本,也没有地产。他一出生就由社会分配指定专门从事雇佣劳动。但是这种指定本身是资本和地产作为独立的生产要素存在的结果。

就整个社会来看,从一方面说,分配似乎先于生产,并且决定生产,似忽是先经济的事实。一个征服者民族在征服者之间分配土地,因而造成了地产的一定的分配和形式,由此决定了生产。或者,它使被征服的民族成为奴隶,于是使奴隶劳动成为生产的基础。或者,一个民族经过革命把大地产粉碎成小块,从而通过这种新的分配使生产有了一种新的性质。或者,立法使地产永远属于一定的家庭,或者,把劳动[当作]世袭的特权来分配,因而把它像等级一样地固定下来。在所有这些历史上有过的情况下,似乎不是生产安排和决定分配,而相反地是分配安排和决定生产。

照最浅薄的理解,分配表现为产品的分配,因此它彷佛离开生产很远,对生产是独立的。但是,在分配是产品的分配之前,它是(1)生产工具的分配,(2)社会成员在各类生产之间的分配(个人从属于一定的生产关系)-这是上述同一关系的进一步规定。这种分配包含在生产过程本身中并且决定生产的结构,产品的分配显然只是这种分配的结果。如果在考察生产时把包含在其中的这种分配撇开,生产显然只是一个空洞的抽象;反过来说,有了这种本来构成生产的一个要素的分配,力求在一定的社会结构中来理解现代生产并且主要是研究生产的经济学家李嘉图,不是把生产而是把分配说成现代经济学的本题。从这里,又一次显出了那些把生产当作永恒真理来论述而把历史限制在分配范围之内的经济学家是多么荒诞无稽。

这种决定生产本身的分配究竟和生产处于怎么样的关系,这显然是属于生产本身内部的问题。如果有人说,既然生产必须从生产工具的一定分配出发,至少在这个意义上分配先于生产,成为生产的前提,那末就应该答复他说,生产实际上有它的条件和前提,这些条件和前题构成生产的要素。这些要素最初可能表现为自然发生的东西。通过生产过程本身,它们就从自然发生的东西变成历史的东西了,如果它们对于一个时期表现为生产的自然前提,对于另一个时期就是生产的历史结果了。它们在生产内部不断地改变。例如,机器的应用既改变了生产工具的分配,也改变了产品的分配。现代大土地所有制本身既是现代商业和现代工业的结果,也是现代工业在农业上应用的结果。

上面提出的一些问题,归根到底就是:一般历史条件在生产上是怎样起作用的,生产和一般历史运动的关系又是怎样的。这个问题显然属于对生产本身的讨论和分析。

然而,这些问题即使照上面那样平庸的提法,也可以同样给予简短的回答。所有的征服有三种可能。征服民族把自己的生产方式强加于被征服的民族(例如,本世纪英国人在爱尔兰所做的,部份地在印度所做的);或者是征服民族让旧生产方式维持下去,自己满足于征收贡赋(如土耳其人和罗马人);或者是发生一种相互作用,产生一种新的,综合的生产方式(日耳曼人的征服中一部分就是这样)。在所有的情况下,生产方式,不论是征服民族的,被征服民族的,还是两者混合形成的,总是决定新出现的分配。因此,虽然这种分配对于新的生产时期表现为前提,但它本身又是生产的产物,不仅是一般历史生产的产物,而且是一定历史生产的产物。

例如,蒙古人把俄罗斯弄成一片荒凉,这样做是适合于他们的生产,畜牧的,大片无人居住的地带是畜牧的主要条件。在日耳曼蛮族,用农奴耕作是传统的生产,过的是乡村的孤独生活,他们能够非常容易地让罗马各省服从于这些条件,因为那里发生的土地所有权的集中已经完全推翻了旧的农业关系。

有一种传统的观念,认为在某些时期人们只靠劫掠生活。但是要能够劫掠,就要有可以劫掠的东西,因此就要有生产。而劫掠方式本身又决定生产方式。例如,劫掠一个从事证券投机的民族就不能同劫掠一个游牧民族一样。

奴隶直接被剥夺了生产工具。但是奴隶受到剥夺的国家的生产必须安排得容许奴隶劳动,或者必须建立一种适于使用奴隶的生产方式(如在南美等)。

法律可以使一种生产资料,例如土地,永远属于一定家庭。这些法律,只有当大土地所有权适合于社会生产的时候,如像在英国那样,才有经济意义。在法国,尽管有大土地所有权,但经营的是小土地农业,因而大土地所有权就被革命摧毁了。但是,土地析分的状态是否例如通过法律永远固定下来了呢?尽管有这种法律,土地的所有权却又集中起来了。法律在巩固分配关系方面的影响和它们由此对生产发生的作用,要专门加以确定。

\textbf{(c)最后,交换和流通}

流通本身只是交换的一定要素,或者也是从总体上看的交换。

既然交换只是生产以及由生产决定的分配一方和消费一方之间的媒介要素,而消费本身又表现为生产的一个要素,交换当然也就当做生产的要素包含在生产之内。

首先很明显,在生产本身之中发生的各种活动和各种能力的交换,直接属于生产,并且从本质上组成生产。第二,这同样适用于产品交换,只要产品交换是用来制造供直接消费的成品的手段。在这个限度内,交换本身是包含在生产之中的行为。第三,所谓企业家之间的交换,从它的组织方面看,既完全决定于生产,且本身也是生产行为。只有在最后阶段上,当产品直接为了消费而交换的时候,交换才表现为独立于生产之外,与生产漠不相干。但是,(1)如果没有分工,不论这种分工是自然发生的或者本身已经是历史的成果,也就没有交换;(2)私的交换以私的生产为前提;(3)交换的深度,广度和方式都是由生产的发展和结构决定的。例如,城乡之间的交换,乡村中的交换,城市中的交换等等。可见,交换就其一切要素来说,或者是直接包含在生产当中,或者是由生产决定。

我们得到的结论并不是说,生产,分配,交换,消费是同一的东西,而是说,它们构成一个总体的各个环节,一个统一体内部的差别。生产既支配着生产的对立规定上的自身,也支配着其它要素。过程总是从生产重新开始。交换和消费是不能支配作用的东西,那是自明之理。分配,作为产品的分配,也是这样。而作为生产要素的分配,它本身就是生产的一个要素。因此,一定的生产决定一定的消费,分配,交换和这些不同要素相互间的一定关系。当然,生产就其片面形式来说也决定于其它要素。例如,当市场扩大,即交换范围扩大时,生产的规模也就增大,生产也就分得更细。随着分配的变动,例如,随着资本的集中,随着城乡人口的不同的分配等等,生产也就发生变动。最后,消费的需要决定着生产。不同要素之间存在着相互作用。每一个有机整体都是这样。

\subsubsection{3~政治经济学的方法}

当我们从政治经济学方面观察某一国家的时候,我们从该国的人口,人口的阶级划分,人口在城乡海洋的分布,在不同生产部门的分布,输入和输出,全年的生产和消费,商品价格等等开始。

从实在和具体开始,从现实的前提开始,因而,例如在经济学上从做为全部社会生产行为的基础和主体开始,似乎是正确的。但是,更仔细地考察起来,这是错误的。如果我抛开构成人口的阶级,人口就是一个抽象。如果我不知道这些阶级所依据的因素,如雇佣劳动,资本等等,阶级又是一句空话。而这些因素是以交换,分工,价格等等为前提的。比如资本,如果没有雇佣劳动,价值,货币,价格等等,它就什么也不是。因此,如果我从人口着手,那末这就是一个混沌的关于整体的表象,经过更切进的规定之后,我就会在分析中达到越来越简单的概念;从表象中的具体达到越来越稀薄的抽象,直到我达到一些最简单的规定。于是行程又得从那里回过头来,直到我最后又回到人口,但是这回人口已不是一个混沌的关于整体的表象,而是一个具有许多规定和关系的丰富的总体了。第一条道路是经济学在它产生时期在历史上走的道路。例如,十七世纪的经济学家总是从生动的整体,从人口,民族,国家,若干国家等等开始;但是他们最后总是从分析中找出一些具有决定意义的抽象的一般的关系,如分工,货币,价值等等。这些个别要素一旦多少确定下来和抽象出来,从劳动,分工,需要交换价值等等这些简单的东西上升到国家,国际交换和世界市场的各种经济学体系就开始出现了。后一种显然是科学上正确的方法。具体之所以具体,因为它是许多规定的综合,因而是多样性的统一。因此它在思维中表现为综合的过程,表现为结果,而不是表现为起点,虽然它是现实中的起点,因而也是直观和表象的起点。在第一条道路上,完整的表象蒸发为抽象的规定;在第二条道路上,抽象的规定在思维行程中导致具体的再现。因而黑格尔陷入幻觉,把实在理解为自我综合,自我深化和自我运动的思维的结果,其实,从抽象上升到具体的方法,只是思维用来掌握具体并把它当作一个精神上的具体再现出来的方式。但决不是具体本身的产生过程。举例来说,最简单的经济范畴,如交换价值,是以人口,以在一定关系中进行生产的人口为前提的;也是以某种形式的家庭,公社或国家等为前提的。它只能做为一个既与的,具体的,生动的整体的抽象片面的关系而存在。相反,做为范畴,交换价值却有一种洪水期前的存在。因此,在意识看来-而哲学意识就是被这样规定的:在它看来,正在理解着的思维是现实的人,因而,被理解的世界本身才是现实的世界-范畴的运动表现为现实的生产行为(只可惜它从外界取得一种推动),而世界是这种生产行为的结果;这-不过又是一个同义反复-只有在下面这个限度内才是正确的:具体总体做为思维总体,做为思维具体,事实上是思维的,理解的产物;但是,决不是处于直观和表象之外或驾乎其上而思维着的,自我产生着的概念的产物,而是把直观和表象加工成概念这一过程的产物。整体,当它在头脑中作为被思维的整体而出现时,是思维着的头脑的产物,这个头脑用它所专有的方式掌握世界,而这种方式是不同于对世界的艺术的,宗教的,实践-精神的掌握的。实在主体仍然是在头脑之外保持着它的独立性;只要这个头脑还仅仅是思辨地,理论地活动着。因此,就是在理论方法上,主体,即社会,也一定要经常作为前提浮现在表象面前。

但是,这些简单的范畴在比较具体的范畴以前是否也有一种独立的历史存在或自然存在呢?要看情况而定。比如,黑格尔论法哲学,是从主体的最简单的法的关系即占有开始的,这是对的。但是,在家庭或主奴关系这些具体的多的关系之前,占有并不存在。相反,如果说有这样的家庭和氏族,它们还只是占有,而没有所有权,这倒是对的。所以,这种比较简单的范畴,表现为简单的家庭或氏族的公社在所有权方面的关系。它在比较高级的社会中表现为一个发达的组织的比较简单的关系。但是那个以占有为关系的具体的基础总是前提。可以设想一个孤独的野人占有东西,但是在这种情况下,占有并不是法的关系。说占有在历史上发展为家庭,是错误的。占有倒总是以这个”比较具体的法的范畴”为前提的。但是,不管怎样总可以说,简单范畴是这样一些关系的表现,在这些关系中,不发展的具体可以已经实现,而那些通过较具体的范畴在精神上表现出来的较多方面的联系和关系还没有产生;而比较发展的具体则把这个范畴当作一种从属关系保存下来。在资本存在之前,银行存在之前,雇佣劳动存在之前,货币能够存在,而且在历史上存在过。因此,从这一方面看来,可以说,比较简单的范畴可以表现一个比较不发展的整体的处于支配地位的关系,或者可以表现一个比较发展的整体的从属关系,后面这些关系,在整体向着一个比较具体的范畴表现出来的方面发展之前,在历史上已经存在。在这个限度内,从最简单上升到复杂这个抽象思维的进程符合现实的历史过程。

另一方面,可以说,有一些十分发展的,但在历史上还不成熟的社会形式,其中有最高级的经济形式,如协作,发达的分工等等,却不存在任何货币,秘鲁就是一个例子。就在斯拉夫公社中,货币以及作为货币的条件的交换,也不是或者很少是出现在个别公社内部,而是出现在它的边界上,出现在与其它公社的交往中,因此,把同一公社内部的交换当作原始构成因素,是完全错误的。相反地,与其说它起初发生在同一公社内部的成员间,不如说它发生在不同公社的相互关系中。其次,虽然货币很早就全面地发生作用,但是在古代它只是片面发展的民族即商业民族中才是处于支配地位的因素。甚至在最文明的古代,在希腊人和罗马人那里,货币的充份发展-在现代的资产阶级社会中这是前提-只是在他们解体的时期。因此,这个十分简单的范畴,在历史上只有在最发达的社会状态下才表现出它的充份的力量。它决没有历尽一切经济关系。例如,在罗马帝国,在它最发达的时期,实物税和实物租仍然是基础。那里,货币制度原来只是在军队中得到充份发展。它也从来没有掌握劳动的整个领域。可见,比较简单的范畴,虽然在历史上可以在比较具体的范畴之前存在,但是,它的充分深入而广泛的发展恰恰只能属于一个复杂的社会形式,而比较具体的范畴在一个比较不发达的社会形式中有过比较充份的发展。

劳动似乎是一个十分简单的范畴。它在这种一般性-作为劳动一般-上的表象也是古老的。但是,在经济学上从这种简单性上来把握的”劳动”,和产生这个简单抽象的那些关系一样,是现代的范畴。例如,货币主义把财富看成还是完全客观的东西,看成存在于货币中的物。同这个观点相比,重工主义或重商主义把财富的源泉从对象转到主体的活动-商业劳动和工业劳动,已经是很大的进步,但是,他们仍然只是局限地把这种活动本身理解为取得货币的活动。同这个学派相对立的重农学派把劳动的一定形式-农业-看作创造财富的劳动,不再把对象本身看做裹在货币的外衣之中,而是看做产品一般,看做劳动的一般成果了。这种产品还与活动的局限性相应而仍然被看做自然规定的产品-农业的产品,主要还是土地的产品。

亚当.斯密大大地前进了一步,他抛开了创造财富的活动的一切规定性,-干脆就是劳动,既不是工业劳动,又不是商业劳动,也不是农业劳动,而既是这种劳动,又是那种劳动,有了创造财富的活动的抽象一般性,也就有了被规定为财富的对象的一般性,这就是产品一般,或者说又是劳动一般,然而是作为过去的,物化的劳动。这一步跨得多么艰难,多么远,只要看看连亚当.斯密本人还时时要回到重农学派的观点上去,就可想见了。这会造成一种看法,好象由此只是替人-不论在哪种社会形式下-做为生产者在其中出现的那种最简单,最原始的关系找到了一个抽象表现。从这一方面来看这是对的,从另一方面看来就不是这样。
  对任何种类劳动的同样看待,以一个十分发达的实在劳动种类的总体为前提,在这些劳动种类中,任何一种劳动都不再是支配一切的劳动。所以,最一般的抽象只产生在最丰富的具体的发展的地方,在那里,一种东西为许多东西所共有,为一切所共有。这样一来,它就不再只是再特殊形式上才能加以思考了。另一方面,劳动一般这个抽象,不仅仅是具体的劳动总体的精神结果。对任何种类的劳动的同样看待,适合于这样一种社会形式,在这种社会形式中,个人很容易从一种劳动转到另一种劳动,一定种类的劳动对他们来说是偶然的,因而是无差别的。这里,劳动不仅在范畴上,而且在现实中都是创造财富一般的手段,它不再是在一种特殊性上同个人结合在一起的规定了。在资产阶级社会的最现代的存在形式-美国,这种情况最为发达。所以,在这里,”劳动”,”劳动一般”,直截了当的劳动这个范畴的抽象,这个现代经济学的起点,才成为实际真实的东西。人们也许会说,在美国表现为历史产物的东西-对任何劳动同样看待-在俄罗斯人那里,比如说,就表现为天生的素质了。但是,首先,是野蛮人具有适应一切的素质还是文明人自动去适应一切,是大有区别的。并且,在俄罗斯人那里,实际上同对任何种类劳动同样看待这一点相适应的,是传统地固定在一种十分确定的劳动上的状态,他们只是由于外来的影响才从这种状态中解放出来。

劳动这个例子确切地表明,哪怕是最抽象的范畴,虽然正是由于它们的抽象而适用于一切时代,但是就这个抽象的规定性本身来说,同样是历史关系的产物,而且只有对这于些关系并在这些关系之内才具有充份的意义。

资产阶级社会是历史上最发达的和最复杂的生产组织。因此,那些表现它的各种关系的范畴以及对于它的结构的理解,同时也能使我们透视一切已经覆灭的社会形式的结构和生产关系。资产阶级借这些社会形式的残片和因素建立起来,其中一部分是还未克服的遗物,继续在这里存留着,一部分原来只是征兆的东西,发展到具有充份意义,等等。人体解剖对于猴类解剖是一把钥匙。低等动物身上表露的高等动物的征兆,反而只有在高等动物本身已被认识之后才能理解。因此,资产阶级经济为古代经济等等提供了钥匙,但是,决不是像那些抹杀一切历史差别,把一切社会形式都看成资产阶级社会形式的经济学家所理解的那样。人们认识了地租,什一税等等。但是不应当把它们等同起来。

其次,因为资产阶级社会本身只是发展的一种对抗的形式,所以,那些早期形式的各种关系,在它里面常常只以十分萎缩的或者漫画式的形式出现。公社所有制就是个例子。因此,如果说资产阶级经济的范畴包含着一种适用于一切其它社会形式的真理这种说法是对的,那末,这也只能在一定意义上来理解。这些范畴可以在发展了的,萎缩的了,漫画式的种种形式上,然而总是在有本质区别的形式上,包含着这些社会形式。所谓的历史发展总是建立在这样的基础上的:最后的形式总是把过去的形式看成是向着自己发展的各个阶段,并且因为它很少而且只是在特定条件下才能够进行自我批判,-这里当然不是指做为崩溃时期出现的那样的历史时期,-所以总是对过去的形式做片面的理解。基督教只有在它的自我批判在一定程度上,所谓在可能范围内准备好时,才有助于对早期神话作客观的理解。同样,资产阶级经济只有在资产阶级社会的自我批判已经开始时,才能理解封建社会,古代社会和东方社会.在资产阶级经济没有把自己神话化而同过去完全等同起来时。它对于前一个社会,即它还得与之直接斗争的封建社会的批判,是与基督教对异教的批判或者新教对旧教的批判相似的。

在研究经济范畴的发展时,正如在研究任何历史科学,社会科学时一样,应当时刻把握住:无论在现实中或在头脑中,主体-这里是现代资产阶级社会-都是既与的;因而范畴表现这一定社会的,这个主体的存在形式,存在规定,常常只是个别的侧面;因此,这个一定社会在科学上也决不是把它当作这样一个社会来谈论的时候才开始存在的。这必须把握住,因为这对于分篇直接具有决定的意义。例如,从地租开始,从土地所有制开始,似乎是再自然不过的,因为它是同土地结合着的,而土地是一切生产的源泉,并且它又是同农业结合着的,而农业是一切多少固定的社会的最初的生产方式。但是,这是最错误不过的了。在一切社会形式中都有一种一定的生产支配着其它一切生产的地位和影响。这是一种普照的光,一切其它色彩都隐没其中,它使它们的特点变了样。这是一种特殊的以太,它决定着它里面显露出来的一切存在的比重。以畜牧民族为例(纯粹的渔猎民族还处于真正发展的起点之外)。在他们中间出现一定形式的,即偶然的耕作。土地所有制由此决定了。它是公有的,这种形式依这些民族保持传统的多少而或多或少地遗留下来,斯拉夫人中的公社所有制就是个例子。而在从事定居耕作-这种定居已是一大进步-的民族那里,像在古代社会和封建社会,耕作处于支配地位,那里连工业,工业的组织以及与工业相应的所有制形式都或多或少带着土地所有制的性质;或者像在古代罗马人中那样工业完全附属于耕作;或者像中世纪那样工业在城市中和在城市的各种关系上摹仿着乡村的组织。在中世纪,甚至资本-只要不是纯粹的货币资本-做为传统的手工工具等等,也带着这种土地所有制的性质。

在资产阶级社会中情况则相反。农业越来越变成仅仅是一个工业部门,完全由资本支配。地租也是如此。在土地所有制居于支配地位的一切社会形式中,自然联系还占优势。在资本居于支配地位的社会形式中,社会,历史所创造的因素占优势。不懂资本便不能懂地租。不懂地租却完全可以懂资本。资本是资产阶级社会的支配一切的经济权力。它必须成为起点又成为终点,必须放在土地所有制之前来说明。分别考察了两者之后,必须考察它们的相互关系。

因此,把经济范畴按它们在历史上起作用的先后次序来安排是不行的,错误的。它们的次序倒是由他们在现代资产阶级社会中的相互关系决定的,这种关系同看来是它们的合乎自然次序或者符合历史发展次序的东西恰好相反。问题不在于各种经济关系在不同社会形式的相继更替的序列中在历史上占有什么地位,更不在于它们在”观念上”(蒲鲁东)(在历史运动的一个模糊表象中)的次序。而在于它们在现代资产阶级社会内部的结构。
古代世界中的商业民族-腓尼基人,迦太基人-表现的单纯性(抽象规定性);正是由农业民族占优势这种情况决定的。做为商业资本和货币资本的资本,在资本还没有成为社会的支配因素的地方,正是在这种抽象中表现出来。伦巴第人和犹太人对于经营农业的中世纪社会,也是处于这种地位。

还有一个例子,说明同一些范畴在不同的社会阶段有不同的地位,这就是资产阶级社会的最新形式之一:股份公司。但是,它还在资产阶级社会初期就曾以特权的,有垄断权的大公司的形式出现。

国民财富这个概念,在十七世纪经济学家看来,无形中是说财富的创造仅仅是为了国家,而国家的实力是与这种财富成比例的,-这种观念在十八世纪的经济学家中还部份地保留着。这是一种不自觉的伪善形式,在这种形式下财富本身和财富的生产被宣布为现代国家的目的,而现代国家被看成只是生产财富的手段。

显然,应当这样来分篇:

(1)一般的抽象的规定,因此它们或多或少属于一切社会形式,不过是在上面所分析过的意义上。

(2)形成资产阶级社会内部结构并且成为基本阶级的依据的范畴。资本,雇佣劳动,土地所有制。它们相互之间的关系。城市和乡村。三大社会阶级。它们之间的交换。流通。信用事业(私的)。

(3)资产阶级社会在国家形式上的概括。就它本身来考察。”非生产”阶级。税。国债。公的信用。人口。殖民地。向外国移民。

(4)生产的国际关系。国际分工,国际交换。输出和输入。汇率。

(5)世界市场和危机。

\subsubsection{4~生产、生产资料和生产关系。生产关系和交往关系。国家形式和意识形式同生产关系和交往关系的关系。法的关系,家庭关系。}

注意:应该在这里提到而不该忘记的各点:

(1)战争比和平发达的早;某些经济关系,如雇佣劳动,机器等等,怎样在战争和军队等等中比在资产阶级社会内部发展的早。生产力和交往关系的关系在军队中也特别显着。

(2)历来的观念的历史编纂法同现实的历史编纂法的关系。特别是所谓文化史,旧时的宗教使和政治史。(顺便也可以说一下历来的历史编纂法的各种不同方式。所谓客观的,主观的(伦理的等等)。哲学的。)

(3)第二级的和第三级的东西,总之,派生的,转移来的,非原生的生产关系。国际关系在这里的影响。

(4)对这种见解中的唯物主义的种种非难;同自然唯物主义的关系。

(5)生产力(生产资料)的概念和生产关系的概念的辨证法,这样一种辨证法,它的界限应当确定,它不抹杀现实差别。

(6)物质生产的发展例如同艺术生产的不平衡关系。进步这个概念决不能在通常的抽象意义上去理解。现代艺术等等。这种不平衡在理解上还不是像在实际社会关系本身内部那样如此重要和如此困难。例如教育。美国同欧洲的关系。可是,这里要说明的真正困难之点是:生产关系作为法的关系怎样进入了不平衡的发展。例如罗马私法(在刑法和公法中这种情形较少)同现代生产的关系。

(7)这种见解表现为必然的发展。但承认偶然。怎样。(对自由等也是如此。)(交通工具的影响。世界史不是过去一直存在的;作为世界史的历史是结果。)

(8)出发点当然是自然规定性;主观地和客观地。部落,种族等。

关于艺术,大家知道,它的一定繁盛时期决不是同社会的一般发展成比例的,因而也决不是同彷佛是社会组织的骨骼的物质基础的一般发展成比例的。例如,拿希腊人或莎士比亚同现代人相比。就某些艺术形式,例如史诗来说,甚至谁都承认:当艺术生产一旦作为艺术生产出现,他们就再不能以那种在世界史上画时代的,古典的形式创造出来;因此,在艺术本身的领域内,某些有重大意义的艺术形式只有在艺术发展的不发达阶段上才是可能的。如果说在艺术本身的领域内部的不同艺术种类的关系中有这种情形,那末,在整个艺术领域同社会一般发展的关系上有这情形,就不足为奇了。困难只在于对于这些矛盾作一般的表述。一旦它们的特殊性被确定了,它们也就被解释明白了。

我们先拿希腊艺术同现代的关系作例子,然后再说莎士比亚同现代的关系。大家知道,希腊神话不只是希腊艺术的武库,而且是它的土壤。成为希腊人的幻想的基础,从而成为希腊[神话]的基础的那种对自然的观点和对社会关系的观点,能够同自动纺机,铁道,机车和电报并存吗?在罗伯茨公司面前,武尔坎又在哪里?在避雷针面前,邱必特又在哪里?在动产信用公司面前,海尔梅斯又在哪里?任何神话都是用想象和借助想象以征服自然力,支配自然力,把自然力加以形象化;因而,随着这些自然力之实际上被支配,神话也就消失了。在印刷所广场旁边,法玛还成什么?希腊艺术的前提是希腊神话,也就是已经通过人民的幻想用一种不自觉的艺术方式加工过的自然和社会形式本身。这是希腊艺术的素材。不是随便一种神话,就是说,不是对自然(这里指一切对象,包括社会在内)的随便一种不自觉的艺术加工。埃及神话决不能成为希腊艺术的土壤和母胎。但是无论如何总得是一种神话。因此,决不是这样一种社会发展,这种发展排斥一切神话地对待自然的态度和一切把自然神话化的态度;并因而要求艺术家具备一种与神话无关的幻想。

从另一方面看:阿基利斯能同火药和弹丸并存吗?或者,《伊利亚特》能够同活字盘甚至印刷机并存吗?随着印刷机的出现,歌谣,传说和诗神谬斯岂不是必然要绝迹,因而史诗的必要条件岂不是要消失吗?

但是,困难不在于理解希腊艺术和史诗同一定社会发展形式结合在一起。困难的是,他们何以仍然能够给我们以艺术享受,而且就某方面说还是一种规范和高不可及的范本。

一个成人不能再变成儿童,否则就变得稚气了。但是,儿童的天真不使它感到愉快吗?他自己不该努力在一个更高的阶梯上把自己的真实再现出来吗?在每一个时代,它的固有的性格不是在儿童的天性中纯真地复活着吗?为什么历史上的人类童年时代,在它发展的最完美的地方,不该作为永不复返的阶段而显示出永久的魅力呢?有粗野的儿童,有早熟的儿童。古代民族中有许多是属于这一类的。希腊人是正常的儿童。他们的艺术对我们所产生的魅力,同它在其中生长的那个不发达的社会并不矛盾。它倒是这个社会阶段的结果,并且是同它在其中产生而且只能在其中产生的那些未成熟的社会条件永远不能复返这一点分不开的。
\newpage
\subsection{2.笔记}
马克思在导言的前半部分着重讨论了\textbf{生产}与\textbf{生产一般}。马克思指出,\textbf{生产一般}是一个抽象,将生产在不同时代的共同点提取出来的\textbf{合理的}抽象。就这一抽象而言,它是多样性的统一,笔者认为这种抽象是马克思政治经济学研究的哲学起点,或至少是一个必要的前提。为了让读者们更加清晰地体会马克思所进行的这种抽象的必要性,笔者在这里引用一段马克思的原文:

\begin{fangsong}
    “生产一般是一个抽象,但是只要它真正把共同点提出来,定下来,免得我们重复,它就是一个合理的抽象。不过,这个一般,或者说,经过比较而抽出来的共同点,本身就是有许多组成部分的、分为不同规定的东西。其中有些属于一切时代,另一些是几个时代共有的。[有些]规定是最新时代和最古时代共有的。没有它们,任何生产都无从设想;但是如果说最发达的语言和最不发达的语言共同具有一些规律和规定,那么,构成语言发展的恰恰是有别于这个一般和共同点的差别。对生产一般适用的种种规定所以要抽出来,也正是为了不致因为有了统一(主体是人,客体是自然,这总是一样的,这里已经出现了统一)而忘记本质的差别。那些证明现存社会关系永存与和谐的现代经济学家的全部智慧,就在于忘记这种差别。”
\end{fangsong}

在此我们可以看出,马克思从生产一般出发,并不是意味着马克思从普遍性原则出发。因为在马克思看来,普遍性并不是推动事物发展的原因,而否定性、差异性才是原因。\textbf{如果仅仅从一般的抽象(普遍性)中考察社会的发展,那末社会便是不发展的,以至于将社会的发展归结于偶然性因素。}可见,若是从普遍性原则出发去试图解释社会历史的运动,往往会得到充满偶然性的结论。事实上,对这种差异性的统一的理解是非常困难的,为了让读者能够体会到马克思的用意(或至少能够体会到笔者所认为的马克思的用意),在这里笔者想举一个例子。对于一个人,从生物学角度上而言,他是一定会死去的,如果不出意外的话,他的身体一定会经历不同的发展阶段。如果我们对青年时期的那个人进行检查,我们会发现他的心脏很健康,但是我们不能由此就断言他的心脏会一直健康下去,因为从宏观上来说,他的机体是在不断变化的,当他的机体处在老年时期,他的心脏或许就不健康了,因此“健康”对于“心脏”而言,并不是永恒的,而是一个具有历史性的规定。这样的类比同样适用于国民经济学,“资本作为客体化的劳动”这一规定性,仅仅是历史发展的一个阶段\footnote{换句话说,“客体化的劳动”不一定是资本,但在资本主义社会,它的确\textbf{往往}是作为资本而存在的。},否则全部的社会历史都仅是“资本”这一范畴的自我展开过程,这显然是与历史发展的事实相违背的。\footnote{就资本而言,它是一种生产关系,是产生于特定的历史条件的,这种生产关系的历史本身不能被看作人类历史的全部。}

马克思随后论述了生产与消费、分配、流通和交换之间的关系。首先,马克思认为生产与消费是统一的,从消费方面来看,消费使生产得以实现,即使生产出的产品得以成为其自身,同时消费还为生产提供目的与需要,为生产提供前提。从生产方面来看,生产为消费提供对象并创造着消费的形式。需要注意的是,马克思还指出,\textbf{生产不仅为主体生产对象,而且也为对象生产主体。}笔者认为这一点至关重要\footnote{笔者认为部分研究者忽略了这一点,这一点可以和张一兵教授的“我生产故历史在”的命题相联系起来,并可以用此来解释马克思早期的异化理论,其中蕴含着一种很微妙的关系,请读者务必仔细思考我下面的论述。},这里充分体现了马克思的唯物主义哲学思想——即\textbf{对象性活动}的原则。就“生产为主体生产对象”这前半句而言,大家应该是很好理解的,生产为消费主体创造出消费的对象——即生产为\textbf{人}创造\textbf{物}。但是应如何理解马克思的下半句“生产也为对象生产主体”呢?马克思在他的原文中举了一个具象化例子,他是这么说的:

\begin{fangsong}
“消费对于对象所感到的需要,是对于对象的知觉所创造的。 艺术对象创造出懂得艺术和具有审美能力的大众,——任何其他产品也都是这样。”
\end{fangsong}

从中我们可以体会到马克思想表达的一点是,即生产不仅能够创造客体,同时它也在创造着主体。但让我们继续深入反思一下,“生产”这一范畴本身究竟是以谁为主体?换句话说,“生产”是“谁的生产”?事实上,我们完全有理由将“生产”理解为\textbf{“主体的生产”}\footnote{这里的“主体”是“生产”的定语,而不是“生产”的谓语。},理解为\textbf{“人的对象性的活动”}。因此,在这个意义上而言,马克思所认为的\textbf{“生产为对象生产主体”}这一规定便可以转译为\textbf{“主体生产着主体”}换句话说,\textbf{“主体生产着自身”}。

而“主体生产着主体”又意味着什么?事实上,笔者认为这里体现了\textbf{历史唯物主义}的精髓。“主体生产着主体”意味着\textbf{人}的自我创造,也即是说,在马克思看来,\textbf{人}并不是先在的存在物,因而\textbf{人的历史}同样不是\textbf{先在的、既定的}存在物,人和他的历史是通过对象性活动创造(建构)的。因此就这一点来说,历史既不是既定的必然性的展开过程,也不是纯粹的偶然性的叠加,历史是一种需要被补充的状态,换句话说,历史是\textbf{需要}被建构的,而建构他的材料则是人的对象性活动——广义的生产。在弄清了这一原则之后,让我们再次返回到马克思在《1844年经济学哲学手稿》中异化劳动理论,人的异化是谁造成的?答案是人自身,因此马克思会提到诸如“共产主义”是“人的本质的真正占有”、“人向人自身的复归”等的词句。但是,所谓的“人的本质的占有”、“人的复归”等的词句仅仅是在哲学的高度对破除异化之后的应然状态的概括,因此,就实际行动而言,问题并不在于如何关注人的本质,问题在于改变现存的生产关系,但似乎一些西方马克思主义学者们并没有意识到这一点\footnote{笔者认为,西马是对马克思异化理论的深入探索,他们走了马克思曾经想走但后来转而批判的路径,在之后我会系统地谈论一些西马学者的观点。}。

\newpage

\section{《政治经济学批判》序言}
\subsection{1.原文}
我考察资产阶级经济制度是按照以下的顺序:\textbf{资本、土地所有制、雇佣劳动;国家、对外贸易、世界市场}。在前三项下,我研究现代资产阶级社会分成的三大阶级的经济生活条件;其他三项的相互联系是一目了然的。第一册论述资本,其第一篇由下列各章组成:(1)商品,(2)货币或简单流通,(3)资本一般。前两章构成本分册的内容。我面前的全部材料形式上都是专题论文,它们是在相隔很久的几个时期内写成的,目的不是为了付印,而是为了自己弄清问题,至于能否按照上述计划对它们进行系统整理,就要看环境如何了。

我把已经起草好的一篇总的导言压下了,因为仔细想来,我觉得顸先说出正要证明的结论总是有妨害的,读者如果真想跟着我走,就要下定决心,从个别上升到一般。不过在这里倒不妨谈一下我自己研究政治经济学的经过。

我学的专业本来是法律,但我只是把它排在哲学和历史之次当作辅助学科来研究。1842—1843年间,我作为《莱茵报》的编辑,第一次遇到要对所谓物质利益发表意见的难事。莱茵省议会关于林木盗窃和地产析分的讨论,当时的莱茵省总督冯·沙培尔先生就摩泽尔农民状况同《菜茵报》展开的官方论战,最后,关于自由贸易和保护关税的辩论,是促使我去研究经济问题的最初动因。另一方面,在善良的“前进”愿望大大超过实际知识的当时,在《莱茵报》上可以听到法国社会主义和共产主义的带着微弱哲学色彩的回声。我曾表示反对这种肤浅言论,但是同时在和奥格斯堡《总汇报》的一次争论中坦率承认,我以往的研究还不容许我对法兰西思潮的内容本身妄加许判。我倒非常乐意利用《莱茵报》发行人以为把报纸的态度放温和些就可以使那已经落在该报头上的死刑判决撤销的幻想,以便从社会舞台退回书房。

为了解决使我苦恼的疑问,我写的第一部著作是对黑格尔法哲学的批判性的分析,这部著作的导言曾发表在1844年巴黎出版的《德法年鉴》上。我的研究得出这样一个结果:法的关系正像国家的形式一样,既不能从它们本身来理解,也不能从所谓人类精神的一般发展来理解,相反,它们根源于物质的生活关系,这种物质的生活关系的总和,黑格尔按照18世纪的英国人和法国人的先例,概括为“市民社会”,而对市民社会的解剖应该到政治经济学中去寻求。我在巴黎开始研究政治经济学。后来因基佐先生下令驱逐移居布鲁塞尔,在那里继续进行研究。我所得到的、并且一经得到就用于指导我的研究工作的总的结果,可以简要地表述如下:人们在自己生活的社会生产中发生一定的、必然的、不以他们的意志为转移的关系,即同他们的物质生产力的一定发展阶段相适合的生产关系。这些生产关系的总和构成社会的经济结构,即有法律的和政治的上层建筑竖立其上并有一定的社会意识形式与之相适应的现实基础。物质生活的生产方式制约着整个社会生活、政治生活和精神生活的过程。不是人们的意识决定人们的存在,相反,是人们的社会存在决定人们的意识。社会的物质生产力发展到一定阶段,便同它们一直在其中运动的现存生产关系或财产关系(这只是生产关系的法律用语)发生矛盾。于是这些关系便由生产力的发展形式变成生产力的桎梏。那时社会革命的时代就到来了。随着经济基础的变更,全部庞大的上层建筑也或慢或快地发生变革。在考察这些变革时,必须时刻把下面两者区别开来:一种是生产的经济条件方面所发生的物质的、可以用自然科学的精确性指明的变革,一种是人们借以意识到这个冲突并力求把它克服的那些法律的、政治的、宗教的、艺术的或哲学的,简言之,意识形态的形式。我们判断一个人不能以他对自己的看法为根据,同样,我们判断这样一个变革时代也不能以它的意识为根据;相反,这个意识必须从物质生活的矛盾中,从社会生产力和生产关系之间的现存冲突中去解样。无论哪一个社会形态,在它所能容纳的全部生产力发挥出来以前,是决不会灭亡的:而新的更高的生产关系,在它的物质存在条件在旧社会的胎胞里成熟以前,是决不会出现的。所以人类始终只提出自已能解决的任务,因为只要仔细考察就可以发现,任务本身,只有在解决它的物质条件已经存在或者至少是在生成过程中的时候,才会产生。大体说来、亚细亚的、古代的、封建的和现代资产阶级的生产方式可以看做是经济的社会形态演进的几个时代。资产阶级的生产关系是社会生产过程的最后一个对抗形式,这里所说的对抗,不是指个人的对抗,而是指从个人的社会生活条件中生长出来的对抗;但是,在资产阶级社会的胎胞里发展的生产力,同时又创造着解决这种对抗的物质条件。因此,人类社会的史前时期就以这种社会形态而告终。

自从弗里德里希·恩格斯批判经济学范畴的天才大纲\footnote{指恩格斯的《国民经济学批判大纲》。}(在《德法年鉴》上)发表以后,我同他不断通讯交换意见,他从另一条道路(参看他的《英国工人阶级状况》)得出同我一样的结果,当1845年春他也住在布鲁塞尔时,我们决定共同阐明我们的见解与德国哲学的意识形态的见解的对立,实际上是把我们从前的哲学信仰清算一下。这个心愿是以批判黑格尔以后的哲学的形式来实现的。两厚册八开本的原稿\footnote{指马克思和恩格斯的《德意志意识形态》。}早已送到威斯特伐里亚的出版所,后来我们才接到通知说,由于情况改变,不能付印。既然我们已经达到了我们的主要目的——自己弄清问题,我们就情愿让原稿留给老鼠的牙齿去批判了。在我们当时从这方面或那方面向公众表达我们见解的各种著作中,我只提出恩格斯与我合著的《共产党宣言》和我自己发表的《关于自由贸易的演说》。我们见解中有决定意义的论点,在我的1847年出版的为反对蒲鲁东而写的著作《哲学的贫困》中第一次作了科学的、虽然只是论战性的概述。我用德文写的关于《雇佣劳动》\footnote{即《雇佣劳动与资本》。}一书,汇集了我在布鲁塞尔德意志工人协会上对于这个问题的讲演,这本书的印刷由于二月革命和我因此被迫离开比利时而中断。

1848年和1849年《新莱茵报》的出版以及随后发生的一些事变,打断了我的经济研究工作,到1850年我在伦敦才能重新进行这一工作。不列颠博物馆中堆积看政治经济学史的大量资料,伦敦对于考察资产阶级社会是一个方便的地点,最后,随着加利福尼亚和澳大利亚金矿的发现,资产阶级社会看来进人了新的发展阶段,这一切决定我再从头开始,批判地仔细钻研新的材料。这些研究一部分自然要涉及似乎完全属于本题之外的学科,在这方面不得不多少费些时间。但是使我所能够支配的时间特别受到限制的,是谋生的迫切需要。八年来,我一直为第一流英文的美国报纸《纽约每日论坛报》撰稿(写作真正的报纸通讯在我只是例外),这使我的研究工作必然时时间断。然而,由于评论英国和大陆突出经济事件的论文在我的投稿中占着很大部分,我不得不去熟悉政治经济科学本身范围以外的实际的细节。

我以上简短地叙述了自己在政治经济学领域进行研究的经过,这只是要证明,我的见解,不管人们对它怎样评论,不管它多么不合乎统治阶级的自私的偏见,却是多年诚实研究的结果。但是在科学的入口处,正像在地狱的入口处一样,必须提出这样的要求:

\begin{fangsong}
“这里必须根绝一切犹豫;

这里任何怯懦都无济于事。”\footnote{但丁《神曲·地狱篇》第3部第14-15行。} 
\end{fangsong}
\newpage
\subsection{2.笔记}
马克思在序言中指出,他是按照\textbf{“资本、土地所有制、雇佣劳动;国家、对外贸易、世界市场”}的顺序去考察资本主义经济制度的。前三项对应着现代资本主义社会三大阶级(资产阶级、农民阶级、工人阶级)的经济生活条件。马克思介绍了他早年的研究历程,在这里我们可以看出,1844年是马克思的一个转折点,马克思在其著作《黑格尔法哲学批判<导言>》中得出了\textbf{“法的关系根源于物质的生活关系”}的结论,并认为\textbf{“对市民社会的解剖应该到政治经济学中去寻求”}。很显然,这是马克思的思想由唯心主义转向唯物主义的一个标志。

此外,在序言中,马克思还提出了“两个绝不会”的命题。关于这个命题笔者无需过多解释,马克思在此想表达的内容是很清晰明了的。关于这一命题背后所蕴含的思想,已经成为了众多马克思主义理论家在分析政治问题时所坚持的一个基本原则。

最后,关于马克思在序言中所提到的社会形态演进的几个阶段,以及马克思所认为“资产阶级社会是社会生产过程的最后一个对抗形式”这些内容,笔者认为需要去深入地去研究其中的具体细节。也即是说,从马克思的论述中蕴含着资本主义必然灭亡的观点,但对于资本主义的批判并不能以此为依据,应从一定历史阶段的特定的具体的矛盾中去考察资本主义制度本身,正如优秀的医生面对一个罹患癌症而死去的病人,能够从这个病人的机体内部去分析他的死因,而不是从藏传佛教的教义\footnote{藏传佛教认为,生与死是轮回的,死亡是生命的一部分。}中寻找对“死亡”的解释。

\newpage
\section{雇佣劳动与资本}

\subsection{1.原文节选}
资本包括原料、劳动工具和各种生活资料,这些东西是用以生产新的原料、新的劳动工具和新的生活资料的。资本的所有这些组成部分都是劳动的创造物,劳动的产品,积累起来的劳动。作为进行新生产的手段的积累起来的劳动就是资本。

经济学家们就是这样说的。

什么是黑奴呢?黑奴就是黑种人。上面的说明和这个说明是一样的。

黑人就是黑人。只有在一定的关系下,他才成为奴隶。纺纱机是纺棉花的机器。只有在一定的关系下,它才成为资本。脱离了这种关系,它也就不是资本了,就像黄金本身并不是货币,沙糖并不是沙糖的价格一样。

人们在生产中不仅仅同自然界发生关系\footnote{在1891年的版本中,“不仅仅与自然界发生关系”改为“不仅仅影响自然界,而且也互相影响”。——编者注}。他们如果不以一定方式结合起来共同活动和互相交换其活动,便不能进行生产。为了进行生产,人们便发生一定的联系和关系;只有在这些社会联系和社会关系的范围内,才会有他们对自然界的关系,\footnote{在1891年的版本中,“对自然界的关系”改为“对自然界的影响”。——编者注}才会有生产。

生产者相互发生的这些社会关系,他们借以互相交换其活动和参于共同生产的条件,当然依照生产资料的性质而有所不同。随着新作战工具即射击火器的发明,军队的整个内部组织就必然改变了,各个人借以组成军队并能作为军队行动的那些关系就改变了,各个军队相互间的关系也发生了变化。

总之,各个人借以进行生产的社会关系,即社会生产关系,是随着物质生产资料、生产力的变化和发展而变化和改变的。生产关系总合起来就构成为所谓社会关系,构成为所谓社会,并且是构成为一个处于一定历史发展阶段上的社会,具有独特的特征的社会。古代社会、封建社会和资产阶级社会都是这样的生产关系的总和,而其中每一个生产关系的总和同时又标志着人类历史发展中的一个特殊阶段。

资本也是一种社会生产关系。这是资产阶级的生产关系,是资产阶级社会的生产关系。构成资本的生活资料、劳动工具和原料,难道不是在一定的社会条件下,不是在一定的社会关系下生产出来和积累起来的吗?难道这一切不是在一定的社会条件下,在一定的社会关系内被用来进行新生产的吗?并且,难道不正是这种一定的社会性质把那些用来进行新生产的产品变为资本的吗?

资本不仅包括生活资料、劳动工具和原料,不仅包括物质产品。并且还包括交换价值。资本所包括的一切产品都是商品。所以,资本不仅是若干物质产品的总和,并且也是若干商品或若干交换价值或若干社会定量的总和。

不论我们是以棉花代替羊毛也好,是以米代替小麦也好,是以轮船代替铁路也好,只要这些体现资本的棉花、米和轮船同原先体现资本的羊毛、小麦和铁路具有同样的交换价值即同样的价格,那末资本依然还是资本。资本的肉体可以经常改变,但不会使资本性质有丝毫改变。

虽然任何资本都是一些商品即交换价值的总和,然而远不是任何一些商品即交换价值的总和都是资本。

任何一些交换价值的总和都是一个交换价值。任何单个交换价值都是一些交换价值的总和。例如,值一千法郎的一座房子是一千法郎的交换价值。值一生丁[注:在1891年的版本中,“生丁”改为“分尼”。——编者注]的一印张纸是100/100生丁的交换价值的总和。能同别的产品交换的产品就是商品。这些产品由以交换的一定比率就是它们的交换价值,如果这种比率是用货币来表示的,就是它们的价格。这些产品的数量多少丝毫不能改变它们成为商品,或者表现交换价值,或者具有一定价格的这种性能。一株树不论其大小如何,终究是一株树。我们拿铁同别的产品交换时不是以两为单位,而是以公担为单位,难道铁作为商品,作为交换价值的性能竟会因此而改变吗?铁作为一种商品,只是依其数量多少而具有大小不同的价值,高低不同的价格。

一些商品即一些交换价值的总和究竟是怎样成为资本的呢?

它成为资本,是由于它作为一种独立的社会力量,即作为一种属于社会一部分的力量,借交换直接的、活的劳动[注:在1891年的版本中,“劳动”改为“劳动力”。——编者注]而保存下来并增殖起来。除劳动能力以外一无所有的阶级的存在是资本的必要前提。

只是由于积累起来的、过去的、物化的劳动支配直接的、活的劳动,积累起来的劳动才变为资本。

资本的实质并不在于积累起来的劳动是替活劳动充当进行新生产的手段。它的实质在于活劳动是替积累起来的劳动充当保存自己并增加其交换价值的手段。

资本和雇佣劳动\footnote{在1891年的版本中,“资本和雇佣劳动”改为“资本家和雇佣工人”。——编者注}是怎样进行交换的呢?

工人拿自己的劳动\footnote{在1891年的版本中,“劳动”改为“劳动力”。——编者注}换到生活资料,而资本家拿归他所有的生活资料换到劳动,即工人的生产活动,亦即创造力量。这种力量不仅能补偿工人所消费的东西,并且还使积累起来的劳动具有比以前更大的价值。工人从资本家那里得到一部分现有的生活资料。这些生活资料对工人有什么用处呢?用于直接消费。可是,如果我不把靠这些生活资料维持我的生活的一段时间用来生产新的生活资料,即在消费的同时用我的劳动创造新价值来补偿那些因消费而消失了的价值,那末我一把这些生活资料消费完,它们对于我就算是完全白耗费了。但是,工人为了换到生活资料,正是把这种贵重的再生产力量让给了资本家。因此,对于工人本身来说,这种力量是白耗费了。

举一个例子来说吧。有个农场主每天付给他的一个短工五银格罗申。这个短工为得到这五银格罗申,就整天在农场主的田地上干活,保证农场主能得到十银格罗申的收入。农场主不但收回了他付给短工的价值,并且还把它增加了一倍。可见,他有成效地、生产性地使用和消费了他付给短工的五银格罗申。他拿这五银格罗申买到的正是一个短工的能生产出双倍价值的农产品并把五银格罗申变成十银格罗申的劳动和力量。短工则拿他的生产力(他正是把这个生产力让给了农场主)换到五银格罗申,并用它们换得迟早要消费掉的生活资料。所以,这五银格罗申的消费有两种方法:对资本家来说,是有生产性的,因为他用这五银格罗申换来的劳动力使他得到了十银格罗申;对工人来说,是非生产性的,因为他用这五银格罗申换来的生活资料永远消失了,他只有再和农场主进行同样的交换才能重新取得这些生活资料的价值。这样,资本以雇佣劳动为前提,而雇佣劳动又以资本为前提。两者相互制约;两者相互产生。

一个棉纺织厂的工人是不是只生产棉织品呢?不是,他生产资本。他生产重新供人利用去支配他的劳动并借他的劳动创造新价值的价值。

资本只有同劳动[注:在1891年的版本中,“劳动”改为“劳动力”。——编者注]交换,只有引起雇佣劳动的产生,才能增加起来。雇佣劳动[注:在1891年的版本中,“雇佣劳动”改为“雇佣工人的劳动力”。——编者注]只有在它增加资本,使奴役它的那种权力加强时,才能和资本交换。因此,资本的增加就是无产阶级即工人阶级的增加。

所以,资产者及其经济学家们断言,资本家和工人的利益是一致的。千真万确呵!工人若不受雇于资本家就会灭亡。资本若不剥削劳动[注:在1891年的版本中,“劳动”改为“劳动力”。——编者注]就会灭亡,而要剥削劳动[注:在1891年的版本中,“劳动”改为“劳动力”。——编者注],资本就得购买劳动[注:在1891年的版本中,“劳动”改为“劳动力”。——编者注]。投入生产的资本即生产资本增殖愈快,也就是说,产业愈繁荣,资产阶级愈发财,生意愈兴隆,资本家需要的工人也就愈多,工人出卖自己的价格也就愈高。

原来,生产资本的尽快增加竟是工人能勉强过活的必要条件。

但是,生产资本的增加又是什么意思呢?就是积累起来的劳动对活劳动的支配权力的增加,就是资产阶级对工人阶级的统治力量的增加。雇佣劳动生产着对它起支配作用的他人财富,也就是说生产着同它敌对的力量——资本,而它从资本那里取得就业手段,即取得生活资料,是以雇佣劳动又会变成资本的一部分,又会变成使资本加速增殖的杠杆为条件的。

断言资本的利益和劳动的利益[注:在1891年的版本中,“劳动的利益”改为“工人的利益”。——编者注]是一致的,事实上不过是说资本和雇佣劳动是同一种关系的两个方面罢了。一个方面制约着另一个方面,就如同高利贷者和挥霍者相互依存一样。

当雇佣工人仍然是雇佣工人的时候,他的命运是取决于资本的。所谓工人和资本家的利益一致就是这么一回事。

\textbf{(中间部分略过)} 

然而,是不是像资产阶级的经济学家们所说的那样,生产资本的增加真的和工资的提高密不可分呢?我们不应当听信他们的话。我们甚至于不能相信他们的这种说法:似乎资本长得越肥,它的奴隶也吃得越好。资产阶级太开明了,太会打算了,它没有封建主的那种以奴仆的衣着华丽夸耀于人的偏见。资产阶级的生存条件迫使它锱铢必较。

因此我们就应当更仔细地研究一个问题:

生产资本的增长是怎样影响工资的?

如果资产阶级社会的生产资本整个说来是在不断增长,那末劳动的积累就是更多方面的了。资本的数目和资本的数额\footnote{在1891年的版本中,“资本的数目和资本的数额”改为“资本家的数目和他们的资本的数额”。——编者注}日益增加。资本的增殖加剧资本家之间的竞争。资本数额的增加,就使得有可能把装备着火力更猛烈的斗争武器的更强大的工人大军抛入产业战场。

一个资本家只有在自己更便宜地出卖商品的情况下,才能把另一个资本家逐出战场,并占有他的资本。可是,要能够贱卖而又不破产,他就必须廉价生产,就是说,必须尽量增加劳动的生产力。而增加劳动的生产力的首要办法是更细地分工,更全面地运用和经常地改进机器。内部实行分工的工人大军愈庞大,应用机器的规模愈广大,生产费用相对地就愈迅速缩减,劳动就更有效率。因此,资本家之间就发生了各方面的竞争:他们竭力设法扩大分工和增加机器,并尽可能大规模地使用机器。

可是,假如某一个资本家由于更细地分工、更多地采用新机器并改进新机器,由于更有利和更广泛地利用自然力,因而有可能用同样多的劳动或积累起来的劳动生产出比他的竞争者更多的产品(即商品),比如说,在同一劳动时间内,他的竞争者只能织出半尺麻布,他却能织出一尺麻布,那末他会怎样办呢?

他可以继续按照原来的市场价格出卖每半尺麻布,但是这样他就不能把自己的敌人逐出战场,就不能扩大自己的销路。可是随着他的生产的扩大,他对销路的需要也增加了。固然,他所采用的这些更有力更贵重的生产资料使他能够廉价出卖商品,但是这种生产资料又使他不得不出卖更多的商品,为自己的商品争夺更大得多的市场。因此,这个资本家出卖半尺麻布的价格就要比他的竞争者便宜些。

虽然这个资本家生产一尺麻布的费用并不比他的竞争者生产半尺麻布的费用多,但是他不会以他的竞争者出卖半尺麻布的价格来出卖一尺麻布。不然他就占不到任何便宜,而只是通过交换把自己的生产费用收回罢了。如果他的收入终究还是增加了,那只是因为他动用了更多的资本,而不是因为他的资本比别人的资本更多地增加了自己的价值。而且只要他把他的商品价格定得比他的竞争者低百分之几,他追求的目的也就达到了。他压低价格就能把他的竞争者挤出市场,或者至少也能夺取他的竞争者的一部分销路。最后,我们再提一下,现时价格总是高于或低于生产费用,这取决于该种商品是在产业的旺季出卖的还是在淡季出卖的。一个采用了生产效能更高的新生产资料的资本家所能得到的超出他的实际生产费用的百分率,是依每尺麻布的市场价格高于或低于迄今的一般生产费用为转移的。

可是这个资本家的特权不会长久,因为同他竞争的资本家也会采用同样的机器,实行同样的分工,并以同样的或更大的规模采用这些机器的分工。这些新措施将得到广泛的推广,直到麻布价格不仅跌到原先的生产费用以下,而且跌到新的生产费用以下为止。

这样,资本家的相互关系又会像采用新生产资料以前那样了;如果说他们由于采用这种生产资料曾经能够用以前的价格供给加倍的产品,那末现在他们已不得不按低于以前的价格出卖加倍的产品了。在这种新生产费用的水平上,同样一场钩心斗角的斗争又重新开始。又有人实行更细的分工,又有人增加机器数量,利用这种分工的范围和采用这些机器的规模日益扩大。而竞争又对这个结果发生反作用。

由此可见,生产方式和生产资料总在不断变更,不断革命化;分工必然要引起更进一步的分工;机器的采用必然要引起机器的更广泛的采用;大规模的生产必然要引起更大规模的生产。

这是一个规律,这个规律一次又一次地把资产阶级的生产甩出原先的轨道,并迫使资本加强劳动的生产力,因为它以前就加强过劳动的生产力;这个规律不让资本有片刻的停息,老是在它耳边催促说:前进!前进!

这个规律正就是那个在商业的周期性波动中必然使商品价格和商品生产费用趋于一致的规律。

不管一个资本家运用了效率多么高的生产资料,竞争总使这种生产资料的采用成为普遍的现象,而当这种生产资料的采用一旦成为普遍的现象时,他的资本具有更大效率的唯一后果就只能是:要取得原有的价格,他就必须供给比以前多十倍、二十倍、一百倍的商品。可是,因为现在他必须售出也许比以前多一千倍的商品,才能靠增加所售产品数量的办法来弥补由于售价降低所受的损失;因为他现在必须卖出更多的商品不仅是为了得到利润\footnote{在1891年的版本中,“得到利润”改为“得到更多的利润”。——编者注},并且也是为了抵补生产费用(我们已经说过,生产工具本身也日益昂贵);因为此时这种大量出卖不仅对于他而且对于他的竞争对方都成了生死问题,所以先前的斗争就因已经发明的生产资料的生产效率愈大而愈残酷无情地激烈起来。所以,分工和机器的采用又将以更大得无比的规模发展起来。

不管已被采用的生产资料的力量多么强大,竞争总是要把资本从这种强大力量中得到的黄金果实夺去,使商品的价格降低到生产费用的水平;也就是说,只要有可能更廉价的生产,即有可能用同一数量的劳动生产更多的产品,竞争就使廉价生产即按原先价格供给日益增多的产品数量成为确定不移的规律。可见,资本家努力的结果,除了必须在同一劳动时间内生产出更多的商品以外,换句话说,除了使他的资本的价值增殖的条件恶化以外,并没有得到任何好处。因此,虽然竞争经常以其生产费用的规律迫使资本家坐卧不宁,把他制造出来对付竞争者的一切武器倒转来针对着他自己,但资本家总是想方设法在竞争中取胜,孜孜不倦地采用价钱较贵但能进行廉价生产的新机器,实行新分工,以代替旧机器和旧分工,并且不等到竞争使这些新措施过时,就这样做了。

现在我们若是想像一下这种狂热的激发状态同时笼罩了整个世界市场,那我们就会明白,资本增殖、积累和集聚的结果,如何导向了不断地、日新月异地、更大规模地实行分工,采用新机器,改进旧机器。

这些同生产资本的增殖分不开的情况又怎样影响工资的确定呢?

更进一步的分工使一个工人能做五个、十个乃至二十个人的工作,因而就使工人之间的竞争加剧五倍、十倍乃至二十倍。工人中间的竞争不只表现于一个工人把自己出卖得比另一个工人贱些,而且还表现于一个工人做五个、十个乃至二十个人的工作。而资本所实行的和经常扩展的分工就迫使工人进行这种竞争。

其次,分工愈细,劳动就愈简单化。工人的特殊技巧失去任何价值。工人变成了一种简单的、单调的生产力,就不需要体力上或智力上的特别本事和技能了。他的劳动成为人人都能从事的劳动了。因此,工人受到四面八方的排挤;我们还要提醒一下,一种工作愈简单,就愈容易学会,为学会这种工作所需要的生产费用愈少,工资也就愈降低,因为工资像一切商品的价格一样,是由生产费用决定的。

总之,劳动愈是不能给人以乐趣,愈是令人生厌,竞争也就愈激烈,工资也就愈减少。工人想维持自己的工资总额,就得多劳动:多工作几小时或者在一小时内造出更多的产品。这样一来,工人为贫困所迫,就愈加重分工的极危险的后果。结果就是:他工作得愈多,他所得的工资就愈少。这里的原因很简单:他工作得愈多,他给自己的工友们造成的竞争就愈激烈,因而就使自己的工友们变成他自己的竞争者,这些竞争者也像他一样按同样恶劣的条件出卖自己。所以,原因同样很简单:他归根到底是自己给自己,即自己给作为工人阶级一员的自己造成竞争。

机器也发生同样的影响,而且影响的规模更大得多,因为机器用不熟练的工人代替熟练工人,用女工代替男工,用童工代替成年工;因为在最先使用机器的地方,机器就把大批手工工人抛到街头上去,而在机器日益完善、改进或为生产效率更高的机器所替换的地方,机器又把一批一批的工人排挤出去。我们在前面大略地描述了资本家相互间的产业战争。这种战争有一个特点,就是致胜的办法与其说是增加劳动大军,不如说是减少劳动大军。统帅们即资本家们相互竞赛,看谁能解雇更多的产业士兵。

不错,经济学家们告诉我们说,似乎因采用机器而成为多余的工人可以在新的工业部门里找到工作。

他们不敢干脆地肯定说,在新的劳动部门中找到栖身之所的就是那些被解雇的工人。事实最无情地粉碎了这种谎言。其实,他们不过是肯定说,在工人阶级的其他组成部分面前,譬如说,在一部分已准备进入那种衰亡的产业部门的青年工人面前,出现了新的就业门路。这对于不幸的工人当然是一个很大的安慰。资本家老爷们是不会缺少可供剥削的新鲜血肉的,于是他们就让死人们去埋葬自己的尸体。这种安慰,与其说是对工人的安慰,不如说是对资本家本身的安慰。要知道,假若机器消灭了整个雇佣工人阶级,那末资本的最可怕的时刻就会到来,因为资本没有雇佣劳动就不再成为资本了!

就假定那些直接被机器从一个产业部门排挤出去的工人以及原已指望受雇于该产业部门的那一部分青年工人都能找到新工作。是否可以相信新工作的报酬会和已失去的工作的报酬同样高呢?要是这样,那就是违反了一切经济规律。我们说过,现代产业经常是用简单的和低级的工作来代替较复杂和较高级的工作的。

既是这样,被机器从一个产业部门排挤出去的一大批工人若不甘愿领取更低更坏的报酬,又怎能在别的部门找到栖身之所呢?

有人说制造机器本身的工人是一种例外。他们说,既然产业需要并使用更多的机器,机器的数量就必然增加,因而机器的生产也必然增加,而在这个生产部门中工作的工人人数也必然随之增加;况且这个产业部门的工人是熟练工人,而且还是受过教育的工人。

从1840年起,这种原先也只有一半正确的论点已经毫无正确的影子了,因为机器生产部门也完全和棉纱生产部门一样,日益多方面地采用机器,而机器生产部门的工人,比起极完善的机器来,只能起着极不完善的机器的作用。

可是,在一个男工被机器排挤出去以后,工厂方面也许会雇佣三个童工和一个女工!难道先前一个男工的工资不是应该足够养活三个孩子和一个妻子吗?难道先前最低工资不是应该足够维持工人生活和繁殖工人后代吗?资产阶级爱说的这些话在这里究竟证明了什么呢?只证明了一点:现在要得到维持一个工人家庭生活的工资,就得消耗比以前多三倍的工人生命。

总括起来说:生产资本愈增加,分工和采用机器的范围就愈扩大。分工和采用机器的范围愈扩大,工人之间的竞争就愈剧烈,他们的工资就愈减少。

加之,工人阶级还从较高的社会阶层中得到补充;降落到无产阶级队伍里来的有大批小产业家和小食利者,他们除了赶快跟工人一起伸手乞求工作,毫无别的办法。这样,伸出来乞求工作的手像森林似地愈来愈稠密,而这些手本身则愈来愈消瘦。

不言而喻,小产业家是支持不住这种战争\footnote{在1891年的版本中,“战争”改为“斗争”。——编者注}的:这种战争的首要条件之一就是生产的规模经常扩大,也就是说必须要做大产业家而绝不能做一个小产业家。

当然,还有一点也是用不着进一步说明的:资本愈增殖,资本的总量和数目愈增加,资本的利息也就愈减少;因此,小食利者就不可能再依靠利息来维持生活,必须投到产业方面去,即补充小产业家的队伍,从而增加无产者的候补人数。

最后,上述发展进程愈迫使资本家以日益扩大的规模使用既有的巨大的生产资料,并为此而动用一切信贷机构,而“地震”\footnote{在1891年的版本中,“地震”改为“产业方面的地震”。——编者注}也来得愈来愈频繁,在每次地震中,商业界只是由于埋葬一部分财富、产品以至生产力才维持下去,——也就是说,危机来得愈益剧烈了。这种危机之所以来得愈频繁和愈剧烈,就是因为随着产品总量的增加,亦即随着对扩大市场的需要的增长,世界市场变得愈加狭窄了,剩下可供榨取的市场\footnote{在1891年的版本中,“市场”改为“新市场”。——编者注}愈益减少了,因为先前发生的每一次危机都把一些新市场或以前只被微微榨取过的市场卷入了世界贸易。但是,资本不光靠剥削劳动来生活。像显贵的野蛮的奴隶主一样,资本也要他的奴隶们陪葬,即在危机时期要使大批的工人死亡。由此可见:如果说资本增长得迅速,那末工人之间的竞争就增长得更迅速无比,就是说,资本增长得愈迅速,工人阶级的就业手段即生活资料就相对地缩减得愈厉害;虽然如此,资本的迅速增长对雇佣劳动却是最有利的条件。

\newpage

\subsection{2.笔记}

这部《雇佣劳动与资本》是马克思为数不多的几本较为通俗的读物。一个核心的论点:资本愈强大,工人阶级愈脆弱。

\newpage
\section{马克思主义经济学的数理分析}

\subsection{一般利润率下降规律(LTRPF)}
一般利润率趋于下降规律(The Law of the Tendency of the Rate of Profit to Fall, LTRPF),是马克思资本主义批判理论的核心。这一规律反映了资本主义生产方式自身的内在矛盾,暴露了资本主义生产方式的局限性。
\subsubsection{马克思的观点}
这里笔者简要地概述一下马克思是如何得出LTRPF的。马克思在《资本论》第三卷中探讨了利润率平均化的过程,指出了价值向生产价格转形的资本主义市场机制运行结果,尽管马克思的分析存在着一定的缺陷,事实上马克思本人也意识到了其分析的缺陷之处,但是这并不影响对于资本主义社会生产方式的总体性考察。在这里,笔者通过简化的双部门模型来说明马克思关于LTRPF的逻辑思路。

首先,假设市场上存在着互相竞争的两个部门,且这两个部门的产品占据全部市场。因此部门1与部门2全部商品的价值构成可以记为:
\begin{equation}
    \begin{cases}
        W_1=C_1+V_1+M_1\\
        W_2=C_2+V_2+M_2
    \end{cases}\notag
\end{equation}
其中$W_i$为第$i$部门商品的总价值,$C_i$、$V_i$和$M_i$分别为第$i$部门商品中蕴含的不变资本、可变资本与剩余价值的量。

按照马克思的观点,全部利润来源于剩余价值,各部门的生产成本来源于其不变资本与可变资本\footnote{事实上,马克思这里存在着一个缺陷,各部门的生产成本不一定等于其不变资本与可变资本的和,但这里笔者先按下不表。}。因此,$\Sigma L_i=\Sigma M_i$,$ K_i =  C_i+V_i$,其中$L_i$为第$i$部门的利润,$K_i$为第$i$部门的成本。
在此基础上,利润率$r$可写作:
\begin{equation}
    r=\frac{M_1+M_2}{C_1+V_1+C_2+V_2}\notag
\end{equation}

接下来我们对其中的部门1进行分析,由于在资本主义市场经济体制中,市场主体的行为往往具有对称性,因此对于部门1的分析可以扩大到部门2。
让我们思考,对于部门1的资本家而言,由于资本逐利的本性,因而其生产的直接目的是为了获取利润。更确切地说,是为了获取超额利润,因此,对于部门1的资本家而言,其生产的直接动机可以用数学形式写作:
\begin{equation}
    L_1>M_1\tag{1}
\end{equation}
上式即是说,对于部门1的资本家而言,获取超额利润等价于(意味着)实际利润大于部门内部生产出的剩余价值,即部门2的部分剩余价值转移到部门1之中了\footnote{事实上,这是一个很常见的现象,就像张朝阳说的\href{https://www.bilibili.com/video/BV1ze4y1J7ZA/?spm_id_from=333.337.search-card.all.click&vd_source=33015fad0cdb4da4b0f44b1d5fc3e8bf}{“市场剥削资本家”}。}。
其中$L_1=r(C_1+V_1)$,$r=\frac{M_1+M_2}{C_1+V_1+C_2+V_2}$,因此$L_1>M_1$可以转化为:
\begin{equation}
    \frac{M_1+M_2}{C_1+V_1+C_2+V_2} (C_1+V_1)>M_1\tag{1.a}
\end{equation}
在这里,让我们令第$i$部门的资本有机构成为$\xi_i=\frac{C_i}{V_i}$,令第$i$部门的剩余价值率(剥削率)为$\mu_i = \frac{m_i}{v_i}$。因此,式(1.a)可转化为\footnote{笔者省略了推导过程,试着算一下吧。}:
\begin{equation}
    \mu_2(\xi_1+1)>\mu_1(\xi_2+1)\tag{1.b}
\end{equation}

根据我们的推导,式(1)等价于式(1.b),而式(1.b)是一个仅仅关于资本有机构成与剩余价值率的不等式。这意味着,资本主义市场经济体制中资本家逐利的行为,可能会使得社会的有机构成或剩余价值率发生变化。现在让我们对该不等式做进一步的分析:

1.当两部门有机构成相等时,部门1资本家若是要获取超额利润,便需要使本部门的剩余价值率小于部门2的剩余价值率。也即是说,当各部门劳动生产率\footnote{这里可以理解为机械化程度}水平一致时,对工人的超额压迫并不会为其带来额外的收益。因此,对于一定历史阶段的社会生产而言,资本家的首要任务并不是超额压迫工人,而是要提高实际的劳动生产率。

2.当两部门剩余价值率相等时\footnote{事实上,马克思便是这么假设的,因为对于工人而言,其自身具有的流动性会导致剩余价值率的平均化。},部门1资本家若是要获取超额利润,便需要提高本部门的资本有机构成。

因此总的来说,在资本主义社会中的某一特定的历史阶段下,资本家逐利的本性会导致其首要地提高本部门资本的有机构成,而对于剩余价值率即对工人的剥削程度而言,是次要的\footnote{但这绝不意味着剩余价值率不会发生变化,事实上剩余价值率的提高同样是一个历史的趋势,但笔者在这里强调的是有机构成提高相较于剩余价值率提高的优先级。}。因此,对于平均利润率$r = \frac{M}{C+V}=\frac{\mu}{\xi+1}$而言,可以认为$\xi $呈现出上升趋势,而$\mu $相对较为平稳,因此可以得到:
\begin{equation}
    r \downarrow= \frac{\mu\rightarrow}{\xi\uparrow+1}\notag
\end{equation}
这便是马克思关于LTRPF的完整逻辑思考过程。


\newpage
\subsection{马克思两部类生产模型的数理化分析}
每个部类,总价值记为$W_i$,其价值构成为\(W_i=C_i+V_i+M_i=\)不变资本+可变资本+剩余价值。
当n=2的时候,则有:
\begin{equation}
 \begin{cases}
W_1=C_1+V_1+M_1\\
W_2=C_2+V_2+M_2    
\end{cases}\notag   %无编号
\end{equation}
\subsubsection{1.简单再生产}

简单再生产指的是全部剩余价值用于资本家个人消费,而不是扩大再生产,同时保障总产值不增不减。在这种情况下,均衡的条件是$W_1=C_1+V_1+M_1$,即$V_1+M_1=C_2$。

下面对这个均衡条件做出证明。为了方便叙述,假定两个部门各有一个工人和一个资本家。第一部类生产两种机器,一种用于部门1,一种用于部门2。所有消费品同质。

(1)开始时,部门1的资本家拥有价值为$C_1+V_1$的资本,其中$C_1$的资本是机器的价值,$V_1$的资本是支付工资的可变资本,同理可知部门2的初始状况。在这里假设一期生产之后固定资本C完全转移到产品中。

(2)一期生产结束后,部门1拥有价值量为$W_1=C_1+V_1+M_1$的产品,由于部门1是生产生产资料的部门,因而$W_1$的价值量代表生产出的机器的价值量;部门2拥有价值量为$W_2=C_2+V_2+M_2$的产品,又由于部门2是生产消费资料的部门,因而$W_2$的价值量代表生产出的消费品的价值量。

(3)为了使得简单再生产可持续,在一期生产结束后,部门1的资本家需要在总产品中抽出一部分补偿消耗的机器(这部分机器的价值量为$C_1$,由于部门1的产品为机器,因此这部分的补偿不需要同部门2进行交换),还需将一部分产品(机器)卖给部门2以换取补偿工人劳动力与部门1资本家的消费资料(这部分消费资料的价值量为$V_1+M_1$,由于部门1生产的产品为机器,因此对于消费资料的补偿需要同部门2之间进行交换)。在一期生产结束后,部门2的资本家同样需要补偿消耗的机器(这部分机器的价值为$C_2$,需要同部门1之间进行交换),还需补偿工人劳动力与部门2资本家所需的消费资料(由于部门2的产品就是消费资料,因而这部分的补偿是在部门2内部进行的)。

因此,为了使得部门1与部门2之间的交换达到简单再生产的均衡,则需要满足部门1所需同部门2交换的价值量正好等于部门2所需同部门1相交换的价值量,即$V_1+M_1=C_2$,唯有在此情况下,能够满足第二期生产以同等规模的程度进行。

\subsubsection{2.扩大再生产}

扩大再生产意味着$M_i$的一部分用于积累从而扩大生产规模。
记资本有机构成为\(\xi_i=\frac{\Delta C_i}{\Delta V_i}\);
剩余价值率为\(\mu_i=\frac{M_i}{V_i}\)(假设各部门一致);
积累率为$\alpha_i=\frac{\Delta C_i + \Delta V_i}{M_i}$;
增长率为$g_i = \frac{\Delta C_i + \Delta V_i}{C_i + V_i}
$
(注:当资本有机构成$\xi_i$不变时,$g_i=\frac{\Delta C_i + \Delta V_i}{C_i + V_i}=\frac{\Delta C_i}{C_i}=\frac{\Delta V_i}{V_i}$)\footnote{这里附上证明:$g_i=\frac{\Delta C_i + \Delta V_i}{C_i + V_i } = \frac{\Delta V_i (1+\xi_i)}{ V_i (1+\xi_i)} =\frac{\Delta C_i(1+\frac{1}{\xi_i})}{C_i(1+\frac{1}{\xi_i})} =\frac{\Delta V_i}{V_i} = \frac{\Delta C_i}{C_i} $};资本家的消费部分记为$U_i$。

扩大再生产同简单再生产之间的区别在于,资本家消费的总量小于剩余价值的总量,即$U_i<M_i$,在这里总剩余价值等于资本家用来消费的量加上扩大生产资本的量,即$M_i=\Delta C_i +\Delta V_i + U_i$。

当n=2时,我们可以得到两部类扩大再生产处于均衡时的一般条件:
\begin{equation}
 \begin{cases}
 W_1 = C_1+C_2+\Delta C_1+\Delta C_2\\ 
  W_2 = V_1+V_2+\Delta V_1+\Delta V_2\\
U = (1-\alpha_1)M_1+(1-\alpha_2)M_2     
 \end{cases} \notag   
\end{equation}

可见,扩大再生产处于均衡时的条件可以被看作马克思再生产的一般均衡条件,因为当积累率$\alpha = 0$时,便是简单再生产的均衡条件。

接下来我们探讨一下在马克思扩大再生产的两部类模型中,第一部类与第二部类增长率$g_1$和$g_2$之间的关系。(这里假定1.两部类各自资本的有机构成不变,以及2.两部类之间剩余价值率相等)\footnote{对于1而言,是考虑到扩大再生产首先表现为规模效应而假设的;对于2而言,是考虑到工人的流动性会导致剩余价值率的平均化而假设的。}

在这里引入参数$t$,$t$表示生产的期数。因此,
对于第二部类而言,其增长率为:
$$
g_{2(t)}=\frac{\Delta C_{2(t)} + \Delta V_{2(t)}}{C_{2(t)} + V_{2(t)}}=\frac{\Delta V_{2(t)}}{V_{2(t)}} = \frac{\Delta C_{2(t)}}{C_{2(t)}} 
$$

根据扩大再生产的均衡条件:
$$
  W_1 = C_{1(t)}+C_{2(t)}+\Delta C_{1(t)}+\Delta C_{2(t)}
$$
则:
$$
\Delta C_2=W_1 -[C_{1(t)}+C_{2(t)}+\Delta C_{1(t)}]
$$

将$\Delta C_2$带入$g_{2(t)}$中,则有:
$$
g_{2(t)}= \frac{\Delta C_{2(t)}}{C_{2(t)}}=\frac{W_1 -[C_{1(t)}+C_{2(t)}+\Delta C_{1(t)}]}{C_{2(t)}}
$$

同时,根据第一部类价值构成,则有:
$$
W_1=C_{1(t)}+V_{1(t)}+M_{1(t)}
$$

将其带入$g_{2(t)}$中,则有:
$$
g_{2(t)}= \frac{W_1=C_{1(t)}+V_{1(t)}+M_{1(t)}-[C_{1(t)}+C_{2(t)}+\Delta C_{1(t)}]}{C_{2(t)}}
$$

将$C_{1(t)}=\xi_{1} V_{1(t)}$和$M_1=\mu V_{1(t)}$带入上式,最终有\footnote{这里笔者省略了推导过程,不是很复杂,大家可以试着自己算一下!}:
$$
g_{2(t)}= [\frac{\mu + 1}{\xi_1}-g_{1(t)}]\frac{C_{1(t)}}{C_{2(t)}}-1
$$

在这里,我们假定第一部类处在均衡增长的路径上(或认为第一部类的增长率是外生的),即$g_{1(t)}=g_{1(t+1)}$。那么让我们算一下$g_{2(t+1)}$吧,你会发现其中的神奇之处!
$$
g_{2(t+1)}=g_{2(t)}= [\frac{\mu + 1}{\xi_1}-g_{1(t+1)}]\frac{[1+g_{1(t)}]C_{1(t)}}{[1+g_{2(t)}]C_{2(t)}}-1
$$

将$g_{1(t)}=g_{1(t+1)}$和$g_{2(t)}= [\frac{\mu + 1}{\xi_1}-g_{1(t)}]\frac{C_{1(t)}}{C_{2(t)}}-1$带入上式中,可以最终得到\footnote{推导过程仍然省略,这里也不是特别复杂,大家试着算一下吧!}:
$$
g_{2(t+1)}=g_{1(t)}
$$

上式表明了这样的一个结论:若1部类自律地决定自己的资本增长率,则2部类的资本增长率会晚1期而追随其后。马克思扩大再生产图式具有强稳定性,1部类对2部类的生产起决定性作用!

\subsection{转形问题}
\subsubsection{写在前面的话}
转形问题指的是价值如何转向生产价格的问题。
关于转形问题的争论,已经一百多年了,关于这一争论的详细历史起因,有兴趣的读者可以看一下\href{https://www.marxists.org/chinese/reference-books/howard-1883-1929/index.htm}{《马克思主义经济学史(1883—1929)》}中的前三章。目前就笔者个人的了解而言,很难说有一个公认的关于转形问题的完美解法。考虑到各位读者大部分是非专业人士,因此笔者在这里会尽量以一种通俗易懂的方式介绍马克思本人的转形理论,以及马克思的转形理论所忽略的地方,并讲述以鲍特凯维茨的转形理论为基础的古典解法\footnote{事实上,转形理论在后续发展过程中变得较为复杂,为了减少读者们的阅读难度,笔者在本书中只介绍古典偏离系数解法。}。



\subsubsection{一个简短的历史回顾:恩格斯“有奖征文竞赛”}

《资本论》第一卷出版后,德国经济学家\textbf{洛贝尔图斯}指责马克思“剽窃”他的剩余价值理论。恩格斯在《资本论》第二卷序言中对其给予了驳斥,他认为马克思的“剩余价值理论”在逻辑上优于洛贝尔图斯。恩格斯指出洛贝尔图斯的理论并没有解决\textbf{李嘉图难题}\footnote{笔者在这里对李嘉图难题进行一个简要的概述:李嘉图作为古典经济学的代表人物,他坚持劳动创造价值。在他的理论中,等量的劳动创造等量的价值,且利润=新价值,因此如果两个资本分别使用不同量的活劳动,那末这两个资本产生的价值便不相同,利润也应是不同的。但现实情况恰好相反,利润的大小往往取决的不是资本所所推动的活劳动量的大小,而是取决于资本的总量,即等额资本获取等额利润。这便和李嘉图所认为的等额的活劳动量获取等额的利润发生矛盾,这种矛盾便是李嘉图难题。},而是选择了回避。
因此,为了彻底的让洛贝尔图斯闭嘴\footnote{正如前文所述,恩格斯认为洛贝尔图斯并没有解决李嘉图难题,而马克思在《资本论》第三卷中解决了。},恩格斯发起了著名的\textbf{“有奖征文竞赛”}。这一竞赛的内容是如何在劳动价值论的的基础上说明等量资本获取等量利润的机制。更细致的说,在\textbf{劳动价值论}的基础上探究等量资本获取等量利润的机制意味着“等额资本获取等额利润”、“总利润等于总剩余价值”与“总价格等于总价值”这三者不发生矛盾(或具有\textbf{逻辑}上的一致性,且这一\textbf{逻辑}的出发点是劳动价值论)。

第一位参赛者,是德国的统计学与经济学家\textbf{威·莱克西斯}。莱克西斯认为对李嘉图难题的解决,唯一可能的方法便是允许价格和劳动价值之间存在着偏离,且这种偏离是由于剩余价值从使用劳动力数量相对大的资本家转移到使用较少劳动力数量的资本家的形式而实现的。

第二位参赛者\footnote{从这开始,无法细致地通过历史材料证明谁是第几位参赛者了,但为了清晰地叙述,笔者在后续依然以第N位参赛者来介绍参与恩格斯有奖征文竞赛的人物。},是德国经济学家\textbf{康拉德·施米特}。施米特将社会总产品分为两个部分:(1)作为可变资本与不变资本的产品,(2)作为剩余的产品。施米特认为,按照马克思的价值理论,只有社会必要劳动创造价值,剩余产品中的物化劳动不是社会必要劳动,因而它无法决定商品的价值。很显然,施米特的解释是站不住脚的,他混淆了马克思的社会必要劳动的概念,也混淆了价值与价格的概念。
为了让读者能够充分理解施米特理论的实质,这里举一下施米特自己所引用的例子:

\begin{fangsong}
    “假定生产了100单位的商品,其中50单位代表资本家在不变资本和可变资本上的支出,其余的50单位是剩余产品。第一部分的价值(用黄金表示)是500英镑,或者每单位产品10英镑。施米特进一步假定,使用的资本的价值是400英镑(这意味着平均周转期不足一年),平均利润率为20\% 。从而资本家的利润为400英镑的20\%,即80英镑。这也是代表剩余产品的50单位商品的价格;从而每单位剩余产品的价格为1.60英镑。全部产品售出时得到(500+80)=580英镑,小于其价值(100×10=1000英镑),同样的,每单位商品的价格(5.80英镑)小于单位商品的价值(10英镑)。”
\end{fangsong}

第三位参赛者,是纽约的医生\textbf{乔治.C.斯蒂贝林}。斯蒂贝林认为有机构成不同的等量资本,无法产生等量的价值和剩余价值。有机构成越高,劳动生产率就越高,从而剥削率也就越高。通过剥削率的非均等化,不同的有机构成,可以带来平均的利润率。他使用统计资料论证了他的观点:1880年美国生产普查数据说明有机构成高的产业具有较高的剥削率。但事实上,他的解释是一种偏经验化的解释,我们并没有理由认为剥削率的提高是由于资本有机构成提高所引起的。 并且,斯蒂贝林的统计并没有说明“总价格=总价值”,而是潜在地假设单位商品的价格=单位商品的价值,在这个意义上,他已经远离了转形问题!(他的主要贡献是从经验的角度上说明剩余价值率与资本有机构成同向的提高。)



第四位参赛者,是苏黎世大学经济学教授\textbf{乌斯·沃尔弗}。沃尔弗的逻辑和斯蒂贝林很相似,可以概述为:
有机构成$\uparrow ~\Rightarrow$ 劳动生产率 $\uparrow ~\Rightarrow$ 剩余价值率(剥削率)$\uparrow ~\Rightarrow$不同部门之间剩余价值的均等化。沃尔弗举了这样的一个数字例:

\begin{table}[H]%  H是让表格自适应宽度,需要添加宏包\usepackage{float}
\resizebox{\linewidth}{!}{
\begin{tabular}{|l|l|l|l|l|l|l|}
\hline
    & 不变资本c & 可变资本v & 剩余价值s & 总价值c+v+s & 剥削率s/v & 利润率s/(c+v) \\ \hline
资本\uppercase\expandafter{\romannumeral1}  & 5     & 5     & 1     & 11       & 20\%   & 10\%       \\ \hline
资本\uppercase\expandafter{\romannumeral2}  & 10    & 5     & 1.5   & 16.5     & 30\%   & 10\%       \\ \hline
\end{tabular}
}
\end{table}

可见,对于资本\uppercase\expandafter{\romannumeral2} 而言,由于它的有机构成小于资本\uppercase\expandafter{\romannumeral1},因此,它的剩余价值率同样小于资本\uppercase\expandafter{\romannumeral1},当利润率平均化之后,单个资本的利润还是等于剩余价值的量。但是,沃尔弗的解释仍然是有缺陷的(换言之,是牵强的),他没有阐明剥削率和资本有机构成之间确切的计量关系,这就会使人产生疑问,对于上述数值例而言,为什么资本\expandafter{\romannumeral2}的剩余价值率是30\%?如果不是30\%的话,沃尔弗的解释便“失效”了。

事实上,沃尔弗同样远离了价值转形问题,他同样混淆了价值和价格之间的区别,固执地认为价值=价格,在预设单位商品的价值=价格这一前提的基础上,强行凑出了一个数值例。但问题在于,只要认为价值=价格,就一定会凑出相应的数值例,只要人为地变动剩余价值率使得最终各部门的剩余价值量相等即可了!

第五位参赛者,是意大利学者\textbf{阿基尔·洛里亚}。
洛里亚的解释大体说来是引入了一个货币资本家部门,这一部门的利润是靠收取其他产业资本家部门的货币利息而实现的。洛里亚认为在产业资本家部门资本有机构成不同,但剩余价值率相同的情况下,产业资本家把剩余价值的一部分用于支付给货币资本家的利息,并使得最终产业资本家各部门的利润率均等化,且这一均等化的利润率等于货币资本家的利润率。

事实上,洛里亚的解释是混乱的,为什么产业资本家的利润率会趋于货币资本家的利润率?洛里亚并没有对此进行解释,并且各产业资本家部门的货币利息率之间的差异是如何形成的,洛里亚同样没有解释。

第六位参赛者,是移居美国的俄国化学家\textbf{彼·法尔曼}。法尔曼认为价格和价值之间差异的产生是分配因素造成的,他指出:“两种商品相交换时,一种商品的价格高于它的价值的大小,必然等于另一种商品的价格低于它的价值的大小,反之亦然\footnote{笔者注:这里的“两种商品交换”应该指的是有机构成高于社会平均部门与有机构成低于社会平均部门的商品之间的交换,因为单独的两种商品相交换时,二者都有可能同时高于或低于其自身的价值。}”。事实上,在这里,法尔曼的观点已经非常接近马克思了。

第七位参赛者,是\textbf{米尔普福特·沃尔夫冈}。
沃尔夫冈也是用代数方法解释的,但是笔者认为沃尔夫冈的解释较为复杂,且不伦不类,放在这里容易影响读者们的阅读兴趣,使读者更加陷入混乱,因此笔者在这里将沃尔夫冈的方法略过。


最后一位参赛者,是慕尼黑的教授\textbf{莱尔}博士。莱尔博士的最重要的贡献是,他尝试用代数形式表述转形问题,虽然是不成功的,但启发了后续的关于价值转形问题的数学分析思路(特别是古典偏离系数法)。

莱尔用$k_1, k_2...$和$v_1, v_2...$分别表示产业1,2...中使用的不变和可变资本的价值,用$m_1, m_2...$表示剩余价值。对整个经济而言,相应的加总分别为$K, V$和$M$,他采用了和马克思相同的立场,用$M/(K+V)=r$表示平均利润率。“各单位商品的交换价值”用$t_1, t_2,...$表示。基于此,莱尔列出了如下等式:




\begin{equation}
\begin{cases}
m_1+m_2+m_3+...=M \\
t_1m_1+t_2m_2+t_3m_3+...=M\\
k_1+v_1+k_2+v_2+k_3+v_3...=K+V\\
(k_1+v_1)r=t_1m_1\\
(k_2+v_2)r=t_2m_2\\
......\\
(k_1+v_1+m_1)=(k_1+v_1)(1+r)\\
(k_2+v_2+m_2)=(k_2+v_2)(1+r)\\
......
\end{cases}\tag{1}%编号
\end{equation}%方程组编码

根据莱尔博士所列出的方程组,“单位商品的交换价值”$t_i$可被理解为生产价格和价值之间的比率,方程组中的前两个式子:
\begin{equation}
\begin{cases}
m_1+m_2+m_3+...=M \\
t_1m_1+t_2m_2+t_3m_3+...=M\\
\end{cases}\tag{1.1}
\end{equation}
实际上想表达的是“总利润=总剩余价值”这个命题。这里是没有什么问题的,$m_it_i$就其定义而言是被修正过了的剩余价值,即利润(生产价格形式)。

但是,后续方程组中的等式便出现了问题。对于$\Sigma (k_i+v_i)=K+V$而言,这是莱尔博士对社会总生产资本量的定义,这一点是无可厚非的。但是,对于方程组中的最后一部分等式而言:
\begin{equation}
    \begin{cases}
        (k_1+v_1)r=t_1m_1\\
(k_2+v_2)r=t_2m_2\\
......\\
(k_1+v_1+m_1)=(k_1+v_1)(1+r)\\
(k_2+v_2+m_2)=(k_2+v_2)(1+r)\\
......
    \end{cases}\tag{1.2}
\end{equation}
第一,没有理由认为单一部门生产成本本身就是以价值量$k_i+v_i$的形式出现的,按照实际情况而言,$(1+r)$前面乘的成本量应该是成本的价格形式,价格相较于价值而言,是一个修正了的概念,莱尔博士在这里对二者的认识发生了混淆。第二,更没有理由认为个别部门以价值形式转化的价格等于该部门产品的总价值,但是莱尔博士直接用$\Sigma(k_i+v_i+m_i)=(k_i+v_i)(1+r)$这一等式组说明利润率平均化后个别部门的实际商品的总价格量等于其总价值量,这是站不住脚的!单个部门内部价值量和价格量并不是相等的,莱尔博士的分析同样远离了转形理论。


\newpage







\end{document}
