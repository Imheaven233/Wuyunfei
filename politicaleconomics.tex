\documentclass[a4paper,twoside,12pt]{ctexart}
\newcommand\specialsectioning{\setcounter{secnumdepth}{-2}}
\specialsectioning
\usepackage[center]{titlesec}
\usepackage{indentfirst}
\setlength{\parindent}{2em}
\usepackage{fancyhdr}
\pagestyle{fancy}%fancy style
\fancyhf{}%清空页眉页脚
\fancyhead[LE,RO]{\thepage}%页码位置:偶数页居左,奇数页居右
\fancyfoot[RO,RE]{\textit{Wuyunfei's note}}% 设置页脚:在每页的右下脚以斜体显示书名

\setlength{\headheight}{15pt}%解决页眉warnings

\renewcommand{\headrulewidth}{0pt} % 页眉与正文之间的水平线粗细
\renewcommand{\footrulewidth}{0pt}

\usepackage{changepage}%设置引用段落左右侧缩进
\newCJKfontfamily\mincho{IPAexMincho}%日语明朝体

\usepackage{hyperref}%设置超链接

\title{马克思主义著作学习笔记}
\author{吴云飞}
\date{}
\begin{document}
\maketitle

\newpage

\tableofcontents

\newpage

\section{序言}
\begin{adjustwidth}{2em}{2em}
\qquad\fangsong 笔者是一名马克思主义理论的硕士研究生,本书记载了笔者学习马克思主义原著的过程,由于笔者个人的专业性有限,因而本书的主要目的是记录笔者的思考过程。本书以马克思主义的原著为原本,在此基础上,添加一些笔者对于某些论述的感悟与理解。

本书从恩格斯的国民经济学批判大纲开始,逐一探讨马克思恩格斯所著的短篇文章。由于笔者的兴趣点集中于政治经济学的方面,因而对某些内容的理解难免会具有一定的片面性,又由于笔者个人学术水平的稚嫩,因而对某些观点的叙述难免会显得重复啰嗦,望请读者见谅。

无产阶级在进行斗争的过程中需要先进的思想作为自身的武器,知识不应被特权阶级(知识资产阶级)所垄断。且由于本书会随着笔者学习的深入而不断更新其内容,故决定将本书开源放置在本人的github上,读者们可以随时获取最新版本\footnote{本书采用LaTeX进行编写

\quad github地址:\url{https://github.com/Imheaven233/Wuyunfei}}。

\end{adjustwidth}

\newpage

\section{国民经济学批判大纲}

\subsection{1.原文}

国民经济学的产生是商业扩展的自然结果,随着它的出现,一个成熟的允许欺诈的体系、一门完整的发财致富的科学代替了简单的不科学的生意经。\footnote[0]{弗·恩格斯大约写于1843年9月底或10月初—1844年1月中,载于1844年2月《德法年鉴》,原文是德文。}

这种从商人的彼此妒忌和贪婪中产生的国民经济学或发财致富的科学,在额角上带有最令人厌恶的自私自利的烙印。人们还有一种幼稚的看法,以为金银就是财富,因此必须到处从速禁止“贵”金属出口。各国像守财奴一样相互对立,双手抱住自己珍爱的钱袋,怀着妒忌心和猜疑心注视着自己的邻居。他们使用一切手段尽可能多地骗取那些与自己通商的民族的现钱,并使这些侥幸赚来的钱好好地保持在关税线以内。

如果完全彻底地实行这个原则,那就会葬送商业。因此,人们便开始跨越这个最初的阶段。他们意识到,放在钱柜里的资本是死的,而流通中的资本会不断增殖。于是,人们变得比较友善了,人们开始把自己的杜卡特\footnote{14—19世纪欧洲许多国家通用的金币。——编者注 }当做诱鸟放出去,以便把别人的杜卡特一并引回来,并且认识到,多花一点钱买甲的商品一点也不会吃亏,只要能以更高的价格把它卖给乙就行了。

重商主义体系就建立在这个基础上。商业的贪婪性已多少有所遮掩;各国多少有所接近,开始缔结通商友好条约,彼此做生意,并且为了获得更大的利润,甚至尽可能地互相表示友爱和亲善。但是实质上还是同从前一样贪财和自私,当时一切基于商业角逐而引起的战争就时时露出这种贪财和自私。这些战争也表明:贸易和掠夺一样,是以强权为基础的;人们只要认为哪些条约最有利,他们就甚至会昧着良心使用诡计或暴力强行订立这些条约。

贸易差额论是整个重商主义体系的要点。正因为人们始终坚持金银就是财富的论点,所以他们认为只有那最终给国家带来现金的交易才是赢利交易。为了说明这一点,他们以输出和输入作比较。如果输出大于输入,那么他们就认为这个差额会以现金的形式回到本国,国家也因这个差额而更富裕。因此经济学家的本事就是要设法使输出和输入到每年年底有一个顺差。为了这样一个可笑的幻想,竟有成千上万的人被屠杀!商业也有了它的十字军征讨和宗教裁判所。

18世纪这个革命的世纪使经济学也发生了革命。然而,正如这个世纪的一切革命都是片面的并且停留在对立的状态中一样,正如抽象的唯物主义和抽象的唯灵论相对立,共和国和君主国相对立,社会契约和神权相对立一样,经济学的革命也未能克服对立。到处依然存在着下述前提:唯物主义不抨击基督教对人的轻视和侮辱,只是把自然界当做一种绝对的东西来代替基督教的上帝而与人相对立;政治学没有想去检验国家的各个前提本身;经济学没有想去过问\textbf{私有制的合理性}的问题。因此,新的经济学只前进了半步;它不得不背弃和否认它自己的前提,不得不求助于诡辩和伪善,以便掩盖它所陷入的矛盾,以便得出那些不是由它自己的前提而是由这个世纪的人道精神得出的结论。这样,经济学就具有仁爱的性质;它不再宠爱生产者,而转向消费者了;它假惺惺地对重商主义体系的血腥恐怖表示神圣的厌恶,并且宣布商业是各民族、各个人之间的友谊和团结的纽带。一切都显得十分辉煌壮丽,可是上述前提马上又充分发挥作用,而且创立了与这种伪善的博爱相对立的马尔萨斯人口论,这种理论是迄今存在过的体系中最粗陋最野蛮的体系,是一种彻底否定关于仁爱和世界公民的一切美好言词的绝望体系;这些前提创造并发展了工厂制度和现代的奴隶制度,这种奴隶制度就它的无人性和残酷性来说不亚于古代的奴隶制度。新的经济学,即以亚当·斯密的《国富论》\footnote{亚·斯密《国民财富的性质和原因的研究》1776年伦敦版。——编者注}为基础的自由贸易体系,也同样是伪善、前后不一贯和不道德的。这种伪善、前后不一贯和不道德目前在一切领域中与自由的人性处于对立的地位。

可是,难道说亚当·斯密的体系不是一个进步吗?当然是进步,而且是一个必要的进步。为了使私有制的真实的后果能够显露出来,就有必要摧毁重商主义体系以及它的垄断和它对商业关系的束缚;为了使当代的斗争能够成为普遍的人类的斗争,就有必要使所有这些地域的和国家的小算盘退居次要的地位;有必要使私有制的理论抛弃纯粹经验主义的、仅仅是客观主义的研究方法,并使它具有一种也对结果负责的更为科学的性质,从而使问题涉及全人类的范围;有必要通过对旧经济学中包含的不道德加以否定的尝试,并通过由此产生的伪善——这种尝试的必然结果——而使这种不道德达于极点。这一切都是理所当然的。我们乐于承认,只有通过对贸易自由的论证和阐述,我们才有可能超越私有制的经济学,然而我们同时也应该有权指出,这种贸易自由并没有任何理论价值和实践价值。

我们所要评判的经济学家离我们的时代越近,我们对他们的判决就必定越严厉。因为斯密和马尔萨斯所看到的现成的东西只不过是一些片断,而在新近的经济学家面前却已经有了一个完整的体系;一切结论已经作出,各种矛盾已经十分清楚地显露出来,但是,他们仍不去检验前提,而且还是对整个体系负责。经济学家离我们的时代越近,离诚实就越远。时代每前进一步,为把经济学保持在时代的水平上,诡辩术就必然提高一步。因此,比如说,\textbf{李嘉图}的罪过比\textbf{亚当·斯密}大,而\textbf{麦克库洛赫}和\textbf{穆勒}的罪过又比李嘉图大。

新近的经济学甚至不能对重商主义体系作出正确的评判,因为它本身就带有片面性,而且还受到重商主义体系的各个前提的拖累。只有摆脱这两种体系的对立,批判这两种体系的共同前提,并从纯粹人的、普遍的基础出发来看问题,才能够给这两种体系指出它们的真正的地位。那时大家就会明白,贸易自由的捍卫者是一些比旧的重商主义者本身更为恶劣的垄断者。那时大家就会明白,在新经济学家的虚伪的人道背后隐藏着旧经济学家闻所未闻的野蛮;旧经济学家的概念虽然混乱,与攻击他们的人的口是心非的逻辑比较起来还是单纯的、前后一贯的;这两派中任何一派对另一派的指责,都不会不落到自己头上。因此,新的自由主义经济学也无法理解李斯特为什么要恢复重商主义体系\footnote{弗·李斯特《政治经济学的国民体系》第 1卷《国际贸易、贸易政策和德国关税同盟》1841年斯图加特—蒂宾根版。——编者注},而这件事我们却觉得很简单。前后不一贯的和具有两面性的自由主义经济学必然要重新分解为它的基本组成部分。正如神学不回到迷信,就得前进到自由哲学一样,贸易自由必定一方面造成垄断的恢复,另一方面造成私有制的消灭。

自由主义经济学达到的唯一肯定的进步,就是阐述了私有制的各种规律。这种经济学确实包含这些规律,虽然这些规律还没有被阐述为最后的结论,还没有被清楚地表达出来。由此可见,在涉及确定生财捷径的一切地方,就是说,在一切严格意义的经济学上的争论中,贸易自由的捍卫者们是正确的。当然,这里指的是与支持垄断的人争论,而不是与反对私有制的人争论,因为正如英国社会主义者早就在实践中和理论上证明的那样\footnote{指约·弗·布雷、威·汤普森、约·瓦茨和他们的著作:布雷《劳动的不公正现象及其解决办法,或强权时代和公正时代》1839年利兹版;汤普森《最能促进人类幸福的财富分配原理的研究》1824年伦敦版;瓦茨《政治经济学家的事实和臆想:科学原则述评,去伪存真》1842年曼彻斯特—伦敦版。——编者注},反对私有制的人能够从经济的观点比较正确地解决经济问题。

因此,我们在批判国民经济学时要研究它的基本范畴,揭露自由贸易体系所产生的矛盾,并从这个矛盾的两个方面作出结论。 

国民财富这个用语是由于自由主义经济学家努力进行概括才产生的。只要私有制存在一天,这个用语便没有任何意义。英国人的“国民财富”很多,他们却是世界上最穷的民族。人们要么完全抛弃这个用语,要么采用一些使它具有意义的前提。国民经济学,政治经济学,公共经济学等用语也是一样。在目前的情况下,应该把这种科学称为私经济学,因为在这种科学看来,社会关系只是为了私有制而存在。 

私有制产生的最直接的结果就是\textbf{商业},即彼此交换必需品,亦即买和卖。在私有制的统治下,这种商业与其他一切活动一样,必然是经商者收入的直接源泉;就是说,每个人必定要尽量设法贱买贵卖。因此,在任何一次买卖中,两个人总是以绝对对立的利益相对抗;

这种冲突带有势不两立的性质,因为每一个人都知道另一个人的意图,知道另一个人的意图是和自己的意图相反的。因此,商业所产生的第一个后果是:一方面互不信任,另一方面为这种互不信任辩护,采取不道德的手段来达到不道德的目的。例如,商业的第一条原则就是对一切可能降低有关商品的价格的事情都绝口不谈,秘而不宣。由此可以得出结论:在商业中允许利用对方的无知和轻信来取得最大利益,并且也同样允许夸大自己的商品本来没有的品质。总而言之,商业是合法的欺诈。任何一个商人,只要他说实话,他就会证明实践是符合这个理论的。

重商主义体系在某种程度上还具有某种纯朴的天主教的坦率精神,它丝毫不隐瞒商业的不道德的本质。我们已经看到,它怎样公开地显露自己卑鄙的贪婪。18世纪民族间的相互敌视、可憎的妒忌以及商业角逐,都是贸易本身的必然结果。社会舆论既然还不具有人道精神,那么何必要掩饰从商业本身的无人性的和充满敌意的本质中所产生的那些东西呢?

但是,当\textbf{经济学的路德}\footnote{马克思在《1844年经济学哲学手稿》中对这个提法作了解释,见本卷第178—179页。——编者注},即亚当·斯密,批判过去的经济学的时候,情况大大地改变了。时代具有人道精神了,理性起作用了,道德开始要求自己的永恒权利了。强迫订立的通商条约、商业战争、民族间的严重孤立状态与前进了的意识异常激烈地发生冲突。新教的伪善代替了天主教的坦率。斯密证明,人道也是由商业的本质产生的,商业不应当是“纠纷和敌视的最丰产的源泉”,而应当是“各民族、各个人之间的团结和友谊的纽带”(参看《国富论》第4卷第3章第2节);理所当然的是,商业总的说来对它的一切参加者都是有利的。

斯密颂扬商业是人道的,这是对的。世界上本来就没有绝对不道德的东西;商业也有对道德和人性表示尊重的一面。但这是怎样的尊重啊!当中世纪的强权,即公开的拦路行劫转到商业时,这种行劫就变得具有人道精神了;当商业上以禁止货币输出为特征的第一个阶段转到重商主义体系时,商业也变得具有人道精神了。现在连这种体系本身也变得具有人道精神了。当然,商人为了自己的利益必须与廉价卖给他货物的人们和高价买他的货物的人们保持良好的关系。因此,一个民族要是引起它的供应者和顾客的敌对情绪,就太不明智了。它表现得越友好,对它就越有利。这就是商业的人道,而滥用道德以实现不道德的意图的伪善方式就是自由贸易体系引以自豪的东西。伪君子叫道:难道我们没有打倒垄断的野蛮吗?难道我们没有把文明带往世界上遥远的地方吗?难道我们没有使各民族建立起兄弟般的关系并减少了战争次数吗?不错,这一切你们都做了,然而你们是怎样做的啊!你们消灭了小的垄断,以便使一个巨大的根本的垄断,即所有权,更自由地、更不受限制地起作用;你们把文明带到世界的各个角落,以便赢得新的地域来扩张你们卑鄙的贪欲;你们使各民族建立起兄弟般的关系——但这是盗贼的兄弟情谊;你们减少了战争次数,以便在和平时期赚更多的钱,以便使各个人之间的敌视、可耻的竞争战争达到登峰造极的地步!你们什么时候做事情是从纯粹的人道出发,是从普遍利益和个人利益之间的对立毫无意义这种意识出发的呢?你们什么时候讲过道德,而不图谋私利,不在心底隐藏一些不道德的、利己的动机呢?

自由主义的经济学竭力用瓦解各民族的办法使敌对情绪普遍化,使人类变成一群正因为每一个人具有与其他人相同的利益而互相吞噬的凶猛野兽——竞争者不是凶猛野兽又是什么呢?自由主义的经济学做完这个准备工作之后,只要再走一步——使家庭解体——就达到目的了。为了实现这一点,它自己美妙的发明即工厂制度助了它一臂之力。共同利益的最后痕迹,即家庭的财产共有被工厂制度破坏了,至少在这里,在英国已处在瓦解的过程中。孩子一到能劳动的时候,就是说,到了九岁,就靠自己的工钱过活,把父母的家只看做一个寄宿处,付给父母一定的膳宿费。这已经是很平常的事了。还能有别的什么呢?从构成自由贸易体系的基础的利益分离,还能产生什么别的结果呢?一种原则一旦被运用,它就会自行贯穿在它的一切结果中,不管经济学家们是否乐意。

然而,经济学家自己也不知道他在为什么服务。他不知道,他的全部利己的论辩只不过构成人类普遍进步的链条中的一环。他不知道,他瓦解一切私人利益只不过替我们这个世纪面临的大转变,即人类与自然的和解以及人类本身的和解开辟道路。

商业形成的第一个范畴是价值。关于这个范畴和其他一切范畴,在新旧两派经济学家之间没有什么争论,因为直接热衷于发财致富的垄断主义者没有多余时间来研究各种范畴。关于这类论点的所有争论都出自新近的经济学家。

靠种种对立活命的经济学家当然也有一种双重的价值:抽象价值(或实际价值)和交换价值。关于实际价值的本质,英国人和法国人萨伊进行了长期的争论。前者认为生产费用是实际价值的表现,后者则说什么实际价值要按物品的效用来测定。这个争论从本世纪初开始,后来停息了,没有得到解决。这些经济学家是什么问题也解决不了的。

这样,英国人——特别是麦克库洛赫和李嘉图——断言,物品的抽象价值是由生产费用决定的。请注意,是抽象价值,不是交换价值,不是 exchangeable value,不是商业价值;至于商业价值,据说完全是另外一回事。为什么生产费用是价值的尺度呢?请听!请听!因为在通常情况下,如果把竞争关系撇开,没有人会把物品卖得低于它的生产费用。没有人会卖吧?在这里,既然不谈商业价值,我们谈“卖”干什么呢?一谈到“卖”,我们就要让我们刚才要撇开的商业重新参加进来,而且是这样一种商业!一种不把主要的东西即竞争关系考虑在内的商业!起初我们有一种抽象价值,现在又有一种抽象商业,一种没有竞争的商业,就是说有一个没有躯体的人,一种没有产生思想的大脑的思想。难道经济学家根本没有想到,一旦竞争被撇开,那就保证不了生产者正是按照他的生产费用来卖自己的商品吗?多么混乱啊!

还不仅如此!我们暂且认为,一切都像经济学家所说的那样。假定某人花了很大的力气和巨大的费用制造了一种谁也不要的毫无用处的东西,难道这个东西的价值也同生产费用一样吗?经济学家回答说,绝对没有,谁愿意买这种东西呢?于是,我们立刻不仅碰到了萨伊的声名狼藉的效用,而且还有了随着“买”而来的竞争关系。经济学家是一刻也不能坚持他的抽象的——这是做不到的。不仅他所竭力避开的竞争,而且连他所攻击的效用,随时都可能突然出现在他面前。抽象价值以及抽象价值由生产费用决定的说法,恰恰都只是抽象的非实在的东西。

我们再一次暂且假定经济学家是对的,那么在不把竞争考虑在内的情况下,他又怎样确定生产费用呢?我们研究一下生产费用,就可以看出,这个范畴也是建立在竞争的基础上的。在这里又一次表明经济学家是无法贯彻他的主张的。

如果我们转向萨伊的学说,我们也会发现同样的抽象。物品的效用是一种纯主观的根本不能绝对确定的东西,至少它在人们还在对立中徘徊的时候肯定是不能确定的。根据这种理论,生活必需品应当比奢侈品具有更大的价值。在私有制统治下,竞争关系是唯一能比较客观地、似乎能大体确定物品效用大小的办法,然而恰恰是竞争关系被撇在一边。但是,只要容许有竞争关系,生产费用也就随之产生,因为没有人会卖得低于他自己在生产上投入的费用。因此,在这里也是对立的一方不情愿地转到另一方。

让我们设法来澄清这种混乱吧!物品的价值包含两个因素,争论的双方都要强行把这两个因素分开,但正如我们所看到的,这是徒劳的。价值是生产费用对效用的关系。价值首先是用来决定某种物品是否应该生产,即这种物品的效用是否能抵偿生产费用。然后才谈得上运用价值来进行交换。如果两种物品的生产费用相等,那么效用就是确定它们的比较价值的决定性因素。

这个基础是交换的唯一正确的基础。可是,如果以这个基础为出发点,那么又该谁来决定物品的效用呢?单凭当事人的意见吗?这样总会有一人受骗。或者,是否有一种不取决于当事人双方、不为当事人所知悉、只以物品固有的效用为依据的规定呢?这样,交换就只能强制进行,并且每一个人都认为自己受骗了。不消灭私有制,就不可能消灭物品固有的实际效用和这种效用的规定之间的对立,以及效用的规定和交换者的自由之间的对立;而私有制一旦被消灭,就无须再谈现在这样的交换了。到那个时候,价值概念的实际运用就会越来越限于决定生产,而这也是它真正的活动范围。

然而,目前的情况怎样呢?我们看到,价值概念被强行分割了,它的每一个方面都叫嚷自己是整体。一开始就为竞争所歪曲的生产费用,应该被看做是价值本身。纯主观的效用同样应该被看做是价值本身,因为现在不可能有第二种效用。要把这两个跛脚的定义扶正,必须在两种情况下都把竞争考虑在内;而这里最有意思的是:在英国人那里,竞争代表效用而与生产费用相对立,在萨伊那里则相反,竞争带来生产费用而与效用相对立。但是,竞争究竟带来什么样的效用和什么样的生产费用!它带来的效用取决于偶然情况、时尚和富人的癖好,它带来的生产费用则随着需求和供给的偶然比例而上下波动。

实际价值和交换价值之间的差别基于下述事实:物品的价值不同于人们在买卖中为该物品提供的那个所谓等价物,就是说,这个等价物并不是等价物。这个所谓等价物就是物品的价格,如果经济学家是诚实的,他就会把等价物一词当做“商业价值”来使用。但是,为了使商业的不道德不过于明显地暴露出来,他总得保留一点假象,似乎价格和价值以某种方式相联系。说价格由生产费用和竞争的相互作用决定,这是完全正确的,而且是私有制的一个主要的规律。经济学家的第一个发现就是这个纯经验的规律;接着他从这个规律中抽去他的实际价值,就是说,抽去竞争关系均衡时、供求一致时的价格,这时,剩下的自然只有生产费用了,经济学家就把它称为实际价值,其实只是价格的一种规定性。但是,这样一来,经济学中的一切就被本末倒置了:价值本来是原初的东西,是价格的源泉,倒要取决于价格,即它自己的产物。大家知道,正是这种颠倒构成了抽象的本质。关于这点,请参看费尔巴哈的著作。\footnote{路·费尔巴哈《关于哲学改革的临时纲要》,见《德国现代哲学和政论界轶文集》1843年苏黎世—温特图尔版第64—71页。——编者注}

在经济学家看来,商品的生产费用由以下三个要素组成:生产原材料所必需的土地的地租,资本及其利润,生产和加工所需要的劳动的报酬。但人们立即就发现,资本和劳动是同一个东西,因为经济学家自己就承认资本是“积蓄的劳动”\footnote{亚·斯密《国民财富的性质和原因的研究》1828年爱丁堡版第2卷第94页。——编者注}。这样,我们这里剩下的就只有两个方面,自然的、客观的方面即土地和人的、主观的方面即劳动。劳动包括资本,并且除资本之外还包括经济学家没有想到的第三要素,我指的是简单劳动这一肉体要素以外的发明和思想这一精神要素。经济学家与发明的精神有什么关系呢?难道没有他参与的一切发明就不会落到他手里吗?有哪一件发明曾经使他花费过什么?因此,他在计算他的生产费用时为什么要为这些发明操心呢?在他看来,财富的条件就是土地、资本、劳动,除此以外,他什么也不需要。科学是与他无关的。尽管科学通过贝托莱、戴维、李比希、瓦特、卡特赖特等人送了许多礼物给他,把他本人和他的生产都提到空前未有的高度,可是这与他有何相干呢?他不懂得重视这些东西,科学的进步超出了他的计算。但是,在一个超越利益的分裂——正如在经济学家那里发生的那样——的合理状态下,精神要素自然会列入生产要素,并且会在经济学的生产费用项目中找到自己的位置。到那时,我们自然会满意地看到,扶植科学的工作也在物质上得到报偿,会看到,仅仅詹姆斯·瓦特的蒸汽机这样一项科学成果,在它存在的头50年中给世界带来的东西就比世界从一开始为扶植科学所付出的代价还要多。

这样,我们就有了两个生产要素——自然和人,而后者还包括他的肉体活动和精神活动。现在我们可以回过来谈谈经济学家和他的生产费用。

经济学家说,凡是无法垄断的东西就没有价值。这个论点以后再详细研究。如果我们说:凡是无法垄断的东西就没有价格,那么,这个论点对于以私有制为基础的状态而言是正确的。如果土地像空气一样容易得到,那就没有人会支付地租了。既然情况不是这样,而是在一种特殊情况下被占有的土地的面积是有限的,那人们就要为一块被占有的即被垄断的土地支付地租或者按照售价把它买下来。令人感到奇怪的是,在这样弄明白了土地价值的产生以后,还得听经济学家说什么地租是付租金的土地的收入和值得费力耕种的最坏的土地的收入之间的差额。大家知道,这是李嘉图第一次充分阐明的地租定义。\footnote{大·李嘉图《政治经济学和赋税原理》1817年伦敦版第54页。——编者注}当人们假定需求的减少马上影响地租并立刻使相应数量的最坏耕地停止耕种的时候,这个定义实际上是正确的。但情况并不是这样,因此这个定义是有缺陷的;况且这个定义没有包括地租产生的原因,仅仅由于这一点,这个定义就已经站不住脚了。反谷物法同盟盟员托·佩·汤普森上校在反对这个定义时,又把亚当·斯密的定义\footnote{亚·斯密《国民财富的性质和原因的研究》1828年爱丁堡版第1卷第237—242页。——编者注}搬了出来并加以论证。据他说,地租是谋求使用土地者的竞争和可支配的土地的有限数量之间的关系。在这里,这至少又回到地租产生的问题上来了;但是,这个解释没有包括土壤肥力的差别,正如上述的定义忽略了竞争一样。\footnote{托·佩·汤普森《真正的地租理论,驳李嘉图先生等》,见他的《政治习作及其他》1842年伦敦版第4卷第404页。——编者注}

这样一来,同一个对象又有了两个片面的因而是不完全的定义。正如研究价值概念时一样,在这里我们也必须把这两个定义结合起来,以便得出一个正确的、来自事物本身发展的、因而包括了实践中的一切情况的定义。地租是土地的收获量即自然方面(这方面又包括自然的肥力和人的耕作即改良土壤所耗费的劳动)和人的方面即竞争之间的相互关系。经济学家会对这个“定义”摇头;当他们知道这个定义包括了有关这个问题的一切时,他们会大吃一惊的。

土地占有者无论如何不能责备商人。

他靠垄断土地进行掠夺。他利用人口的增长进行掠夺,因为人口的增长加强了竞争,从而抬高了他的土地的价值。他把不是通过他个人劳动得来的、完全偶然地落到他手里的东西当做他个人利益的源泉进行掠夺。他靠出租土地、靠最终攫取租地农场主的种种改良的成果进行掠夺。大土地占有者的财富日益增长的秘密就在于此。

认定土地占有者的获得方式是掠夺,即认定人人都有享受自己的劳动产品的权利或不播种者不应有收获,这样的公理\footnote{亚·斯密《国民财富的性质和原因的研究》1828年爱丁堡版第1卷第85—86页。——编者注}并不是我们的主张。第一个公理排除抚育儿童的义务;第二个公理排除任何世代的生存权利,因为任何世代都得继承前一世代的遗产。确切地说,这些公理都是由私有制产生的结论。要么实现由私有制产生的一切结论,要么抛弃私有制这个前提。

甚至最初的占有本身,也是以断言老早就存在过共同占有权为理由的。因此,不管我们转向哪里,私有制总会把我们引到矛盾中去。

土地是我们的一切,是我们生存的首要条件;出卖土地,就是走向自我出卖的最后一步;这无论过去或直至今日都是这样一种不道德,只有自我出让的不道德才能超过它。最初的占有土地,少数人垄断土地,所有其他的人都被剥夺了基本的生存条件,就不道德来说,丝毫也不逊于后来的土地出卖。

如果我们在这里再把私有制撇开,那么地租就恢复它的本来面目,就归结为实质上可以作为地租基础的合理观点。这时,作为地租而与土地分离的土地价值,就回到土地本身。这个价值是依据面积相等的土地在花费的劳动量相等的条件下所具有的生产能力来计算的;这个价值在确定产品的价值时自然是作为生产费用的一部分计算在内的,它像地租一样是生产能力对竞争的关系,不过是对真正的竞争,即对某个时候会展开的竞争的关系。

我们已经看到,资本和劳动最初是同一个东西;其次,我们从经济学家自己的阐述中也可以看到,资本是劳动的结果,它在生产过程中立刻又变成了劳动的基质、劳动的材料;可见,资本和劳动的短暂分开,立刻又在两者的统一中消失了;但是,经济学家还是把资本和劳动分开,还是坚持这两者的分裂,他只在资本是“积蓄的劳动”这个定义\footnote{亚·斯密《国民财富的性质和原因的研究》1828年爱丁堡版第2卷第94页。——编者注}中承认它们两者的统一。由私有制造成的资本和劳动的分裂,不外是与这种分裂状态相应的并从这种状态产生的劳动本身的分裂。这种分开完成之后,资本又分为原有资本和利润,即资本在生产过程中所获得的增长额,虽然实践本身立刻又将这种利润加到资本上,并把它和资本投入周转中。甚至利润又分裂为利息和本来意义上的利润。在利息中,这种分裂的不合理性达到顶点。贷款生息,即不花劳动单凭贷款获得收入,是不道德的,虽然这种不道德已经包含在私有制中,但毕竟还是太明显,并且早已被不持偏见的人民意识看穿了,而人民意识在认识这类问题上通常总是正确的。所有这些微妙的分裂和划分,都产生于资本和劳动的最初的分开和这一分开的完成,即人类分裂为资本家和工人。这一分裂正日益加剧,而且我们将看到,它必定会不断地加剧。但是,这种分开与我们考察过的土地同资本和劳动分开一样,归根结底是不可能的。我们根本无法确定在某种产品中土地、资本和劳动各占多少分量。这三个量是不可通约的。土地出产原材料,但这里并非没有资本和劳动;资本以土地和劳动为前提,而劳动至少以土地,在大多数场合还以资本为前提。这三者的作用截然不同,无法用任何第四种共同的尺度来衡量。因此,如果在当前的条件下,将收入在这三种要素之间进行分配,那就没有它们固有的尺度,而只有由一个完全异己的、对它们来说是偶然的尺度即竞争或者强者狡诈的权利来解决。地租包含着竞争;资本的利润只有由竞争决定,至于工资的情况怎样,我们立刻就会看到。

如果我们撇开私有制,那么所有这些反常的分裂就不会存在。利息和利润的差别也会消失;资本如果没有劳动、没有运动就是虚无。利润把自己的意义归结为资本在决定生产费用时置于天平上的砝码,它仍是资本所固有的部分,正如资本本身将回到它与劳动的最初统一体一样。

劳动是生产的主要要素,是“财富的源泉”\footnote{亚·斯密《国民财富的性质和原因的研究》1828年爱丁堡版第1卷第9—10页。——编者注},是人的自由活动,但很少受到经济学家的重视。正如资本已经同劳动分开一样,现在劳动又再度分裂了;劳动的产物以工资的形式与劳动相对立,它与劳动分开,并且通常又由竞争决定,因为,正如我们所看到的,没有一个固定的尺度来确定劳动在生产中所占的比重。只要我们消灭了私有制,这种反常的分离就会消失;劳动就会成为它自己的报酬,而以前被让渡的工资的真正意义,即劳动对于确定物品的生产费用的意义,也就会清清楚楚地显示出来。

我们知道,只要私有制存在一天,一切终究会归结为竞争。竞争是经济学家的主要范畴,是他最宠爱的女儿,他始终娇惯和爱抚着她,但是请看,在这里出现的是一张什么样的美杜莎的怪脸。

私有制的最直接的结果是生产分裂为两个对立的方面:自然的方面和人的方面,即土地和人的活动。土地无人施肥就会荒芜,成为不毛之地,而人的活动的首要条件恰恰是土地。其次,我们看到,人的活动又怎样分解为劳动和资本,这两方面怎样彼此敌视。这样,我们已经看到的是这三种要素的彼此斗争,而不是它们的相互支持;现在,我们还看到私有制使这三种要素中的每一种都分裂。一块土地与另一块土地对立,一个资本与另一个资本对立,一个劳动力与另一个劳动力对立。换句话说,因为私有制把每一个人隔离在他自己的粗陋的孤立状态中,又因为每个人和他周围的人有同样的利益,所以土地占有者敌视土地占有者,资本家敌视资本家,工人敌视工人。在相同利益的敌对状态中,正是由于利益的相同,人类目前状态的不道德已经达到极点,而这个极点就是竞争。

竞争的对立面是垄断。垄断是重商主义者战斗时的呐喊,竞争是自由主义经济学家厮打时的吼叫。不难看出,这个对立面也是完全空洞的东西。每一个竞争者,不管他是工人,是资本家,或是土地占有者,都必定希望取得垄断地位。每一个较小的竞争者群体都必定希望为自己取得垄断地位来对付所有其他的人。竞争建立在利益基础上,而利益又引起垄断;简言之,竞争转为垄断。另一方面,垄断挡不住竞争的洪流;而且,它本身还会引起竞争,正如禁止输入或高额关税直接引起走私一样。竞争的矛盾和私有制本身的矛盾是完全一样的。单个人的利益是要占有一切,而群体的利益是要使每个人所占有的都相等。因此,普遍利益和个人利益是直接对立的。竞争的矛盾在于:每个人都必定希望取得垄断地位,可是群体本身却因垄断而一定遭受损失,因此一定要排除垄断。此外,竞争已经以垄断即所有权的垄断为前提——这里又暴露出自由主义者的虚伪——,而且只要所有权的垄断存在着,垄断的所有权也同样是正当的,因为垄断一经存在,它就是所有权。可见,攻击小的垄断,保留根本的垄断,这是多么可鄙的不彻底啊!前面我们已经提到过经济学家的论点,凡是无法垄断的东西就没有价值,因此,凡是不容许垄断的东西就不可能卷入这个竞争的斗争;如果我们再把经济学家的这个论点引到这里来,那么我们关于竞争以垄断为前提的论断,就被证明是完全正确的了。

竞争的规律是:需求和供给始终力图互相适应,而正因为如此,从未有过互相适应。双方又重新脱节并转化为尖锐的对立。供给总是紧跟着需求,然而从来没有达到过刚好满足需求的情况;供给不是太多,就是太少,它和需求永远不相适应,因为在人类的不自觉状态下,谁也不知道需求和供给究竟有多大。如果需求大于供给,价格就会上涨,因而供给似乎就会兴奋起来;只要市场上供给增加,价格又会下跌,而如果供给大于需求,价格就会急剧下跌,因而需求又被激起。情况总是这样;从未有过健全的状态,而总是兴奋和松弛相更迭——这种更迭排斥一切进步——一种达不到目的的永恒波动。这个规律永远起着平衡的作用,使在这里失去的又在那里获得,因而经济学家非常欣赏它。这个规律是他最大的荣誉,他简直百看不厌,甚至在一切可能的和不可能的条件下都对它进行观察。然而,很明显,这个规律是纯自然的规律,而不是精神的规律。这是一个产生革命的规律。经济学家用他那绝妙的供求理论向你们证明“生产永远不会过多”\footnote{亚·斯密《国民财富的性质和原因的研究》1828年爱丁堡版第1卷第97页。——编者注},而实践却用商业危机来回答,这种危机就像彗星一样定期再现,在我们这里现在是平均每五年到七年发生一次。 80年来,这些商业危机像过去的大瘟疫一样定期来临,而且它们造成的不幸和不道德比大瘟疫所造成的更大(参看威德《中等阶级和工人阶级的历史》1835年伦敦版第211页)。当然,这些商业革命证实了这个规律,完完全全地证实了这个规律,但不是用经济学家想使我们相信的那种方式证实的。我们应该怎样理解这个只有通过周期性的革命才能为自己开辟道路的规律呢?这是一个以当事人的无意识活动为基础的自然规律。如果生产者自己知道消费者需要多少,如果他们把生产组织起来,并且在他们中间进行分配,那么就不会有竞争的波动和竞争引起危机的倾向了。你们有意识地作为人,而不是作为没有类意识的分散原子进行生产吧,你们就会摆脱所有这些人为的无根据的对立。但是,只要你们继续以目前这种无意识的、不假思索的、全凭偶然性摆布的方式来进行生产,那么商业危机就会继续存在;而且每一次接踵而来的商业危机必定比前一次更普遍,因而也更严重,必定会使更多的小资本家变穷,使专靠劳动为生的阶级人数以增大的比例增加,从而使待雇劳动者的人数显著地增加——这是我们的经济学家必须解决的一个主要问题——,最后,必定引起一场社会革命,而这一革命,经济学家凭他的书本知识是做梦也想不到的。

由竞争关系造成的价格永恒波动,使商业完全丧失了道德的最后一点痕迹。至于价值就无须再谈了。这种似乎非常重视价值并以货币的形式把价值的抽象推崇为一种特殊存在物的制度,本身就通过竞争破坏着一切物品所固有的任何价值,而且每日每时改变着一切物品相互的价值关系。在这个漩涡中,哪里还可能有建立在道德基础上的交换呢?在这种持续地不断涨落的情况下,每个人都必定力图碰上最有利的时机进行买卖,每个人都必定会成为投机家,就是说,都企图不劳而获,损人利己,算计别人的倒霉,或利用偶然事件发财。投机者总是指望不幸事件,特别是指望歉收,他们利用一切事件,例如,当年的纽约大火灾\footnote{指1835年12月16日在纽约发生的火灾。——编者注 };而不道德的顶点还是交易所中有价证券的投机,这种投机把历史和历史上的人类贬低为那种用来满足善于算计或伺机冒险的投机者的贪欲的手段。但愿诚实的、“正派的”商人不以“我感谢你上帝”等表面的虔诚形式摆脱交易所投机。这种商人和证券投机者一样可恶,他也同他们一样地投机倒把,他必须投机倒把,竞争迫使他这样做,所以他的买卖也与证券投机者的勾当一样不道德。竞争关系的真谛就是消费力对生产力的关系。在一种与人类相称的状态下,不会有除这种竞争之外的别的竞争。社会应当考虑,靠它所支配的资料能够生产些什么,并根据生产力和广大消费者之间的这种关系来确定,应该把生产提高多少或缩减多少,应该允许生产或限制生产多少奢侈品。但是,为了正确地判断这种关系,判断从合理的社会状态下能期待的生产力提高的程度,请读者参看英国社会主义者的著作\footnote{约·弗·布雷《劳动的不公正现象及其解决办法,或强权时代和公正时代》1839年利兹版;威·汤普森《最能促进人类幸福的财富分配原理的研究》1824年伦敦版;约·瓦茨《政治经济学家的事实和臆想:科学原则述评,去伪存真》1842年曼彻斯特—伦敦版。——编者注}并部分地参看傅立叶的著作\footnote{沙·傅立叶《关于四种运动和普遍命运的理论》1841年巴黎第2版和《经济的和协作的新世界,或按情欲分类的引人入胜的和合乎自然的劳动方式的发现》1829年巴黎版。——编者注}。

在这种情况下,主体的竞争,即资本对资本、劳动对劳动的竞争等等,被归结为以人的本性为基础并且到目前为止只有傅立叶作过差强人意的说明的竞赛\footnote{沙·傅立叶《关于四种运动和普遍命运的理论》1841年巴黎第2版第175、244—245、265和434—436页。——编者注},这种竞赛将随着对立利益的消除而被限制在它特有的和合理的范围内。

资本对资本、劳动对劳动、土地对土地的斗争,使生产陷于高烧状态,使一切自然的合理的关系都颠倒过来。要是资本不最大限度地展开自己的活动,它就经不住其他资本的竞争。要是土地的生产力不经常提高,耕种土地就会无利可获。要是工人不把自己的全部力量用于劳动,他就对付不了自己的竞争者。总之,卷入竞争斗争的人,如果不全力以赴,不放弃一切真正人的目的,就经不住这种斗争。一方的这种过度紧张,其结果必然是另一方的松弛。在竞争的波动不大,需求和供给、消费和生产几乎彼此相等的时候,在生产发展过程中必定会出现这样一个阶段,在这个阶段,生产力大大过剩,结果,广大人民群众无以为生,人们纯粹由于过剩而饿死。长期以来,英国就处于这种荒诞的状况中,处于这种极不合理的情况下。如果生产波动得比较厉害——这是这种状态的必然结果——,那么就会出现繁荣和危机、生产过剩和停滞的反复交替。经济学家从来就解释不了这种怪诞状况;为了解释这种状况,他发明了人口论,这种理论和当时这种贫富矛盾同样荒谬,甚至比它更荒谬。经济学家不敢正视真理,不敢承认这种矛盾无非是竞争的结果,因为否则他的整个体系就会垮台。

在我们看来,这个问题很容易解释。人类支配的生产力是无法估量的。资本、劳动和科学的应用,可以使土地的生产能力无限地提高。按照最有才智的经济学家和统计学家的计算(参看艾利生的《人口原理》第1卷第 1、2章),“人口过密”的大不列颠在十年内,将使粮食生产足以供应六倍于目前人口的需要。资本日益增加,劳动力随着人口的增长而增长,科学又日益使自然力受人类支配。这种无法估量的生产能力,一旦被自觉地运用并为大众造福,人类肩负的劳动就会很快地减少到最低限度。要是让竞争自由发展,它虽然也会起同样的作用,然而是在对立之中起作用。一部分土地进行精耕细作,而另一部分土地——大不列颠和爱尔兰的3 000万英亩好地——却荒芜着。一部分资本以难以置信的速度周转,而另一部分资本却闲置在钱柜里。一部分工人每天工作 14或16小时,而另一部分工人却无所事事,无活可干,活活饿死。或者,这种分立现象并不同时发生:今天生意很好,需求很大,这时,大家都工作,资本以惊人的速度周转着,农业欣欣向荣,工人干得累倒了;而明天停滞到来,农业不值得费力去经营,大片土地荒芜,资本在正在流动的时候凝滞,工人无事可做,整个国家因财富过剩、人口过剩而备尝痛苦。

经济学家不能承认事情这样发展是对的,否则,他就得像上面所说的那样放弃自己的全部竞争体系,就得认识到自己把生产和消费对立起来、把人口过剩和财富过剩对立起来是荒诞无稽的。但是,既然事实是无法否认的,为了使这种事实与理论一致,就发明了人口论。

这种学说的创始人马尔萨斯断言,人口总是威胁着生活资料,一当生产增加,人口也以同样比例增加,人口固有的那种其繁衍超过可支配的生活资料的倾向,是一切贫困和罪恶的原因。因此,在人太多的地方,就应当用某种方法把他们消灭掉:或者用暴力将他们杀死,或者让他们饿死。可是这样做了以后,又会出现一个空隙,这个空隙又会马上被另一次繁衍的人口填满,于是,以前的贫困又开始到来。据说在任何条件下都是如此,不仅在文明的状态下,而且在自然的状态下都是如此;新荷兰\footnote{澳大利亚的旧称。——编者注}平均每平方英里只有一个野蛮人,却也和英国一样,深受人口过剩的痛苦。简言之,要是我们愿意首尾一贯,那我们就得承认:当地球上只有一个人的时候,就已经人口过剩了。从这种阐述得出的结论是:正因为穷人是过剩人口,所以,除了尽可能减轻他们饿死的痛苦,使他们相信这是无法改变的,他们整个阶级的唯一出路是尽量减少生育,此外就不应该为他们做任何事情;或者,如果这样做不行,那么最好还是像“马尔库斯”所建议的那样,建立一种国家机构,用无痛苦的办法把穷人的孩子杀死;按照他的建议,每一个工人家庭只能有两个半小孩,超过此数的孩子用无痛苦的办法杀死。施舍被认为是犯罪,因为这会助长过剩人口的增长;但是,把贫穷宣布为犯罪,把济贫所变为监狱——这正是英国通过“自由的”新济贫法已经做的——,却算是非常有益的事情。的确,这种理论很不符合圣经关于上帝及其创造物完美无缺的教义,但是“动用圣经来反驳事实,是拙劣的反驳!”\footnote{托·卡莱尔《宪章运动》1840年伦敦版第109页。——编者注}

我是否还需要更详尽地阐述这种卑鄙无耻的学说,这种对自然和人类的恶毒诬蔑,并进一步探究其结论呢?在这里我们终于看到,经济学家的不道德已经登峰造极。一切战争和垄断制度所造成的灾难,与这种理论相比,又算得了什么呢?要知道,正是这种理论构成了自由派的自由贸易体系的拱顶石,这块石头一旦坠落,整个大厦就倾倒。因为竞争在这里既然已经被证明是贫困、穷苦、犯罪的原因,那么谁还敢对竞争赞一词呢?

艾利生在上面引用过的著作中动摇了马尔萨斯的理论,他诉诸土地的生产力,并用以下的事实来反对马尔萨斯的原理:每一个成年人能够生产出多于他本人消费所需的东西。如果不存在这一事实,人类就不可能繁衍,甚至不可能生存;否则成长中的一代依靠什么来生活呢?\footnote{阿·艾利生《人口原理及其和人类幸福的关系》1840年爱丁堡—伦敦版第33—82页。——编者注}可是,艾利生没有深入事物的本质,因而他最后也得出了同马尔萨斯一样的结论。他虽然证明了马尔萨斯的原理是不正确的,但未能驳倒马尔萨斯据以提出他的原理的事实。

如果马尔萨斯不这样片面地看问题,那么他必定会看到,人口过剩或劳动力过剩是始终与财富过剩、资本过剩和地产过剩联系着的。只有在整个生产力过大的地方,人口才会过多。从马尔萨斯写作时起\footnote{托·罗·马尔萨斯《人口原理》第1版于1798年在伦敦出版。——编者注},任何人口过剩的国家的情况,尤其是英国的情况,都极其明显地证实了这一点。这是马尔萨斯应当从总体上加以考察的事实,而对这些事实的考察必然会得出正确的结论;他没有这样做,而是只选出一个事实,对其他事实不予考虑,因而得出荒谬的结论。他犯的第二个错误是把生活资料和就业手段混为一谈。人口总是威胁着就业手段,有多少人能够就业,就有多少人出生,简言之,劳动力的产生迄今为止由竞争的规律来调节,因而也同样要经受周期性的危机和波动,这是事实,确定这一事实是马尔萨斯的功绩。\footnote{托·罗·马尔萨斯《人口原理》1826年伦敦版第1卷第18—21页。——编者注}然而,就业手段并不就是生活资料。就业手段由于机器力和资本的增加而增加,这是仅就其最终结果而言;而生活资料,只要生产力稍有提高,就立刻增加。这里暴露出经济学的一个新的矛盾。经济学家所说的需求不是现实的需求,他所说的消费只是人为的消费。在经济学家看来,只有能够为自己取得的东西提供等价物的人,才是现实的需求者,现实的消费者。但是,如果事实是这样:每一个成年人生产的东西多于他本人所消费的东西;小孩像树木一样能够绰绰有余地偿还花在他身上的费用——难道这不是事实?——,那么就应该认为,每一个工人必定能够生产出远远多于他所需要的东西,因此,社会必定会乐意供给他所必需的一切;同时也应该认为,大家庭必定是非常值得社会向往的礼物。但是,由于经济学家观察问题很粗糙,除了以可触摸的现金向他支付的东西以外,他不知道还有任何别的等价物。他已深陷在自己的对立物中,以致连最令人信服的事实也像最科学的原理一样使他无动于衷。

我们干脆用扬弃矛盾的方法消灭矛盾。只要目前对立的利益能够融合,一方面的人口过剩和另一方面的财富过剩之间的对立就会消失,关于一国人民纯粹由于富裕和过剩而必定饿死这种不可思议的事实,这种比一切宗教中的一切奇迹的总和更不可思议的事实就会消失,那种认为土地无力养活人们的荒谬见解也就会消失。这种见解是基督教经济学的顶峰,——而我们的经济学本质上是基督教经济学,这一点我可以用任何命题和任何范畴加以证明,这个工作在适当的时候我会做的;马尔萨斯的理论只不过是关于精神和自然之间存在着矛盾和由此而来的关于二者的堕落的宗教教条在经济学上的表现。我希望也在经济学领域揭示这个对宗教来说并与宗教一起早就解决了的矛盾的虚无性。同时,如果马尔萨斯理论的辩护人事先不能用这种理论的原则向我解释,一国人民怎么能够纯粹由于过剩而饿死,并使这种解释同理性和事实一致起来,那我就不会认为这种辩护是站得住脚的。

可是,马尔萨斯的理论却是一个推动我们不断前进的、绝对必要的中转站。我们由于他的理论,总的来说由于经济学,才注意到土地和人类的生产力,而且我们在战胜了这种经济学上的绝望以后,就保证永远不惧怕人口过剩。我们从马尔萨斯的理论中为社会变革汲取到最有力的经济论据,因为即使马尔萨斯完全正确,也必须立刻进行这种变革,原因是只有这种变革,只有通过这种变革来教育群众,才能够从道德上限制繁殖本能,而马尔萨斯本人也认为这种限制是对付人口过剩的最有效和最简易的办法。\footnote{托·罗·马尔萨斯《人口原理》1826年伦敦版第2卷第255—269页。——编者注}我们由于这个理论才开始明白人类的极端堕落,才了解这种堕落依存于竞争关系;这种理论向我们指出,私有制如何最终使人变成了商品,使人的生产和消灭也仅仅依存于需求;它由此也指出竞争制度如何屠杀了并且每日还在屠杀着千百万人;这一切我们都看到了,这一切都促使我们要用消灭私有制、消灭竞争和利益对立的办法来消灭这种人类堕落。

然而,为了驳倒对人口过剩普遍存在的恐惧所持的根据,让我们再回过来谈生产力和人口的关系。马尔萨斯把自己的整个体系建立在下面这种计算上:人口按几何级数1+2+4+8+16+32……增加,而土地的生产力按算术级数1+2+3+4+5+6增加。\footnote{同上,第1卷第11页。——编者注 }差额是明显的、触目惊心的,但这是否对呢?在什么地方证明过土地的生产能力是按算术级数增加的呢?土地的扩大是受限制的。好吧。在这个面积上使用的劳动力随着人口的增加而增加。即使我们假定,由于增加劳动而增加的收获量,并不总是与劳动成比例地增加,这时仍然还有一个第三要素,一个对经济学家来说当然是无足轻重的要素——科学,它的进步与人口的增长一样,是永无止境的,至少也是与人口的增长一样快。仅仅一门化学,光是汉弗莱·戴维爵士和尤斯图斯·李比希两人,就使本世纪的农业获得了怎样的成就?可见科学发展的速度至少也是与人口增长的速度一样的;人口与前一代人的人数成比例地增长,而科学则与前一代人遗留的知识量成比例地发展,因此,在最普通的情况下,科学也是按几何级数发展的。而对科学来说,又有什么是做不到的呢?当“密西西比河流域有足够的荒地可容下欧洲的全部人口”\footnote{约·瓦茨《政治经济学家的事实和臆想》1842年曼彻斯特—伦敦版第21页。——编者注}的时候,当地球上的土地才耕种了三分之一,而这三分之一的土地只要采用现在已经人所共知的改良耕作方法,就能使产量提高五倍、甚至五倍以上的时候,谈论什么人口过剩,岂不是非常可笑的事情。

这样,竞争就使资本与资本、劳动与劳动、土地占有与土地占有对立起来,同样又使这些要素中的每一个要素与其他两个要素对立起来。力量较强的在斗争中取得胜利。要预卜这个斗争的结局,我们就得研究一下参加斗争的各方的力量。首先,土地占有或资本都比劳动强,因为工人要生活就得工作,而土地占有者可以靠地租过活,资本家可以靠利息过活,万不得已时,也可以靠资本或资本化了的土地占有过活。其结果是:劳动得到的仅仅是最必需的东西,仅仅是一点点生活资料,而大部分产品则为资本和土地占有所得。此外,较强的工人把较弱的工人,较大的资本把较小的资本,较大的土地占有把小土地占有从市场上排挤出去。实践证实了这个结果。大家都知道,大厂主和大商人比小厂主和小商人占优势,大土地占有者比只有一摩尔根土地的占有者占优势。其结果是:在通常情况下,按照强者的权利,大资本和大土地占有吞并小资本和小土地占有,就是说,产生了财产的集中。在商业危机和农业危机时期,这种集中就进行得更快。一般说来,大的财产比小的财产增长得更快,因为从收入中作为占有者的费用所扣除的部分要小得多。这种财产的集中是一个规律,它与所有其他的规律一样,是私有制所固有的;中间阶级必然越来越多地消失,直到世界分裂为百万富翁和穷光蛋、大土地占有者和贫穷的短工为止。任何法律,土地占有的任何分割,资本的任何偶然的分裂,都无济于事,这个结果必定会产生,而且就会产生,除非在此之前全面变革社会关系、使对立的利益融合、使私有制归于消灭。

作为当今经济学家主要口号的自由竞争,是不可能的事情。垄断至少具有使消费者不受欺骗的意图,虽然它不可能实现这种意图。消灭垄断就会为欺骗敞开大门。你们说,竞争本身是对付欺骗的办法,谁也不会去买坏的东西;照这样说来,每个人都必须是每一种商品的行家,而这是不可能的,由此可见,垄断是必要的,这种必要性也在许多商品中表现出来。药房等等必须实行垄断。最重要的商品即货币恰好最需要垄断。每当流通手段不再为国家所垄断的时候,这种手段就引起商业危机,因此,英国的经济学家,其中包括威德博士,也认为在这里有实行垄断的必要。\footnote{约·威德《中等阶级和工人阶级的历史》1835年伦敦第3版第152—160页。——编者注 }但是,垄断也不能防止假币。随便你站在问题的哪一方面,一方面的困难与另一方面的困难都不相上下。垄断引起自由竞争,自由竞争又引起垄断;因此,二者一定都失败,而且这些困难只有在消灭了产生这二者的原则时才能消除。

竞争贯穿在我们的全部生活关系中,造成了人们今日所处的相互奴役状况。竞争是强有力的发条,它一再促使我们的日益陈旧而衰退的社会秩序,或者更正确地说,无秩序状况活动起来,但是,它每努力一次,也就消耗掉一部分日益衰败的力量。竞争支配着人类在数量 上的增长,也支配着人类在道德上的进步。谁只要稍微熟悉一下犯罪统计,他就会注意到,犯罪行为按照特有的规律性年年增加,一定的原因按照特有的规律性产生一定的犯罪行为。工厂制度的扩展到处引起犯罪行为的增加。我们能够精确地预计一个大城市或者一个地区每年会发生的逮捕、刑事案件,以至凶杀、抢劫、偷窃等事件的数字,在英国就常常这样做。这种规律性证明犯罪也受竞争支配,证明社会产生了犯罪的需求,这个需求要由相应的供给来满足;它证明由于一些人被逮捕、放逐或处死所形成的空隙,立刻会有其他的人来填满,正如人口一有空隙立刻就会有新来的人填满一样;换句话说,它证明了犯罪威胁着惩罚手段,正如人口威胁着就业手段一样。别的且不谈,在这种情况下对罪犯的惩罚究竟公正到什么程度,我让我的读者去判断。我认为这里重要的是:证明竞争也扩展到了道德领域,并表明私有制使人堕落到多么严重的地步。

在资本和土地反对劳动的斗争中,前两个要素比劳动还有一个特殊的优越条件,那就是科学的帮助,因为在目前情况下连科学也是用来反对劳动的。例如,几乎一切机械发明,尤其是哈格里沃斯、克朗普顿和阿克莱的棉纺机,都是由于缺乏劳动力而引起的。对劳动的渴求导致发明的出现,发明大大地增加了劳动力,因而降低了对人的劳动的需求。1770年以来英国的历史不断地证明了这一点。棉纺业中最近的重大发明——自动走锭纺纱机——就完全是由于对劳动的需求和工资的提高引起的;这项发明使机器劳动增加了一倍,从而把手工劳动减少了一半,使一半工人失业,因而也就降低另一半工人的工资;这项发明破坏了工人对工厂主的反抗,摧毁了劳动在坚持与资本作力量悬殊的斗争时的最后一点力量(参看尤尔博士《工厂哲学》第2卷\footnote{安·尤尔《工厂哲学:或论大不列颠工厂制度的科学、道德和商业的经济》1835年伦敦修订第2版第366—373页。——编者注})。诚然,经济学家说,归根结底,机器对工人是有利的,因为机器能够降低生产费用,因而替产品开拓新的更广大的市场,这样,机器最终还能使失业工人重新就业。这完全正确,但是,劳动力的生产是受竞争调节的;劳动力始终威胁着就业手段,因而在这些有利条件出现以前就已经有大量寻求工作的竞争者等待着,于是有利的情况形同虚构,而不利的情况,即一半工人突然被剥夺生活资料而另一半工人的工资被降低,却决非虚构,这一点为什么经济学家就忘记了呢?发明是永远不会停滞不前的,因而这种不利的情况将永远继续下去,这一点为什么经济学家就忘记了呢?由于我们的文明,分工无止境地增多,在这种情况下,一个工人只有在一定的机器上被用来做一定的细小的工作才能生存,成年工人几乎在任何时候都根本不可能从一种职业转到另一种新的职业,这一点为什么经济学家又忘记了呢?

考虑到机器的作用,我有了另一个比较远的题目即工厂制度;但是,现在我既不想也没有时间来讨论这个题目。不过,我希望不久能够有机会来详细地阐述这个制度的极端的不道德,并且无情地揭露经济学家在这里表现得十分出色的那种伪善。 

\newpage

\subsection{2.笔记}
恩格斯认为国民经济学的产生就是商业扩展的结果,是一种自然的现象,是现实的商业发展的理论化的表现。而这种早期的理论表现便是重商主义。

恩格斯指出,\textbf{贸易差额论}是整个重商主义体系的要点。恩格斯说,要从“纯粹人的、普遍的基础出发来看问题”。但是“纯粹人的、普遍的”基础本身或许就不存在(笔者注)。

恩格斯认为在批判国民经济学时要研究它的基本范畴,揭露自由贸易体系所产生的矛盾。(批判要深入实质,从内部瓦解敌人)

在这里,恩格斯提出了著名的“两个和解”。原文是这么说的:

“\begin{fangsong}
经济学家自己也不知道他在为什么服务。他不知道,他的全部利己的论辩只不过构成人类普遍进步的链条中的一环。他不知道,他瓦解一切私人利益只不过替我们这个世纪面临的大转变,即人类与自然的和解以及人类本身的和解开辟道路。    
\end{fangsong}”

事实上,恩格斯想要论述的是,在私有制下,生产分裂为两个对立的方面:自然的方面与人的方面,即土地与人的活动。问题的本质不在于要实现“两个和解”,而在于变革那使得人与自然之间的对立存在的社会条件。
\newpage
\section{1844年经济学哲学手稿(节选)}
\subsection{1.原文节选}
\subsubsection{私有财产的关系}
〔……〕〔XL〕构成他的资本的利息。因此,资本是完全失去自身的人这种情况在工人身上主观地存在着,正像劳动是失去自身的人这种情况在资本身上客观地存在着一样。但是,工人不幸而成为一种活的、因而是贫困的资本,这种资本只要一瞬间不劳动便失去自己的利息,从而也失去自己的生存条件。作为资本,工人的价值按照需求和供给而增长,而且,从肉体上来说,他的存在、他的生命也同其它任何商品一样,过去和现在都被看成是商品的供给。工人生产资本,资本生产工人,因而工人生产自身,而且人做为工人、做为商品就是这整个运动的产物。人只不过是工人,并且做为工人,他的特性只有在这些特性对异己的资本来说是存在的时候才存在。但是,因为资本和工人彼此是异己的,从而处于漠不关心的、外部的和偶然的相互关系中,所以这种异己性也必然现实地表现出来。因此,资本一但想到-不管是必然地还是任意地想到-不再对工人存在,工人自己对自己说来便不再存在:他没有工作,因而也没有工资,并且因为他不是作为人,而是作为工人存在,所以他就会被人埋葬,会饿死,等等。工人有当他对自己作为资本存在的时候,才作为工人存在;而他只有当某种资本对他存在的时候,才作为资本存在。资本的存在便是他的存在、他的生活,资本的存在以一种他无法干预的方式来规定他的生活的内容。因此,国民经济学不知道有失业的工人,即处于这种劳动关系之外的劳动人。小偷、骗子、乞丐,失业的、快饿死的、贫穷的和犯罪的劳动人,他们都是些在国民经济学看来并不存在,而只有在其它人眼中,在医生、法官、掘墓人、乞丐管理人等等的眼中才存在的人物;他们是一些国民经济学领域之外游荡的幽灵。因此,在国民经济学看来,工人的需要不过是维持工人在劳动期间的生活的需要,而且只限于保持工人后代不致死绝的程度。因此,工资就与其它任何生产工具的保养和维修,与资本连同利息的再生产所需要的一般资本的消费,与为了保持车轮运转而加的润滑油,具有完全相同的意义。可见,工资是资本和资本家的必要费用之一,并且不得不超出这个必要的需要。因此,英国工厂主在1834年实行济贫法以前,把工人靠济贫税得到的社会救济金从他的工资中扣除,并且把这种救济金看作工资的一个组成部分,这种做法是完全合乎逻辑的。

生产不仅把人当作商品、当作商品人、当作具有商品的规定的人生产出来;它依照这个规定把人当作即在精神上又在肉体上非人化的存在物生产出来。-工人和资本家的不道德、退化、愚钝。-这种生产的产品是自我意识的和自主活动的商品……商品人……李嘉图、穆勒等人比斯密和萨伊进了一大步,他们把人的存在-人这种商品的或高或低的生产率-说成是无关紧要的,甚至是有害的。在他们看来,生产的真正目的不是一笔资本养活多少工人,而是它带来多少利息,每年总共积攒多少钱。同样,现代英国国民经济学的一个合乎逻辑的大进步是,它把劳动提高为国民经济学的惟一原则,同时十分清楚地揭示了工资和资本利息之间的反比例关系,指出资本家通常只有通过降低工资才能增加收益,反之则降低收益。不是对消费者诈取,而是资本家和工人彼此诈取,才是正常的关系。

私有财产的关系潜在地包含着作为劳动的私有财产的关系和作为资本的私有财产的关系,以及这两种表现的相互关系。一方面是作为劳动,即作为对自身、对人和自然界因而也对意识和生命表现说来完全异己的活动的人的活动的生产,是人作为单纯的劳动人的抽象存在,因而这种劳动人每天都可能由他的充实的无沦为绝对的无,沦为他的社会的从而也是现实的非存在。另一方面是作为资本的人的活动的对象的生产,在这里,对象的一切自然的社会的规定性都消失了,在这里私有财产丧失了自己的自然的和社会的特质(因而也丧失了一切政治的和社会的幻象,甚至连表面上的人的关系也没有了),在这里同一个资本在各种不同的自然的和社会的存在中始终是同一的,而完全不管它的现实内容如何。劳动和资本的这种对立一到达极端,就必然成为全部私有财产关系的顶点、最高阶段和灭亡。因此,现代英国国民经济学的又一重大成就是:它指明了地租是最坏耕地的利息和最好耕地的利息之间的差额,揭示了土地所有者的浪漫主义想象-他的所谓社会重要性和他的利益同社会利益的一致性,而这一点是亚当·斯密继重农学派之后主张过的;它预料到并且准备了这样一个现实的运动:使土地所有者变成极其普通的、平庸的资本家,从而使对立简化和尖锐化,并加速这种对立的消灭。这样一来,作为土地的土地,作为地租的地租,就失去自己的等级的差别而变成毫无意义的,或者毋宁说,只表示货币意义的资本和利息。

资本和土地的差别,利润和地租的差别,这二者和工资的差别,工业和农业之间、私有的不动产和动产之间的差别,仍然是历史的差别,而不是基于事物本质的差别。这种差别是资本和劳动之间的对立形成和产生的一个固定环节。同不动的地产相反,在工业等等中只表现出工业产生的方式以及工业在其中得到发展的那个与农业的对立。这种差别只要在下述情况下就作为特殊种类的活动,作为一个本质的、重要的、包括全部生活的差别而存在:工业(城市生活)同地产(贵族生活(封建生活))对立而形成,并且工业本身在垄断、行会、同业公会和社团等形式中还带有自己对立面的封建性质;而在这些形式的规定内,劳动还具有表面上的社会意义,现实的共同体的意义,还没有达到对自己的内容漠不关心,没有达到完全自为的存在的地步,就是说,还没有从其它一切存在中抽象出来,从而也还没有成为获得行动自由的资本。

但是,获得行动自由的、本身有单独构成的工业和获得行动自由的资本是劳动的必然发展。工业对它的对立面的支配立即表现在作为真正工业活动的农业的产生上,而过去农业是把主要工作交给土地和耕种这块土地的奴隶去做的。随着奴隶转化为自由工人即雇佣工人,地主本身便实际上转化为工厂主、资本家,而这种转化最初是通过租地农场主这个中介环节实现的。但是,租地农场主是土地所有者的代表,是土地所有者的公开秘密;只有依靠租地农场主,土地所有者才有经济上的存在,才能作为私有者存在,-因为他的土地的地租只有依靠租地农场主的竞争才能获得。因此,地主通过租地农场主本质上已经变成普通的资本家。而这种情况也必然在现实中发生:经营农业的资本家即租地农场主必然要成为地主,或者相反。租地农场主的以产业形式牟利就是土地所有者的以产业形式牟利,因为前者的存在设定后者的存在。

但是,当土地所有者和资本家想起自己的对立面的产生,回想起自己的来历时:土地所有者把资本家看做自己的骄傲起来的、发了财的、昨天的奴隶,并且看出他对自己这个资本家的威胁;而资本家则把土地所有者看作自己游手好闲的、残酷无情的(自私自利的)、昨天的主人;他知道土地所有者会使他这个资本家受损害,虽然土地所有者今天的整个社会地位、财产和享受都应归功于工业;资本家把土地所有者看成自由的工业和摆脱任何自然规定的自由的资本的直接对立面。他们之间的这种对立是极其激烈的,并且双方各自说出对方的真相。只要看一看不动产对动产的攻击,并且反过来看一看,动产对不动产的攻击,就对双方的卑鄙行有一个明确的概念。土地所有者炫耀他的财产的贵族渊源、封建的往昔纪念(怀旧)、他的诗意的回忆、他的耽于幻想的气质、他的政治上的重要性等等,而如果他用国民经济学的语言来表达,那末他就会说:只有农业才是生产的。同时,他把自己的对手描绘为狡黠诡诈的,兜售叫卖的,吹毛求疵的,坑蒙拐骗的,贪婪成性的,见钱眼开的,图谋不轨的,没有心干和丧尽天良的,背离社会和出卖社会利益的,放高利贷的,牵线撮合的,奴颜婢膝的,阿谀奉承的,圆滑世故的,招摇撞骗的,冷漠生硬的,制造、助长和纵容竞争、赤贫和犯罪的、败坏一切社会纲纪的,没有廉耻、没有原则、没有诗意、没有实体、心灵空虚的贪财恶棍(此外,见其中的重农学派贝尔加斯的著作,对他,卡米耶·德穆兰在自己的杂志《法国革命和布拉班特革命》中曾经严厉地批评过;并见冯·芬克、兰齐措勒、哈勒、莱奥\footnote{见爱好夸张的老年黑格尔派神学家丰克的著作,他满眼含泪,按照莱奥先生的说法讲述了在废除农奴制时一个奴隶如何不肯不再充当贵族的财产。并见尤斯图斯·莫泽尔的《爱国主义的幻想》,这些幻想的特色是它们一刻也没有超出循规蹈矩的庸人的那种小市民的、“家传的”、平庸的狭隘眼界;虽然如此,它们仍不失为纯粹的幻想。这个矛盾也使这些幻想如此投合德国人的口味。}、科泽加滕。)

动产也显示工业和运动的奇迹,它是现代之子,现代的合法的嫡子;它很遗憾自己的对手是一个不理解自己本质(而这是完全对的),想用粗野的、不道德的暴力和农奴制来代替道德的资本和自由的劳动的蠢人;它把他描绘成用率直坦诚、一本正经、为普遍利益服务、坚贞不渝这些假面具来掩盖缺乏活动能力、贪得无餍的享乐欲、自私自利、斤斤计较和居心不良的唐·吉诃德。它宣布他的对手是诡计多端的垄断者;它从历史发展上并用嘲讽的口气历数他的以罗曼蒂克的城堡为作坊的下流、残忍、挥霍、淫逸、寡廉鲜耻、无法无天和大逆不道,来给他的怀旧、他的诗意、他的幻想浇冷水。

据说,动产已经使人民获得了政治的自由,解脱了束缚市民社会的桎梏,把各领域彼此联成一体,创造了博爱的商业、纯粹的道德、温文尔雅的教养;它给人民以文明的需要来代替粗陋的需要,并提供了满足需要的手段;而土地所有者这个无所事事的、只会碍事的粮食投机商则抬高人民最必须的生活资料的价格,从而迫使资本家提高工资而不能提高生产力;因此,土地所有者妨碍国民年收入的增长,阻碍资本的积累,从而减少人民就业和国家增加财富的可能性;最后使这种可能性完全消失,引起普遍的衰退,并且像高利贷一样剥削现代文明的一切利益,而没有对它做丝毫贡献,甚至不放弃自己的封建偏见。最后,让土地所有者来看一看自己的租地农场主-对土地所有者来说,农业和土地本身仅仅作为赐给他的财源而存在,-并且让他说说,他是不是这样一个一本正经的、非凡的、狡猾的无赖:不管他曾怎样反对工业和商业,也不管他曾怎样絮絮叨叨地数说历史的回忆以及伦理的和政治的目的,他早已在心理并且在实际上属于自由的工业和可爱的商业了。土地所有者实际上提出的有利于自己的一切,只有用在耕作者(资本家和雇农)身上才是符合事实的,而土地所有者不如说是耕作者的敌人;因此,土地所有者作了不利于自身的论证。据说,没有资本,地产就是死的、无价值的物质。资本的文明的胜利恰恰在于,资本发现并促进使人的劳动代替死的物而成为财富的源泉。(见保罗·路易·库里埃、圣西门、加尼耳、李嘉图、穆勒、麦克库洛赫、德斯杜特·德·特拉西和米歇尔·舍伐利埃的著作。)

从现实的发展进程中(这里插一句)必然产生出资本家对土地所有者的胜利,即发达的私有财产对不发达的、不完全的私有财产的胜利,正如一般说来运动必然战胜不动,公开的、自觉的卑鄙行为必然战胜隐蔽的、不自觉的卑鄙行为,贪财欲必然战胜享乐欲,直认不讳的、老于世故的、孜孜不息的、精明机敏的开明利己主义必然战胜眼界狭隘的的、一本正经的、懒散的、幻想的、迷信利己主义,货币必然战胜其它形式的私有财产一样。

那些多少觉察到完成的自由工业、完成的纯粹道德和完成的博爱商业的危险的国家,企图阻止地产变成资本,却完全白费力气。

与资本不同,地产是还带有地域的和政治的偏见的私有财产、资本,是还没有完全摆脱同周围世界的纠缠而达到自身的资本,即还没有完成的资本。它必然要在它的世界发展过程中达到它的抽象的即纯粹的表现。

私有财产的关系是劳动、资本以及二者的关系。这个关系的各个成分所必定经历的运动是:

第一:二者直接的或间接的统一。

起初,资本和劳动还是统一的;后来,他们虽然分离和异化,却作为积极的条件而互相促进和互相推动。

〔第二〕:二者的对立。它们互相排斥;工人知道资本家是自己的非存在,反过来也是这样;每一方都力图剥夺另一方的存在。

〔第三〕:二者各自同自身对立。资本=积累的劳动=劳动。作为这样的东西,资本分解为自身和自己的利息,而利息又分解为利息和利润。资本家完全成为牺牲品。他沦为工人阶级,正像工人-但只是例外地-成为资本家一样。劳动是资本的要素,是资本的费用,因而,工资是资本的牺牲。

劳动分解为自身和工资。工人本身是资本、商品。

敌对性的相互对立。

\subsubsection{国民经济学中反映的私有财产的本质}
私有财产的主体本质,作为自为地存在着的活动、作为主体、作为个人的私有财产,就是劳动。因此,十分明显,只有把劳动视为自己的原则——亚当·斯密——,也就是说,不再认为私有财产仅仅是人之外的一种状态的国民经济学,只有这种国民经济学才应该被看成私有财产的现实能量和现实运动的产物(这种国民经济学是私有财产的在意识中自为地形成的独立运动,是现代工业本身),现代工业的产物;而另一方面,正是这种国民经济学促进并赞美了这种工业的能量和发展,使之变成意识的力量。因此,在这种揭示了-在私有制范围内-财富的主体本质的启蒙国民经济学看来,那些认为私有财产对人来说仅仅是对象性的本质的货币主义体系和重商主义体系地拥护者,是拜物教徒、天主教徒。所以,恩格斯有理由把亚当·斯密称作国民经济学的路德。正像路德认为宗教、信仰为外部世界的本质并以此反对天主教异教一样,正像他把宗教观念变成人的内在本质,从而扬弃了外在的宗教观念一样,正像他把教士移到俗人心中,因而否定了俗人之外存在的教士一样,由于私有财产体现为在人本身中,而人本身被认为是私有财产的本质,因而在人之外并且不依赖于人的财富,也就是只以外在方式来保存和保持的财富被扬弃了,换言之,财富这种外在的、无思想的对象性就被扬弃了,但正因为这个缘故,人本身被当成了私有财产的规定,就像在路德那里被当成了宗教的规定一样。因此,以劳动为原则的国民经济学,在承认人的假象下,毋宁说不过是彻底实现对人的否定而已,因为人本身已不再同私有财产的外在本质处于外部的紧张关系中,而人本身却成了私有财产的这种紧张的本质。以前是人之外的存在——人的实际外化——的东西,现在仅仅变成外化的行为,变成了外在化。因此,如果说上述国民经济学是从表面上承认人、人的独立性、自主活动等等开始,并由于把私有财产转为人自身的本质而能够不再束缚于作为存在于人之外的本质的私有财产的那些地域性的、民族的等等的规定,从而发挥一种世界主义的、普遍的、摧毁一切界限和束缚的能量,以便自己作为惟一的政策、普遍性、界限和束缚取代这些规定,-那末,国民经济学在它往后的发展过程中必定抛弃这种伪善性,而使自己的犬儒主义充分表现出来。它实际上也是这样做的——它不顾这种学说使它陷入的那一切表面上的的矛盾——,它十分片面地,因而也更加明确和彻底地发挥了关于劳动是财富的惟一本质的论点,然而它表明,这个学说的结论与上述原来的观点相反,不如说是敌视人的;最后,它还致命地打击了私有财产和财富的最后的个别的、自然的、不依赖于劳动运动存在的形式即地租,打击了这种已经完全成了经济的东西因而对国民经济学无法反抗的封建所有的制的表现。(李嘉图学派。)从斯密经过萨伊到李嘉图、穆勒等等,国民经济学的犬儒主义不仅相对地增长了(因为工业所造成的后果在后面这些人面前以更发达和更充满矛盾的形式表现出来),而且肯定地说,他们总是自觉地在人的异化方面比他们的先驱者走的更远,但这只是因为他们的科学发展的更加彻底、更加真实罢了。因为他们把具有活动形式的私有财产变为主体,就是说,既使人成为本质,又同时使作为某种非存在物〔Unwesen〕的人成为本质,所以,现实中的矛盾就完全符合他们视为原则的那个充满矛盾的本质。支离破碎的〔II〕工业现实不仅没有推翻,相反地,却证实了他们的自身支离破碎的原则。他们的原则本来就是这种支离破碎状态的原则。

魁奈医生的重农主义学说是从重商主义体系到亚当·斯密的过渡。重农学派直接是封建所有制在国民经济学上的解体,但正因为如此,它同样直接是封建所有制在国民经济学上的变革、恢复,不过它的语言这时不再是封建的,而且是经济学的了。全部财富被归结为土地和耕作(农业)。土地还不是资本,它还是资本的一种特殊的存在形式,这种存在形式应当在它的自然特殊性中并且由于它的这种自然特殊性才具有意义。但是,土地毕竟是一种普遍的自然的要素,而重商主义体系只承认贵金属是财富的存在。因此,财富的对象、财富的材料立即获得了自然界范围内的最高普遍性,因为它们作为自然界仍然是直接对象性的财富。而土地只有通过劳动、耕种才对人存在。因而,财富的主体本质已经移入劳动中。但农业同时是唯一的生产的劳动。因此,劳动还不是从它的普遍性和抽象性上被理解的,它还是同一种作为它的材料的特殊自然要素结合在一起的,因而,它也还是仅仅在一种特殊的、自然规定的存在形式中被认识的。因此,劳动不过是人的一种特定的、特殊的外化,正像劳动产品还被看作一种特定的财富-与其说来源于劳动本身,不如说来源于自然界的财富。在这里,土地还被看作不依赖于人的自然存在,还没有被看作资本,也就是说,还没有被看作劳动本身的要素。相反地,劳动却表现为土地的要素。但是,因为这里把过去的仅仅作为对象存在的外部财富的拜物教归结为一种极其简单的自然要素,而且已经承认-虽然只是部分地、以一种特殊的方式承认-财富的本质就在于财富的主体存在,所以,认出财富的普遍本质,并因此把具有完全绝对性即抽象性的劳动提高为原则,是一个必要的进步。人们向重农学派证明,从经济学观点即唯一合理的观点来看,农业同任何其它一切生产部门毫无区别,因此,财富的本质不是某种特定的劳动,不是与某种特殊要素结合在一起的、某种特殊的劳动表现,而是一般劳动。

重农学派既然把劳动宣布为财富的本质,也就否定了特殊的、外在的、仅仅是对象性的财富。但是,在重农学派看来,劳动首先只是地产的主体本质(重农学派是以那种在历史上占统治地位并得到公认的财产为出发点的);他们认为,只有地产才成为外化的人。他们既然把生产(农业)宣布为地产的本质,也就消除了地产的封建性质;但由于他们宣布农业是唯一的生产,他们对工业世界持否定态度,并且承认封建制度。

十分明显,那种与地产相对立的、即作为工业而确立下来的工业的主体本质一旦被理解,那末这种本质就同时也包含着自己的那个对立面。因为正像工业包含着已被扬弃了的地产一样,工业的主体本质也同时包含着地产的主体本质。

地产是私有财产的第一个形式,而工业在历史上最初仅仅作为财产的一个特殊种类与地产相对立,或者不如说它是地产的被释放了的奴隶,同样,在科学地理解私有财产的主体本质,理解劳动时,这一过程也在重演。而劳动起初只作为农业劳动出现,然后才作为一般劳动得到承认。

一切财富都成了工业的财富,成了劳动的财富,而工业是完成了的劳动,正像工厂制度是工业即劳动的发达的本质,而工业资本是私有财产的完成了的客观形式一样。

我们看到,只有这时私有财产才能完成它对人的统治,并以最普遍的形式成为世界历史性的力量。

\subsubsection{共产主义}
但是,无产和有产的对立,只要还没有把它理解为劳动和资本的对立,它还是一种无关紧要的对立,一种没有从它的能动关系上、它的内在关系上来理解的对立,还没有作为矛盾来理解的对立。这种对立即使没有私有财产的进一步的运动也能以最初的形式表现出来,如在古罗马、土耳其等。所以它还不表现为私有财产本身设定的对立。但是,作为财产之排除的劳动,即私有财产的主体本质,和作为劳动之排除的资本,即客体化的劳动,-这就是作为上述对立发展到矛盾关系的、因而促使矛盾得到解决的能动关系的私有财产。

补入同一页。自我异化的扬弃同自我异化走的是同一条道路。最初,对私有财产只是从它的客体方面来考察,-但是劳动仍然被看成它的本质。因此,它的存在方式就是“本身“应被消灭的资本(蒲鲁东。)或者,劳动的特殊方式,即划一的、分散的因而是不自由的劳动,被理解为私有财产的有害性和它同人相异化的存在的根源-傅立叶,他和重农学派一样,也把农业劳动看成至少是最好的劳动,而圣西门则相反,他把工业劳动本身说成本质,因此他渴望工业家独占统治,渴望改善工人状况。最后,共产主义是扬弃私有财产的积极表现;起先它是作为普遍的私有财产出现的。共产主义是从私有财产的普遍性来看私有财产关系,因而共产主义

(1)在它的最初的形式中不过是私有财产关系的普遍化和完成。这样的共产主义以双重的形式表现出来:首先,物质的财产对它的统治力量如此之大,以致它想把不能被所有人作为私有财产占有的一切都消灭;它想用强制的方式把才能等等舍弃。在它看来,物质的直接占有是生活和存在的惟一目的;工人这个范畴并没有被取消,而是被推广到一切人身上;私有财产关系仍然是整个社会同实物世界的关系;最后,用普遍的私有财产来反对私有财产的这个运动是以一种动物的形式表现出来的:用公妻制(也就是把妇女变成公有的和共有的财产)来反对婚姻(它确实是一种排它性的私有财产的形式)。人们可以说,公妻制这种思想暴露了这个完全粗陋的和无思想的共产主义的秘密。正像妇女从婚姻转向普遍卖淫\footnote{卖淫不过是工人普遍卖的一个特殊表现而已,因为这种卖淫是一种不仅包括卖淫者,而且包括逼人卖淫的关系,而且后者的下流无耻远为严重,所以,资本家等等,也包括到卖淫这一范畴中。}一样,财富即人的对象性的本质的整个世界也从它同私有者的排它性的婚姻关系转向它同整个社会的普遍卖淫关系。这种共产主义,由于到处否定人的个性,只不过是私有财产的彻底表现,私有财产就是这种否定。普遍的和作为权力形成起来的忌妒,是贪财欲所采取的并且仅仅是用另一种方式来满足自己的隐蔽形式。任何私有财产,就它本身而言,至少都对较富裕的私有财产怀有忌妒和平均化欲望,这种忌妒和平均化欲望甚至构成竞争的本质。粗陋的共产主义不过是这个忌妒和这种从想象的最低限度出发的平均化的顶点。它具有一个特定的、有限制的尺度。对整个文化和文明的世界的抽象否定,向贫穷的、需求不高的人-他不仅没有超越私有财产的水平,甚至从来没有达到私有财产的水平-的非自然的〔IV〕简单状态的倒退,恰恰证明私有财产的这种扬弃决不是真正的占有。

共同性只是劳动的共同性以及由共同的资本即作为普遍的资本家的共同体所支付的工资的平等的共同性。这种关系的两个方面被提高到想象的普遍性的程度:劳动是每个人的本分,而资本是共同体的公认的普遍性和力量。

拿妇女当作共同淫欲的虏获物和婢女来对待,这表现了人在对待自身方面的无限的退化,因为这种关系的秘密在男人对妇女的关系上,以及在对直接的、自然的、类的关系的理解方式上,都毫不含糊地、确凿无疑地、明显地、露骨地表现出来的。人和人之间的直接的、自然的、必然的关系是男人对妇女的关系。在这种自然的、类的关系中,人同自然的关系直接就是人和人之间的关系,而人和人之间的关系直接就是人同自然的关系,就是他自己的关于自然的规定。因此,这种关系通过感性的形式,作为一种显而易见的事实,表现出人的本质在何种程度上对人来说成为自然,或者自然在何种程度上成了人具有的人的本质。因而,从这种关系就可以判断人的整个文化教养程度。从这种关系的性质就可以看出,人在何种程度上对自己来说成为并把自身理解为类存在物、人。男人对妇女的关系是人和人之间最自然的关系。因此,这种关系表明人的自然的行为在何种程度上成为了人的行为,或者,人的本质在何种程度上对人来说成了自然的本质,他的人的本性在何种程度上对他来说成为了自然。这种关系还表明,人具有的需要在何种程度上成为了人的需要,也就是说,别人作为人在何种程度上对他说来成了需要,他作为个人的存在在何种程度上同时又是社会存在物。

由此可见,对私有财产的最初积极的扬弃,即粗陋的共产主义,不过是想把自己作为积极的共同体确定下来的私有财产的卑鄙性的一种表现形式。

(2)共产主义(a)按政治性质是民主的或专制的;(b)是废除国家的,但同时是尚未完成的,并且仍然处于私有财产即人的异化的影响下。这两种形式的共产主义都已经把自己理解为人向自身的还原或复归,理解为人的自我异化的扬弃;但是它还没有弄清楚私有财产的积极的本质,也还不理解所具有的人的本性,所以它还受私有财产的束缚和感染。它虽然已经理解私有财产这一概念,但是还不理解它的本质。

(3)共产主义是私有财产即人的自我异化的积极的扬弃,因而是通过人并且为了人而对人的本质的真正占有;因此,它是人向自身、向社会的即合乎人性的人的复归,这种复归是完全的,自觉的和在以往发展的全部财富的范围内生成的。这种共产主义,作为完成了的自然主义,等于人道主义,而作为完成了的人道主义,等于自然主义,它是人和自然界之间、人和人之间的矛盾的真正解决,是存在和本质、对象化和自我确证、自由和必然、个体和类之间的斗争的真正解决。它是历史之谜的解答,而且知道自己就是这种解答。

〔V〕因此,历史的全部运动,既是这种共产主义的现实的产生活动即它的经验存在的诞生活动,同时,对它的能思维的意识说来,又是它的被理解到和被认识到的生成运动。而上述尚未完成的共产主义从各个同私有财产相对立的历史形式中为自己寻找历史的证明,从现存的事物中寻找证明,同时从运动中抽出个别环节(卡贝、维尔加德尔等人尤其喜欢卖弄这一套),把它们作为自己的历史的纯种的证明固定下来;但是,它这样做恰好证明:历史运动的绝大部分是同它的论断相矛盾的,如果说它曾经存在过,那末它的这种过去的存在恰恰反驳了对本质的奢求。

不难看到,整个革命运动必然在私有财产的运动中,即在经济的运动中,为自己既找到经验的基础,也找到理论的基础。

这种物质的、直接感性的私有财产,是异化了的人的生命的物质的、感性的表现。私有财产的运动-生产和消费-是迄今为止全部生产的运动的感性展现,也就是说,是人的实现或人的现实。宗教、家庭、国家、法、道德、科学、艺术等等,都不过是生产的一些特殊的方式,并且受生产的普遍规律的支配。因此,对私有财产的积极的扬弃,作为对人的生命的占有,是对一切异化的积极的扬弃,从而是人从宗教、家庭、国家等等向自己的人的即社会的存在的复归。宗教的异化本身只是发生在人内心深处的意识领域中,而经济的异化则是现实生活的异化,-因此异化的扬弃包括两个方面。不言而喻,在不同的民族那里,这一运动从哪个领域开始,这要看一个民族的真正的、公认的生活主要是在意识领域中还是外部世界中进行,这种生活更多地是观念的生活还是现实的生活。共产主义是从无神论开始的(欧文),而无神论最初还远不是共产主义;那种无神论毋宁说还是一个抽象。所以,无神论的博爱最初还只是哲学的、抽象的博爱,而共产主义的博爱则从一开始就是现实的和直接追求实效的。

我们已经看到,在被积极扬弃的私有财产的前提下,人如何生产人-他自己和别人;直接体现他的个性的对象如何是他自己为别人的存在,同时是这个别人的存在,而且也是这个别人为他的存在。但是,同样,无论劳动的材料是作为主体的人,都既是运动的结果,又是运动的出发点(并且二者必须是这个出发点,私有财产的历史必然性就在于此)。因此,社会性质是整个运动的普遍性质;正像社会本身生产作为人的人一样,人也生产社会。活动和享受,无论就其内容或就其存在方式来说,都是社会的,是社会的活动和社会的享受。自然界的人的本质只有对社会的人来说才是存在的;因为只有在社会中,自然界对人来说才是人与人联系的纽带,才是他为别人的存在和别人为他的存在,才是人的现实的生活要素;只有在社会中,自然界才是人自己的人的存在的基础。只有在社会中,人的自然的存在对他说来才是他的人的存在,而自然界对他来说才成为人。因此,社会是人同自然界的完成了的本质的统一,是自然界的真正复活,是人的实现了的自然主义和自然界的实现了的人道主义。

〔VI〕社会的活动和社会的享受决不仅仅存在于直接共同的活动和直接共同的享受这种形式中,虽然共同的活动和共同的享受,即直接通过同别人的实际交往表现出来和得到确证的那种活动和享受,在社会性的上述直接表现以这种活动或这种享受的内容的本质为根据并且符合其本性的地方都会出现。

甚至当我从事科学之类的活动,即从事一种我只是在很少情况下才能同别人进行直接联系的活动的时候,我也是社会的,因为我是作为人活动的。不仅我的活动所需的材料,甚至思想家用来进行活动的语言本身,都是作为社会的产品给予我的,而且我本身的存在就是社会的活动;因此,我从自身所做出的东西,是我从自身为社会做出的,并且意识到我自己是社会存在物。

我的普遍意识不过是以现实共同体、社会存在物为生动形式的那个东西的理论形式,而在今天,普遍意识是现实生活的抽象,并且作为这样的抽象是与现实生活相敌对的。因此,我的普遍意识的活动本身也是我作为社会存在物的理论存在。

首先应当避免重新把“社会“当作抽象的东西同个人对立起来。个人是社会存在物。因此,他的生命表现,即使不采取共同的、同其它人一起完成的生命表现这种直接形式,也是社会生活的表现和确证。人的个人生活和类生活并不是各不相同的,尽管个人生活的存在方式是——必然是——类生活的较为特殊的或者较为普遍的方式,而类生活必然是较为特殊的或者较为普遍的个人生活。

作为类意识,人确证自己的现实的社会生活,并且只是在思维中复现自己的现实存在;反之,类存在则在类意识中确证自己,并且在自己的普遍性中作为思维着的存在物自为地存在着。

因此,人是一个特殊的个体,并且正是他的特殊性使他成为一个个体,成为一个现实的、单个的社会存在物,同样,他也是总体,观念的总体,被思考和被感知的社会的自为的主体存在,正如他在现实中既作为对社会存在的直观和现实感受而存在,又作为人的生命表现的总体而存在一样。

可见,思维和存在虽有区别,但同时彼此又处于统一中。

死似乎是类对特定的个体的冷酷无情的胜利,并且似乎是同它们的统一相矛盾的;但是特定的个体不过是一个特定的类存在物,而作为这样的存在物是迟早要死的。

(4)私有财产不过是下述情况的感性表现:人变成了对自己来说是对象性的,同时,确切的说,变成了异己的和非人的对象;他的生命表现就是他的生命的外化,他的现实化就是他的非现实化,就是异己的现实。同样,私有财产的积极的扬弃,也就是说,为了人并且通过人对人的本质和人的生命、对象性的人和人的作品的感性的占有,不应当仅仅被理解为所有、拥有,不应当仅仅被理解为直接的、片面的享受。人以一种全面的方式,也就是说,作为一个完整的人,占有自己的全面的本质。人同世界的任何一种人的关系-视觉、听觉、嗅觉、味觉、触觉、思维、直观、情感、愿望、活动、爱-总之,他的个体的一切器官,正像在形式上直接是社会的器官的那些器官一样,〔IVII〕是通过自己的对象性关系,即通过自己同对象的关系而对对象的占有,对人的现实的占有;这些器官同对象的关系,是人的现实的实现\footnote{因此,正像人的本质规定和活动是多种多样的一样,人的现实也是多种多样的。},是人的能动和人的受动,因为按人的方式来理解的受动,是人的一种自我享受。

私有制使我们变得如此愚蠢和片面,以致一个对象,只有当它为我们拥有的时候,也就是说,当它对我们来说作为资本而存在,或者它被我们直接占有,被我们吃、喝、穿、住等等的时候,简言之,在它被我们使用的时候,才是我们的。尽管私有制本身又把占有的这一切直接实现仅仅看作生活手段,而它们作为手段为之服务的那种生活是私有制的生活-劳动和资本化。

因此,一切肉体的和精神的感觉都被这一切感觉的单纯异化即拥有的感觉所代替。人这个存在物必须被归结为这种绝对的贫困,这样他才能从自身产生出他的内在丰富性。(关于拥有这个范畴,见《二十一印张》文集中赫斯的论文。)

因此,对私有财产的扬弃,是人的一切感觉和特性的彻底解放;但这种扬弃之所以是这种解放,正是因为这些感觉和特性无论在主体上还是在客体上都变成人的。眼睛变成了人的眼睛,正像眼睛的对象变成了社会的、人的、由人并为了人创造出来的对象一样。因此,感觉通过自己的实践直接变成了理论家。感觉为了物而同物发生关系,但物本身却是对自身和对人的一种对象性的、人的关系\footnote{只有当物按人的方式同人发生关系时,我才能在实践上按人的方式同物发生关系。},反过来也是这样。因此,需要和享受失去了自己的利己主义性质,而自然界失去了自己的纯粹的有用性,因为效用成了人的效用。

同样,别人的感觉和享受也形成了我自己的占有。因此,除了这些直接的器官外,还以社会的形式形成社会的器官。例如,直接同别人交往的活动等等,成了我的生命表现的器官和对人的生命的一种占有方式。

不言而喻,人的眼睛和野性的、非人的眼睛得到的享受不同,人的耳朵与野性的耳朵得到的享受不同,如此等等。

我们知道,只有当对象对人来说成为人的对象或者说成为对象性的人的时候,人才不致在自己的对象中丧失自身。只有当对象对人来说成为社会的对象,人本身对自己来说成为社会的存在物,而社会在这个对象中对人来说成为本质的时候,这种情况才是可能的。

因此,一方面,随着对象性的现实在社会中对人说来到处成为人的本质力量的现实,成为人的现实,因而成为人自己的本质力量的现实,一切对象对他说来也就成为他自身的对象化,成为确证和实现他的个性的对象,成为他的对象,而这就是说,对象成为他自身。对象如何对他来说成为他的对象,这取决于对象的性质以及与之相适应的本质力量的性质;因为正是这种关系的规定性形成一种特殊的、现实的肯定方式。眼睛对对象的感觉不同于耳朵,眼睛的对象是不同于耳朵的对象的。每一种本质力量的独特性,恰好就是这种本质力量的独特的本质,因而也是它的对象化的独特方式,它的对象性的、现实的、活生生的存在的独特方式。因此,人不仅通过思维,〔VIII〕而且以全部感觉在对象中肯定自己。

另一方面,即从主体方面来看:只有音乐才能激起人的音乐感;对于没有音乐感的耳朵说来,最美的音乐也毫无意义,不是对象,因为我的对象只能是我的一种本质力量的确证,也就是说,它只能像我的本质力量作为一种主体能力自为地存在着那样才对我而存在,因为任何一个对象对我的意义(它只是对那个与它相适应的感觉来说才有意义)恰好都以我的感觉所及的程度为限。所以社会的人的感觉不同于非社会的人的感觉。只是由于人的本质客观地展开的丰富性,主体的、人的感性的丰富性,如有音乐感的耳朵、能感受形式美的眼睛,总之,那些能成为人的享受的感觉,即确证自己是人的本质力量的感觉,才一部份发展起来,一部分产生出来。因为,不仅五官感觉,而且所谓精神感觉、实践感觉(意志、爱等等),一句话,人的感觉、感觉的人性,都是由于它的对象的存在,由于人化的自然界,才产生出来的。

五官感觉的形成是以往全部世界历史的产物。囿于粗陋的实际需要的感觉,也只具有有限的意义。对于一个挨饿的人来说并不存在人的食物形式,而只有作为食物的抽象存在;食物同样也可能具有最粗糙的形式,而且不能说,这种进食活动与动物的进食活动有什么不同。忧心忡忡的、贫穷的人甚至对最美丽的景色都没有什么感觉;贩卖矿物的商人只看到矿物的商业价值,而看不到矿物的美和独特性;他没有矿物学的感觉。因此,一方面为了使人的感觉成为人的,另一方面为了创造同人的本质和自然界的本质的全部丰富性相适应的人的感觉,无论从理论方面还是从实践方面来说,人的本质的对象化都是必要的。

通过私有财产及其富有和贫困-或物质的和精神的富有和贫困-的运动,正在生成的社会发现这种形式所需的全部材料;同样,已经生成的社会,创造着具有人的本质的这种全部丰富性的人,创造着具有丰富的、全面而深刻的感觉的人作为这个社会的恒久的现实。

我们看到,主观主义和客观主义,唯灵主义和唯物主义,活动和受动,只是在社会状态中才失去它们彼此间的对立,并从而失去它们作为这样的对立面的存在;我们看到,理论的对立本身的解决,只有通过实践方式,只有借助于人的实践力量,才是可能的;因此,这种对立的解决决不只是认识的任务,而是一个现实生活的任务,而哲学未能解决这个任务,正是因为哲学把这仅仅看作理论的任务。

我们看到,工业的历史和工业的已经产生的对象性的存在,是一本打开了的关于人的本质力量的书,是感性地摆在我们面前的人的心理学;对这种心理学人们至今还没有从它同人的本质的联系,而总是仅仅从有用性这种外在关系来理解,因为在异化范围内活动的人们仅仅把人的普遍存在、宗教、或者具有抽象普遍性质的历史,如政治、艺术和文学等等,〔IX〕理解为人的本质力量的现实性和人的类活动。在通常的、物质的工业中(人们可以把这种工业看成是上述普遍运动的一部分,正像可以把这个运动本身看成是工业的一个特殊部分一样,因为全部人的活动迄今都是劳动,也就是工业,就是同自身相异化的活动)人的对象化的本质力量以感性的、异己的、有用的对象的形式,以异化的形式呈现在我们面前。如果心理学还没有打开这本书即历史的这个恰恰最容易感知的、最容易理解的部份,那末这种心理学就不能成为内容确实丰富的和真正的科学。如果科学从人的活动的如此广泛的丰富性中只知道那种可以用“需要“、“一般需要!”的话来表达的东西,那末人们对于这种高傲地撇开人的劳动的这一巨大部分而不感觉自身不足的科学究竟应该怎样想呢?

自然科学展开了大规模的活动并且占有了不断增多的材料。但是哲学对自然科学也始终是疏远的,正像自然科学对哲学也始终是疏远的一样。过去把它们暂时结合起来,不过是离奇的幻想。存在着结合的意志,但缺少结合的能力。甚至历史学也只是顺便地考虑到自然科学,仅仅把它看作是启蒙、有用性和某些伟大发现的因素。然而,自然科学却通过工业日益在实践上进入人的生活,改造人的生活,并为人的解放做准备,尽管它不得不直接地使非人化充分发展。工业是自然界同人之间,因而也是自然科学同人之间的现实的历史关系。因此,如果把工业看成人的本质力量的公开的展示,那末,自然界的人的本质,或者人的自然的本质,也就可以理解了;因此,自然科学将失去它的抽象物质的或者不如说是唯心主义的方向,并且将成为人的科学的基础,正像它现在已经-尽管以异化的形式-成了真正人的生活的基础一样;至于说生活还有别的什么基础,科学还有别的什么基础-这根本就是谎言。在人类历史中即在人类社会的形成过程中生成的自然界是人的现实的自然界;因此,通过工业-尽管以异化的形式-形成的自然界,是真正的、人本学的自然界。

感性(见费尔巴哈)必须是一切科学的基础。科学只有从感性意识和感性需要这两种形式的感性出发,因而,科学只有从自然界出发,才是现实的科学。可见,全部历史是为了使“人“成为感性意识的对象和使“人作为人“的需要成为“自然的、感性的“需要而做准备的历史(发展史)。历史本身是自然史的即自然界生成为人这一过程的一个现实部分。自然科学往后将包括关于人的科学,正像关于人的科学包括自然科学一样:这将是一门科学。

〔X〕人是自然科学的直接对象;因为直接的感性自然界,对人说来直接地就是人的感性(这是同一个说法),直接地就是另一个对他来说感性地存在着的人;因为他自己的感性,只有通过另一个人,才对他本身说来是人的感性。但是自然界是关于人的科学的直接对象。人的第一个对象-人-就是自然界、感性;而那些特殊的、人的、感性的本质力量,正如它们只有在自然对象中才能得到客观的实现一样,只有在关于一般自然界的科学中才能获得它们的自我认识。思维本身的要素,思想的生命表现的要素,即语言,是感性的自然界。自然界的社会的现实,和人的自然科学或关于人的自然科学,是同一个说法。

我们看到,富有的人和富有的人的需要代替了国民经济学上的富有和贫困。富有的人同时就是需要有总体的人的生命表现的人,在这样的人的身上,他自己的实现作为内在的必然性、作为需要而存在。不仅人的富有,而且人的贫困,——在社会主义的前提下——同样具有人的因而是社会的意义。贫困是被动的纽带,它使人感觉到需要最大的财富即别人。因此,对象性的本质在我身上的统治,我的本质活动的感性爆发,是激情,从而激情在这里就成了我的本质的活动。

(5)任何一个存在物只有当它用自己的双脚站立的时候,才认为自己是独立的,而且只有当它依靠自己而存在的时候,它才是用自己的双脚站立的。靠别人恩典为生的人,把自己看成一个从属的存在物。但是,如果我不仅靠别人维持我的生活,而且别人还创造了我的生活,别人还是我的生活的泉源,那末,我就完全靠别人的恩典为生;如果我的生活不是我自己的创造,那末,我的生活就必定在自身之外有这样一个根源。所以,创造是一个很难从人民意识中排除的观念。自然界和人的通过自身的存在,对人民意识来说是不能理解的,因为这种存在是同实际生活的一切明摆着的事实相矛盾的。

大地创造说,受到了地球构造学(即说明地球的形成、生成是一个过程、一种自我产生的科学)的致命打击。自然发生说是对创世说的唯一实际的驳斥。

对个别人讲讲亚理士多德已经说过的下面这句话,当然是容易的:你是你的父亲和你的母亲所生;这就是说,在你身上,两个人的性的结合即人的类行为生产了人。因而,你看到,人的肉体的存在也要归功于人。所以,你应该不是仅仅注意一个方面即无限的过程,由于这个过程你会进一步发问:谁生出了我的父亲?谁生出了他的祖父?等等。你还应该紧紧盯住这个无限过程中的那个可以直接感觉到的循环运动,由于这个运动,人通过生儿育女使自身重复出现,因而人始终是主体。

但是,你会回答说:我承认这个循环运动,那末你也要向我承认那个无限的过程,这过程使我不断追问,直到我提出问题,谁生出了第一个人和整个自然界?我只能对你做如下的回答:你的问题本身就是抽象的产物。请你问一下自己,你是怎样想到这个问题的;请你问一下自己,你的问题是不是来自一个因为荒谬而使我无法回答的观点。请你问一下自己,那个无限的过程本身对理性的思维说来是否存在。既然你提出自然界和人的创造问题,那末你也就把人和自然界抽象掉了。你假定它们是不存在的,然而你却希望我向你证明它们是存在的。那我就对你说:放弃你的抽象,你也就会放弃你的问题,或者,你想坚持自己的抽象,你就要贯彻到底,如果你设想人和自然界是不存在的,〔XI〕那末你就要设想你自己也是不存在的,因为你自己也是自然界和人。不要那样想,也不要那样向我提问,因为你一旦那样想,那样提问,你就会把自然界的和人的存在抽象掉,这是没有任何意义的。也许你是一个假定一切都不存在,而自己却想存在的利己主义者吧?

你可能反驳我说:我并不想假定自然界等等不存在;我是问你自然界的形成过程,正像我问解剖学家骨骼如何形成等等一样。

但是,因为在社会主义的人看来,整个所谓世界历史不外是人通过人的劳动而诞生的过程,是自然界对人来说的生成过程,所以,关于他通过自身而诞生、关于他的形成过程,他有直观的、无可辩驳的证明。因为人和自然界的实在性,即人对人说来作为自然界的存在以及自然界对人来说作为人的存在,已经成为实际的、可以通过感觉直观的,所以,关于某种异己的存在物,关于凌驾于自然界和人之上的存在物的问题,即包含着对自然界和人的非实在性的承认的问题,实际上已经成为不可能的了。无神论,作为对这种非实在性的否定,已不再有任何意义,因为无神论是对神的否定,并且正是通过这种否定而肯定人的存在;但是,社会主义作为社会主义,已经不再需要这样的中介;它是从把人和自然界看作本质这种理论上和实践上的感性认识开始的。社会主义是人的不再以宗教的扬弃为中介的积极的自我意识,正像现实生活是人的不再以私有财产的扬弃即共产主义为中介的积极的现实一样。共产主义是作为否定的否定的肯定,因此,它是人的解放和复原的一个现实的、对下一段历史发展说来是必然的环节。共产主义是最近将来的必然的形式和有效的原则。但是,共产主义本身并不是人的发展的目标,并不是人的社会的形式。

\subsubsection{对黑格尔的辩证法和整个哲学的批判}
〔XI〕(6)在这一部分,为了便于理解和论证,对黑格尔的整个辩证法,特别是《现象学》和《逻辑学》中有关辩证法的叙述,以及最后对最近的批判运动同黑格尔的关系略作说明,也许是适当的。

现代德国的批判,着意研究旧世界的内容,而且批判的发展完全拘泥于所批判的材料,以致对批判的方法采取完全非批判的态度,同时,对于我们如何对待黑格尔辩证法,这一表面上看来是形式的问题,而实际上是本质的问题,则完全缺乏认识。对于现代的批判同黑格尔的整个哲学,特别是同辩证法的关系问题是如此缺乏认识,以致像施特劳斯和布鲁诺·鲍威尔这样的批判家——前者是完完全全地,后者在自己的《符类福音作者》中(与施特劳斯相反,它在这里用抽象的人的“自我意识“代替了“抽象的自然界“的实体),甚至在《基督教真相》中,至少有可能完全地——仍然拘泥于黑格尔的逻辑学。例如《基督教真相》一书中说︰

\begin{fangsong}
“自我意识设定世界、设定差别,并且在它所创造的东西中创造自身,因为它重新扬弃了它的创造物同它自身的差别。因为它只是在创造活动中和运动中才是自己本身,——这个自我意识在这个运动中似乎就没有自己的目的了“,等等。或者说:“他们〈法国唯物主义者〉还未能看到,宇宙的运动只有作为自我意识的运动,才能实际成为自为的运动,从而达到同自身的统一。”    
\end{fangsong}

这些说法连语言上都和黑格尔的观点毫无区别,而且毋宁说是在逐字逐句重述黑格尔的观点。

〔XII〕鲍威尔在他的《自由的正义事业》一书中对格鲁培先生提出的“那末逻辑学的情况如何呢?”这一唐突的问题避而不答,却让他去问未来的批评家。这表明,鲍威尔在进行批判活动(鲍威尔《复类福音作者》)时对于黑格尔辩证法关系是多么缺乏认识,而且在物质的批判活动之后也还缺乏这种认识。

但是,即使现在,在费尔巴哈不仅在收入《轶文集》的《纲要》中,而且更详细地在《未来哲学》中从根本上推翻了旧的辩证法和哲学之后;在不能完成这一事业的上述批判,反而认为这一切事业已经完成,并且宣称自己是“纯粹的、坚决的、绝对的、洞察一切的批判“之后;在批判以唯灵论的狂妄自大态度把整个历史运动归结为世界的其他部分(它把这个世界与它自身对立起来而归入“群众”这一范畴)和它自身的关系,并且把一切独断的对立销融于它自身的聪明和世界的愚蠢之间、批判的基督和作为“群氓”的人类之间的一个独断的对立中之后;在批判每日每时以群众的愚钝来证明它本身的超群出众之后;在批判终于宣称这样一天——那时整个正在堕落的人类将聚集在批判面前,由批判加以分类,而每一人类都将得到一份贫困证明书——即将来临,即以这种形式宣告批判的末日审判之后;在批判于报刊上宣布它既对人的感觉又·对它自己独标一格地雄踞其上的世界具有优越性,而且只是不时从它那好讥讽嘲笑的口中发出奥林帕斯诸神的哄笑声之后,——在以批判的形式消逝着的唯心主义(青年黑格尔主义)做出这一切滑稽可笑的动作之后,这种唯心主义甚至一点也没想到现在已经到了同自己的母亲,即黑格尔辩证法批判地画出界限的时候,甚至也〔丝毫〕未能表明它对费尔巴哈辩证法的批判态度。这是对自身持完全非批判的态度。

费尔巴哈是唯一对黑格尔辩证法采取严肃的、批判的态度的人;只有他在这个领域内作出了真正的发现,总之他真正克服了旧哲学。费尔巴哈成就的伟大以及他把这种成就贡献给世界时所表现的那种谦虚纯朴,同批判所持的相反的态度恰成惊人的对照。

费尔巴哈的伟大功绩在于︰

(1)证明了哲学不过是变成思想的并且经过思维加以阐明的宗教,不过是人的本质的异化的另一种形式和存在方式;因此,哲学同样应当受到谴责。

(2)创立了真正的唯物主义和现实的科学,因为费尔巴哈也使“人与人之间的“社会关系成了理论的基本原则。

(3)他把基于自身并且积极地以自身为根据的肯定的东西,同自称是绝对肯定的东西的那个否定的否定对立起来。

费尔巴哈这样解释了黑格尔的辩证法(从而论证了要从肯定的东西,即从感觉确定的东西出发);

黑格尔从异化出发(在逻辑上就是从无限的东西、抽象的普遍的东西出发),从实体出发,从绝对的和不变的抽象出发,就是说,说得更通俗些,他从宗教和神学出发。

第二,他扬弃了无限的东西,设定了现实的、感性的、实在的、有限的、特殊的东西(哲学,对宗教和神学的扬弃)。

第三,他重新扬弃了肯定的东西,重新恢复了抽象、无限的东西。宗教和神学的恢复。

由此可见,费尔巴哈把否定的否定仅仅看作哲学同自身的矛盾,看作在否定神学(超验性等等)之后又肯定神学的哲学,即同自身相对立而肯定神学的哲学。

否定的否定所包含的肯定,或自我肯定和自我确证,被认为是对自身还不能确信,因而自身还受对立面影响的、对自身怀疑因而需要证明的肯定,即被认为是还没有用自己的存在证明自身的、还没有被承认的〔XIII〕肯定;可见,感觉确定的、以自身为基础的肯定,是同这种肯定直接地而非间接地对立着的。\footnote{马克思在这里加了一句话:“费尔巴哈把否定的否定、具体概念看做在思维中超越自身的和作为思维而想直接成为直观、自然界、现实的思维。”——编者注}

费尔巴哈还把否定的否定、具体概念看作在思维中超越自身的和作为思维而想直接成为直观、自然界、现实的思维

但是,由于黑格尔根据否定的否定所包含的肯定方面把否定的否定看成真正的和惟一的肯定的东西,而根据它所包含的否定方面把它看成一切存在的唯一真正的活动和自我实现的活动,所以他只是为那种历史的运动找到抽象的、逻辑的、思辨的表达,这种历史还不是作为既定的主体的人的现实的历史,而只是人的产生的活动、人的形成的历史。

我们既要说明这一运动在黑格尔那里所采取的抽象形式,也要说明这一和现代的批判相反的运动,同费尔巴哈的《基督教的本质》一书所描述的同一过程的区别;或者更正确些说,要说明这一在黑格尔那里还是非批判的运动所具有的批判形式。

现在看一看黑格尔的体系。必须从黑格尔的《现象学》即从黑格尔哲学的真正诞生地和秘密开始。

现象学

(A)自我意识。

Ⅰ.意识。(α)感性确定性,或“这一个”和意谓。(β)知觉,或具有特性的事物和幻觉。(γ)力和知性,现象和超感觉世界。

Ⅱ.自我意识。自身确定性的真理。(a)自我意识的独立性和非独立性,主人和奴隶(b)自我意识的自由。斯多葛主义,怀疑主义,苦恼的意识。

Ⅲ.理性。理性的确定性和真理。(a)观察的理性;对自然界和自我的意识的观察。(b)理性的自我意识通过自身来实现。快乐和必然性。心的规律和自大狂。德行和世道。(c)自在和自为地实在的个性。精神的动物界和欺骗,或事情本身。立法的理性。审核法律的理性。

(B)精神。

Ⅰ.真的精神;伦理。Ⅱ.自我异化的精神,教养。Ⅲ.确定自身的精神,道德。

(C)宗教。自然宗教,艺术宗教,启示宗教。

(D)绝对知识。

因为黑格尔的《哲学全书》以逻辑学,以纯粹的思辨的思想开始,而以绝对知识,以自我意识的、理解自身的哲学或绝对的即超人的抽象精神结束,所以整整一部《哲学全书》不过是哲学精神的展开的本质,是哲学精神的自我对象化;而哲学精神不过是在它的自我异化内部通过思维理解,即抽象地理解自身话的、异化的宇宙精神。逻辑学是精神的货币,是人和自然界的思辨的、思想的价值——人和自然界的同一切现实的规定性毫不相干的生成的因而是非现实的本质,——是外化的因而是从自然界和现实的人抽象出来的思维,即抽象思维。——这种抽象思维的外在性就是……自然界,就像自然界对这种抽象思维所表现的那样。自然界对抽象思维说来是外在的,是抽象思维的自我丧失;而抽象思维也是外在地把自然界作为抽象的思想来理解,然而是作为外化的、抽象的思维来理解。——最后,精神,这个回到自己的诞生地的思维,这种思维在它终于发现自己和肯定自己就是绝对知识,因而就是绝对的即抽象的精神之前,在它获得自己的自觉的、与自身相符合的存在之前,它作为人类学的、现象学的、心理学的、伦理的、艺术的、宗教的精神,总还不是自身,因为它的现实存在就是抽象。

黑格尔有双重错误。

第一个错误在黑格尔哲学的诞生地《现象学》中表现的最为明显。例如,当他把财富、国家权力等等看成同人的本质相异化的本质时,这只是就它们的思想形式而言……它们是思想本质,因而只是纯粹的即抽象的哲学思维的异化。因此,整个运动是以绝对知识结束的。这些从对象中异化出来的并且以现实性自居而与之对立的,恰恰是抽象的思维。哲学家——他本身是异化的人的抽象形象——把自己变成异化的世界的尺度。因此,全部外化历史和外化的全部消除,不过是抽象的、绝对的〔XVII〕思维的生产史,即逻辑的思辨的思维的生产史。因而,异化——它从而构成这种外化的以及这种外化之扬弃的真正意义——是在自在和自为之间、意识和自我意识之间、客体和主体之间的对立,就是说,是抽象思维同感性的现实,或现实的感性在思想本身范围内的对立。其它一切对立及其运动,不过是这种唯一有意义的对立的外观、外壳、公开形式,这些唯一有意义的对立构成其它世俗对立的含义。在这里,不是人的本质以非人的方式同自身对立的对象化,而是人的本质以不同于抽象思维的方式,并且同抽象思维对立的对象化,被当作异化的被设定的和应该扬弃的本质。

〔XVIII〕因此,对于人的已成为对象而且是异己对象的本质力量的占有,首先不过是那种在意识中、在纯思维中即在抽象中发生的占有,是对这些作为思想和思想运动的对象的占有;因此,在《现象学》中,尽管已有一个完全否定的和批判的外表,尽管实际上已包含着往往早在后来发展之前就先进行的批判,黑格尔晚期著作的那种非批判的实证主义,和同样非批判的唯心主义——现有经验在哲学上的分解和恢复——已经已一种潜在的方式,作为萌芽、潜能和秘密存在着了。其次,因此,要求把对象世界归还给人——例如,有这样一种理解︰感性意识不是抽象的感性意识,而是人的感性的意识;宗教、财富等等不过是人的对象化的异化了的现实,是客体化了的人的本质力量的异化了的现实;因此,宗教、财富等等不过是通向真正的人的现实的道路,——这种对人的本质力量的占有或对这一过程的理解,在黑格尔那里是这样表现的:感性、宗教、国家权力等等是精神的本质,因为只有精神才是人的真正的本质,而精神的真正的形式则是思维着的精神,逻辑的、思辨的精神。自然界的人性和历史所创造的自然界——人的产品——的人性,就表现在它们是抽象精神的产物,所以,在这个限度内,它们是精神的环节即思想本质。可见,《现象学》是一种隐蔽的、自身还不清楚的、神秘化的批判;但是,由于《现象学》紧紧抓住人的异化,——尽管人只是以精神的方式出现的,——所以它潜在地包含着批判的一切要素,而且这些要素往往已经以远远超过黑格尔观点的方式准备好和加过工了。关于“苦恼的意识”、“诚实的意识”、“高尚的意识和卑鄙的意识”的斗争等等、等等这些章节,包含着对宗教、国家、市民生活等整个领域的批判的要素,但还是通过异化的形式。正像本质、对象表现为思想本质一样,主体也始终是意识或自我意识,或者更正确些说,对象仅仅表现为抽象的意识,而人仅仅表现为自我意识。因此,在《现象学》中出现的异化的各种不同形式,不过是意识和自我意识的不同形式,正像抽象的意识本身(对象就被看成这样的意识)仅仅是自我意识的一个差别环节一样,这一运动的结果表现为自我意识和意识的同一,绝对知识,那种已经不是向外部而是仅仅在自身内部进行的抽象思维活动,也就是说,其结果是纯思想的辩证法。〔XVIII〕

〔XXIII〕因此,黑格尔的《现象学》及其最后成果——作为推动原则和创造原则的否定性的辩证法——的伟大之处首先在于,黑格尔把人的自我产生看作一个过程,把对象化看作非对象化,看作外化和这种外化的扬弃;因而,他抓住了劳动的本质,把对象性的人、现实的因而是真正的人,理解为他自己的劳动的成果。人同作为类存在物的自身发生现实的、能动的关系,或者说,人作为现实的类存在物即作为人的存在物的实现,只有通过下述途径才有可能:人实际上把自己的类的力量统统发挥出来(这又是只有通过人类的全部活动、只有作为历史的结果才有可能),并且把这些力量当作对象来对待,而这首先又是只有通过异化的形式才有可能。

我们将以《现象学》的最后一章——绝对知识——来详细说明黑格尔的片面性和局限性。这一章既概括地阐述了《现象学》的精神、包含《现象学》同思辨的辩证法的关系,也概括地阐述了黑格尔对这二者及其相互关系的理解。

且让我们先指出一点︰黑格尔站在现代国民经济学家的立场上。他把劳动看作人的本质,看作人的自我确证的本质;他只看到劳动的积极的方面,而没有看到它的消极的方面。劳动是人在外化范围内或者作为外化的人的自为的生成。黑格尔唯一知道并承认的劳动是抽象的精神的劳动。因此,黑格尔把一般说来构成哲学的本质的那个东西,即知道自身的人的外化,或者思考自身的、外化的科学看成劳动的本质;因此,同以往的哲学相反,他能把哲学的各个环节加以总括,并且把自己的哲学描述成这种哲学。其它哲学家做过的事情——把自然界和人类生活的各个环节看作自我意识的,以至抽象的自我意识的环节,黑格尔则认为是哲学本身所做的事情。因此,他的科学是绝对的。

现在让我们转到我们的本题上来。

绝对知识。《现象学》的最后一章。

主要之点就在于︰意识的对象无非就是自我意识;或者说,对象不过是对象化的自我意识、作为对象的自我意识(把人和自我意识等同起来)。

因此,问题就在于克服意识的对象。对象性本身被认为是人的异化了的、同人的本质(自我意识)不相适应的关系。因此,重新占有在异化规定下作为异己的东西产生的、人的对象性的本质,这不仅具有扬弃异化的意义,而且有扬弃对象性的意义,这就是说,因此,人被看成非对象性的、唯灵论的存在物。

黑格尔对克服意识的对象的运动作了如下的描述︰

对象不仅表现为向自我〔das Selbst〕复归的东西(在黑格尔看来,这是对第一运动的片面的,即只抓住了一个方面的理解)。把人和自我等同起来。而自我不过是被抽象地理解和通过抽象产生出来的人。人是自我的〔selbstisch〕。人的眼睛、人的耳朵等等都是自我的;人的每一种本质力量在人身上都具有自我性这种特性。但正因为这样,说自我意识具有眼睛、耳朵、本质力量,就完全错了。毋宁说,自我意识是人的自然的即人的眼睛等等的质,而并非人的自然是〔XXIV〕自我意识的质。

被抽象化和被固定化的自我,就是作为抽象的利己主义者的人,就是在自己的纯粹抽象中被提升到思维的利己主义(下文还要提到这一点)。

人的本质,人,在黑格尔看来是和自我意识等同的。因此,人的本质的一切异化都不过是自我意识的异化。自我意识的异化没有被看作人的本质的现实异化的表现,即在知识和思维中反映出来的这种异化的表现。相反地,现实的即真实出现的异化,就其潜藏在内部最深处的——并且只有哲学才能揭示出来的——本质来说,不过是现实的、人的本质即自我意识的异化现象。因此,掌握了这一点的科学就叫作现象学。因此,对异化了的对象性本质的全部重新占有,都表现为把这种本质合并于自我意识:掌握了自己本质的人,仅仅是掌握了对象性本质的自我意识。因此,对象向自我的复归就是对象的重新占有。

意识的对象的克服可全面表述如下:

(1)对象本身对意识说是正在消逝的东西;

(2)自我意识的外化就是设定物性;

(3)这种外化不仅有否定的意义,而且有肯定的意义;

(4)它不仅对我们有这种意义或者说自在地有这种意义,而且对意识本身也有这种意义;

(5)对象的否定,或对象的自我扬弃,对意识所以有肯定的意义(或者说,它所以知道对象的这种虚无性),是由于意识把自身外化了,因为意识在这种外化中把自身设定为对象,或者说,由于自为的存在的不可分割的统一性,而把对象设定为自身。

(6)另一方面,这里同时包含着另一个环节,即意识扬弃这种外化和对象性,同样也把它们收回到自身,因而,它在自己的异在本身中也就是在自己那里;

(7)这就是意识的运动,因而也就是意识的各个环节的总体;

(8)意识必须依据自己的各个规定的总体对待对象,同样也必须依据这个总体的每一个规定来把握对象。意识的各个规定的这种总体使对象自在地成为精神的本质,而对于意识来说,对象所以真正成为精神的本质,是由于把对象(这个总体)的每一个别规定理解为自我的规定,或者说,是由于对这些规定采取了上述的精神的态度。

关于(1)。——所谓对象本身对意识来说是正在消逝的东西,就是上面提到的对象向自我的复归。

关于(2)。——自我意识的外化设定物性。因为人等于自我意识,所以人的外化的、对象性的本质即物性(即对他来说是对象的那个东西,而只有对他来说是本质的对象,并因而是他的对象性的本质的那个东西,才是他的真正的对象。既然被当作主体的不是现实的人本身,因而也不是自然——人是人的自然——而只是人的抽象,即自我意识,所以,物性只能是外化的自我意识),等于外化的自我意识,而物性是由这种外化设定的。一个有生命的、自然的、具备并赋有对象性的即物质的本质力量的存在物,既拥有他的本质的现实的、自然的对象,而他的自我外化又设定一个现实的、但以外在性的形式表现出来因而不属于他的本质的、极其强大的对象世界,这是十分自然的。这里并没有什么不可捉摸的和神秘莫测的东西。相反的情况倒是神秘莫测的。但是,同样明显的是,自我意识通过自己的外化所能设定的只是物性,即只是抽象物、抽象的物,而不是现实的物。〔XXVI〕此外还很明显的是:物性因此对自我意识来说绝不是什么独立的、实质的东西,而只是纯粹的创造物,是自我意识所设定的东西,这个被设定的东西并不证实自己,而只是证实设定这一行动,这一行动在一瞬间把自己的能力作为产物固定下来,使它表面上具有独立的、现实的本质的作用——但仍然只是一瞬间。

当现实的、肉体的的、站在坚实的呈圆形的地球上呼出和吸入一切自然力的人,通过自己的外化把自己现实的、对象性的本质力量设定为异己的对象时,这种设定并不是主体;它是对象性的本质力量的主体性,因此这些本质力量的活动也必须是对象性的活动。对象性的存在物进行对象性活动,如果它的本质规定中不包含对象性的东西,它就不进行对象性活动。它所以只创造或设定对象,因为它本身是被对象所设定的,因为它本来就是自然界。因此,并不是它在设定这一行动中从自己的“纯粹的活动”转而创造对象,而是它的对象性的产物仅仅证实了它的对象性活动,证实了它的活动是对象性的自然存在物的活动。

我们在这里看到,彻底的自然主义或人道主义,既不同于唯心主义,也不同于唯物主义,同时又是把这两者结合的真理。我们同时也看到,只有自然主义能够理解世界历史的行动。
 
人直接地是自然存在物。人作为自然存在物,而且作为有生命的自然存在物,一方面具有自然力、生命力,是能动的自然存在物;这些力量作为天赋和才能、作为欲望存在于人身上;另一方面,人作为自然的、肉体的、感性的、对象性的存在物,同动植物一样,是受动的、受制约的和受限制的存在物,也就是说,他的欲望的对象是作为不依赖于他的对象而存在于他之外的;但这些对象是他的需要的对象;是表现和确证他的本质力量所不可缺少的、重要的对象。说人是肉体的、有自然力的、有生命的、现实的、感性的、对象性的存在物,这就等于说,人是有现实的、感性的对象作为自己本质的即自己生命表现的对象;或者说,人只有凭借现实的、感性的对象才能表现自己的生命。说一个东西是对象性的、自然的、感性的,又说,在这个东西之外有对象、自然界、感觉,或者说,它本身对于第三者来说是对象、自然界、感觉,这都是同一个意思。饥饿是自然的需要;因而为了使自身得到满足,使自身解除饥饿,它需要自身之外的自然界、自身之外的对象。饥饿是我的身体对某一对象的公认的需要,这个对象存在于我的身体之外、是使我的身体得以充实并使本质得以表现所不可缺少的。太阳是植物的对象,是植物所不可缺少的、确证它的生命的对象,正像植物是太阳的对象,是太阳的唤醒生命的力量的表现,是太阳的对象性的本质力量的表现一样。

一个存在物如果在自身之外没有自己的自然界,就不是自然存在物,就不能参加自然界的生活。一个存在如果在自身之外没有对象,就不是对象性的存在物。一个存在物如果本身不是第三存在物的对象,就没有任何存在物作为自己的对象,也就是说,它没有对象性的关系,它的存在就不是对象性的存在。

〔XXVII〕非对象性的存在物是非存在物〔Unwesen〕

假定一种存在物本身既不是对象,又没有对象。这样的存在物首先将是一个唯一的存在物,在它之外没有任何东西存在,它孤零零地独自存在着。因为,只要有对象存在于我之外,只要我不是独自存在着,那末我就是和在我之外存在的对象不同的它物、另一个现实。因而,对这个第三对象来说,我是和他不同的另一个现实,也就是说,我是它的对象。这样,一个存在物如果不是另一个存在物的对象,那末就要以不存在任何一个对象性的存在物存在为前提。只要我有一个对象,这个对象就以我作为它的对象。但是,非对象性的存在物,是一种非现实的、非感性的、只是思想上的,即只是虚构出来的存在物,是抽象的东西。说一个东西是感性的即现实的,这是说,它是感觉的对象,是感性的对象,从而在自己之外有感性的对象,有自己的感性的对象。说一个东西是感性的,就是指它是受动的。

因此,人作为对象性的、感性的存在物,是一个受动的存在物;因为它感到自己是受动的,所以是一个有激情的存在物。激情、热情是人强烈追求自己的对象的本质力量。

但是,人不仅仅是自然存在物,而且是人的自然存在物,就是说,是自为地存在着的存在物,因而是类存在物。他必须既在自己的存在中也在自己的知识中确证并表现自身。因此,正像人的对象不是直接呈现出来的自然对象一样,直接地存在着的、客观地存在着的人的感觉,也不是人的感性、人的对象性。自然界,无论是客观的还是主观的,都不是直接同人的存在物相适合地存在着。

正像一切自然必须产生一样,人也有自己的形成过程即历史,但历史对人来说是被认识到的历史,因而它作为形成过程是一种有意识地扬弃自身的形成过程。历史是人的真正的自然史。——(关于这一点以后还要回过头来谈。)

第三,由于物性的这种设定本身不过是一种外观,一种与纯粹活动的本质相矛盾的行为,所以这种设定必然重新被扬弃,物性必然被否定。

关于第(3)、(4)、(5)、(6)。——(3)意识的这种外化不仅有否定的意义,而且也有肯定的意义。(4)它不仅对我们有这种肯定的意义或者说自在地有这种肯定的意义,而且对它即意识本身也有这种肯定的意义。(5)对象的否定,或对象的自我扬弃,对意识所以有肯定的意义,或者说,它所以知道对象的这种虚无性,是由于意识把自身外化了,因为意识在这种外化中知道自己是对象,或者说,由于自为存在的的不可分割的统一性而知道对象就是它自身。(6)另一方面,这里同时包含着另一个环节,即意识既扬弃这种外化和对象性,同样也把它们收回到自身,因此,它在自己的异在本身中也就是在自身。

我们已经看到,异化的对象性本质的占有,或在异化——它必然从漠不关心的异己性发展到现实的、敌对的异化——这个规定下的对象性的扬弃,在黑格尔看来,同时或甚至主要地具有扬弃对象性的意义,因为并不是对象的一定的性质,而是它的对象性的性质本身,对自我意识来说成为一种障碍和异化。因此,对象是一种否定的东西、自我扬弃的东西,是一种虚无性。对象的这种虚无性对意识来说不仅有否定的意义,而且有肯定的意义,因为对象的这种虚无性正是它自身的非对象性的即〔XXVIII〕抽象的自我确证。对于意识本身来说,对象的虚无性所以有肯定的意义,是因为意识知道这种虚无性、这种对象性本质是它自己的自我外化,知道这种虚无性只是由于它的自我外化才存在……

意识的存在方式,以及对意识说来某个东西的存在方式,这就是知识。知识是意识的惟一的行动。因此,只要意识知道某个东西,那末这个东西对意识来说就生成了。知识是意识的惟一的、对象性的关系。——意识所以知道对象的虚无性,就是说知道对象同它没有区别,对象对它说来是非存在,是因为意识知道对象是它的自我外化,就是说,意识所以知道自己(作为对象的知识),是因为对象只是对象的外观、障眼的云雾,而就它的本质来说不过是知识本身,知识把自己同自身对立起来,从而把某种虚无性,即在知识之外没有任何对象性的某种东西同自己对立起来;或者说,知识知道,当它与某个对象发生关系时,它只是在自身之外,使自身外化;它知道它本身只表现为对象,或者说,对它来说表现为对象的那个东西仅仅是它本身。

另一方面,用黑格尔的话来说,这里同时还包含着另一个环节,即自我意识既扬弃这种外化和对象性,同样也把它们收回到自身,因此,它在自己的异在本身中就是在自身。

这段议论汇集了思辨的一切幻想。

第一,意识、自我意识在自己的异在本身中就是在自身。因此,自我意识——或者,如果我们在这里撇开黑格尔的抽象而用人的自我意识来代替自我意识,——从而可以说人的自我意识在自己的异在本身中,也就是在自身。这里首先包含着:意识,也就是作为知识的知识、作为思维的思维,直接地冒充为异于自身的他物,冒充为感性、现实、生命,——在思维中超越自身的思维(费尔巴哈)。这里所以包含着这一方面,是因为仅仅作为意识的意识,所碰到的障碍不是异化的对象性,而是对象性本身。

第二,这里包含着:因为有自我意识的人认为精神世界——或人的世界在精神上的普遍存在——是自我外化并加以扬弃,所以他仍然重新通过这个外化的形态确证精神世界,把这个世界冒充为自己的真正的存在,恢复这个世界,假称他在自己的异在本身中也就是在自身。因此,在扬弃例如宗教之后,在承认宗教是自我外化的产物之后,他仍然在作为宗教的宗教中找到自身的确证。黑格尔的虚假的实证主义,即他那只是徒有其表的批判主义的根源就在于此,这也就是费尔巴哈所说的宗教或神学的设定、否定和恢复,然而这应当以更一般的形式来表述。因此,理性在作为非理性的非理性中也就是在自身。一个认识到自己在法、政治等等中过着外化生活的人,就是在这种外化生活本身中过着自己的真正的人的生活。因此,与自身相矛盾的,既与知识又与对象的本质相矛盾的自我肯定、自我确证,是真正的知识和真正的生活。

因此,现在不用再谈黑格尔对宗教、国家等等的适应了,因为这种谎言是他的原则的谎言。

〔XXIX〕如果我知道宗教是外化的人的自我意识,那末我也知道,在作为宗教的宗教中得到确证的不是我的自我意识,而是我的外化的自我意识。这就是说,我知道我的属于自身的、属于我的本质的自我意识,不是在宗教中,倒是在被消灭、被扬弃的宗教中得到确证的。

因而,在黑格尔那里,否定的否定不是通过否定假象本质来确证真正的本质,而是通过否定假象本质来确证假象本质,或者说,来确证同自身相异化的本质,换句话说,否定的否定是否定作为在人之外的、不依赖于人的对象性本质的这种假象本质,并使它转化为主体。

因此,把否定和保存即肯定结合起来的扬弃起着一种独特的作用。

例如,在黑格尔法哲学中,扬弃了的私法等于道德,扬弃了的道德等于家庭,扬弃了的家庭等于市民社会,扬弃了的市民社会等于国家,扬弃了的国家等于世界历史。在现实中,私法、道德、家庭、市民社会、国家等等依然存在着,它们只是变成环节,变成人的存在和存在方式,这些存在方式不能孤立地发挥作用,而是互相消融,互相产生等等。它们是运动的环节。

在它们的现实存在中,它们的这种运动的本质是隐蔽的。这种本质只是在思维中、在哲学中才表露、显示出来;因此,我的真正的宗教存在是我的宗教哲学的存在,我的真正的政治存在是我的法哲学的存在,我的真正的自然存在是我的自然哲学的存在,我的真正艺术存在是我的艺术哲学的存在,我的真正的人的存在是我的哲学的存在。同样,宗教、国家、自然界、艺术的真正存在,就是宗教哲学、自然哲学、国家哲学、艺术哲学。但是,如果只有宗教哲学等等对我来说才是真正的宗教存在,那末我就只有作为宗教哲学家才算是真正信教的,而这样一来我就否定了现实的宗教信仰和现实的信教的人。但是我同时又确证了它们:一方面,是在我自己的存在中或在我使之与它们相对立的那个异己的存在中,因为异己的存在仅仅是它们本身的哲学的表现,另一方面,则是在它们自己的最初形式中,因为在我看来它们不过是虚假的异在、譬喻,是隐蔽在感性外壳下面的它们自己的真正存在即我的哲学的存在的形式。

同样地,扬弃了的质等于量,扬弃了的量等于度,扬弃了的度等于本质,扬弃了的本质等于现象,扬弃了的现象等于现实,扬弃了的现实等于概念,扬弃了的概念等于客观性,扬弃了的客观性等于绝对观念,扬弃了的绝对观念等于自然界,扬弃了的自然界等于主观精神,扬弃了的主观精神等于伦理的客观精神,扬弃的伦理精神等于艺术,扬弃了的艺术等于宗教,扬弃了的宗教等于绝对知识。

一方面,这种扬弃是思想上的本质的扬弃,也就是说,思想上的私有财产在道德的思想中的扬弃。而且因为思维自以为直接就是和自身不同的另一个东西,即感性的现实,从而认为自己的活动也是感性的现实的活动,所以这种思想上的扬弃,在现实中没有触动自己的对象,却以为实际上克服了自己的对象;另一方面,因为对象对于思维说来现在已成为一个思维环节,所以对象在自己的现实中也被思维看作思维本身的即自我意识的、抽象的自我确证。

〔XXIX〕因此,从一方面说,黑格尔在哲学中加以扬弃的存在,并不是现实的宗教、国家、自然界、而是已经成为知识的对象的宗教本身,即教义学;法学、国家学、自然科学也是如此。因此,从一方面来说,黑格尔既同现实的本质相对立,也同直接的、非哲学的科学或这种本质的非哲学的概念相对立。因此,黑格尔是同它们的通用的概念相矛盾的。

另一方面,信奉宗教等等的人可以在黑格尔那里找到自己的最后的确证。

现在应该考察——在异化这个规定之内——黑格尔辩证法的积极的环节。

(a)扬弃是把外化收回到自身的、对象性的运动。——这是在异化的范围内表现出来的关于通过扬弃对象性本质的异化来占有对象性本质的见解;这是异化的见解,它主张人的现实的对象化,主张人通过消灭对象世界的异化的规定、通过在对象世界的异化存在中扬弃对象世界而现实地占有自己的对象性本质,正像无神论作为神的扬弃,就是理论的人道主义的生成,而共产主义作为私有财产的扬弃,就是要求归还真正人的生命即人的财产,就是实践的人道主义的生成一样;或者说,无神论是以扬弃宗教作为自己的中介的人道主义,共产主义则是以扬弃私有财产作为自己的中介的人道主义。只有通过扬弃这种中介——但这种中介是一个必要的前提——积极地从自身开始的即积极的人道主义才能产生。

然而,无神论、共产主义绝不是人所创造的对象世界的消逝、舍弃和丧失,即绝不是人的采取对象形式的本质力量的消逝、舍弃和丧失,绝不是返回到非自然的、不发达的简单状态去的贫困。恰恰相反,它们倒是人的本质的或作为某种现实东西的人的本质的现实的生成,对人来说的真正的实现。

这样,因为黑格尔理解到——尽管又是通过异化的方式——有关自身的否定具有的积极意义,所以同时也把人的自我异化、人的本质的外化、人的非对象化和非现实化理解为自我获得、本质的表现、对象化、现实化。简单地说,他——在抽象的范围内——把劳动理解为人的自我产生的行动,把人对自身的关系理解为对异己存在物的关系,把作为异己存在物的自身的实现理解为生成着的类意识和类生活。

(b)但是,撇开上述颠倒的说法不谈,或者更正确地说,作为上述颠倒说法的结果,在黑格尔看来,这种行动,第一,仅仅具有形式的性质,因为它是抽象的,因为人的本质本身仅仅被看作抽象的、思维着的本质,即自我意识。

第二,因为这种观点是形式的和抽象的,所以外化的扬弃成为外化的确证,或者说,在黑格尔看来,自我产生、自我对象化的运动,作为自我外化和自我异化的运动,是绝对的因而也是最后的、以自身为目的的、安于自身的、达到自己本质的人的生命表现。

因此,这个运动在其抽象〔XXXI〕形式上,作为辩证法,被看成真正人的生命;而因为它毕竟是人的生命的抽象、异化,所以它被看成神性的过程,然而是人的神性的过程,——一个与人自身有区别的、抽象的、纯粹的、绝对的本质本身所经历的过程。

第三,这个过程必须有一个承担者、主体;但主体只作为结果出现;因此,这个结果,即知道自己是绝对自我意识的主体,就是神,就是绝对精神,就是知道自己并且实现自己的观念。现实的人和现实的自然界不过是成为这个隐蔽的非现实的人和这个非现实的自然界的谓语、象征。因此,主语和宾语之间的关系被绝对地相互颠倒了:这就是神秘的主体——客体,或笼罩在客体上的主体性,作为过程的绝对主体,作为使自身外化并且从这种外化返回到自身的、但同时又把外化收回到自身的主体,以及作为这一过程的主体;这就是在自身内部的纯粹的、不停息的旋转。

关于第一点:对人的自我产生的或自我对象化的行动的形式的和抽象的理解。

因为黑格尔把人和自我意识等同起来,所以人的异化了的对象,人的异化了的、本质的现实性,不外就是异化的意识,只是异化的思想,是异化的抽象的因而无内容的和非现实的表现,即否定。因此,外化的扬弃也不外是对这种无内容的抽象进行抽象的、无内容的扬弃,即否定的否定。因此,自我对象化的内容丰富的、活生生的、感性的、具体的活动,就成为这种活动的纯粹抽象——绝对的否定性,而这种抽象也被抽象地固定下来并且被想象为独立的活动,即干脆被想象为活动。因为这种所谓否定性无非就是上述现实的、活生生的行动的抽象的无内容的形式,所以它的内容也只能是形式的、抽去一切内容而产生的内容。因此,这就是普遍的,抽象的,适合任何内容的,从而既超脱任何内容同时又恰恰对任何内容都有效的,脱离现实的精神和现实的自然界的抽象形式、思维形式、逻辑范畴。(下文我们将阐明绝对的否定性的逻辑内容。)

黑格尔在这里——在它的思辨的逻辑学里——所完成的积极的东西在于;独立于自然界和精神的特定概念、普遍的固定的思维形式,是人的本质普遍异化的必然结果,因而也是人的思维普遍异化的必然结果;因此,黑格尔把它们描绘成抽象过程的各个环节,并且把它们连贯起来了。例如,扬弃了的存在是本质,扬弃了的本质是概念,扬弃了的概念……是绝对观念。然而,绝对观念究竟是什么呢?如果绝对观念不愿意再去重头经历全部抽象活动,不想再满足于充当种种抽象的总体或充当理解自我的抽象,那末,绝对观念也要再一次扬弃自身。但是,把自我理解为抽象的抽象,知道自己是无;它必须放弃自身,放弃抽象,从而达到那恰恰是它的对立面的本质,达到自然界。因此,全部逻辑学都证明,抽象思维本身是无,绝对观念本身是无,只有自然界才是某物。

〔XXXII〕绝对观念、抽象观念

\begin{fangsong}
“从它与自身统一这一方面来考察就是直观”(黑格尔《全书》第3版第222页),它“在自己的绝对真理中决心把自己的特殊性这一环节,或最初的规定和异在这一环节,即作为自己的反应的直接观念,从自身释放出去,也就是说,把自身作为自然界从自身释放出去”
\end{fangsong}

举动如此奇妙而怪诞、使黑格尔分子伤透了脑筋的这整个观念,无非始终是抽象,即抽象思维者,这种抽象由于经验而变得聪明起来,并且弄清了它的真相,于是在某些——虚假的甚至还是抽象的——条件下决心放弃自身,而用自己的异在,即特殊的东西、特定的东西,来代替自己的在自身的存在(非存在),代替自己的普遍性和不确定性;决心把那只是作为抽象、作为思想物而隐藏在它里面的自然界从自身释放出去,也就是说,决心抛弃抽象而去看一看摆脱了它的自然界。直接成为直观的抽象观念,无非始终是那种放弃自身并且决心成为直观的抽象思维。从逻辑学到自然哲学的这整个过度,无非是对抽象思维者来说如此难以实现、因而由他作了如此牵强附会的描述的从抽象到直观的过渡。有一种神秘的感觉驱使哲学家从抽象思维转向直观,那就是厌烦,就是对内容的渴望。

(同自身相异化的人,也就是同自己的本质即同自己的自然的和人的本质相异化的思维者。因此,他的那些思想是居于自然界和人之外的僵化的精灵。黑格尔把这一切僵化的精灵统统禁锢在他的逻辑学里,先是把它们每一个都看成否定,即人的思维的外化,然后又把它们看成否定的否定,即看成这种外化的扬弃,看成人的思维的现实的表现;但是,这种否定的否定——尽管仍然被束缚在异化中——,它一部分是使原来那些僵化的精灵在它们的异化中恢复,一部分是停留于最后的行动中,也就是在作为这些僵化的精灵的真实存在的外化中自身同自身相联系\footnote{这就是说,黑格尔用那在自身内部绕圈的抽象行动来代替这些僵化的抽象概念;于是,他就有了这样的贡献:他指明了原来属于各个哲学家的一切不适当的概念的诞生地,把他们综合起来,并且创造出一个在自己整个范围内穷尽一切的抽象作为批判的对象,以代替某种特定的抽象。(我们在下面将会看到,黑格尔为什么把思维同主体分隔开来;但就是现在也已经很清楚:如果没有人,那末人的本质表现也不可能是人的,因此思维也不能被看作是人的本质表现,即在社会、世界和自然界生活的有眼睛、耳朵等等的人和自然的主体的本质表现。)
};一部分则由于这种抽象理解了自身并且对自身感到无限的厌烦,所以,在黑格尔那里放弃抽象的、只在思维中运动的思维,即无眼、无牙、无耳、无一切的思维,便表现为决心承认自然界是本质并且转而致力于直观。)

〔XXXIII〕但是,被抽象地理解的,自为的,被确定为与人分割开来的自然界,对人来说也是无。不言而喻,这位决心转向直观的抽象思维者是抽象地直观自然界的。正向自然界曾经被思维者禁锢于他的这种对他来说也是隐密的和不可思议的形式即绝对观念、思想物中一样,现在,当他把自然界从自身释放出去时,他实际上从自身释放出去的只是这个抽象的自然界,只是自然界的思想物——不过现在具有这样一种意义,即这个自然界是思想的异在,是现实的、被直观的、有别于抽象思维的自然界——,只是自然界的思想物。或者,如果用人的语言来说,抽象思维者在他直观自然界时了解到,他在神性的辩证法中以为是从无、从纯抽象中创造出来的那些本质——在自身中转动的并且在任何地方都不向现实看一看的思维劳动的纯粹产物——无非是自然界诸规定的抽象概念。因此,对他来说整个自然界不过是在感性的、外在的形式下重复逻辑的抽象概念而已。他重新把自然界分解为这些抽象概念。因此,他对自然界的直观不过是他把对自然界的直观抽象化的确证行动,不过是他有意识地重复他的抽象概念的产生过程。例如,时间等于自身同自身相联系的否定性(前引书;第238页)。扬弃了的运动即物质——在自然形式中——同扬弃了的生成即定在相符合。光是反射于自身的自然形式。像月亮和彗星这样的物体,是对立物的自然形式,按照《逻辑学》,这种对立物一方面是以自身为根据的肯定的东西,而另一方面又是以自身为根据的否定的东西。地球是作为对立物的否定性统一的逻辑根据的自然形式,等等。

作为自然界的自然界,这是说,就它还在感性上不同于它自身所隐藏的神秘的意义而言,与这些抽象概念分割开来并并与这些抽象概念不同的自然界,就是无,是证明自己为无的无,是无意义的,或者只具有应被扬弃的外在性的意义。

\begin{fangsong}
“有限的目的论的观点包含着一个正确的前提,即自然界本身并不包含着绝对的目的。”(第225页)
\end{fangsong}

自然界的目的就在于对抽象的确证。

\begin{fangsong}
“结果自然界成为具有异在形式的观念。既然观念在这里表现为对自身的否定或外在于自身的东西,那末自然界并非只在相对的意义上对这种观念说来是外在的,而是外在性构成这样的规定,观念在其中表现为自然界。”(第227页)
\end{fangsong}

这里不应该把外在性理解为显露在外的并且对光、对感性的人敞开的感性;在这里应该把外在性理解为外化,理解为不应有的偏差、缺陷。因为真实的东西毕竟是观念。自然界不过是观念的异在的形式。而既然抽象思维是本质,那末外在于它的东西,就其本质来说,不过是某种外在的东西。抽象思维者同时承认感性、同在自身中转动的思维相对立的外在性,是自然界的本质。但是,它同时又把这种对立说成这样,即自然界的这种外在性,自然界同思维的对立,是自然界的缺陷;就自然界不同于抽象而言,自然界是个有缺陷的存在物。〔XXXIV〕一个不仅对我来说、在我的眼中有缺陷而且本身就有缺陷的存在物,在它自身之外有一种为它所缺少的东西。这就是说,它的本质是不同于它自身的另一种东西。因此,对抽象思维者说来,自然界必须扬弃自身,因为他已经把自然界设定为潜在地被扬弃的本质。

\begin{fangsong}
“对我们来说,精神以自然界为自己的前提,精神是自然界的真理,因而对自然界来说,精神也是某种绝对第一性的东西。在这个真理中自然界消逝了,结果精神成为达到其自为的存在的观念,而概念则既是观念的客体,又是观念的主体。这种同一性是绝对的否定性,因为概念在自然界中有自己的完满的外在的客观性,但现在它的这种外化被扬弃了。而概念在这种外化中成了与自己同身的东西。因此,概念只有作为从自然界的回归才是这种同一性。”(第392页)

“启示,作为抽象观念,是向自然界的直接的过渡,是自然界的生成,而作为自由精神的启示,则是自由精神把自然界设定为自己的世界,——这种设定,作为反思,同时又是把世界假定为独立的自然界。概念中的启示,是精神把自然界创造为自己的存在,而精神在这个存在中获得自己的自由的确证和真实性。”“绝对的东西是精神;这是绝对的东西的最高定义。”〔XXXIV〕
\end{fangsong}

\newpage
\subsection{2.笔记}
马克思认为私有财产的关系是劳动、资本以及二者的关系。他指出劳动与资本之间的关系经历了三个阶段:(1)直接的或间接的统一(2)对立(3)各自同自身的对立。

在这里(1)和(2)是很好理解的:资本和劳动之间的关系最初便是一体的,资本就是劳动,劳动就是资本,而随着历史的发展,资本与劳动之间显现出了对立性的关系,劳动服从于资本。但是为什么马克思又说劳动与资本又会经历各自同自身相对立的阶段呢?对于(3)而言,马克思原文是这么叙述的:

\begin{fangsong}
    “〔第三〕:二者各自同自身对立。资本=积累的劳动=劳动。作为这样的
东西,资本分解为自身和自己的利息,而利息又分解为利息和利润。资本家
完全成为牺牲品。他沦为工人阶级,正像工人-但只是例外地-成为资本
家一样。劳动是资本的要素,是资本的费用,因而,工资是资本的牺牲。”
\end{fangsong}

事实上,在分析这段话的时候,我们要明白马克思在这一语境中所规定的前提,马克思在指出劳动与资本之间关系所经历的阶段之时,他是在\textbf{私有财产}这一框架内讲述的。也即是说,在(1)(2)(3)这三个阶段中,劳动与资本之间发生的关系一直处于\textbf{私有财产关系}的自身运动之中。

因此,我们可以这么理解马克思所指出的(3)这一阶段,即(3)是(2)的尖锐化。马克思在这里并没有认为对于\textbf{私有财产}而言,它自身蕴含着解决问题的答案,相反,马克思认为私有财产的自我运动所指向的是其自身的毁灭(即自我矛盾的不可克服)。换句话说,马克思认为私有财产的发展本身就是一种历史的异化过程,资本起初作为一种人的力量(资本与劳动的统一)逐步走向同人的对立。在这一对立的过程中,资本起初是同作为代表着劳动的劳动者相对立,后来甚至同\textbf{一切人}相对立:1.就劳动者阶级而言,资本对于劳动的需求导致劳动者之间互相竞争,劳动阶级内部便产生了互相对立的因素。2.就资产阶级而言,资本的逐利性导致资本之间的斗争,资本家阶级内部便出现了区域性的斗争\footnote{诚然,在面对阶级整体利益时,资产阶级还是比较团结的,但是他们内部的斗争依旧存在着,因为对于资产阶级中的个体而言,\textbf{竞争的目的就是为了垄断},这在现实社会中已经有许多经验性的例子了。},部分资本家便作为牺牲品而下落到无产阶级的队伍中。

因此,从这个意义上来说,马克思甚至不仅仅是站在无产阶级的立场上去谈论私有财产的运动,而是站在整个人类社会的角度上谈论私有财产的运动。在私有财产的框架下,资本不仅对于无产者而言是一种束缚,且对于资本家而言也是他们的枷锁。\textbf{因为一切人都被异化了。}

马克思认为在国民经济学中反映的是私有财产的本质,这和恩格斯的《国民经济学批判大纲》中的观点类似。私有财产的主体本质就是劳动,也即是说私有财产的本质是某种\textbf{人的属性}。因此,马克思进一步指出,那些仅仅认为私有财产是某种对人而言的对象物的观念是\textbf{拜物教}。在这里马克思的原文是这么说的:

\begin{fangsong}
“因此,
在这种揭示了——在私有制范围内——财富的主体本质的启蒙国民经济学看来,
那些认为私有财产对人来说仅仅是对象性的本质的货币主义体系和重商主
义体系地拥护者,是拜物教徒、天主教徒。所以,恩格斯有理由把亚当·斯
密称作国民经济学的路德。正像路德认为宗教、信仰为外部世界的本质并
以此反对天主教异教一样,正像他把宗教观念变成人的内在本质,从而扬
弃了外在的宗教观念一样,正像他把教士移到俗人心中,因而否定了俗人
之外存在的教士一样,由于私有财产体现为在人本身中,而人本身被认为
是私有财产的本质,因而在人之外并且不依赖于人的财富,也就是只以外
在方式来保存和保持的财富被扬弃了,换言之,财富这种外在的、无思想的
对象性就被扬弃了,但正因为这个缘故,人本身被当成了私有财产的规定,
就像在路德那里被当成了宗教的规定一样。”
\end{fangsong}

马克思在这里用了类比的手法形象地说明了私有财产与人之间的关系。私有财产与人之间的关系恰如宗教与人之间的关系,私有财产的神秘面纱——即它的\textbf{至高性}\footnote{这里笔者用了“至高性”一词,可以这么理解:私有财产相对于人而言所具有控制力,似乎它天生就具有某种对人的控制力量,仿佛至高的天神一样。}原则,并非某种\textbf{脱离人之外}的虚幻力量,相反,私有财产的本质就是人的本质,国民经济学做的工作正是揭露了私有财产与人之间关系的面纱。但是国民经济学所做的工作并不是真正具有革命性的,因为它仅仅揭露了一种既定的现实,而并没有对这种现实进行更深入的批判。国民经济学使私有财产与人之间的\textbf{分离}转变成了私有财产与人之间的\textbf{统一},但这种\textbf{统一}是\textbf{颠倒的统一}。因此,对于国民经济学自身而言,它是矛盾的、伪善的,它没有看到它所声称的私有财产与人的统一与现实的私有财产与人的统一二者是主客颠倒的,它声称财富的本质是人的本质,承认人的独立性与个性,但现实情况是人的本质似乎变成了财富的本质,人不具备独立性与个性。我们可以很清晰地看到马克思在原文中是这么叙述的:

\begin{fangsong}
“……因此,以劳动为原则的国民经
济学,在承认人的假象下,毋宁说不过是彻底实现对人的否定而已,因为
人本身已不再同私有财产的外在本质处于外部的紧张关系中,而人本身却成了私有财产的这种紧张的本质。以前是人之外的存在——人的实际外化
——的东西,现在仅仅变成外化的行为,变成了外在化。因此,如果说上述
国民经济学是从表面上承认人、人的独立性、自主活动等等开始,并由于把
私有财产转为人自身的本质而能够不再束缚于作为存在于人之外的本质的
私有财产的那些地域性的、民族的等等的规定,从而发挥一种世界主义的、
普遍的、摧毁一切界限和束缚的能量,以便自己作为惟一的政策、普遍性、
界限和束缚取代这些规定,-那末,国民经济学在它往后的发展过程中必
定抛弃这种伪善性,而使自己的犬儒主义充分表现出来。”
\end{fangsong}



\end{document}
